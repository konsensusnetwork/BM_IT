% Options for packages loaded elsewhere
\PassOptionsToPackage{unicode}{hyperref}
\PassOptionsToPackage{hyphens}{url}
%
\documentclass[
  a5paper,
  smalldemyvopaper,10pt,twoside,onecolumn,openright,extrafontsizes,hidelinks]{memoir}

\usepackage{amsmath,amssymb}
\usepackage{iftex}
\ifPDFTeX
  \usepackage[T1]{fontenc}
  \usepackage[utf8]{inputenc}
  \usepackage{textcomp} % provide euro and other symbols
\else % if luatex or xetex
  \usepackage{unicode-math}
  \defaultfontfeatures{Scale=MatchLowercase}
  \defaultfontfeatures[\rmfamily]{Ligatures=TeX,Scale=1}
\fi
\usepackage{lmodern}
\ifPDFTeX\else  
    % xetex/luatex font selection
\fi
% Use upquote if available, for straight quotes in verbatim environments
\IfFileExists{upquote.sty}{\usepackage{upquote}}{}
\IfFileExists{microtype.sty}{% use microtype if available
  \usepackage[]{microtype}
  \UseMicrotypeSet[protrusion]{basicmath} % disable protrusion for tt fonts
}{}
\makeatletter
\@ifundefined{KOMAClassName}{% if non-KOMA class
  \IfFileExists{parskip.sty}{%
    \usepackage{parskip}
  }{% else
    \setlength{\parindent}{0pt}
    \setlength{\parskip}{6pt plus 2pt minus 1pt}}
}{% if KOMA class
  \KOMAoptions{parskip=half}}
\makeatother
\usepackage{xcolor}
\setlength{\emergencystretch}{3em} % prevent overfull lines
\setcounter{secnumdepth}{5}
% Make \paragraph and \subparagraph free-standing
\makeatletter
\ifx\paragraph\undefined\else
  \let\oldparagraph\paragraph
  \renewcommand{\paragraph}{
    \@ifstar
      \xxxParagraphStar
      \xxxParagraphNoStar
  }
  \newcommand{\xxxParagraphStar}[1]{\oldparagraph*{#1}\mbox{}}
  \newcommand{\xxxParagraphNoStar}[1]{\oldparagraph{#1}\mbox{}}
\fi
\ifx\subparagraph\undefined\else
  \let\oldsubparagraph\subparagraph
  \renewcommand{\subparagraph}{
    \@ifstar
      \xxxSubParagraphStar
      \xxxSubParagraphNoStar
  }
  \newcommand{\xxxSubParagraphStar}[1]{\oldsubparagraph*{#1}\mbox{}}
  \newcommand{\xxxSubParagraphNoStar}[1]{\oldsubparagraph{#1}\mbox{}}
\fi
\makeatother


\providecommand{\tightlist}{%
  \setlength{\itemsep}{0pt}\setlength{\parskip}{0pt}}\usepackage{longtable,booktabs,array}
\usepackage{calc} % for calculating minipage widths
% Correct order of tables after \paragraph or \subparagraph
\usepackage{etoolbox}
\makeatletter
\patchcmd\longtable{\par}{\if@noskipsec\mbox{}\fi\par}{}{}
\makeatother
% Allow footnotes in longtable head/foot
\IfFileExists{footnotehyper.sty}{\usepackage{footnotehyper}}{\usepackage{footnote}}
\makesavenoteenv{longtable}
\usepackage{graphicx}
\makeatletter
\def\maxwidth{\ifdim\Gin@nat@width>\linewidth\linewidth\else\Gin@nat@width\fi}
\def\maxheight{\ifdim\Gin@nat@height>\textheight\textheight\else\Gin@nat@height\fi}
\makeatother
% Scale images if necessary, so that they will not overflow the page
% margins by default, and it is still possible to overwrite the defaults
% using explicit options in \includegraphics[width, height, ...]{}
\setkeys{Gin}{width=\maxwidth,height=\maxheight,keepaspectratio}
% Set default figure placement to htbp
\makeatletter
\def\fps@figure{htbp}
\makeatother

% typographical packages
\usepackage{microtype}
\usepackage{setspace}
\tolerance=6000
\hyphenpenalty=1000

% typographical settings for the body text
\setlength{\parskip}{0em}
\setlength{\parindent}{1em}
\linespread{1.3}

% DEFINITIONS TITLE PAGE / COPYRIGHT
\newcommand{\titleoriginal}{Title}
\newcommand{\subtitleoriginal}{Subtitle}
\newcommand{\yearoriginal}{Year}
\newcommand{\subtitletranslation}{Translation Subtitle}
\newcommand{\yeartranslation}{Translation Year}
\newcommand{\stringtranslation}{Translation String}
\newcommand{\stringlicense}{Translation License String.}
\newcommand{\stringpublisher}{Published by String}
\newcommand{\ISBNHC}{978-9916-}
\newcommand{\ISBNSC}{978-9916-}
\newcommand{\ISBNEBOOK}{978-9916-}
%\newcommand{\ISBNAUDIO}{978-9916-}
\newcommand{\press}{Konsensus Network}
\newcommand{\translatorone}{Translator 1}
\newcommand{\translators}{
\large\textit{\stringtranslation:}\\
\translatorone\\
}

% PHYSICAL DOCUMENT SETUP
\setstocksize{210mm}{148mm}
\settrimmedsize{210mm}{148mm}{*}
\setbinding{5mm}
\setlrmarginsandblock{8mm}{15mm}{*}
\setulmarginsandblock{25mm}{26mm}{*}

% FONTS
\usepackage{fontspec}
\setmainfont{stone-serif}[
    Path=./fonts/stone-serif-itc-pro/,
    Scale=0.86,
    Extension=.OTF,
    UprightFont=*-Regular,
    BoldFont=*-SemiBd,
    ItalicFont=*-MediumIt,
    BoldItalicFont=*-SemiBdIt
    ]

\setsansfont{stone-sans}[
    Path=./fonts/stone-sans/,
    Scale=0.82,
    Extension=.otf,
    UprightFont=*-Medium,
    BoldFont=*-Semibold,
    ItalicFont=*-MediumItalic,
    BoldItalicFont=*-SemiBoldItalic
    ]

\usepackage{lettrine}
\setcounter{DefaultLines}{3}
\renewcommand{\DefaultLoversize}{0.1}
\renewcommand{\DefaultLraise}{0}
\renewcommand{\LettrineTextFont}{}
\setlength{\DefaultFindent}{\fontdimen2\font}
\setlength{\DefaultNindent}{0em}

%\usepackage[footskip=8mm]{geometry}

% custom second title page
\makeatletter
\newcommand*\halftitlepage{\begingroup % Misericords, T&H p 153
  \setlength\drop{0.1\textheight}
  \begin{center}
  \vspace*{\drop}
  \rule{\textwidth}{0in}\par
  {\Large\sffamily\thetitle\par}
  \rule{\textwidth}{0in}\par
  \vfill
  \end{center}
\endgroup}
\makeatother

% custom title page
\makeatletter
\newlength\drop
\newcommand*\titleM{\begingroup % Misericords, T&H p 153
  \setlength\drop{0.15\textheight}
  \begin{center}
  \vspace*{\drop}
  {\huge\sffamily\thetitle\par}
  \vspace{2em}
  %{\normalsize\sffamily\textit\subtitletranslation\par}
  %\vspace{2em}
  \rule{5.5cm}{0.3mm}\par
  \vspace{2em}
  {\Large\sffamily\textit\theauthor\par}
  \vspace{3em}
  %{\footnotesize\sffamily\textit\translators\par}
  \vfill
  \includegraphics[width=3.5cm]{figures/knw.png}\par
  \end{center}
\endgroup}
\makeatother

% copyright page
\makeatletter
\newcommand*\copyrightpage{\begingroup
  \setlength\drop{0.1\textheight}
  \vphantom{just for the drop}
  \vfill
  \begin{scriptsize}
  \noindent \copyright\space \yearoriginal: \theauthor
  \par\noindent \textit{\titleoriginal}
  \vspace{0.5\baselineskip}
  %\par\noindent \copyright\space \yeartranslation\space \stringtranslation: \translatorone
  %\par\noindent \textit{\thetitle: \subtitletranslation}
  \vspace{\baselineskip}
  \par\noindent \textit{\stringlicense}
  \vspace{0.5\baselineskip}
  \par\noindent \stringpublisher: \href{https://konsensus.network}{\textit{konsensus.network}}
  \vspace{0.5\baselineskip}
  \par\noindent v1.0.0
  \vspace{0.5\baselineskip}
  \setlength{\parindent}{2em}% default 20pt
  \par\noindent ISBN \ISBNHC \:Hardcover
  \par\hspace{0.28\parindent}\ISBNSC \:Paperback
  \par\hspace{0.28\parindent}\ISBNEBOOK \:E-book\par
  \setlength{\parindent}{0pt}
  \end{scriptsize}
  \vspace{3em}
  \par\noindent \href{https://konsensus.network}{\large\MakeUppercase \press \hspace{3em} \includegraphics[width=1cm]{figures/freestarfish.png}}
  \setcounter{footnote}{0}
  \clearpage
\endgroup}
\makeatother

% HEADER AND FOOTER MANIPULATION
% for normal pages
\nouppercaseheads
\headsep = 10mm
\makepagestyle{mystyle} 
\makeevenhead{mystyle}{\scriptsize\sffamily\mdseries\thepage}{}{}
\makeoddhead{mystyle}{{\scriptsize\sffamily\mdseries\leftmark}}{}{\scriptsize\sffamily\mdseries\thepage}
\makeevenfoot{mystyle}{}{}{}
\makeoddfoot{mystyle}{}{}{}
\makeatletter

% for pages where chapters begin
\makepagestyle{plain}
\makerunningwidth{plain}{\headwidth}
\makeevenfoot{plain}{}{}{}
\makeoddfoot{plain}{}{\scriptsize\sffamily\mdseries\thepage}{}
\pagestyle{mystyle}

\newif\ifmainmatter
\appto\mainmatter{\mainmattertrue}
\appto\backmatter{\mainmatterfalse}
\appto\appendix{\mainmatterfalse}

\renewcommand\chaptermark[1]{%
  \markboth{\MakeUppercase{%
    \ifmainmatter~\oldstylenums\thechapter.~\fi#1}}{}}%

% TOC
\usepackage[]{tocloft}
\renewcommand{\cftsectiondotsep}{\cftnodots}
\renewcommand{\cftpartfont}{\Large\sffamily\MakeUppercase}
\renewcommand{\cftchapterfont}{\small\sffamily}
\renewcommand{\cftsectionfont}{\Small\sffamily}
\renewcommand{\cftpartpagefont}{\Large\sffamily}
\renewcommand{\cftchapterpagefont}{\small}
\renewcommand{\cftchapterpresnum}{HOOFDSTUK\space}
\renewcommand{\cftchapternumwidth}{7em}
\setlength{\cftchapterindent}{0em}
\setlength{\cftsectionindent}{7em}
\setlength{\cftbeforechapterskip}{-0.2em}
\setsecnumdepth{chapter}
\setcounter{tocdepth}{0}


% Redefine footnote presentation
\makeatletter
\renewcommand\@makefntext[1]{%
  \noindent\hb@xt@2em{% <-- Box of fixed size for footnote number and space
    \@thefnmark\quad}% <-- Footnote number followed by a quad space
  \parbox[t]{\dimexpr\linewidth-2em}{#1}% <-- Parbox to control the width of footnote content
}
\makeatother

% layout check and fix
\checkandfixthelayout

% COUNTERS FOOTNOTES
\usepackage{chngcntr}
\counterwithout*{footnote}{chapter}

% TITLE FORMATTING
\usepackage{titlesec}
\titleformat
    {\chapter}[display]
    {\huge\sffamily}
    {\Large\sffamily\chaptertitlename\space\thechapter}
    {0pt}
    {\vspace{28pt}}

\titleformat
  {\section}[block]
  {\sffamily\large\bfseries}
  {}
  {0pt}
  {}
  
\titlespacing*{\section}{0pt}{2em}{0.5em}

\titleformat{\subsection}{\sffamily\bfseries}{}{}{}
\titlespacing*{\subsection}{0pt}{2em}{0em}

% QUOTE FORMATTING
\renewenvironment{quote}%
               {\list{}{\rightmargin=.6cm\leftmargin=.6cm}%
                \itshape \item[]}% <- The effect of \samepage is local!!!
               {\endlist}

% LAYOUT CHECK AND FIX
\checkandfixthelayout

% CUSTOM TITLE PAGE
\makeatletter
\def\@maketitle{%
  % the half title page
  \pagestyle{empty}
  \halftitlepage
  \cleardoublepage

  % the title page
  \titleM
  \clearpage

  % the copyright page
  \copyrightpage
  \cleardoublepage
  \pagestyle{mystyle}
}
\makeatother
% END PREAMBLE
\makeatletter
\@ifpackageloaded{bookmark}{}{\usepackage{bookmark}}
\makeatother
\makeatletter
\@ifpackageloaded{caption}{}{\usepackage{caption}}
\AtBeginDocument{%
\ifdefined\contentsname
  \renewcommand*\contentsname{Indice}
\else
  \newcommand\contentsname{Indice}
\fi
\ifdefined\listfigurename
  \renewcommand*\listfigurename{Elenco delle Figure}
\else
  \newcommand\listfigurename{Elenco delle Figure}
\fi
\ifdefined\listtablename
  \renewcommand*\listtablename{Elenco delle Tabelle}
\else
  \newcommand\listtablename{Elenco delle Tabelle}
\fi
\ifdefined\figurename
  \renewcommand*\figurename{Figura}
\else
  \newcommand\figurename{Figura}
\fi
\ifdefined\tablename
  \renewcommand*\tablename{Tabella}
\else
  \newcommand\tablename{Tabella}
\fi
}
\@ifpackageloaded{float}{}{\usepackage{float}}
\floatstyle{ruled}
\@ifundefined{c@chapter}{\newfloat{codelisting}{h}{lop}}{\newfloat{codelisting}{h}{lop}[chapter]}
\floatname{codelisting}{Lista}
\newcommand*\listoflistings{\listof{codelisting}{Elenco degli Elenchi}}
\makeatother
\makeatletter
\makeatother
\makeatletter
\@ifpackageloaded{caption}{}{\usepackage{caption}}
\@ifpackageloaded{subcaption}{}{\usepackage{subcaption}}
\makeatother

\ifLuaTeX
\usepackage[bidi=basic]{babel}
\else
\usepackage[bidi=default]{babel}
\fi
\babelprovide[main,import]{italian}
% get rid of language-specific shorthands (see #6817):
\let\LanguageShortHands\languageshorthands
\def\languageshorthands#1{}
\ifLuaTeX
  \usepackage{selnolig}  % disable illegal ligatures
\fi
\usepackage{bookmark}

\IfFileExists{xurl.sty}{\usepackage{xurl}}{} % add URL line breaks if available
\urlstyle{same} % disable monospaced font for URLs
\hypersetup{
  pdftitle={Book Title},
  pdfauthor={Book Author},
  pdflang={it},
  hidelinks,
  pdfcreator={LaTeX via pandoc}}


\title{Book Title}
\author{Book Author}
\date{+0100-01-01}

\begin{document}
\frontmatter
\maketitle

\renewcommand*\contentsname{Inhoudsopgave}
{
\setcounter{tocdepth}{0}
\tableofcontents
}

\mainmatter
\bookmarksetup{startatroot}

\chapter*{Over dit boek}\label{over-dit-boek}
\addcontentsline{toc}{chapter}{Over dit boek}

\markboth{Over dit boek}{Over dit boek}

\bookmarksetup{startatroot}

\chapter{Introduzione}\label{introduzione}

Nel settembre 2022 un'ondata di persone comuni si è spinta a rapinare
banche in Libano. Ciò che rese questi eventi particolarmente degni di
nota, rispetto alle classiche rapine in banca, fu il fatto che la
maggior parte degli individui coinvolti intendeva semplicemente
recuperare i propri risparmi. A causa di una crisi finanziaria, le
istituzioni bancarie libanesi avevano infatti impedito ai clienti di
accedere al denaro depositato per lunghi periodi.

Tra questi casi spicca quello di una giovane decoratrice d'interni, che
fece notizia dopo aver aggredito una banca a Beirut utilizzando quella
che si rivelò essere una pistola finta dall'aspetto sorprendentemente
realistico. Il suo intento era prelevare i risparmi della famiglia,
necessari per finanziare il trattamento della sorella afflitta da una
grave forma di cancro, poiché i fondi erano stati bloccati dall'istituto
bancario. Sebbene questo episodio rappresenti forse l'esempio più
eclatante, si sono verificati numerosi altri casi, alcuni dei quali con
l'impiego anche di armi reali, in cui gli investitori cercavano
semplicemente di recuperare il denaro che legalmente gli spettava.

Questi eventi, pur essendo strettamente legati a un determinato contesto
nazionale e temporale, rientrano in una narrazione di portata globale.

La Nigeria, un paese con oltre 200 milioni di abitanti, ha registrato un
tasso d'inflazione annualizzato del 13\% nell'ultimo
decennio.\footnote{FMI, ``Prezzi al consumo, fine periodo,'' Datamapper.}
Nel 2021 è stata lanciata una valuta digitale della banca centrale, la
eNaira, che fino ad oggi ha riscosso un'adozione estremamente limitata,
mentre le criptovalute -- in particolare bitcoin e le stablecoin
ancorate al dollaro statunitense -- hanno registrato un tasso di
adozione da un ordine di grandezza superiore all'interno del paese,
nonostante il taglio dei collegamenti con il sistema bancario nazionale.
Successivamente, il governo nigeriano ha adottato una serie di politiche
volte a ridurre la disponibilità di contanti fisici e a incentivare
l'uso dei pagamenti digitali, operazione che ha contribuito a generare
un periodo di turbolenze politiche e rivolte sociali.

L'Egitto, nell'autunno del 2016, ha improvvisamente dimezzato il valore
della propria moneta rispetto al dollaro statunitense, costringendo
circa 100 milioni di persone a vedere annientati anni di risparmi
accumulati. Negli anni 2022 e 2023, il paese ha nuovamente effettuato
ripetuti e marcati devalutazioni della propria valuta rispetto al
dollaro, determinando un ulteriore dimezzamento del tasso di cambio.
Conosco persone in Egitto che acquistano dollari fisici sul mercato nero
e li detengono come protezione contro questo fenomeno ricorrente. In
questo processo, essi devono corrispondere considerevoli commissioni di
conversione, senza tuttavia percepire alcun interesse sui dollari
cartacei posseduti. Quando si verificano tali devalutazioni, spetta
immediatamente a tutti i lavoratori nel paese cercare di negoziare
stipendi maggiori, al fine di recuperare parte del potere d'acquisto
perso, poiché i loro compensi continuano ad essere erogati nella valuta
locale svalutata.

Sia la Turchia che l'Argentina, membri del G20 e con una popolazione
complessiva superiore a 130 milioni di abitanti, stanno affrontando
negli ultimi anni un'inflazione incontrollata. La Turchia ha raggiunto
un tasso d'inflazione del 85\% su base annua nel 2022, mentre in
Argentina l'inflazione ha superato ben il 100\% nel 2023.\footnote{Zeynep
  Dierks, ``Tasso di inflazione CPI in Turchia,'' \emph{Statista}, 3
  marzo 2023;
  \href{https://www.bloomberg.com/authors/ATpJM4rxzpI/patrick-gillespie}{Patrick~Gillespie},
  ``L'inflazione in Argentina supera il 100\% mentre incombe una
  recessione economica,'' \emph{Bloomberg}, 14 marzo 2023.}

Negli anni '90, il Brasile ha vissuto un'iperinflazione sfrenata, pur
essendo al tempo stesso il quinto paese al mondo per popolazione. Quando
si immagina l'iperinflazione, è consueto evocare la Germania degli anni
'20 o alcuni stati falliti contemporanei, ma un numero sorprendentemente
elevato di Paesi --- a partire dagli anni '80 in poi --- tra cui
Brasile, Argentina, Jugoslavia, Zimbabwe, Venezuela, Polonia,
Kazakhstan, Perù, Bielorussia, Bulgaria, Ucraina, Libano e altri, ha
attraversato periodi iperinflazionistici. Altre nazioni, come Israele,
Messico, Vietnam, Ecuador, Costa Rica e Turchia, hanno registrato tassi
di inflazione a tre cifre (cioè quasi un'iperinflazione) nello stesso
arco temporale.

Dal 2016 al 2021, molti mercati obbligazionari nelle nazioni occidentali
più ricche, in Europa e in Giappone, hanno offerto rendimenti nominali
prossimi allo zero o addirittura negativi, con un ammontare complessivo
di oltre 18 trilioni di dollari in obbligazioni a rendimento negativo al
picco della crisi.\footnote{Cormac Mullen e John Ainger, ``L'ammasso
  mondiale di debito a rendimento negativo raggiunge il record di 18
  trilioni di dollari,'' \emph{Bloomberg}, 11 dicembre 2020.} In tali
circostanze, chi prestava denaro a governi e grandi corporazioni doveva,
in realtà, pagare per il privilegio di prestare, invertendo il normale
meccanismo di remunerazione per i prestiti. Di conseguenza, gli
incentivi presenti nel sistema finanziario si sono ribaltati. Nei
successivi anni, un'ondata globale di inflazione ha ridotto
drasticamente il potere d'acquisto di coloro che detenevano tali
obbligazioni.

Nel corso degli anni 2010 numerosi alti funzionari della U.S. Federal
Reserve affermarono ripetutamente che l'economia era al di sotto del
loro obiettivo medio d'inflazione per troppo tempo e che, di
conseguenza, era necessario puntare a una maggiore inflazione. Durante
un'audizione al Congresso agli inizi del 2021, quando il tasso headline
inflazionistico era pari all'1,7\%, un parlamentare chiese al presidente
della Federal Reserve cosa potesse significare il balzo del 25\% su base
annua dell'ampia massa monetaria -- il valore più alto dal 1940 --
derivante dai recenti stimoli fiscali, e quali eventuali ripercussioni
potesse comportare sull'inflazione o sul valore del dollaro. Il
presidente respinse tali preoccupazioni, sostenendo che un tale aumento
dell'aggregato monetario probabilmente non avrebbe effetti economici
rilevanti e che, forse, sarebbe stato necessario ``disimparare'' l'idea
secondo cui gli aggregati monetari abbiano un impatto fondamentale
sull'economia.\footnote{Howard Schneider, ``Powell's Econ 101: posti di
  lavoro, non inflazione. E dimenticate l'offerta di moneta,''
  \emph{Reuters}, 23 febbraio 2021.}

Quando, più tardi nel 2021, l'inflazione dei prezzi cominciò a farsi
sentire seriamente, il presidente la liquidò inizialmente come un
fenomeno transitorio, mentre la Federal Reserve continuava ad espandere
la base monetaria mediante operazioni di quantitative easing. Tuttavia,
con l'insorgere nel 2022 di tassi d'inflazione ai massimi degli ultimi
quattro decenni, il presidente insieme ad altri vertici della Federal
Reserve entrò in preda al panico, modificando radicalmente la politica
monetaria e identificando l'inflazione come il problema più urgente da
affrontare. Nel tentativo di domare l'inflazione, i tassi d'interesse
vennero aumentati in maniera così aggressiva -- riducendo nel contempo
la base monetaria a un ritmo record nell'anno successivo -- da generare
perdite non realizzate per oltre un trilione di dollari per le banche,
soprattutto su titoli del Tesoro e altri asset a basso rischio.
L'estrazione accelerata dei depositi dal sistema bancario contribuì,
inoltre, a determinare alcune delle crisi bancarie più gravi della
storia americana. Entro il 2023 le banche di tutto il paese avevano
riportato rapporti patrimoniali fortemente compromessi a causa del
brusco aumento dei tassi, al punto che, per la prima volta nella storia
moderna, persino la Federal Reserve registrava una perdita operativa,
dovuta al differenziale tra gli elevati interessi pagati sulle passività
e quelli percepiti sugli asset.\footnote{Erica Jiang et al.,
  ``Rafforzamento monetario e fragilità delle banche statunitensi nel
  2023.''} Queste decisioni, prese manualmente e in maniera soggettiva
da un gruppo di appena dodici persone, influenzano le condizioni
monetarie per 330 milioni di americani e per miliardi di persone
all'estero.

Attualmente esistono circa 160 valute differenti nel mondo,\footnote{XE.com.
  ``Codici ISO 4217 delle valute.''} ognuna con il proprio monopolio
locale all'interno della rispettiva giurisdizione, e la maggior parte di
esse gode di una scarsa o nulla accettazione fuori dai confini
nazionali. In questo senso l'assetto finanziario globale si caratterizza
quasi come un sistema basato sul baratto. Solo una manciata di
principali valute viene detenuta come riserva da altre banche centrali e
beneficia di un certo grado di riconoscimento internazionale; queste
però perdono valore lentamente nel tempo e i tassi d'interesse ad esse
associati non sono riusciti a tenere il passo con l'inflazione per anni.
La maggior parte delle altre valute, invece, è maggiormente soggetta a
bruschi deprezzamenti, a lunghi periodi di inflazione a doppia cifra e,
in alcuni casi, a episodi di iperinflazione, spesso senza godere di
alcun riconoscimento estero. Nei paesi in cui la moneta locale
appartiene a questa seconda categoria, i cittadini sono soliti cercare
di accaparrarsi valute estere, come il dollaro, per proteggere i propri
risparmi, non potendosi affidare in modo sufficiente alle banche locali.

Anche mettere da parte dei risparmi risulta una sfida in alcune delle
giurisdizioni monetarie più stabili e, se una persona dovesse nascere
nella ``giurisdizione sbagliata'', si troverebbe ad affrontare una
battaglia estremamente ardua.

Come siamo giunti a questa situazione? Perché il nostro denaro non è
\emph{migliore} di così?

Il sistema finanziario globale ha mostrato notevoli inefficienze per i
paesi in via di sviluppo nel corso della storia moderna e, nelle ultime
decadi, ha generato squilibri seri anche tra le nazioni più sviluppate.
Le fondamenta su cui poggia non risultano più solide, in parte a causa
dell'obsolescenza delle tecnologie centrali che lo supportano.

Sostengo che l'ascesa del populismo, riscontrabile negli Stati Uniti, in
Europa e in numerosi paesi in via di sviluppo sin dalla crisi
finanziaria globale del 2008, derivi in larga misura da questa realtà.
Sia gli elettori dell'estrema sinistra che quelli dell'estrema destra
percepiscono che qualcosa non quadra, che il sistema è ``truccato''
contro di loro, pur non riuscendo sempre a identificarne le ragioni
precise. Un elemento fondamentale di questo enigma risiede nel fatto che
il sistema finanziario tradizionale non funziona più.

Abbiamo osservato, in decenni precedenti, come i sistemi finanziari
globali si disgregano gradualmente a causa dell'accumularsi di squilibri
economici, dei cambiamenti negli assetti geostrategici e
dell'introduzione di nuove tecnologie. Quando ciò avviene, l'ordine
precedente viene parzialmente o completamente ricostruito, dando vita a
un nuovo assetto che questo volume analizza attraverso numerosi esempi.
Tutti gli indicatori suggeriscono che l'assetto finanziario in vigore
dagli anni '70 stia ormai giungendo alla fase terminale e si stia
aprendo a un processo di ristrutturazione e riallineamento.

Questo libro affronta il tema del denaro attraverso la lente degli
sviluppi tecnologici. Esamina l'evoluzione delle forme monetarie del
passato, le ragioni per cui le attuali tecnologie e istituzioni
monetarie si rivelano inadeguate nel presente, nonché alcune delle
possibili soluzioni ai problemi monetari che oggi ci troviamo ad
affrontare. Il testo è redatto in linguaggio chiaro e si struttura in
moduli, in modo da consentire ai lettori di concentrarsi agevolmente sui
temi di loro maggiore interesse.

\textbf{Parte 1} guida il lettore attraverso antichi registri contabili
e le forme di denaro basate sulle materie prime, con l'obiettivo di
analizzare perché il denaro sia emerso in maniera naturale e quali forme
abbiano prevalso rispetto ad altre. Questo percorso ci aiuta a
individuare le proprietà ideali che caratterizzano il denaro e a
comprendere perché tali attributi tendano a ripresentarsi in modo
indipendente nel corso della storia. Viene inoltre esaminato il rapporto
tra credito sociale e denaro-merce, offrendo così una sintesi tra due
scuole economiche tradizionalmente in contrasto.

\textbf{Parte 2}\\
Questa sezione si concentra sui primi servizi proto-bancari e
sull'ascesa delle banche a servizio completo. L'autore analizza come una
serie di innovazioni tecnologiche abbia accelerato le transazioni
monetarie, allontanandole dal lento iter delle liquidazioni fisiche.
Tale processo ha portato con sé numerosi vantaggi, sebbene non privo di
alcune criticità. Si conclude illustrando come il progressivo divario in
termini di rapidità tra le transazioni e le liquidazioni, all'alba
dell'era delle telecomunicazioni, abbia conferito un'enorme influenza
alle banche e alle banche centrali, diventando queste ultime i
principali attori in grado di trasmettere in tempo reale fondi a livello
globale.

\textbf{Parte 3}\\
Qui viene descritto come il sistema finanziario globale si sia
strutturato fin dai primi del Novecento, soffermandosi anche sulle
dinamiche geopolitiche che hanno guidato la sua creazione e sui
cambiamenti intervenuti nel tempo. Il testo ripercorre il periodo in cui
i legami con il gold standard cominciarono a crollare, in concomitanza
con la Prima Guerra Mondiale, passando al sistema di Bretton Woods
attivo dagli anni '40 fino ai primi anni '70 e successivamente al
modello Eurodollar/Petrodollar, che ha sostituito il precedente a
partire dagli anni '70 fino ai giorni nostri. Infine, si spiega in che
modo alcuni aspetti problematici dell'attuale sistema abbiano generato
squilibri strutturali a livello mondiale nelle ultime decadi.

\textbf{Parte 4}\\
Questa sezione analizza nel dettaglio come il denaro venga creato
all'interno dell'odierno sistema finanziario e come, nel tempo, il
debito contribuisca intrinsecamente a destabilizzarlo. Si esaminano,
inoltre, gli squilibri e gli incentivi problematici derivanti dalla
continua svalutazione delle unità monetarie, che spingono i
risparmiatori a ricercare altre forme di asset non monetari per
proteggere il proprio potere d'acquisto. Viene inoltre messo in luce
come i legislatori, potenziati dalla flessibilità di un registro
pubblico, abbiano la possibilità di finanziare spese in modo opaco, ad
esempio impegnandosi in guerre senza ricorrere alla tassazione o
realizzando salvataggi selettivi attraverso la svalutazione dei risparmi
altrui.

\textbf{Parte 5}\\
Questa parte si sofferma sulle innovazioni monetarie digitali del XXI
secolo, spaziando tra Bitcoin, stablecoin, smart contracts e le valute
digitali emesse dalle banche centrali. Essendo il segmento più
speculativo del libro, esso si concentra sul presente e sul futuro
piuttosto che sul passato. Il testo descrive le nuove tecnologie a
nostra disposizione, analizzandone in modo approfondito i compromessi e
i rischi che accompagnano le opportunità che esse possono offrire.

\textbf{Parte 6}\\
L'ultima sezione esplora l'etica del denaro e della comunicazione,
considerate le due componenti fondamentali del commercio. Si discute il
ruolo della crittografia -- elemento critico per le infrastrutture
bancarie moderne e di internet -- e si confrontano le reti finanziarie
aperte con quelle chiuse, per arrivare infine a considerare
l'intersezione tra tecnologia finanziaria e diritti umani.

Al centro, il denaro è fondamentalmente un registro. Il denaro-merce
funge da registro governato dalla natura, quello bancario opera come
registro sotto il controllo degli Stati nazionali, mentre il denaro open
source rappresenta un registro regolato dagli utenti. Come esplorato nel
libro, l'evoluzione tecnologica trasforma di epoca in epoca le strutture
di potere e gli incentivi che circondano il denaro.

Il mio percorso professionale è il frutto di una combinazione di
ingegneria e finanza, e adotto un approccio ingegneristico sistemico
nell'analisi di vari aspetti del sistema finanziario globale.
L'ingegneria dei sistemi, disciplina multidisciplinare, si concentra
sulla progettazione, integrazione, gestione e manutenzione di sistemi
complessi nel loro intero ciclo di vita. Considero il sistema
finanziario globale come un vero e proprio sistema progettato, e ho
constatato che questo metodo analitico porta a conclusioni innovative,
talvolta in contrasto con il pensiero economico convenzionale.

Il mio obiettivo nella stesura di questo libro è contribuire a una
migliore comprensione del funzionamento del denaro e a spiegare perché
l'attuale sistema finanziario globale non performi come una volta. Il
testo non si limita a spiegare le carenze del sistema attuale in termini
temporali -- sia nell'immediato che nel medio termine -- bensì offre
un'analisi approfondita della natura del denaro, del percorso che ci ha
condotti all'attuale situazione e dei problemi fondamentali che lo
affliggono oggi.

Non pretendendo di avere tutte le risposte o di poter prevedere
esattamente quale sarà il panorama finanziario nelle prossime decadi,
intendo condividere quanto emerso dalle mie ricerche affinché i lettori
possano trarre ulteriore ispirazione nella ricerca delle proprie
risposte. Mentre la politica può influenzare le dinamiche a livello
locale e per periodi limitati, la tecnologia ha il potere di incidere in
modo globale e permanente, motivo per cui scelgo di analizzare il denaro
principalmente attraverso la lente tecnologica.

Questo non è un libro sull'oro, un trattato bancario, né un saggio
esclusivamente su bitcoin o sulle questioni politiche. È, piuttosto,
un'indagine sulle tecnologie monetarie nelle loro molteplici
manifestazioni -- passate, presenti e future -- toccando tutti questi
temi e oltre, per permetterci di comprendere meglio le nostre origini e
le possibili direzioni da intraprendere in avanti.

\section{Note a piè di pagina}\label{note-a-piuxe8-di-pagina}

\bookmarksetup{startatroot}

\chapter{Introduzione}\label{introduzione-1}

A settembre 2022, un'ondata di persone comuni ha rapinato banche in
Libano.

Ciò che ha resa questi eventi più degni di notizia rispetto alle tipiche
rapine in banca è stato il fatto che la maggior parte di queste persone
stava rapinando le banche per riavere indietro il \emph{proprio denaro}.
A causa di una crisi finanziaria in Libano, infatti, le banche non
permettevano alle persone di accedere ai propri depositi in contanti per
lungo tempo.

Una delle ``rapinatrici di banche'' che ha fatto notizia era una giovane
donna che lavorava come interior decorator. Ha rapinato una banca a
Beirut usando quella che in seguito si è rivelata essere una pistola
finta dall'aspetto estremamente realistico, con l'intento di prelevare i
risparmi della sua famiglia destinati al trattamento della sorella,
afflitta dal cancro, poiché i risparmi erano stati congelati dalla
banca. Questo è stato forse l'esempio più eclatante, ma ci furono anche
altre rapine durante quel periodo, effettuate da persone che volevano
semplicemente riavere indietro i propri depositi, e alcune di esse
ricorsero anche ad armi reali.

Questi eventi in Libano sono specifici di un determinato paese e momento
storico, ma fanno parte di una storia globale.

Nigeria, un paese con oltre 200 milioni di abitanti, ha registrato
un'inflazione annualizzata del 13\% nell'ultimo decennio.\footnote{FMI,
  ``Prezzi al consumo, fine periodo,'' Datamapper.} Nel 2021 è stata
lanciata una valuta digitale della banca centrale, la eNaira, che finora
ha riscosso un'adozione estremamente bassa, mentre le criptovalute (in
particolare bitcoin e stablecoin ancorate al dollaro statunitense) hanno
registrato un tasso di adozione superiore di un ordine di grandezza
all'interno del paese, pur essendo escluse dal sistema bancario
nazionale. Il governo nigeriano ha successivamente adottato una serie di
politiche volte a ridurre la disponibilità di contante fisico e a
spingere la popolazione verso i pagamenti digitali, contribuendo così a
un periodo di sconvolgimenti politici e a sommersi scontri.

L'Egitto ha improvvisamente dimezzato il valore della propria valuta
rispetto al dollaro statunitense nell'autunno del 2016, compromettendo
anni di risparmi per una popolazione di circa 100 milioni di abitanti.
Nel 2022 e nel 2023 il paese ha nuovamente effettuato diverse brusche
devalutazioni della propria valuta rispetto al dollaro, portando a un
ulteriore dimezzamento del tasso di cambio. Conosco persone in Egitto
che acquistano dollari fisici sul mercato nero e li detengono come
protezione contro questo problema persistente. Pagano commissioni di
conversione considerevoli per farlo, senza però ottenere interessi sui
dollari cartacei detenuti. E quando tali devalutazioni si verificano,
ricade immediatamente su tutti i lavoratori del paese l'onere di cercare
di negoziare stipendi più alti per recuperare parte del potere
d'acquisto perso, poiché i loro salari continuativi sono espressi nella
valuta locale deprezzata.

La Turchia e l'Argentina, entrambi membri delle nazioni del G20 e con
una popolazione complessiva di oltre 130 milioni di abitanti, hanno
dovuto far fronte a un'inflazione incontrollata negli ultimi anni. La
Turchia ha raggiunto un'inflazione anno su anno dell'85\% nel 2022,
mentre l'Argentina ha superato il 100\% di inflazione nel
2023.\footnote{Zeynep Dierks, ``Tasso di inflazione CPI in Turchia,''
  \emph{Statista}, 3 marzo 2023;
  \href{https://www.bloomberg.com/authors/ATpJM4rxzpI/patrick-gillespie}{Patrick~Gillespie},
  ``L'inflazione in Argentina supera il 100\% mentre incombe una
  recessione economica,'' \emph{Bloomberg}, 14 marzo 2023.}

Negli anni '90 il Brasile sperimentò una vera iperinflazione, essendo
allora il quinto paese più popoloso del mondo. Quando si immagina
l'iperinflazione, spesso ci si riferisce alla Germania degli anni '20 o
ad alcuni stati falliti contemporanei, ma un numero sorprendentemente
elevato di paesi l'ha vissuta in un determinato periodo durante la
seconda metà del XX secolo. Già dagli anni '80 in poi, persone in
Brasile, Argentina, Jugoslavia, Zimbabwe, Venezuela, Polonia,
Kazakistan, Perù, Bielorussia, Bulgaria, Ucraina, Libano e altri hanno
sperimentato l'iperinflazione. Altri paesi, come Israele, Messico,
Vietnam, Ecuador, Costa Rica e la Turchia, hanno avuto un'inflazione a
tre cifre (cioè quasi iperinflazione) in quel periodo.

Dal 2016 al 2021, molti mercati obbligazionari di nazioni ricche in
Europa e in Giappone offrivano rendimenti nominali prossimi allo zero o
addirittura negativi, con oltre 18 trilioni di dollari in obbligazioni a
rendimento negativo al picco.\footnote{Cormac Mullen e John Ainger,
  ``L'ammasso mondiale di debito a rendimento negativo raggiunge il
  record di 18 trilioni di dollari,'' \emph{Bloomberg}, 11 dicembre
  2020.} Le persone dovevano pagare per il privilegio di prestare denaro
a governi e grandi imprese, invece di guadagnare interessi. Di
conseguenza, gli incentivi del sistema finanziario furono completamente
rovesciati. Nei successivi anni, tuttavia, un'ondata globale di
inflazione ridusse drasticamente il potere d'acquisto dei detentori di
tali obbligazioni.

Negli anni 2010, diversi alti funzionari della Federal Reserve
statunitense hanno ripetutamente dichiarato che l'economia era troppo a
lungo al di sotto del loro obiettivo medio di inflazione e che
desideravano un'inflazione più alta. Durante un'udienza congressuale
all'inizio del 2021, quando il tasso di inflazione principale negli
Stati Uniti era dell'1,7\%, un deputato chiese al presidente della
Federal Reserve riguardo all'incremento del 25\% su base annua
dell'aggregato monetario (il più alto dagli anni '40) avvenuto a seguito
dei recenti stimoli fiscali, e alle eventuali implicazioni che ciò
potesse avere sull'inflazione o sul valore del dollaro. Il presidente
respinse tali preoccupazioni, affermando che un tale aumento
dell'aggregato monetario probabilmente non avrebbe avuto importanti
conseguenze economiche e che forse sarebbe stato necessario
``dimenticare'' l'idea che gli aggregati monetari abbiano un impatto
rilevante sull'economia.\footnote{Howard Schneider, ``Powell's Econ 101:
  posti di lavoro, non inflazione. E dimenticate l'offerta di moneta,''
  \emph{Reuters}, 23 febbraio 2021.}

Quando, verso la fine del 2021, l'inflazione dei prezzi iniziò a
manifestarsi seriamente, il presidente la liquidò inizialmente come
transitoria e la Federal Reserve continuò ad espandere la base monetaria
attraverso il quantitative easing. Ma poi, con il verificarsi nel 2022
di tassi d'inflazione ai massimi di quattro decenni, il presidente e
altri leader della Federal Reserve entrarono in panico e cambiarono
completamente la loro politica monetaria, individuando l'inflazione dei
prezzi come il problema più urgente da affrontare. Nel tentativo di
domare l'inflazione, aumentarono i tassi in modo così aggressivo --- e
ridussero la base monetaria a un ritmo record nell'anno successivo ---
da finire per creare perdite non realizzate per le banche per un valore
superiore a un trilione di dollari sui loro titoli del Tesoro e su altri
asset a basso rischio. Estraendo depositi dal sistema bancario a un
ritmo così sostenuto, contribuirono ad alcune delle maggiori crisi
bancarie della storia americana. Entro il 2023, le banche di tutto il
paese avevano rapporti di capitale gravemente compromessi a causa
dell'impennata dei tassi di interesse. Per la prima volta nella storia
moderna, persino la Federal Reserve registrava una perdita operativa,
dovuta al pagamento di tassi così elevati sulle proprie passività
rispetto a quanto guadagnato sugli asset.\footnote{Erica Jiang et al.,
  ``Rafforzamento monetario e fragilità delle banche statunitensi nel
  2023.''} Queste decisioni della Federal Reserve influenzano le
condizioni monetarie per 330 milioni di americani e per miliardi di
persone all'estero, eppure vengono prese manualmente e soggettivamente
da un gruppo di sole dodici persone.

Esistono circa 160 valute diverse nel mondo,\footnote{XE.com. ``Codici
  ISO 4217 delle valute.''} ciascuna con un monopolio locale all'interno
della propria giurisdizione, e la maggior parte di esse ha poca o
nessuna accettazione fuori dai propri confini. L'ordine finanziario
globale è, a questo proposito, praticamente un sistema di baratto. Una
manciata di valute principali sono detenute come riserve da altre banche
centrali e godono di un certo grado di accettazione internazionale, ma
perdono valore lentamente nel tempo e i loro tassi d'interesse non hanno
seguito l'inflazione per anni. La maggior parte delle altre valute è
invece più soggetta a brusche svalutazioni, a periodi persistenti di
inflazione a doppia cifra e a occasionali iperinflazioni, pur godendo di
scarsa o nessuna accettazione estera. Per le persone che vivono in
questi paesi, spesso si cerca di procurarsi valute straniere, come il
dollaro, per proteggere i propri risparmi, poiché generalmente non ci si
può fidare delle banche locali per custodirli.

Risparmiare denaro può essere una sfida anche nelle giurisdizioni
monetarie più stabili, e se qualcuno nasce nella ``giurisdizione
sbagliata'', la battaglia per mettere da parte denaro diventa
incredibilmente ardua.

Come siamo arrivati a questo punto? Perché il nostro denaro non è
\emph{meglio} di così?

Il sistema finanziario globale si è dimostrato inadeguato per i paesi in
via di sviluppo lungo tutta la storia moderna e, nelle ultime decadi, ha
accumulato seri squilibri anche nei paesi sviluppati. Le sue fondamenta
non risultano più solide, in parte perché la tecnologia di base è
obsoleta.

Sostengo che l'ascesa del populismo negli Stati Uniti, in Europa e in
numerosi paesi in via di sviluppo, a partire dalla crisi finanziaria
globale del 2008, sia in larga misura attribuibile a questo fatto.
Persone di sinistra e di destra percepiscono che qualcosa non va, che il
sistema è ``truccato'' contro di loro, ma non riescono a comprendere
esattamente il motivo. Un tassello fondamentale di questo enigma è
rappresentato dal fatto che il sistema finanziario, come lo conosciamo,
non funziona più.

Abbiamo visto, nelle decadi passate, come gli ordini finanziari globali
si disintegrino gradualmente a causa dell'accumulo di squilibri
economici, del verificarsi di riallineamenti geopolitici e
dell'introduzione di nuove tecnologie. Quando ciò accade, il vecchio
ordine viene in parte o completamente ricostruito, dando vita a un nuovo
assetto, come si può osservare in numerosi esempi riportati in questo
libro. Tutti gli indizi suggeriscono che l'ordine finanziario vigente
dagli anni '70 stia giungendo ai suoi ultimi anni e stia avviando un
processo di ricostruzione e riallineamento.

Questo è un libro sul denaro esaminato attraverso la lente degli
sviluppi tecnologici. Esso analizza l'evoluzione del denaro nel passato,
il motivo per cui le tecnologie e le istituzioni attuali dedicate al
denaro non ci soddisfano più e alcune delle possibili soluzioni ai
problemi monetari che oggi affrontiamo. È scritto in un linguaggio
semplice ed è strutturato in moduli, così che i lettori possano
concentrarsi sulle parti che più li interessano.

\textbf{Parte 1} del libro accompagna il lettore attraverso antichi
registri contabili e forme di denaro merce, per analizzare perché il
denaro sia emerso in maniera naturale e perché alcune forme di denaro
abbiano prevalso su altre. Questo percorso ci aiuta a discernere quali
siano le proprietà ideali del denaro e perché tali proprietà tendano a
riemergere in maniera indipendente nel corso della storia. Esamina
inoltre il rapporto tra credito sociale e denaro merce, offrendo un
punto di convergenza tra due scuole economiche spesso in contrasto.

\textbf{Parte 2} parla dei primi servizi proto-bancari e dell'ascesa
delle banche a servizio completo. Esamina come vari sviluppi tecnologici
abbiano accelerato le transazioni monetarie e le abbiano disgiunte dal
più lento processo dei regolamenti fisici, comportando numerosi benefici
ma anche alcuni svantaggi. Conclude illustrando come il divario
crescente tra la rapidità delle transazioni e quella dei regolamenti,
all'alba dell'era delle telecomunicazioni, abbia conferito un notevole
potere a banche e banche centrali, divenute le entità primarie capaci di
trasmettere denaro in fretta in tutto il mondo.

\textbf{Parte 3} descrive il sistema finanziario globale così come è
stato strutturato fin dai primi del Novecento, includendo la geopolitica
alla base della sua creazione e il suo evolversi nel tempo. Copre il
periodo dei fallimenti dei peg in oro intorno alla Prima Guerra
Mondiale, il sistema di Bretton Woods, esistito dagli anni '40 agli
inizi degli anni '70, e il sistema Eurodollaro/Petrodollaro che lo ha
sostituito dagli anni '70 fino ai giorni nostri. Infine, spiega come
alcuni aspetti problematici dell'attuale versione del sistema abbiano
condotto a squilibri strutturali in tutto il mondo negli ultimi decenni.

\textbf{Parte 4} analizza in dettaglio come il denaro venga creato nel
moderno sistema finanziario e come il debito, per sua natura,
destabilizzi il sistema nel tempo. Esamina, inoltre, alcuni degli
squilibri e degli incentivi problematici causati dalla costante
svalutazione delle unità monetarie, mentre i risparmiatori cercano di
preservarne il potere d'acquisto acquistando altri asset non monetari.
Mostra come i legislatori siano stati dotati di un registro pubblico
flessibile, che consente loro di intraprendere guerre senza ricorrere
alla tassazione, di effettuare salvataggi selettivi tramite la
svalutazione dei risparmi altrui e, in generale, di finanziare spese in
modo opaco.

\textbf{Parte 5} esamina le innovazioni monetarie digitali del XXI
secolo, includendo Bitcoin, stablecoin, smart contract e le valute
digitali delle banche centrali. Questa è la parte più speculativa del
libro, in quanto tratta del presente e del futuro piuttosto che del
passato. Descrive alcune delle nuove tecnologie a nostra disposizione e
analizza nello specifico i vari compromessi e rischi connessi a queste
tecnologie, accanto alle opportunità che esse possono offrire.

\textbf{Parte 6} esplora l'etica del denaro e della comunicazione, che
rappresentano le due componenti fondamentali del commercio. Discute il
ruolo della crittografia in generale (un elemento cruciale per il
moderno sistema bancario e l'infrastruttura di internet), il confronto
tra reti finanziarie aperte e chiuse, e l'intersezione tra tecnologia
finanziaria e diritti umani.

Fondamentalmente, il denaro è un registro contabile. Il denaro-merce
funge da registro governato dalla natura. Il denaro bancario è un
registro controllato dagli stati nazionali. Il denaro open-source è un
registro gestito dagli utenti. Come esplora il libro, l'evoluzione della
tecnologia cambia le strutture di potere e gli incentivi che circondano
il denaro da un'epoca all'altra.

Il mio percorso professionale unisce ingegneria e finanza, e adotto un
approccio di ingegneria dei sistemi per analizzare vari aspetti del
sistema finanziario globale. L'ingegneria dei sistemi è un campo
multidisciplinare che si concentra sulla progettazione,
sull'integrazione, sul funzionamento e sulla manutenzione di sistemi
complessi nel corso del loro ciclo di vita. Considero il sistema
finanziario globale come il vero sistema ingegnerizzato che è, e ho
constatato che questo metodo di analisi porta a conclusioni innovative
che a volte sfidano il pensiero economico convenzionale.

Il mio obiettivo nel scrivere questo libro è aiutare le persone a
comprendere meglio come funziona il denaro e perché il sistema
finanziario globale non operi più come una volta. Il libro non tratta
esclusivamente del perché il nostro sistema finanziario non funzioni
bene quest'anno o in questo decennio, ma rappresenta un'analisi più
profonda su cosa sia il denaro, su come siamo arrivati alla situazione
attuale e su quali siano i problemi fondamentali attuali.

Non ho tutte le risposte e non posso prevedere come sarà il mondo della
finanza nei prossimi decenni, ma con questo libro intendo condividere le
mie ricerche affinché i lettori possano trovare ulteriori risposte per
conto proprio. La politica può influenzare le cose in maniera locale e
temporanea, ma la tecnologia ha effetti globali e permanenti; per questo
motivo analizzo il denaro prevalentemente attraverso la lente della
tecnologia.

Questo non è un libro sull'oro, né un libro bancario, né un libro sul
bitcoin, né un libro politico. È, invece, un'esplorazione delle
tecnologie monetarie nelle loro innumerevoli forme -- passate, presenti
e future -- che tocca tutti questi temi e oltre, affinché possiamo
comprendere meglio da dove veniamo e quali percorsi potremmo
intraprendere in futuro.

\section{Note a piè di pagina}\label{note-a-piuxe8-di-pagina-1}

\bookmarksetup{startatroot}

\chapter{Part 1}\label{part-1}

\textbf{Cos'è il denaro?}

``\emph{I precursori del denaro, insieme al linguaggio, hanno permesso
agli esseri umani primitivi di risolvere problemi di cooperazione che
gli altri animali non sono in grado di affrontare --- inclusi quelli
dell'altruismo reciproco, dell'altruismo familiare e della mitigazione
dell'aggressività. Questi precursori condividevano con le valute non
fiat caratteristiche molto specifiche --- non si trattava semplicemente
di oggetti simbolici o decorativi.}''\footnote{Nick Szabo,
  \emph{Shelling Out: The Origins of Money.}}

-Nick Szabo

\section{Note}\label{note}

\bookmarksetup{startatroot}

\chapter{}\label{section}

\textbf{Cos'è il denaro?}

``I precursori del denaro, insieme al linguaggio, permisero agli esseri
umani della prima età moderna di superare le problematiche cooperative
che altre specie non sono in grado di risolvere --- tra cui l'altruismo
reciproco, l'altruismo filiale e la mitigazione dell'aggressività. Tali
precursori condividevano con le valute non fiat una serie di
caratteristiche molto specifiche --- non erano meri oggetti simbolici o
decorativi.''\footnote{Nick Szabo, \emph{Shelling Out: The Origins of
  Money.}}

-Nick Szabo

\bookmarksetup{startatroot}

\chapter{Capitolo 1: I registri come fondamento del
denaro}\label{capitolo-1-i-registri-come-fondamento-del-denaro}

Molte persone pensano che il denaro, inteso come concetto, nasca da
qualcosa come monete o conchiglie, ma in realtà la storia inizia ben
prima. Inizia con un registro.

Un registro è una sintesi delle transazioni e viene utilizzato per
tenere traccia di chi possiede cosa. I registri scritti più antichi
conosciuti risalgono a oltre 5.000 anni fa nell'antica Mesopotamia,
sotto forma di tavolette d'argilla. Secondo l'Encyclopedia Britannica,
il sumero è il più antico tipo di scrittura esistente, e i primi esempi
conosciuti di scrittura sumera erano registri in argilla che tenevano
traccia delle merci.\footnote{Ignace Gelb, ``Sumerian Language.''} Essi
mostravano immagini di varie merci con dei puntini accanto che
rappresentavano le quantità. In altre parole, le prime idee che gli
esseri umani sono noti per aver messo per iscritto con i loro
proto-scritti erano elenchi di proprietà, crediti o
transazioni.\footnote{William Goetzmann, \emph{Money Changes Everything:
  How Finance Made Civilization Possible}, 15--25.}

Tuttavia, il concetto di registro può essere anche molto più semplice. E
prima dell'invenzione della scrittura, dovevano esistere registri,
seppur in forma orale e nella memoria. Ogni volta che qualcuno doveva
qualcosa a un'altra persona, in modo formale o informale, si stava in
pratica mantenendo un semplice registro orale.

Ad un livello molto basilare, per fare un esempio moderno, immaginiamo
due fratelli, Alice e Bobby. Sono abbastanza grandi da ricevere
incarichi domestici dai loro genitori e, man mano che crescono e
iniziano a condurre vite più complesse, di tanto in tanto hanno bisogno
di riorganizzare i loro impegni. Ad esempio, Alice potrebbe dover
saltare i lavori domestici una sera per poter uscire con le amiche. Per
fare ciò, può proporre a suo fratello Bobby che, se lui si occupa dei
suoi compiti domestici oggi, lei si occuperà dei suoi domani. Accettando
l'offerta, hanno appena creato un semplice registro mentale e instaurato
una forma di credito. Ora, Alice deve a Bobby un determinato insieme di
compiti. Questo sistema funziona solo grazie alla fiducia e alla
reputazione: se Alice non salda il suo debito, è probabile che Bobby
rifiuti futuri scambi. Se la situazione rimane abbastanza semplice, il
loro piccolo registro sarà solo verbale; ma se i loro impegni
diventeranno sufficientemente complessi e scambieranno regolarmente i
lavori, potrebbero utilizzare un calendario come registro scritto. Non
esiste una specifica unità monetaria associata a questo registro --- è
semplicemente un sistema di baratto. Le uniche unità coinvolte sono i
singoli compiti, e il registro si limita a tenere traccia dei compiti
scambiati nel tempo come forma di credito.

Possiamo anche immaginare un gruppo di cacciatori, forse decine di
migliaia di anni fa in una tribù da qualche parte, che contavano quante
prede ciascuno avesse abbattuto, oppure annotavano in maniera
approssimativa chi avesse fatto un favore a chi. Le tribù di tutto il
mondo avevano (e hanno ancora) diversi modi di scegliere i leader,
formalmente o informalmente, e il processo era spesso in qualche misura
meritocratico. Che lo intendessero o meno, le persone tengono
approssimativamente conto delle imprese e delle reputazioni altrui, per
capire chi fornisce al gruppo un surplus e chi invece costituisce un
peso.

I gruppi sociali primitivi erano generalmente composti da decine di
individui, che formavano un piccolo nucleo. I vari gruppi appartenenti a
una stessa area geografica, con una cultura molto affine, si
riconoscevano spesso come parte di una più ampia cultura tribale
interconnessa. In un contesto in cui tutti si conoscono, il denaro non è
necessario, fatta eccezione per i registri orali e basati sulla memoria.
I favori possono essere annotati in maniera sommaria ed è solitamente
chiaro chi si impegna e chi invece no. Tali gruppi erano solitamente
costituiti da legami di parentela e amicizia, pertanto non era
necessario tenere un conto preciso del ``saldo''. Il registro era
approssimativo, informale e flessibile.\footnote{Justin Pack,
  \emph{Money and Thoughtlessness}, 51--70.}

Nei miei tempi da ingegnere, un gruppo di colleghi ed io andavamo spesso
a pranzo insieme. Tenevamo un conto sommario di chi guidava il gruppo
ogni volta, in modo da bilanciare approssimativamente i favori. Non
veniva messo per iscritto, e non era preciso, ma esisteva un registro
mentale collettivo. Lo stesso valeva per accompagnare i colleghi dal
meccanico o in aeroporto, con l'aspettativa che il favore venisse
ricambiato in seguito (prima che le app per il ride-sharing diventassero
comuni), oppure per prestare una piccola somma di denaro quando ce ne
serviva (ad esempio, quando si divideva il conto in contanti in un
ristorante, cosa che accadeva più spesso allora). Questi favori non
venivano mai formulati come ``Te lo faccio ora, ma poi mi devi
ricambiare in futuro''. Anzi, venivano offerti volentieri come un dono
quando richiesti, e si presumeva che, se in seguito si chiedesse un
favore reciproco, questo sarebbe stato altrettanto offerto con piacere.

Numerose ricerche condotte da antropologi sulle tribù di
cacciatori-raccoglitori hanno evidenziato come il comportamento
orientato al dono rappresenti un tema ricorrente. Sebbene le culture
varino sostanzialmente, gli individui che si conoscono tendono a
scambiarsi regali o favori, aspettandosi naturalmente un
ricambio.\footnote{Vedi ad esempio Marcel Mauss, \emph{The Gift};
  Marshall Sahlins, \emph{Stone Age Economics}; e Paul Einzig,
  \emph{Primitive Money}.} Questo è un aspetto fondamentale
dell'amicizia.\footnote{Elise Berman, ``Avoiding Sharing.''}

La situazione diventa più complicata quando iniziamo ad interagire con
persone che non conosciamo bene, di cui non ci fidiamo o che potremmo
non rivedere mai. Se, per esempio, due gruppi si incontrano in un
ambiente primitivo, si apre il rischio di violenza, ma anche la
possibilità di instaurare rapporti commerciali.

Il trading spot è un passo iniziale ovvio per transare con persone che
non conosciamo bene. Invece di offrire loro una sorta di credito
informale come facciamo con familiari e amici, idealmente vogliamo
concludere ogni transazione sul momento, poiché è molto probabile che
non li incontreremo mai più. Due gruppi si incontrano, entrambi dotati
di risorse ma anche con una certa capacità di ricorrere alla violenza,
se necessario, e attraverso un semplice linguaggio o dei gesti
completano uno scambio. Forse una banda ha un eccesso di lance, ma ha
bisogno di pellicce, mentre l'altra dispone di un surplus di pellicce,
ma necessita di lance. Possono scambiare pellicce per lance sul momento,
e così entrambe le parti ne traggono beneficio. Gli antropologi hanno
documentato numerosi casi di scambi ritualizzati tra diversi gruppi di
cacciatori-raccoglitori, spesso accompagnati dalla prospettiva
dell'accoppiamento.

Se non esiste già un processo rituale consolidato tra gruppi
relativamente equilibrati nella regione, e invece alcuni soggetti si
incontrano in maniera più casuale, c'è un'elevata probabilità che il
tentativo di scambio fallisca a causa della mancata realizzazione della
``doppia coincidenza dei bisogni''. Tale espressione descrive il
requisito economico secondo cui, perché lo scambio abbia successo,
ciascuna parte deve possedere un surplus di ciò che l'altra desidera. Se
entrambe le parti sono in penuria di lance, lo scambio fallirà. Se
entrambe sono carenti di pellicce, lo scambio fallirà. Esistono infatti
più combinazioni che portano al fallimento dello scambio rispetto a
quelle che lo rendono fruttuoso.

È molto più facile scambiare con i membri del nostro stesso gruppo, in
quanto con familiari e amici godiamo del lusso della fiducia e del
tempo, che possiamo considerare una forma di credito sociale flessibile.
Qualcuno può chiedermi un favore e io posso soddisfarlo, anche se al
momento non ho assolutamente bisogno di nulla da quella persona. Potrei
avere a disposizione tutto il cibo in eccesso, le pellicce e gli
strumenti necessari, eppure, quando qualcuno che conosco ha una carenza
o necessita del mio aiuto per qualche motivo, posso fargli un favore e
procurarglielo.\footnote{Paul Seabright, \emph{The Company of Strangers:
  A Natural History of Economic Life}, 2--5, 91--105.}

Oltre al fatto che fa sentire bene, il motivo per cui estendo questo
credito regalo a qualcuno che conosco è che prevedo che, alla fine,
arriverà un momento in cui avrò bisogno di qualcosa. Forse mi ammalerò,
mi infortunerò o diventerò incinta e non potrò raccogliere cibo per un
po', affidandomi così alla persona a cui sto rendendo un favore ora.
Fornendo un surplus di favori, aumento il mio status sociale e, di
conseguenza, la mia sicurezza all'interno del gruppo. La stessa logica
vale tuttora quando aiutiamo amici, vicini e familiari. Naturalmente,
probabilmente non agirò in maniera così meccanica quando eseguo un
favore; potrei farlo semplicemente perché sono biologicamente
predisposto a provare piacere nell'aiutare gli altri, grazie a migliaia
di generazioni di selezione biologica per questa caratteristica, che
hanno permesso ai miei antenati di sopravvivere e prosperare come esseri
sociali intelligenti e generosi. Tuttavia, nel profondo della mente,
calcoli mentali coscienti sono inevitabilmente presenti: facendo un
favore, rafforzo l'intero gruppo, me compreso, e accumulo una sorta di
assicurazione personale o risparmi sociali per me e/o per i miei stretti
legami in futuro. Sto investendo lavoro e risorse nel periodo della mia
abbondanza e, in cambio, sto raccogliendo dei risparmi nel nostro
registro sociale collettivo. Questo credito sociale, questo registro
mentale informale, è la soluzione che il gruppo famigliare e di amici ha
messo in atto per affrontare il problema della ``doppia coincidenza dei
bisogni''. Con un credito sociale flessibile, possiamo facilmente
aiutarci quando una persona ha bisogno di qualcosa, anche se l'altra in
quel momento non necessita di nulla.

In uno studio del 2010 intitolato ``Wealth Transmission and Inequality
Among Hunter-Gatherers'', che ha preso in considerazione una vasta gamma
di letteratura esistente, i ricercatori hanno osservato che, in alcuni
casi, l'assicurazione sociale può basarsi sulla reputazione della
persona in difficoltà e sulla qualità della sua rete sociale.

La maggior parte degli adulti nelle società di cacciatori-raccoglitori
contribuisce attivamente alla produzione e lavorazione del cibo, oltre
che alla fabbricazione e manutenzione degli strumenti. Inoltre, la cura
dei bambini e l'approvvigionamento sono generalmente compiti affidati ai
genitori. La maggior parte di queste attività richiede una notevole
forza, resistenza, acuità visiva e altri aspetti di buona salute. Di
conseguenza, ci si aspetta che il benessere somatico sia di primaria
importanza per il successo e il benessere stesso. D'altra parte, chi
soffre periodicamente di dotazioni somatiche subottimali può solitamente
contare sull'aiuto degli altri sotto forma di condivisione del cibo,
assistenza per la cura dei bambini e protezione nelle controversie.
Questa forma di assicurazione sociale è la norma ed è ampiamente
disponibile, anche se alcune evidenze suggeriscono che la qualità di
tale aiuto vari in base alla ``ricchezza relazionale'' (reputazione,
dimensione e qualità della rete sociale) dell'individuo o del nucleo
familiare bisognoso (Gurven et al.~2000; Wiessner 2002; Nolin
2008).\footnote{Eric Smith et al., \emph{Wealth Transmission and
  Inequality Among Hunter-Gatherers}, 21.}

All'inizio del famoso film \emph{Il Padrino}, un uomo chiede a Vito, il
boss della mafia, un favore, e Vito accetta di esaudirlo. In cambio,
anziché denaro, Vito chiede un favore non specificato per un momento
futuro. In altre parole, egli desidera un credito sociale flessibile.
Ciò avviene perché quell'uomo ha bisogno di qualcosa da Vito, mentre
Vito al momento non ha assolutamente bisogno di nulla da lui, eppure
Vito conosce quell'uomo e riconosce che fa parte della sua comunità più
ampia. Vito si occupa di raccogliere favori per poi richiederli quando
gliene conviene. Più avanti nel film, Vito effettivamente chiede il
favore: sviluppa una necessità che quell'uomo è l'unico in grado di
soddisfare, una necessità che Vito non aveva agli inizi. La storia di
Vito racconta di un uomo che cerca di massimizzare la ricchezza
relazionale della propria famiglia mantenendo un registro esteso di
favori, i quali fungono da una sorta di valuta basata sul credito
nell'economia parallela della mafia.

Tornando al nostro esempio di scambio tra gruppi distinti di persone,
dato che questi non dispongono della possibilità di un credito sociale
flessibile o di registri (non si fidano l'uno dell'altro e potrebbero
non rincontrarsi mai più dopo questo incontro), cosa potrebbero portare
a uno scambio che si aspettano abbia un'alta probabilità di essere
desiderato dall'altra parte? Se mi trovassi nella loro situazione,
riuscirei a pensare a qualcosa da offrire che quasi tutti vogliono,
sempre? In altre parole, esiste un bene che sia il più commerciabile?
Per molte tribù, una risposta iniziale fu rappresentata dalle
conchiglie.

Le conchiglie, specialmente quelle scolpite e lucidate per diventare
perline ornamentali, emersero come asset simili al denaro migliaia di
anni fa in diverse regioni. La loro utilità era prevalentemente
estetica: potevano essere trasformate in braccialetti, cinture, usate
come orecchini, cucite nei vestiti o infilate tra i capelli. Il
vantaggio delle conchiglie nello scambio è che sono piccole, scarse e
resistenti nel tempo. E il vantaggio specifico di indossarle come
collane è che non devono essere trasportate in mano, il che le rende
particolarmente portatili.

Nel suo saggio del 2002, \emph{Shelling Out: The Origins of Money}, Nick
Szabo illustra in dettaglio le ragioni per cui le conchiglie e altri
proto-denari collezionabili probabilmente presero piede, come ha
sintetizzato nel suo abstract:

I precursori del denaro, insieme al linguaggio, permisero agli esseri
umani della prima età moderna di risolvere problemi di cooperazione che
altri animali non sono in grado di affrontare -- inclusi quelli
dell'altruismo reciproco, dell'altruismo di parentela e della
mitigazione dell'aggressività. Questi precursori condividevano con le
valute non fiat caratteristiche ben specifiche: non erano semplicemente
oggetti simbolici o decorativi.\footnote{Szabo, \emph{Shelling Out}.}

Sulla costa pacifica del Nord America, le tribù raccoglievano il
dentalium, termine che indica lunghe conchiglie che ricordano dei denti.
Questi oggetti svolgevano la funzione di denaro e venivano scambiati
fino nell'entroterra, arrivando fino al North Dakota. Essendo tubi
naturali con aperture ad entrambe le estremità, i dentalium venivano
infilati insieme in lunghe file, e alcuni membri delle tribù si
tatuavano le braccia per avere un riferimento di lunghezza da utilizzare
nella misurazione durante le transazioni. Alcune tribù si
specializzavano nel raccoglierli dalle acque profonde.\footnote{Dror
  Goldberg, \emph{Famous Myths of `Fiat Money'}, 962--963.}

Sulla costa atlantica veniva impiegata una diversa tipologia di
conchiglia, chiamata wampum. Queste erano realizzate con conchiglie di
vongole e richiedevano una levigatura approfondita e l'utilizzo di un
trapano ad arco per praticare piccoli fori, necessari per infilarle
insieme. I creatori di queste conchiglie di solito non le consideravano
denaro in senso stretto. Le perle erano onorate per il fatto di essere
state, in un certo senso, creature viventi e venivano spesso impiegate a
scopi cerimoniali, come nella realizzazione di cinture inestimabili per
onorare trattati e altri eventi importanti. Tuttavia, altre tribù -- e
persino i coloniali -- iniziarono ad utilizzarle come denaro, o come
riserva di valore e status. Le tribù dell'entroterra le raccoglievano in
modo estensivo.\footnote{Marc Shell, \emph{Wampum and the Origins of
  American Money}.}

In alcune zone dell'Africa e dell'Asia che si affacciano sull'Oceano
Indiano, le conchiglie cowrie venivano impiegate come denaro per ragioni
similari. I commercianti internazionali le portavano con sé per le
transazioni, e esiste una vasta documentazione storica su questa pratica
fino ai secoli recenti.\footnote{Bin Yang, \emph{The Rise and Fall of
  the Cowrie Shell: The Asian Story}.}

Sebbene le conchiglie siano state tra i proto-denari più comuni,
esistevano anche altri tipi di denaro a perline. A volte venivano
utilizzate perline realizzate con uova di struzzo, o file di denti
provenienti da grandi animali predatori come leoni o lupi, che
ricoprivano un ruolo analogo. In ``Shelling Out'', uno degli esempi di
Szabo, viene citato il caso dei !Kung.

Come la maggior parte dei cacciatori-raccoglitori, gli !Kung trascorrono
gran parte dell'anno in piccoli gruppi dispersi e solo poche settimane
all'anno in aggregazione con altri gruppi. L'aggregazione è simile a una
fiera con funzioni aggiuntive: si realizza il commercio, si cementano
alleanze, si rafforzano partenariati e si arrangiano matrimoni. La
preparazione all'aggregazione è scandita dalla fabbricazione di oggetti
commerciabili, in parte utilitari, ma per lo più da collezione. Il
sistema di scambio, chiamato dagli !Kung \emph{hxaro}, prevede un
consistente scambio di gioielli con perline, inclusi pendenti in guscio
di struzzo, molto simili a quelli rinvenuti in Africa 40.000 anni fa.

Come ci si potrebbe aspettare, il continente africano ospita le perline
più antiche conosciute. Nel sito archeologico della Blombos Cave in Sud
Africa sono stati rinvenuti piccoli gusci di lumaca con minuscoli fori,
stimati a 75.000 anni. La National Science Foundation degli Stati Uniti
riportò questa scoperta nel 2004:

\begin{quote}
I gusci perforati rinvenuti nella Blombos Cave in Sud Africa sembrano
essere stati infilati come perline circa 75.000 anni fa, rendendoli
30.000 anni più antichi di qualunque ornamento personale precedentemente
identificato. Gli archeologi che hanno scavato il sito sulla costa
dell'Oceano Indiano hanno scoperto 41 gusci, tutti con fori e segni di
usura in posizioni simili, in uno strato sedimentario depositatosi
durante il Medio Paleolitico (MSA).
\end{quote}

\begin{quote}
``Le perline della Blombos Cave rappresentano una prova assoluta di
forse il primo deposito di informazioni al di fuori del cervello
umano'', afferma Christopher Henshilwood, direttore del programma del
Progetto Blombos Cave e professore presso il Centre for Development
Studies dell'Università di Bergen in Norvegia.
\end{quote}

\begin{quote}
I gusci, trovati in gruppi fino a 17 perline, provengono da un minuscolo
mollusco necrofago, Nassarius kraussianus, che vive negli estuari.
Devono essere stati trasportati al sito della caverna dai fiumi più
vicini, situati a 20 chilometri a est o a ovest lungo la costa. I gusci
sembrano essere stati selezionati per dimensione e deliberatamente
perforati, il che suggerisce che venissero lavorati per essere
trasformati in perline, direttamente sul posto o prima del trasporto
alla caverna. Tracce di ocra rossa indicano che le perline, o le
superfici contro cui venivano indossate, furono ricoperte con questo
pigmento a base di ossido di ferro ampiamente utilizzato.\footnote{National
  Science Foundation, \emph{Shell Beads from South African Cave Show
  Modern Human Behavior 75,000 Years Ago}.}
\end{quote}

Il cibo marcisce, e così, in un mondo senza congelatori, le persone non
hanno l'incentivo a mantenere riserve superiori alle proprie necessità.
Allo stesso modo, lance e pellicce sono ingombranti da trasportare;
oltre un certo punto non conviene possedere troppe lance o pellicce in
eccesso. Commerciare questi oggetti con altre tribù è difficile perché
ciascuna parte deve disporre esattamente di ciò che l'altra desidera.
Invece, possedere perline in guscio scolpite e levigate risolve il
problema: non marciscono e non sono ingombranti, per cui è conveniente
(e persino auspicabile) collezionarne qualche esemplare in più ogni
volta che le necessità primarie sono soddisfatte. Sono quasi
universalmente desiderate in un mondo caratterizzato da questo livello
basilare di tecnologia. Anche se a qualcuno potrebbe non piacere
indossarle, il coniuge, il fratello o un amico potrebbe apprezzarle. E
sapendo che la maggior parte dei membri delle altre tribù le gradisce,
si aprono nuove opportunità di scambio per il futuro.

Realizzare perline in guscio scolpite e levigate era un processo
estremamente laborioso. In primo luogo, i gusci dovevano essere raccolti
a mano lungo la costa e, successivamente -- a seconda del tipo --
venivano scolpiti, levigati e forati manualmente con una trivella ad
arco, in modo da poter far passare un filo per unirli tra loro o
fissarli a qualche altro oggetto, trasformandoli così in ornamenti
utili. Una volta realizzate, queste perline in guscio duravano a lungo e
possedevano un valore elevato in relazione alle loro dimensioni e al
loro peso, grazie al fascino estetico e all'enorme lavoro impiegato per
realizzarle. Se qualcuno scambiava del cibo in eccesso per delle perline
o impiegava tempo surplus per crearle, poteva conservarle per mesi o
anni, fino a quando non trovava qualcosa che desiderasse o di cui avesse
bisogno e procedeva allo scambio. Nel frattempo, esse risultavano
indossabili e gradevoli esteticamente.

In altre parole, le perle di conchiglia fungono da oggetti che si
possono accumulare, che possono aumentare o sostituire la necessità di
un credito sociale flessibile e che possono rimpiazzare il registro
orale --- almeno quando si tratta di rapporti con persone di cui non ci
si fida o che probabilmente non si rivedranno mai più. Le perle di
conchiglia, essendo un bene quasi universalmente desiderabile e
durevole, consentono lo scambio anche quando non vi è una richiesta
immediata, poiché si possono sempre richiedere altre perle che fungano
da segnaposto finché non si trovi ciò di cui si ha bisogno o si
desidera. Inoltre, è sempre possibile accumularne un numero maggiore
rispetto a quello attuale, perché esse rappresentano un valore portatile
e accumulato da scambiare in futuro per ottenere risorse, sia
all'interno del proprio gruppo che con altri. Rispetto al cibo che
deperisce o a pellicce e lance troppo ingombranti da conservare o
trasportare, queste piccole conchiglie portabili rappresentano, in
effetti, l'invenzione della tecnologia del risparmio a lungo termine ---
ovvero, un modo per trasformare tempo o risorse in eccesso in una sorta
di batteria finanziaria. Le persone possono indossare alcune file di
perle di conchiglia sui polsi, altre al collo, alle caviglie, nei
capelli, come cintura, e così via. Possono anche regalarle ai figli o al
coniuge. Ogni piccolo ornamento in conchiglia è intrinsecamente
desiderabile e testimonia un notevole impegno.

In questo ruolo, come bene maggiormente commerciabile, ogni fila di
perle di conchiglia funziona come uno dei favore futuri, non
specificati, di Vito. Chi, o quale gruppo, ha accumulato numerose perle
investendo tempo e risorse in eccesso (o le ha ereditate dalla
generazione precedente) possiede ora un notevole valore da offrire
qualora in futuro avesse necessità di risorse immediate. E a differenza
di un favore, una fila di perle di conchiglia rappresenta una
transazione definitiva; il suo valore non dipende dal ricordo di chi ha
ricevuto il favore.

Oltre al semplice piacere estetico di indossarle, le perle di conchiglia
erano spesso un segno di status. Chi le possedeva in abbondanza era
infatti non solo ricco in senso letterale, ma anche sociale. In questo
contesto tribale, quando vediamo qualcuno adornato con splendide
cinture, bracciali, collane di conchiglia o con ornamenti cuciti nei
vestiti, possiamo dedurre che in passato abbia offerto molto valore agli
altri per accumulare così tante perle, oppure che sia strettamente
legato a persone altrettanto benestanti. Indossando letteralmente una
serie di favori preziosi accumulati, essa testimonia un lungo periodo di
surplus di risorse. Una persona del genere appare degna di essere
conosciuta, rispettata e, possibilmente, scelta come partner, segnalando
socialmente un passato colmo di abbondanza.

Nello studio menzionato in precedenza --- \emph{Wealth Transmission and
Inequality Among Hunter-Gatherers} --- i ricercatori hanno osservato che
i beni mobili erano solitamente di proprietà individuale nelle società
di cacciatori-raccoglitori, mentre la terra tendeva ad essere posseduta
in comune:

\begin{quote}
I beni materiali mobili, quali strumenti, abiti e oggetti preziosi, sono
generalmente considerati proprietà individuale e vengono spesso
trasmessi ai discendenti. In gran parte delle società di sussistenza,
tuttavia, tali beni possono essere prodotti da ogni adulto del genere
appropriato o ottenuti abbastanza facilmente; le eccezioni riguardano
gli oggetti che richiedono una produzione altamente specializzata o che
si ottengono tramite contatti commerciali limitati, così come i beni di
ricchezza e prestigio in alcune società sedentarie e meno
egualitarie.\footnote{Smith et al., \emph{Wealth Transmission}, 21.}
\end{quote}

È stato rilevato, in particolare, che le ``merci soggette a una
produzione altamente specializzata'' e i ``beni di prestigio'' rientrano
tra le tipologie di proprietà non facilmente ottenibili. In altre
parole, possiedono una scarso effettivo. I ricercatori hanno quindi
concluso che, pur essendo in molti aspetti società basate sulla
comunanza, le comunità di cacciatori-raccoglitori non sono
necessariamente così egualitarie come potremmo immaginarle:

\begin{quote}
Infatti, come dettagliato nell'articolo introduttivo di questo forum di
Bowles et al., un valore β=0,25 implica che un bambino nato nel decile
superiore di ricchezza ha 5 volte più probabilità di rimanere in quel
decile rispetto a un bambino i cui genitori erano nel decile inferiore.
Anche un β pari a 0,1 implica che un bambino nato nel decile superiore
ha il doppio delle probabilità di rimanervi rispetto a uno nato nel
decile inferiore. Questi risultati suggeriscono che nelle popolazioni di
cacciatori-raccoglitori, anche in quelle in cui la condivisione del cibo
e altri meccanismi di livellamento sono diffusi (Cashdan 1982), la prole
dei più abbienti tenderà a rimanere tale, e viceversa.\footnote{Smith et
  al., \emph{Wealth Transmission}, 31.}
\end{quote}

A differenza di un libromastro letterale, nessuna delle parti coinvolte
nella transazione conosce l'aspetto completo del registro dei beads di
conchiglia. Se tu ed io partecipiamo a una transazione, nessuno dei due
sa esattamente quante conchiglie esistano nella nostra regione.
Tuttavia, conosciamo le loro caratteristiche e quanto sia difficile
produrle, e sappiamo con quale frequenza le osserviamo usate da altri,
il che ci aiuta a giudicarne la rarità e a stabilire a cosa potrebbero
essere scambiate.

I beads di conchiglia, e in senso più ampio le valute merce, funzionano
da libromastro decentralizzato della natura. Consegnando delle
conchiglie in cambio di qualcosa di valore, aggiorniamo lo stato del
registro, e proprio attraverso il possesso fisico viene mantenuto e
aggiornato l'intero stato del registro stesso. Tutti i partecipanti
comprendono e interagiscono con parti di questo libromastro naturale, ma
nessuno di noi conosce lo stato completo del registro.

Chi controlla questo registro? Per la maggior parte, la risposta a
questa domanda è: ``la natura.'' E, in termini pratici, ciò significa
che nessun essere umano o gruppo lo controlla. La produzione dei beads
di conchiglia richiede di spendere energia e tempo --- nel modo giusto e
con i materiali adatti --- il che significa che nessuno può barare.
Alcuni partecipanti costieri potrebbero dedicare il loro tempo in
eccesso alla produzione diretta dei beads di conchiglia, mentre altri,
situati nell'entroterra, potrebbero impiegare il tempo accumulando altre
risorse in surplus, per poi scambiare parte di queste risorse in eccesso
con i beads di conchiglia. In ogni caso, i beads di conchiglia
rappresentavano una misura del tempo e delle risorse in surplus, una
misura di risparmio e di valore, spesso con un certo grado di
cerimoniosità nel processo.

Per la parte restante, o nel caso limite, la risposta a chi controlla il
registro è: chi possiede la tecnologia più avanzata controlla il
registro. Questo sistema di registro basato sul denaro merce funziona se
tutti i partecipanti hanno una capacità produttiva abbastanza simile,
come avveniva in gran parte del mondo per migliaia di anni. Se una
civiltà estremamente avanzata arriva dall'altra parte dell'oceano,
dotata di strumenti metallici specializzati, e capisce come funziona il
sistema del denaro fatto di conchiglie, probabilmente sarà in grado di
produrre per unità di lavoro un ordine di grandezza in più di perline di
conchiglie rispetto a chiunque altro. Di conseguenza, potrà svalutare le
conchiglie di tutti inondando il mercato con esse, e raccoglierà molte
risorse nel processo, perché ci vorrà tempo prima che le tribù si
rendano conto che questa nuova civiltà può produrre conchiglie molto più
rapidamente di chiunque altro, e che in generale le perline di
conchiglie stanno diventando meno rare e meno preziose nel corso di mesi
o anni a causa di questa rapida espansione dell'offerta.

Come vedremo nel prossimo capitolo, la storia del denaro merce è una
storia sul progresso tecnologico. Varie forme di denaro merce fungono da
sistemi di registro onesti e giusti fino a quando la tecnologia non
raggiunge un punto in cui un gruppo ottiene un vantaggio ineguale, il
che costringe tutti gli altri ad adattarsi o a perdere.

\section{Note a piè di pagina}\label{note-a-piuxe8-di-pagina-2}

\bookmarksetup{startatroot}

\chapter{Capitolo 1: I registri come fondamento del
denaro}\label{capitolo-1-i-registri-come-fondamento-del-denaro-1}

Molti pensano che il concetto di denaro nasca con l'introduzione delle
monete o delle conchiglie, ma in realtà la storia comincia ben prima,
precisamente con l'idea di un registro contabile.

Un registro, inteso come riepilogo delle transazioni, serve a tenere
traccia dei rapporti di proprietà. I più antichi registri scritti, sotto
forma di tavolette di argilla, risalgono a oltre 5.000 anni fa nella
Mesopotamia antica. Secondo l'Encyclopedia Britannica, il sumero
rappresenta il più antico sistema di scrittura conosciuto, e le
testimonianze più antiche di questa scrittura sono proprio registri
cuneiformi che annotavano il passaggio di merci\footnote{Ignace Gelb,
  ``Sumerian Language.''}. Su queste tavolette figuravano
rappresentazioni pittoriche di vari beni, accompagnate da puntini
indicanti le quantità, i quali costituivano le prime forme di
elencazione della proprietà, del credito o delle transazioni, così come
li sappiamo dai protoscritti degli albori della civiltà\footnote{William
  Goetzmann, \emph{Money Changes Everything: How Finance Made
  Civilization Possible}, 15--25.}.

Il concetto di registro, tuttavia, può essere inteso in termini ancora
più semplici. Anche prima dell'invenzione della scrittura, nei meandri
della memoria e della tradizione orale, esisteva una forma primitiva di
registro. Ogni volta che qualcuno si trovava in debito, formalmente o
meno, con un altro soggetto, veniva così creato, implicitamente, un
registro orale di obblighi e scambi.

A titolo esemplificativo, immaginiamo una situazione contemporanea: due
fratelli, Alice e Bobby, abbastanza grandi da occuparsi di faccende
domestiche, che talvolta devono organizzare i loro impegni. Ad esempio,
Alice potrebbe avere necessità di rinunciare alle faccende una sera per
uscire con gli amici e, per rimediare, proporre a Bobby di occuparsi
delle sue faccende quel giorno in cambio del fatto che lei si faccia
carico delle sue il giorno seguente. Accettando l'accordo, i due creano
così un semplice registro mentale -- una forma basica di credito -- in
cui Alice accumula un debito nei confronti di Bobby relativo a
specifiche faccende. Tale accordo si basa unicamente sulla fiducia e
sulla reputazione: se Alice non adempie al suo debito, Bobby tenderà a
rifiutare futuri scambi. In situazioni semplici il registro rimane
puramente verbale, ma se gli impegni diventano complessi e le faccende
vengono scambiate regolarmente, potrebbe essere utile ricorrere a un
calendario, trasformando il registro in un documento scritto. In questo
contesto non esiste un'unità monetaria specifica: si tratta
semplicemente di un sistema di baratto, in cui le ``unità'' sono
costituite dalle faccende stesse, e il registro è impiegato per tenere
traccia degli scambi nel tempo, fungendo da strumento di credito.

Possiamo immaginare, ad esempio, un gruppo di cacciatori, magari decine
di migliaia di anni fa in una tribù qualsiasi, che contava il numero
delle prede abbattute o che teneva in maniera sommaria il conto dei
favori scambiati. Le tribù di tutto il mondo adottavano (e adottano
tuttora) diverse modalità, sia formali che informali, per scegliere i
loro leader, un processo che spesso assumeva una connotazione in parte
meritocratica. Che sia intenzionale o meno, gli individui annotano
approssimativamente le azioni e le reputazioni altrui, allo scopo di
riconoscere chi apporta un surplus al gruppo e chi, invece, costituisce
un peso.

I primi gruppi sociali umani erano solitamente composti da una dozzina
di persone, costituendo un manipolo. Diversi manipoli all'interno di una
stessa area geografica, accomunati da una cultura simile, tendevano a
riconoscersi come parte di una più ampia cultura tribale interconnessa.
In un contesto in cui tutti si conoscono, il denaro diventa superfluo se
non per registri trasmessi oralmente o basati sulla memoria; i favori
venivano monitorati in modo informale e, di solito, era chiaro chi
facesse la propria parte e chi no. In gruppi costituiti principalmente
da legami familiari e amicali, non era necessario tenere un conto
rigoroso: il ``registro'' risultava approssimativo, flessibile e
informale.\footnote{Justin Pack, \emph{Money and Thoughtlessness},
  51--70.}

Ai tempi in cui lavoravo come ingegnere, un piccolo gruppo di colleghi
ed io pranzavamo insieme e tenevamo in conto in maniera informale chi,
ogni volta, si faceva carico della guida del gruppo, così da bilanciare
gli oneri. Non veniva messo nero su bianco e il conteggio non era
preciso, ma esisteva indubbiamente un registro mentale condiviso. La
stessa dinamica valeva, ad esempio, per chi accompagnava i colleghi dal
meccanico o all'aeroporto -- e riceveva il favore in cambio in un
secondo momento (prima che le app di ride-sharing si diffondessero) --
oppure per chi prestava una piccola somma a un collega momentaneamente
in difficoltà, come quando si divida un conto in contanti al ristorante,
un evento molto più frequente allora. Tali favori non venivano formulati
con l'idea di ``ti farò questo ora, ma tu dovrai ricambiare in futuro'';
venivano offerti volentieri come un dono al momento della richiesta, con
l'implicita aspettativa che, qualora in seguito si richiedesse un favore
analogo, questo sarebbe stato restituito con la stessa generosità.

Numerose ricerche antropologiche condotte sulle tribù di
cacciatori-raccoglitori hanno evidenziato che un comportamento orientato
al dono è un tema ricorrente. Sebbene le culture differiscano in modo
sostanziale, le persone che si conoscono tendono normalmente a
scambiarsi doni o favori, attendendosi naturalmente un reciproco
scambio.\footnote{Vedi ad esempio Marcel Mauss, \emph{The Gift};
  Marshall Sahlins, \emph{Stone Age Economics}; e Paul Einzig,
  \emph{Primitive Money}.} Questo aspetto rappresenta uno degli elementi
fondanti dell'amicizia.\footnote{Elise Berman, ``Avoiding Sharing.''}

La situazione diventa tuttavia più complessa quando si interagisce con
persone poco conosciute, di cui non ci si fida oppure che potremmo non
rivedere mai più. Ad esempio, se due gruppi si incontrano in un ambiente
primordiale, oltre al rischio di conflitto violento, si apre anche la
possibilità di instaurare rapporti di scambio commerciale.

Lo spot trading rappresenta il primo ed evidente passo per effettuare
transazioni con persone con cui non abbiamo legami particolarmente
consolidati. Invece di concedere loro una forma informale di credito --
simile a quello che offriamo a familiari o amici -- l'ideale sarebbe
concludere ogni scambio immediatamente, poiché è molto probabile che non
ci incontreremo nuovamente. In tali situazioni, due gruppi, ciascuno
dotato di risorse e, se necessario, anche di una certa capacità di
impiegare la forza, si incontrano e, grazie a un linguaggio di base o a
gesti semplici, riescono a compiere uno scambio. Ad esempio, una banda
in eccesso di lance ma affamata di pellicce può incontrarsi con un'altra
che possiede in abbondanza pellicce ma necessita di lance, e scambiare i
due beni all'istante, rendendo entrambe le parti migliori. Numerosi
studi antropologici documentano infatti casi di scambi ritualizzati tra
diversi gruppi di cacciatori-raccoglitori, spesso accompagnati dalla
prospettiva di favorire anche legami riproduttivi.

Nel caso in cui non esista un processo ritualizzato di scambio tra
gruppi relativamente equilibrati nella regione -- e invece gli incontri
avvengano in maniera più casuale -- la probabilità che l'iniziativa
commerciale fallisca aumenta notevolmente, a causa dell'insorgenza del
problema della ``doppia coincidenza dei desideri''. Tale concetto
economico, infatti, esige che per uno scambio di successo ciascuna parte
debba possedere in surplus ciò di cui l'altra ha bisogno. Se entrambi
mancano delle lance oppure delle pellicce, l'operazione di scambio non
può andare a buon fine; in sostanza, le combinazioni che portano a uno
scambio fallito sono ben più numerose rispetto a quelle che ne
assicurano il successo.

È ben più semplice scambiare all'interno del proprio gruppo familiare o
tra amici, perché in tali contesti possiamo contare sul lusso della
fiducia e sul tempo, elementi che costituiscono una sorta di credito
sociale flessibile. Qualcuno può chiedermi un favore e io posso
aiutarlo, anche se al momento non ho alcuna particolare esigenza da
parte sua. Potrei disporre di un surplus di cibo, pellicce e strumenti,
eppure se un conoscente si trova in difficoltà o necessita del mio
supporto, sarò in grado di porgere il mio aiuto.\footnote{Paul
  Seabright, \emph{The Company of Strangers: A Natural History of
  Economic Life}, 2--5, 91--105.}

Oltre al piacere intrinseco che deriva dal soccorrere l'altro, il motivo
per cui offro questo credito gratuito a una persona vicina risiede nella
consapevolezza che, ad un certo punto, mi troverò anch'io in difficoltà.
Potrei ammalarmi, subire un infortunio o entrare in stato di gravidanza
da cui non sarò in grado di procurarmi il cibo per un periodo, e dovrò
dunque affidarmi a chi oggi beneficia del mio favore. Offrendo una serie
di aiuti in eccesso, accresco la mia posizione sociale e, di
conseguenza, la mia sicurezza all'interno del gruppo. Questo stesso
ragionamento si applica anche ai tempi moderni, quando si presta
soccorso ad amici, vicini e familiari. Naturalmente, difficilmente agirò
in maniera completamente meccanica quando compio un favore; potrei farlo
semplicemente perché il mio organismo è predisposto, a livello
biologico, a provare soddisfazione nell'aiutare l'altro -- una
caratteristica frutto di migliaia di generazioni di selezione naturale
che ha consentito ai nostri antenati di sopravvivere e prosperare come
esseri sociali intelligenti e generosi. Eppure, nel profondo, si celano
anche calcoli mentali consapevoli: compiendo questo favore rafforzo
l'intero gruppo, me compreso, e accumulo una forma di assicurazione
personale o riserva sociale per il futuro, a vantaggio mio e dei miei
cari. In altre parole, impiego lavoro e risorse in tempi di abbondanza
per ottenere un risparmio nel nostro registro sociale collettivo. Tale
credito sociale, ovvero questo registro mentale informale, costituisce
la soluzione, per il gruppo di amici e parenti, al problema della
``doppia coincidenza dei desideri''. Grazie a un sistema di credito
sociale flessibile, possiamo facilmente venire incontro alle necessità
altrui, anche se l'altra parte, al momento, non ha bisogno di nulla.

In uno studio del 2010 intitolato \emph{Wealth Transmission and
Inequality Among Hunter-Gatherers}, in cui veniva esaminata una vasta
gamma di letteratura esistente, i ricercatori hanno osservato che, in
alcuni casi, l'assicurazione sociale può basarsi sulla reputazione della
persona in difficoltà e sulla qualità della sua rete sociale.\\
\footnote{Paul Seabright, \emph{The Company of Strangers: A Natural
  History of Economic Life}, 2--5, 91--105.}: Vedi riferimento
bibliografico originale per ulteriori dettagli.

Nelle società di cacciatori-raccoglitori la maggior parte degli adulti
contribuisce attivamente alla produzione e lavorazione degli alimenti,
nonché alla fabbricazione e manutenzione degli strumenti. Inoltre, la
cura dei bambini e il provisionamento sono generalmente compiti affidati
ai genitori. Queste attività richiedono notevoli capacità di forza,
resistenza, acuità visiva e altri indicatori di buona salute, pertanto è
lecito aspettarsi che il benessere somatico costituisca un elemento
cruciale per il successo e la qualità della vita. D'altro canto, coloro
che sperimentano periodicamente prestazioni fisiche inferiori possono
solitamente contare sul sostegno degli altri, che si manifesta
attraverso la condivisione del cibo, l'assistenza nella cura dei bambini
e la protezione nei conflitti. Tale forma di assicurazione sociale è
considerata la norma e si rende ampiamente disponibile, anche se alcuni
studi suggeriscono che la qualità di questo supporto varia in funzione
della ``ricchezza relazionale'' (ovvero la reputazione, l'ampiezza e la
qualità della rete sociale) dell'individuo o del nucleo familiare in
difficoltà (Gurven, et al.~2000;~Wiessner 2002;~Nolin 2008).\footnote{Eric
  Smith et al., \emph{Wealth Transmission and Inequality Among
  Hunter-Gatherers}, 21.}

All'inizio del celebre film \emph{The Godfather} un uomo si rivolge a
Vito, il boss mafioso, chiedendogli un favore, a cui Vito acconsente. In
cambio, il prezzo richiesto non consiste in una somma di denaro, bensì
in un favore non definito da prestare in futuro, ovvero in un credito
sociale flessibile. Ciò deriva dal fatto che quell'uomo ha bisogno di
qualcosa da Vito mentre, al momento, Vito non necessita di nulla da lui;
tuttavia, Vito conosce quest'uomo e ne riconosce l'appartenenza alla sua
comunità più ampia. Operando nel campo della riscossione di favori, Vito
li accumula per poi richiederli quando ciò risulta vantaggioso. Più
avanti nel film, egli chiama effettivamente in causa un favore, facendo
emergere un bisogno che solo quell'uomo è in grado di soddisfare,
un'esigenza che non si era manifestata all'inizio della pellicola. La
storia di Vito illustra come un individuo cerchi di massimizzare la
ricchezza relazionale della propria famiglia mantenendo un esteso
registro di favori, che fungono da una sorta di valuta basata sul
credito all'interno dell'economia sommersa della mafia.

Ritornando all'esempio di scambio tra gruppi autonomi, dal momento che
essi non dispongono della possibilità di ricorrere al credito sociale
flessibile o a registri delle transazioni -- poiché non si fidano
reciprocamente e potrebbero non incontrarsi mai più dopo quel singolo
incontro -- cosa potrebbero offrire in un baratto che abbiano la buona
probabilità di essere richiesti dall'altra parte? Se mi trovassi nella
loro situazione, riuscirei a individuare un bene che quasi tutti
desiderino, in ogni circostanza? In altre parole, esiste un bene che
risulti il più facilmente commerciabile? Per molte tribù, una delle
prime risposte a questa domanda fu rappresentata dalle conchiglie.

Le conchiglie, in particolare quelle cesellate e lucidate in perline
ornamentali, emersero come beni simili alla moneta migliaia di anni fa
in diverse regioni del mondo. Il loro impiego si fondava su una funzione
estetica: potevano essere trasformate in braccialetti, cinture,
orecchini, cucite agli indumenti o utilizzate come ornamento per i
capelli. Il vantaggio delle conchiglie nello scambio risiede nel fatto
che sono oggetti di dimensioni ridotte, relativamente scarsi e dotati di
grande durata. Inoltre, il loro impiego in forma di gioielli indossabili
consente di non doverle necessariamente trasportare a mano, aumentando
così la loro portabilità.

Nel saggio del 2002, \emph{Shelling Out: The Origins of Money}, Nick
Szabo approfondisce ampiamente le ragioni per cui le conchiglie e altri
proto-moni collezionabili probabilmente si sono affermati. Come egli
stesso riassume nel suo abstract:

I precursori del denaro, insieme al linguaggio, permisero ai primi
esseri umani moderni di affrontare problemi di cooperazione che altre
specie animali non sono in grado di risolvere -- fra cui quelli
dell'altruismo reciproco, dell'altruismo familiare e della riduzione
dell'aggressività. Questi precursori possedevano caratteristiche ben
definite, molto simili a quelle delle valute non fiat, in quanto non
costituivano semplicemente oggetti simbolici o ornamentali.\footnote{Szabo,
  \emph{Shelling Out}.}

Sulla costa pacifica del Nord America, le tribù raccoglievano il
dentalium, ovvero lunghe conchiglie dalla forma simile a denti,
utilizzate come denaro e scambiate fino a zone interne come il North
Dakota. Trattandosi di tubi naturali dotati di aperture ad entrambe le
estremità, essi venivano infilandosi insieme in lunghe serie e alcuni
membri delle tribù si facevano tatuare le braccia, fissando lungo la
pelle le misure di riferimento per valutare la lunghezza delle serie
acquistate o vendute. Alcune tribù si specializzavano nella raccolta di
questi esemplari provenienti dalle acque profonde.\footnote{Dror
  Goldberg, \emph{Famous Myths of `Fiat Money'}, 962--963.}

Sulla costa atlantica, invece, si utilizzava un diverso tipo di
conchiglia, il wampum. Queste conchiglie di vongola venivano lavorate
intensamente -- lucidate e perforate con un trapano ad arco per poter
essere infilate insieme -- e, sebbene i loro produttori non le
considerassero propriamente ``denaro'', esse erano celebrate per essere
state in origine organismi viventi e adoperate spesso in contesti
cerimoniali, ad esempio per realizzare cinture inestimabili destinate a
celebrare trattati e importanti eventi. Tuttavia, altre tribù -- e
perfino i colonialisti -- iniziarono a utilizzare questi ``grani'' come
denaro, intesi anche come depositi di valore e simboli di status; le
comunità tribali dell'entroterra ne facevano largo uso.\footnote{Marc
  Shell, \emph{Wampum and the Origins of American Money}.}

Nei territori africani e asiatici bagnati dall'Oceano Indiano, per
motivi analoghi, si adottavano le conchiglie di cowrie come moneta. I
mercanti internazionali trasportavano con sé tali conchiglie per le
operazioni di scambio e numerosi documenti attestano che questa prassi
sia proseguita fino a epoche recenti.\footnote{Bin Yang, \emph{The Rise
  and Fall of the Cowrie Shell: The Asian Story}.}

Sebbene le conchiglie rappresentassero una delle forme più comuni di
proto-moneta, esistevano anche altre tipologie di denaro a perline.
Talvolta venivano usate perline ottenute da uova di struzzo oppure
stringhe di denti di grandi animali predatori, come leoni o lupi, che
assumevano un ruolo analogo. In ``Shelling Out'', uno degli esempi
citati da Szabo, si fa riferimento ai !Kung.

Come la maggior parte dei cacciatori-raccoglitori, i !Kung trascorrono
l'intera annualità in piccoli gruppi sparsi, riunendosi solo per alcune
settimane insieme a numerosi altri gruppi. Queste aggregazioni, che
ricordano una fiera arricchita da ulteriori funzionalità, rappresentano
occasioni in cui si svolgono scambi commerciali, si consolidano
alleanze, si rafforzano partenariati e si celebrano matrimoni. La
preparazione a tali incontri è caratterizzata dalla produzione di
oggetti commerciabili, in parte destinati a uno scopo utilitaristico e
in parte concepiti come oggetti da collezione. Il sistema di scambio,
noto dai !Kung come \emph{hxaro}, prevede uno scambio articolato di
gioielli realizzati con perline, tra cui spiccano pendenti in guscio di
struzzo, analoghi a quelli rinvenuti in Africa 40.000 anni fa.

Come ci si potrebbe attendere, il continente africano ospita le perline
più antiche conosciute. Nel sito archeologico della Grotta di Blombos,
in Sudafrica, sono stati rinvenuti piccoli gusci di lumaca, muniti di
minuscoli fori, e datati a circa 75.000 anni fa. La U.S. National
Science Foundation riportò questa scoperta nel 2004:

\begin{quote}
I gusci perforati rinvenuti nella Grotta di Blombos, in Sudafrica,
sembrano essere stati infilati come perline circa 75.000 anni fa --
rendendoli 30.000 anni più antichi di qualsivoglia ornamento personale
precedentemente identificato. Gli archeologi che hanno scavato il sito,
situato sulla costa dell'Oceano Indiano, hanno infatti ritrovato 41
gusci, tutti dotati di fori e segni di usura in posizioni analoghe, in
uno strato di sedimento depositato durante il Medio Paleolitico (MSA).
\end{quote}

\begin{quote}
``Le perline della Grotta di Blombos costituiscono una prova
inconfutabile, forse della più antica conservazione di informazioni al
di fuori del cervello umano'', afferma Christopher Henshilwood,
direttore del Blombos Cave Project e professore presso il Centre for
Development Studies dell'Università di Bergen in Norvegia.
\end{quote}

\begin{quote}
I gusci, rinvenuti in gruppi fino a 17 pezzi, appartengono a un
minuscolo mollusco netturbino, il Nassarius kraussianus, che abita gli
estuari. Essi dovettero essere trasportati al sito della grotta dai
fiumi più vicini, situati a circa 20 chilometri ad est o ad ovest lungo
la costa. La loro selezione per dimensione e la perforazione deliberata
suggeriscono che siano stati trasformati in perline direttamente sul
luogo o in prossimità del trasporto alla grotta. Infine, la presenza di
tracce di ocra rossa indica che le perline stesse o le superfici con le
quali venivano a contatto erano state ricoperte con questo ampiamente
utilizzato pigmento di ossido di ferro.\footnote{National Science
  Foundation, \emph{Shell Beads from South African Cave Show Modern
  Human Behavior 75,000 Years Ago}.}
\end{quote}

Il cibo deperisce e, in un mondo privo di congelatori, le persone non
hanno l'incentivo di accumulare viveri in eccesso rispetto al fabbisogno
immediato. Allo stesso modo, lance e pellicce sono ingombranti e, oltre
una certa quantità, il valore marginale di averne in surplus è
decisamente scarso. Gli scambi tra tribù risultano complicati perché
ciascuna parte deve possedere esattamente ciò che l'altra richiede. La
soluzione a tali criticità risiede proprio nella produzione di perline
in guscio scolpite e levigate: essendo resistenti al deperimento e poco
ingombranti, è conveniente -- anzi, anche auspicabile -- accumularne in
eccesso non appena le altre necessità sono soddisfatte. Queste perline
risultano quasi universalmente desiderate in un contesto tecnologico di
base; anche se a qualcuno può non piacere indossarle, il proprio
coniuge, un fratello o un amico potrebbe apprezzarle. Inoltre, il
riconoscimento di tali ornamenti da parte della maggior parte delle
altre tribù apre ulteriori prospettive di scambio.

La realizzazione di perline in guscio scolpite e levigate rappresentava
un processo estremamente laborioso. I gusci dovevano essere innanzitutto
raccolti manualmente lungo la costa e, successivamente, in base al tipo,
venivano scolpiti, levigati e forati a mano utilizzando un trapano ad
arco, in modo da poter infilarvi un filo per unirli o fissarli ad altri
oggetti, trasformandoli così in ornamenti funzionali. Una volta creati,
questi ornamenti godevano di una lunga durata e assumevano un valore
elevato in relazione alla loro dimensione e peso, grazie all'attrattiva
estetica e all'intenso lavoro impiegato nella loro fabbricazione.
Qualora qualcuno scambiasse un surplus di viveri per ottenere alcune di
queste perline, oppure dedicasse tempo in eccesso alla loro produzione,
potrebbe conservarle per mesi o anni, in attesa di incontrare un oggetto
desiderato o necessario da acquisire in cambio. Nel frattempo, essi
possiedono la doppia qualità di essere indossabili e di grande pregio
estetico.

In altre parole, le perline di conchiglia fungono da elemento
accumulabile, capace di aumentare o sostituire il bisogno di un credito
sociale flessibile e di rimpiazzare il registro orale, almeno nei
rapporti con persone di cui non ci si fida o che probabilmente non si
incontreranno mai più.

Essendo quasi universalmente desiderabili e particolarmente durevoli, le
perline di conchiglia consentono lo scambio anche quando non si ha nulla
di immediato da offrire, in quanto è sempre possibile richiederle come
segnaposto fino a quando non si incrocia l'oggetto del proprio interesse
o bisogno. Inoltre, si può sempre accumulare un numero maggiore di
conchiglie rispetto a quelle attualmente possedute, poiché esse
rappresentano un valore conservabile e portatile, scambiabile in futuro
con risorse, sia all'interno del proprio gruppo che con altri
collettivi.

Rispetto agli alimenti deperibili o a pellicce e lance troppo
ingombranti da immagazzinare o trasportare, queste piccole conchiglie
indossabili rappresentano, in senso lato, l'invenzione di una tecnologia
di risparmio a lungo termine, ossia un metodo per trasformare tempo o
risorse in eccesso in una sorta di riserva finanziaria. Le persone
possono indossare diverse serie di perline ai polsi, al collo, alle
caviglie, intrecciarle tra i capelli o usarle come cintura; possono
persino regalarle ai figli o al coniuge. Ogni piccolo gioiello di
conchiglia è intrinsecamente attraente e testimonia un notevole impegno
produttivo.

In veste di bene altamente commerciabile, ogni serie di perline opera
come uno dei favoreggi non specificati promessi, per così dire, da Vito.
Chi, o quale gruppo, ha accumulato un gran numero di perline investendo
tempo e risorse in eccesso -- oppure le ha ereditate dalla generazione
precedente -- possiede un patrimonio di valore da mettere a disposizione
nel caso in cui in futuro necessiti di risorse immediate. A differenza
di un favore, però, una collana di perline rappresenta una liquidazione
finale: il suo valore continuativo non dipende dalla memoria di chi ne
ha beneficiato.

Oltre al piacere puramente estetico di indossarle, le perline di
conchiglia fungevano spesso da distintivo di status. Chi ne era in
possesso in abbondanza dimostrava una notevole ricchezza, sia in senso
materiale sia sociale. Nel contesto tribale, osservando una persona
adornata con eleganti cinture, bracciali, collane e applicazioni cucite
nei vestiti, si può dedurre che essa abbia fornito in passato un elevato
contributo agli altri per potersi permettere di accumularle in gran
numero, o che mantenga strette relazioni con chi già le possiede. In
sostanza, indossare tali ornamenti equivale a portare addosso una serie
di favori preziosi e conservati, segno evidente di un lungo periodo di
abbondanza di risorse. Una persona così arricchita risulta degna di
conoscenza, rispetto e, perché no, anche di futura unione affettiva,
comunicando socialmente un passato fatto di prosperità.

Nel precedente studio citato -- \emph{Wealth Transmission and Inequality
Among Hunter-Gatherers} -- i ricercatori hanno osservato che, nelle
società di cacciatori-raccoglitori, il patrimonio materiale mobile --
come strumenti, abiti e oggetti preziosi -- è generalmente trattato come
proprietà individuale e trasmesso ai discendenti. Tuttavia, nella
maggior parte delle società foraggiatrici tale patrimonio può essere
prodotto da qualsiasi adulto del sesso appropriato o reperito con
relativa facilità; le eccezioni riguardano quegli oggetti che richiedono
una manifattura altamente specializzata o che si ottengono tramite
contatti commerciali limitati, nonché beni di prestigio in alcune
società sedentarie e meno egualitarie.\footnote{Smith et al.,
  \emph{Wealth Transmission}, 21.}

Notably, ``items involving highly specialized manufacture'' e ``prestige
goods'' sono individuati come alcune delle proprietà che non sono
facilmente ottenibili. In altre parole, possiedono una scarsità reale. I
ricercatori sono giunti alla conclusione che, pur essendo in vari
aspetti caratterizzate da dinamiche di comunanza, le società di
cacciatori--raccoglitori generalmente non risultano tanto egualitarie
quanto spesso si potrebbe pensare:

\begin{quote}
Infatti, come dettagliato nel documento introduttivo di questo forum da
Bowles et al., un valore di β pari a 0,25 implica che un bambino nato
nel decile più alto di ricchezza ha cinque volte più probabilità di
rimanere in quella fascia rispetto a un bambino i cui genitori
appartenevano al decile più basso. Anche un β pari a 0,1 suggerisce che
un bambino nato nel decile superiore abbia il doppio delle probabilità
di mantenersi in quella posizione rispetto a uno nato nel decile
inferiore. Questi risultati indicano che, nelle popolazioni di
cacciatori--raccoglitori, persino laddove esistano ampie pratiche di
condivisione del cibo e altri meccanismi compensativi (Cashdan 1982), la
prole dei soggetti più benestanti tende a mantenere il proprio status, e
viceversa.\footnote{Smith et al., \emph{Wealth Transmission}, 31.}
\end{quote}

A differenza di un registro contabile letterale, nessun partecipante a
una transazione è a conoscenza dell'intero stato del registro composto,
ad esempio, da perline di conchiglie. Se tu e io siamo coinvolti in uno
scambio, nessuno dei due conosce esattamente quante perline esistano
nella nostra regione. Ciò che sappiamo, comunque, sono le proprietà
intrinseche delle conchiglie, il notevole impegno necessario per
produrle, e la frequenza con cui le osserviamo in uso da parte degli
altri---informazioni che ci consentono di valutarne la rarità e
stabilire un criterio d'equivalenza per lo scambio.

Le perline di conchiglie, e più in generale le valute merce, fungono da
registro decentralizzato offerto dalla natura stessa. Scambiando
fisicamente delle conchiglie in cambio di un bene o servizio,
aggiorniamo lo stato di questo registro; è la proprietà materiale
l'elemento mediante il quale l'intero stato del registro viene
conservato e continuamente aggiornato. Tutti i partecipanti comprendono
e interagiscono con frammenti di questo registro naturale, pur non
avendo una visione d'insieme completa.

Chi controlla dunque questo registro? In larga misura, la risposta è
``la natura''. In termini pratici ciò significa che nessun essere umano
o gruppo possiede il controllo centrale sul registro. La produzione di
perline di conchiglie richiede infatti l'impiego di energia e
tempo---eseguito correttamente con i materiali adatti---, rendendo
impossibile qualsiasi tentativo di imbroglio. Alcuni partecipanti
residenti lungo la costa potevano dedicare il loro tempo in eccesso alla
realizzazione diretta di conchiglie, mentre altri, situati
nell'entroterra, accumulavano risorse in surplus per poi scambiarne una
parte e ottenere le perline. In entrambi i casi, le conchiglie
rappresentavano una misura del tempo e delle risorse eccedenti, un
indicatore di risparmio e valore, spesso connesso a rituali formali che
ne accompagnavano il processo di produzione.

Per quanto riguarda il caso limite, la risposta a chi controlla il
registro è semplice: colui che possiede la tecnologia più avanzata
determina il funzionamento del sistema. Questo modello, basato su un
registro di moneta merce, è efficace finché tutti i partecipanti
mantengono capacità produttive abbastanza equilibrate, condizione che ha
caratterizzato gran parte del mondo per migliaia di anni. Tuttavia, se
una civiltà estremamente avanzata dovesse emergere da oltreoceano,
dotata di strumenti metallici specializzati e in grado di comprendere il
meccanismo del sistema di moneta basato sulle conchiglie, essa potrebbe
produrre, per unità di lavoro, un numero di perle di conchiglia di un
ordine di grandezza superiore rispetto a chiunque altro. Di conseguenza,
inondando il mercato con queste conchiglie, potrebbe svalutare il valore
di quelle altrui e accumulare enormi risorse, poiché alle tribù
impiegherebbe tempo per rendersi conto di come questa nuova civiltà
riesca a produrre conchiglie a un ritmo molto più rapido, facendo sì che
tali perle diventino, nel corso di mesi o anni, progressivamente meno
rare e meno preziose.

Come vedremo nel prossimo capitolo, la storia della moneta merce è in
realtà una narrazione del progresso tecnologico. Diverse forme di moneta
merce funzionano efficacemente come sistemi di registro equi e
trasparenti fino a quando la tecnologia non evolve al punto da conferire
a un gruppo un vantaggio sproporzionato, costringendo tutti gli altri a
dover necessariamente adattarsi o a soccombere.

\section{Footnotes}\label{footnotes}

\bookmarksetup{startatroot}

\chapter{\texorpdfstring{Capitolo 2: \textbf{L'evoluzione delle merci
come
denaro}}{Capitolo 2: L'evoluzione delle merci come denaro}}\label{capitolo-2-levoluzione-delle-merci-come-denaro}

Come ha esplorato il capitolo precedente, gli esseri umani appartenenti
a piccoli gruppi di parentela e amicizia non hanno bisogno di denaro;
possono organizzare le risorse tra loro manualmente, a volte avvalendosi
al massimo di registrazioni orali informali. Essi tengono traccia di chi
offre un surplus costante al gruppo e di chi, invece, opera sempre in
deficit. All'interno di piccoli gruppi, le persone risolvono
spontaneamente il problema del baratto attraverso sistemi di credito
sociale flessibili, prima ancora che il problema del baratto sorga.

Tuttavia, i gruppi che commerciano regolarmente con realtà esterne, o
che sviluppano l'agricoltura e raggiungono popolazioni statiche maggiori
rispetto alla dimensione tipica di una tribù, iniziano inevitabilmente a
identificare e utilizzare una qualche forma di denaro. Questo fornisce
loro un'unità contabile più liquida, divisibile, trasportabile e
ampiamente accettata per conservare e scambiare valore anche con persone
sconosciute. Oltre a continuare a utilizzare sistemi di credito sociale,
si affidano anche al registro naturale, riuscendo così a eludere il
problema della doppia coincidenza dei desideri che altrimenti
comprometterebbe il successo degli scambi.

L'impiego di proto-monete da collezione, per via del notevole sforzo
richiesto per produrle, può apparire arbitrario agli osservatori esterni
a quella cultura. Perché impiegare così tanto tempo per realizzare, ad
esempio, perline di conchiglia? Non sarebbe uno spreco di risorse in un
ambiente aspro e a bassa tecnologia, tipico di cacciatori-raccoglitori,
dove ogni risorsa è preziosa e più di un terzo dei bambini non raggiunge
nemmeno l'età adulta? Non andrebbe meglio impiegare il tempo in eccesso
in altre attività? La risposta è che tale lavoro rappresenta un buon
utilizzo delle risorse durante i periodi di abbondanza e finisce per
ripagarsi ampiamente, poiché un mezzo di scambio e una riserva di valore
standardizzati e credibili rendono tutte le altre transazioni economiche
più efficienti.

Man mano che un'economia diventa più complessa, cresce anche il numero
delle possibili combinazioni di baratto tra diversi tipi di beni e
servizi. Ad esempio, se un'economia produce cinque prodotti differenti,
esistono 10 coppie di scambio uniche. Se ne produce 20, le coppie uniche
salgono a 190. Un'economia con 100 prodotti differenti conta ben 4.950
coppie uniche di scambio. A questo punto, qualsiasi baratto, salvo
quello relativo all'essenziale, risulterebbe estremamente inefficiente.

Allora, se una società richiede interazioni più complesse o che non
facciano affidamento sulla fiducia, oltre a quanto consentito dal
credito sociale flessibile, essa ha bisogno di qualche unità di conto
standard --- o denaro --- che funga da un lato della coppia commerciale
con ogni altro bene o servizio.

In particolare, tra gli asset che una società scambia, uno o due dei più
scarsi, divisibili, durevoli, portatili e liquidi tendono a salire in
cima. Un fruttivendolo di mele che necessita di alcuni attrezzi (un
fabbro), carne (un allevatore), lavori di riparazione (un falegname) e
medicinali per i suoi figli (un medico) non può permettersi di perdere
tempo a cercare persone che abbiano ciò di cui ha bisogno e che, al
contempo, desiderino una grande quantità di mele in quel preciso
istante. Un sistema di baratto così esteso tra vicini non si sviluppa
naturalmente. Invece, lei ha semplicemente bisogno di poter vendere le
sue mele (altamente stagionali e di breve durata) per ottenere una
riserva duratura e ampiamente accettata che potrà usare, col tempo, per
acquistare quei beni quando ne avrà bisogno.

Nel 1776, Adam Smith discusse l'emergere del denaro come soluzione al
problema del baratto nel suo \emph{Wealth of Nations}. I teorici del
credito obiettano questo esempio e l'ordine cronologico degli eventi
relativi al baratto in generale, ma tale obiezione e il dibattito più
ampio a riguardo sono affrontati in dettaglio nel Capitolo 4 di questo
libro. Dopo l'esplorazione del tema da parte di Smith, il denaro-merce
come argomento dettaglia diventa fortemente enfatizzato da chi
appartiene alla scuola austriaca di economia, fondata da Carl Menger nel
XIX secolo e ulteriormente sviluppata da Ludwig von Mises, Friedrich
Hayek e molti altri.

In questo modo di pensare, il denaro dovrebbe essere divisibile,
portatile, durevole, fungibile, verificabile e scarso. Di solito (ma non
sempre) possiede anche una certa utilità. Diversi tipi di denaro possono
essere considerati come aventi ``punteggi'' differenti lungo queste
metriche:

\begin{itemize}
\item
  Divisibile significa che il denaro può essere suddiviso in varie unità
  adatte a diverse dimensioni d'acquisto.
\item
  Portatile significa che il denaro è facile da trasportare su lunghe
  distanze, il che implica che debba contenere molto valore in un peso
  ridotto.
\item
  Durevole significa che il denaro è facile da conservare nel tempo; non
  marcisce, non arrugginisce e non si rompe facilmente.
\item
  Fungibile significa che le singole unità di denaro non differiscono in
  modo significativo l'una dall'altra; una vale quanto un'altra.
\item
  Verificabile significa che chi vende beni o servizi in cambio di
  denaro può facilmente accertarsi che il denaro sia effettivamente ciò
  che sembra.
\item
  Scarso significa che l'offerta di denaro non aumenta rapidamente.
\end{itemize}

Utilità significa che il denaro ha un valore intrinseco in qualche modo;
per esempio, può essere consumato o possedere un valore estetico.

Sommando insieme queste caratteristiche, il denaro è il ``bene più
vendibile'' disponibile in una società, ossia il bene in grado di essere
scambiato nel modo più agevole -- il più capace di essere venduto. Il
denaro è quel bene che possiede la massima universalità, nel senso che
le persone lo desiderano oppure si rendono conto di poterlo scambiare e,
a loro volta, scambiarlo facilmente e in modo affidabile con
qualcos'altro che desiderano. Nell'articolo ``On the Origin of Money''
Menger descriveva come il denaro ideale trasporti valore sia nello
spazio che nel tempo, ovvero possa essere trasportato su lunghe distanze
in modo efficiente oppure conservato per essere speso in
futuro.\footnote{Carl Menger, ``On the Origin of Money.''} Inoltre, un
aspetto fondamentale della vendibilità è la liquidità, che indica come
qualcuno debba poter comprare o vendere grandi quantità relativamente
facilmente, senza perdere troppo valore a causa di ampi spread di prezzo
o della mancanza di volumi di scambio sufficienti. In molti sensi, la
liquidità è una misura di accettabilità: più qualcosa è ampiamente
accettato e detenuto, più fluido risulta il suo scambio.

La scarsità spesso determina il vincitore tra due denaro basati su
commodity concorrenti. Tuttavia, non si tratta solo di quanto l'asset
sia raro. In realtà, una rarità estrema può essere negativa per la
liquidità e trasformare una commodity in una forma di denaro poco
vendibile. Un concetto importante da conoscere in questo contesto è il
rapporto stock-to-flow, che misura quanta offerta esistente (lo stock)
vi sia attualmente a livello regionale o globale, divisa per quanta
nuova offerta (il flusso) possa essere prodotta in un anno.

Ad esempio, i minatori d'oro aggiungono ogni anno circa l'1,5\% di nuovo
oro rispetto alla stima dell'offerta attualmente presente sopra il
suolo,\footnote{Nuno Palma, ``The Real Effects of Monetary Expansions:
  Evidence from a Large-Scale Historical Experiment''; Saifedean Ammous,
  \emph{The Bitcoin Standard: The Decentralized Alternative to Central
  Banking}, 28--29.} e, a differenza della maggior parte delle altre
commodity, la maggior parte dell'oro non viene consumata; viene
continuamente fuso e conservato in varie forme e in differenti luoghi.

L'oro non marcisce, non arrugginisce e non si corrode con la facilità
della maggior parte degli altri materiali. È chimicamente inerte e forma
quindi pochissimi composti. Può essere rifuso innumerevoli volte e
persino dissolto in alcuni tipi di acido per poi essere nuovamente
recuperato tramite filtrazione. Può essere fatto esplodere e disperso,
ma quei frammenti non si arrugginiscono fino a scomparire come accade
con altri materiali, e quindi sono recuperabili. A parte tracce che
finiscono nelle schede elettroniche o che affondano sul fondo
dell'oceano in relitti, la maggior parte dell'oro mai estratto è ancora
sotto il controllo umano (ed anche quelle quantità di oro perduto sono
tecnicamente recuperabili, al giusto prezzo). È praticamente
indistruttibile.\footnote{Ammous, \emph{The Bitcoin Standard: The
  Decentralized Alternative to Central Banking}, 27--28.}

La combinazione di estrazione costante dell'oro e della rara perdita del
materiale estratto ha portato l'oro ad avere un rapporto stock-to-flow
pari a circa 100/1,5 = 67 in media, il più alto rapporto stock-to-flow
di qualsiasi commodity. A livello globale, si possiede collettivamente
una quantità pari a 67 anni della produzione annuale media, secondo le
stime del World Gold Council. Il tasso di crescita dell'offerta è
variato tra l'1\% e il 2\% nel corso dell'ultimo secolo, una fascia
sorprendentemente bassa e ristretta.\footnote{World Gold Council,
  ``Above-Ground Stock.''} Anche negli anni '70, quando il prezzo
dell'oro in dollari aumentò di un ordine di grandezza, ciò non riuscì ad
influire in modo significativo sul tasso di crescita annuo dell'offerta
in percentuale alle scorte esistenti. Prima di quel periodo, gli unici
momenti in cui l'offerta raffinatamente disponibile di oro aumentava a
ritmo accelerato erano quelli in cui le società industriali scoprivano
un nuovo continente ed esploravano depositi facilmente accessibili,
oppure quando inventavano nuove tecniche per estrarre a profitti
depositi precedentemente non redditizi.

Se un asset possiede un premio monetario aggiuntivo rispetto al suo puro
valore d'uso, i partecipanti al mercato sono fortemente incentivati a
produrne di più. Solo asset altamente resistenti agli aumenti di
fornitura in rapporto alla quantità totale esistente possono affrontare
questa sfida e, di conseguenza, diventare e rimanere denaro ampiamente
accettato a livello globale.

D'altra parte, se un asset è così raro che pochissimi lo possiedono,
esso può risultare estremamente prezioso se utile, ma ha scarso impiego
come denaro; non è liquido, né diffuso o accettato, e pertanto i costi
di transazione per comprarlo e venderlo risultano elevati. Alcuni
elementi, come il rodio, ad esempio, sono più rari dell'oro ma
possiedono bassi rapporti stock-to-flow perché vengono consumati
dall'industria alla stessa velocità con cui vengono estratti. Una moneta
o una barra di rodio può essere acquistata come oggetto da collezione di
nicchia o riserva di valore, ma non è utile come denaro accettato a
livello societario e, perciò, non emerge naturalmente in tale ruolo. Lo
stesso vale per i meteoriti o altre cose insolitamente rare. A partire
dal 2022 sono stati scoperti 1.878\footnote{Randy Korotev, ``Meteorite
  Numbers in the United States, Canada, and Mexico,'' Washington
  University in St.~Louis.} meteoriti noti negli Stati Uniti, e ce ne
sono decine di migliaia in altre giurisdizioni, cosa che rende i
meteoriti da collezione rari e preziosi, ma non adatti come denaro.
Elementi come le barre di rodio o i meteoriti semplicemente non
possiedono sufficiente liquidità o divisibilità per essere utili in
veste di denaro.

Pertanto, un rapporto stock-to-flow elevato e duraturo tende ad essere
il miglior indicatore di scarsità per qualcosa che deve essere
considerata denaro --- insieme agli altri attributi sopra elencati ---
piuttosto che la semplice rarità assoluta. Una commodity con un alto
rapporto stock-to-flow è difficile da produrre, eppure ne è già stata
prodotta molta, ed è ampiamente distribuita e detenuta, perché o non
viene consumata rapidamente o non viene consumata affatto. Questo
insieme relativamente raro di attributi è ciò che consente a qualcosa di
essere denaro, anziché limitarsi ad essere un oggetto da collezione.

Nel corso della storia, varie pietre, perline, piume, conchiglie, sale,
pellicce, tessuti, zucchero, noci di cocco, bestiame, rame, argento, oro
ed altri materiali hanno servito da denaro. Ciascuno di essi ha ottenuto
differenti ``punteggi'' rispetto ai vari attributi del denaro,
evidenziandone punti di forza e debolezze specifici. Spesso era la norma
che almeno due forme di denaro fossero ampiamente utilizzate
contemporaneamente, poiché nessun tipo di denaro riusciva a soddisfare
perfettamente ogni funzione.

Il sale, ad esempio, è divisibile, durevole, verificabile, fungibile e
possiede un'importante utilità, ma non è molto prezioso per unità di
peso e non è molto raro, quindi non ottiene buoni punteggi in termini di
portabilità e scarsità. La parola ``salary'' deriva dal latino
\emph{salarium}, che indicava un reddito denominato in sale.

L'oro è il migliore sotto quasi ogni punto di vista ed è, di gran lunga,
la merce con il più alto rapporto stock-to-flow. L'unica debolezza
rispetto alle altre commodity è la sua scarsa divisibilità. Anche una
piccola moneta d'oro vale più della maggior parte degli acquisti ed è il
risultato di un immenso lavoro. È il re delle commodity. L'oro, inteso
come forma ideale di ornamento, rappresenta sostanzialmente una versione
tecnologicamente avanzata delle perline di conchiglie. La sua
applicazione più comune è quella di fungere da ricchezza indossabile o
esibibile in molte culture diverse. È denaro che possiamo facilmente
portare con noi e che possiamo utilizzare per segnalare il nostro status
sociale agli altri.

Per gran parte della storia umana, l'argento è stato il vincitore in
termini di uso quotidiano. Ha il secondo miglior punteggio dopo l'oro
per la maggior parte degli attributi monetari, incluso il secondo più
alto rapporto stock-to-flow, e supera l'oro in termini di divisibilità,
poiché le piccole monete d'argento sono ideali per le transazioni
giornaliere. È la regina delle commodity, e come nel gioco degli
scacchi, sebbene il re possa essere il pezzo più importante, la regina è
il pezzo più utile.

Di conseguenza, l'oro veniva spesso detenuto dai ricchi come deposito (e
manifestazione) di valore a lungo termine e come mezzo di scambio per
acquisti molto importanti, mentre l'argento costituiva il denaro più
tattico, utilizzato come deposito di valore e mezzo di scambio dalla
maggior parte delle persone che lavorano. Un sistema monetario
bimetallico era comune in molte regioni del mondo a causa della limitata
divisibilità dell'oro, nonostante le sfide insite in questo approccio a
denaro multiplo.

Perché oro e argento hanno sconfitto tutte le altre forme di denaro
basate sulle commodity per arrivare all'era moderna come denaro
utilizzabile? La risposta è che questi due sono riusciti a mantenere
rapporti stock-to-flow sufficientemente elevati nonostante l'avanzare
della tecnologia umana, anche con un sostanziale premio monetario
applicato su di essi. Hanno potuto conservare la loro rarità nel tempo,
pur essendo ampiamente accettati, diffusi, durevoli, portatili,
divisibili e riconfigurabili.

Il potere d'acquisto di una moneta basata su una merce può essere
concettualmente diviso in due componenti: il valore di utilità e un
premio monetario superiore a tale valore. Il valore di utilità
corrisponde all'effettivo impiego di quella merce per uno scopo
economico (consumo o produzione), mentre il premio monetario rappresenta
un valore aggiuntivo derivante dal fatto che molte persone la detengono
come forma di risparmio, in assenza di alternative migliori. La
differenza tra una merce comune e una moneta-merce risiede nel fatto che
chi possiede unità di una moneta-merce non le utilizza esclusivamente
per il loro fine primario, bensì le accumula come riserva, poiché si
tratta di un bene altamente commerciabile che potrà essere facilmente
rivenduto in futuro. Le merci non monetarie, come il petrolio greggio,
vengono per lo più valutate in termini del loro valore di utilità: vi è
una domanda effettiva e una fornitura produttiva, mentre pochissimi ne
detengono il possesso per lunghi periodi. Le monete-merce predominanti
in una determinata regione, come l'oro, invece, godono di una domanda in
eccesso, dovuta alla diffusione del possesso da parte di soggetti che
non sono consumatori finali, il che ne accresce notevolmente il valore
di mercato complessivo. La gente detiene una moneta d'oro non per
utilizzarla direttamente, ma perché sa che l'oro ha molteplici usi e
che, accumulandone una certa quantità, sta conservando valore in un bene
dotato di elevata liquidità e accettabilità a livello globale.

Questo premio monetario (questo surplus rispetto al valore d'utilità)
funge da massiccia e permanente pubblicità, spingendo le persone a
cercare di produrne di più. Solo le merci più scarse, ossia quelle con
il più alto rapporto stock-flusso, possono sostenere tale ``pubblicità''
nel lungo periodo. I premi monetari possono essere applicati anche ad
altri asset, come le proprietà fronte mare o le opere d'arte, poiché
questi vengono spesso detenuti più come forma di risparmio che per il
loro valore intrinseco. Lo svantaggio di questo approccio risiede nel
fatto che tali asset non monetari mancano intrinsecamente della
portabilità, della liquidità, della fungibilità e della divisibilità
tipiche dell'oro e delle altre monete.

Molti ritengono che il denaro sia una delusione condivisa. Secondo
questo punto di vista, una società può scegliere qualsiasi cosa come
denaro, purché la maggior parte dei suoi membri vi creda. Per esempio,
le fermagli per carta potrebbero diventare denaro se tutti concordassimo
nel riconoscerne il valore. Sebbene ciò possa apparire vero all'inizio,
non è sostenibile: se l'offerta di quel denaro può essere rapidamente
espansa, i risparmi di tutti verranno diluiti altrettanto velocemente.
Inoltre, un premio monetario offre fortissimi incentivi affinché le
persone ne producano di più, se possibile. Pertanto, se una forma di
denaro non viene selezionata con giudizio all'interno di una società,
bastano pochi individui, capaci di staccarsi dalla delusione condivisa,
per rendersi conto che quel denaro non è scarso e per produrne in
quantità maggiori, estrapolando valore da tutti gli altri. In
alternativa, individui appartenenti a un'altra società possono sfruttare
tale illusione collettiva. Di conseguenza, unicamente quelle forme di
denaro che possiedono un notevole grado di genuina scarsità possono
mantenere il loro impiego in una società per lunghi periodi.

Il denaro a conchiglie è durato migliaia di anni in varie regioni, ma
alla fine divenne inadeguato di fronte alla rivoluzione industriale.
Anche pellicce, bestiame, sale, tabacco e altre forme di denaro svolsero
ruoli utili in diversi momenti, ma la crescente capacità tecnica della
civiltà li rese progressivamente inadeguati come mezzi di scambio.
Funzionarono finché la tecnologia non proibì loro di continuare a
operare. Le sezioni che seguono illustrano diversi esempi di questo
concetto.

\textbf{Conchiglie}

Come descritto in precedenza, il denaro conchiglia fu utilizzato per
lunghi periodi in alcune parti delle Americhe, dell'Africa e dell'Asia.
In alcune zone veniva impiegato in maniera più cerimoniale, mentre in
altre aveva un uso più letterale, in senso transazionale come denaro.

La varietà di perline di conchiglie, il wampum, comune sulla costa
orientale del Nord America, era quella destinata a scopi cerimoniali
dalle tribù originarie che lo avevano sviluppato. Tuttavia, i coloni del
New England lo incorporarono nel loro sistema monetario nei primi anni
del 1600, fissando un tasso di cambio in cui un certo numero di
conchiglie equivaleva alle loro monete.\footnote{Claire Priest,
  ``Currency Policy and Legal Development in Colonial New England,''
  \emph{The Yale Law Journal}, 1324--26; Glyn Davies, \emph{A History of
  Money: From Ancient Times to the Present Day}, 40--41.} Il wampum
viola, essendo più raro, veniva valutato al doppio rispetto a quello
bianco.

Col tempo le leggi sul tasso di cambio fisso furono abrogate e il wampum
fu valutato in base al mercato. Nel 1812 John Campbell aprì il Campbell
Wampum Mill nel New Jersey e, grazie a moderne tecniche di perforazione,
fu possibile produrre in serie le perline di wampum a un ritmo molto più
rapido rispetto al passato.\footnote{Kristin Beuscher, ``From Pasack to
  the Plains.'' \emph{Northern Valley Press}, 21 maggio 2019.} John
Jacob Astor, della American Fur Company, acquistò questo wampum prodotto
industrialmente e lo utilizzò per commerciare pellicce con le
popolazioni indigene in Canada.

Nel lungo arco del tempo, il denaro conchiglia, in tutte le sue forme e
in ogni parte del mondo, divenne inadeguato di fronte al potere
dell'industria. Strumenti metallici e nuove tecnologie resero il denaro
conchiglia un mediocre mezzo per conservare il valore. In alcune zone,
tuttora, la tradizione di realizzare il wampum a mano viene mantenuta
viva dai discendenti dei popoli indigeni che lo usavano, riportandolo ai
suoi scopi più cerimoniali e sfumati, come strumento per preservare la
tradizione culturale.

\textbf{Tabacco}

All'inizio del 1600, Virginia, Maryland e North Carolina iniziarono a
utilizzare il tabacco come denaro, anche come moneta legale per decreto
governativo. Col tempo, tuttavia, emersero dei problemi in quel sistema,
analoghi a quelli riscontrati nell'industrializzazione delle conchiglie
di wampum.

Poiché al tabacco veniva attribuito un premio monetario superiore al suo
valore d'uso, sorgeva un notevole incentivo a coltivarlo in quantità
maggiori, cercando di catturare -- e quindi, col tempo, erodere -- quel
premio.\footnote{Milton Friedman, ``Understanding Inflation,''
  3:01--5:28.} A differenza dell'oro, il tabacco non possiede una
scarsità naturale sufficiente a impedire che tale premio venga sfruttato
e dissipato. Il naturale esito di questo meccanismo fu un forte aumento
dell'offerta di tabacco, che determinò una marcata inflazione dei prezzi
di beni e servizi, misurata in termini di tabacco. In risposta a questo
eccesso, i governi coloniali introdussero restrizioni sulla coltivazione
del tabacco, limitando ad esempio la produzione a determinati gruppi,
per creare una scarsità artificiale in una sostanza altrimenti non
scarsa di per sé.\footnote{Ron Mitchener, ``Money in the American
  Colonies,'' \emph{EH.net}.} In altre parole, solo i gruppi favoriti
dal governo potevano operare come ``stampatori di denaro tabaccoso''.
Una soluzione chiaramente imperfetta e difficile da mantenere
all'infinito.

Un ulteriore problema era che il tabacco non è perfettamente fungibile,
essendovi qualità superiori e inferiori. Se tutto il tabacco venisse
valutato a un determinato tasso di cambio, si creava un forte incentivo
a spendere localmente quello di qualità inferiore e a esportare quello
di qualità superiore, dove veniva valutato in maniera più adeguata. Di
conseguenza, si sviluppò la prassi di immagazzinarlo e classificarlo,
con l'emissione di certificati cartacei standardizzati contro il tabacco
detenuto. Così venne creato un sistema a ``standard del tabacco''. Per
chi deteneva il certificato, ciò comportava un rischio di controparte
aggiuntivo al rischio già esistente di devalorizzazione del tabacco
sottostante.\footnote{Sharon Ann Murphy, \emph{Other People's Money: How
  Banking Worked in the Early American Republic}.}

In sintesi, il problema del tabacco era che non riusciva a resistere
all'impulso di produrne di più -- e un premio monetario è, per sua
natura, un forte incentivo a generarne in quantità maggiore -- un
fardello troppo pesante per qualsiasi merce. L'oro è riuscito a superare
questa sfida per migliaia di anni, mentre il tabacco no. Esso ebbe un
ruolo utile per un certo periodo nelle colonie del sud, poiché nelle
loro fasi iniziali erano piccole e poco sviluppate e mancavano di specie
sufficienti, ma il sistema monetario basato sul tabacco cessò di avere
senso una volta che le colonie crebbero e si svilupparono. Diversi
tentativi del governo di prolungare la vita utile di tale denaro
ritardarono il suo inevitabile declino per un certo periodo, ma alla
fine il sistema divenne chiaramente impraticabile rispetto a forme di
denaro più solide e fu completamente abbandonato.\footnote{Farley Grubb,
  ``Colonial Virginia's Paper Money, 1755--1774''; Barry Eichengreen,
  \emph{Exorbitant Privilege: The Rise and Fall of the Dollar and the
  Future of the International Monetary System}, 9--11.}

\textbf{Cocoa}

Nel corso di alcune regioni dell'America centrale e meridionale, le
civiltà usavano il cacao come valuta. Questa pratica era in vigore
all'arrivo degli Europei, e i murales mostrano che risale a molti secoli
fa.\footnote{Stefania Moramarco e Loreto Nemi, ``Nutritional and Health
  Effects of Chocolate,'' 134--35.} I semi di cacao sono piccoli,
relativamente fungibili e possono essere conservati per un buon periodo.
Soprattutto, il loro gusto è molto apprezzato! Queste caratteristiche
rendevano il cacao una forma di denaro adatta.\footnote{Ingrid Fromm,
  ``From Small Chocolatiers to Multinationals to Sustainable Sourcing: A
  Historic Review of the Swiss Chocolate Industry,'' 73.}

Come molte società preindustriali, queste civiltà adottavano un sistema
flessibile di credito sociale e baratto, che tendeva a utilizzare uno o
due beni altamente commerciabili per facilitare gli scambi. Uno o due
beni scarsi e liquidi tendono a diventare denaro in contesti in cui il
credito non basta. Gli Aztechi, ad esempio, impiegavano anche una moneta
in rame, con unità modellate a forma di zappa decorativa o di ascia
opaca ornamentale. Migliaia di semi di cacao potevano essere scambiati
per una di queste unità di rame, qualora fosse necessario effettuare
transazioni più consistenti o conservare ricchezza liquida in un'unità
piccola e facilmente trasportabile per un periodo più lungo.\footnote{Dudley
  Easby, ``Early Metallurgy in the New World,'' 77.}

Quando gli Europei giunsero nelle Americhe, impiegarono il cacao e il
rame come valuta, ma, come in altre parti del mondo, questa pratica fu
infine sostituita da forme di denaro ancora più scarse.

\textbf{Pietre Rai}

Gli abitanti di un'isola del Pacifico meridionale chiamata Yap
utilizzavano enormi pietre come moneta. Queste ``pietre rai'' (o
``pietre fei'') venivano scolpite in dischi circolari di pietra con un
buco nel centro e si presentavano in varie dimensioni, da pochi pollici
di diametro a oltre dieci piedi. Molte di esse misuravano almeno un paio
di piedi, e quindi pesavano centinaia di libbre. Le più grandi
superavano i dieci piedi di diametro e pesavano migliaia di
libbre.\footnote{Milton Friedman, \emph{Monetary Mischief}, 3--7.}

Curiosamente, ho visto questo esempio utilizzato sia da un economista
austriaco che da un economista MMT. Ciò che lo rende interessante è che
queste due scuole di pensiero hanno concezioni molto diverse di che cosa
sia il denaro. Gli economisti austriaci tendono a enfatizzare il denaro
come merce, mentre i chartalisti (e gli economisti MMT nella loro forma
attuale) tendono a vederlo come un registro pubblico.\footnote{William
  Luther e Alexander Salter, ``Synthesizing State and Spontaneous Order
  Theories of Money.''} Queste visioni possono essere riconciliate
comprendendo che le monete-merce vengono impiegate come un registro, con
la natura che funge da amministratore di tale registro. Questa
riconciliazione viene approfondita nel Capitolo 4.

Ciò che rendeva uniche queste pietre rai era il fatto che erano
realizzate in un tipo speciale di calcare non presente sull'isola. Gli
abitanti di Yap viaggiavano per 250 miglia fino a un'isola vicina,
chiamata Palau, per estrarre il calcare e riportarlo a Yap.

Mandavano una squadra composta da molte persone su quell'isola lontana,
estraevano la roccia in enormi lastre e la trasportavano a bordo di
barche di legno. Immaginate di dover trasportare una pietra del peso di
diverse migliaia di libbre per 250 miglia di oceano aperto su una barca
di legno. Nel corso degli anni, alcune persone persero la vita in questo
processo. Richiedeva un'enorme quantità di tempo, sforzo e comportava
notevoli pericoli.

Una volta trasformate in pietre rai a Yap, quelle di grandi dimensioni
non venivano spostate. Si tratta di una piccola isola, e tutte le pietre
venivano catalogate tramite la tradizione orale. Un proprietario poteva
scambiare una pietra con altri beni e servizi importanti, e invece di
spostare la pietra, l'operazione consisteva nell'annunciare alla
comunità che ora quella pietra apparteneva a un'altra
persona.\footnote{Friedman, \emph{Monetary Mischief}, 4.}

In questo senso, le pietre rai rappresentavano un vero e proprio sistema
di registro, non molto diverso dal nostro attuale sistema monetario. Il
registro tiene traccia di chi possiede cosa, e questo particolare
registro era distribuito oralmente, il che ovviamente può funzionare
solo in una piccola comunità.

Già al momento in cui gli europei ne fecero documentazione, su Yap
esistevano migliaia di pietre rai, frutto di generazioni di estrazione,
trasporto e lavorazione. Le pietre rai vantavano così un elevato
rapporto stock-to-flow, motivo principale per cui potevano essere
adoperate come denaro.

Verso la fine del XIX secolo, un irlandese di nome David O'Keefe giunse
sull'isola e ne intuì il funzionamento. Con la sua tecnologia superiore,
poteva facilmente estrarre la pietra da Palau e portarla a Yap per
realizzare le pietre rai, diventando così l'uomo più ricco dell'isola,
capace di far lavorare i locali e di barattare le pietre con svariate
merci. Un articolo del \emph{Smithsonian Magazine} di Mike Dash,
intitolato ``David O'Keefe: The King of Hard Currency'', lo riassumeva
così:

\begin{quote}
Più l'irlandese conosceva Yap, più si rendeva conto che c'era una sola
merce, e una sola, che il popolo desiderava: il ``denaro in pietra'' per
cui l'isola era rinomata e che veniva utilizzato in quasi tutte le
transazioni ad alto valore a Yap. Queste monete venivano estratte
dall'aragonite, un particolare tipo di calcare che scintilla alla luce
ed era prezioso perché non si trovava sull'isola. Il genio di O'Keefe fu
riconoscere che, importando le pietre per i suoi nuovi amici, poteva
scambiarle con la manodopera nelle piantagioni di cocco di Yap. I yapesi
non erano infatti molto interessati a sudare per dei bibelot, che
altrove nel Pacifico fungevano da valuta comune (né, come ammise un
visitatore, avrebbero dovuto essere, dato che ``tutti i cibi, le bevande
e gli indumenti sono facilmente reperibili, quindi non c'è baratto né
debito''), ma sarebbero lavorati come pazzi per ottenere denaro in
pietra.\footnote{Mike Dash, ``David O'Keefe: The King of Hard
  Currency,'' \emph{Smithsonian Magazine}, 28 luglio 2011.}
\end{quote}

In sostanza, la tecnologia avanzata finì per compromettere il rapporto
stock-to-flow delle pietre rai, aumentando drasticamente il flusso.
Estranei come O'Keefe, armati di tecnologie più sofisticate, potevano
portare sull'isola innumerevoli pietre, diventare così i più ricchi e,
di conseguenza, aumentare l'offerta e ridurne il valore economico nel
tempo.

Tuttavia, anche i locali si dimostrarono astuti e, col tempo, riuscirono
a mitigare questo fenomeno. Cominciarono ad attribuire maggiore valore
alle pietre più antiche (quelle che si poteva provare fossero state
estratte a mano decenni o addirittura secoli fa), poiché quel
sottoinsieme rimaneva scarso. È come accade con l'arte: non importa
quanta nuova opera venga prodotta, Vincent van Gogh non ne crea altre, e
così i suoi dipinti tendono ad apprezzarsi anziché essere svalutati
dalla nuova offerta di altri artisti. Tuttavia, il destino era ormai
scritto; le pietre rai non rappresentavano più un sistema monetario
valido.

Le cose presero una piega più oscura. Come descritto ulteriormente
nell'articolo \emph{Smithsonian} di Dash:

\begin{quote}
Con O'Keefe morto e i tedeschi ormai ben radicati, le cose iniziarono ad
andare male per gli abitanti di Yap dopo il 1901. I nuovi governanti
arruolarono gli isolani per scavare un canale attraverso l'arcipelago e,
quando gli Yapesi si mostrarono riluttanti, cominciarono a sequestrare
il loro denaro di pietra, deturpando le monete con croci dipinte di nero
e dicendo ai loro sudditi che potevano essere riscattate solo attraverso
il lavoro. La cosa peggiore fu l'introduzione da parte dei tedeschi di
una legge che vietava agli Yapesi di allontanarsi per oltre 200 miglia
dalla loro isola. Ciò impose un immediato arresto dell'estrazione del
fei, anche se la valuta continuò ad essere utilizzata anche dopo che le
isole furono conquistate dai giapponesi e successivamente occupate dagli
Stati Uniti nel 1945.\footnote{Dash, ``O'Keefe.''}
\end{quote}

Molte delle pietre furono prelevate e usate come ancore improvvisate o
materiali da costruzione dagli invasori giapponesi durante la Seconda
Guerra Mondiale, riducendo il numero di pietre sull'isola.

Le pietre Rai furono una forma notevole di denaro finché esistevano,
poiché non avevano alcuna utilità pratica se non quella estetica. Esse
rappresentavano un mezzo per esibire e registrare la ricchezza, e poco
altro. In sostanza, si trattava di una delle prime versioni di un
registro pubblico formale, poiché molte di esse non venivano spostate e
solo i registri orali (o, successivamente, le marcature fisiche lasciate
dai tedeschi) determinavano chi le possedeva.

\textbf{Piume}

Le piume venivano spesso usate come oggetti simili al denaro da tribù in
tutto il mondo. Uccelli maestosi, come aquile o pappagalli,
caratterizzati da piume eccezionalmente grandi o belle, venivano
raccolti da numerose culture.

Talvolta esse assumevano un valore cerimoniale, come le piume d'aquila
impiegate come copricapo dai capi tribù. Altre volte venivano raccolte
in modo più informale, semplicemente per la loro bellezza e il loro
fascino, per poi essere occasionalmente utilizzate negli
scambi.\footnote{David Jones, \emph{Native North American Armor,
  Shields, and Fortification}, 41.} Uno svantaggio delle piume è la loro
scarsa durata; col tempo è facile che una piuma si consumi e si rovini
--- soprattutto se si è sempre in movimento.

Nelle Isole Salomone, una forma di valuta a base di piume veniva
realizzata da artigiani tribali in rotoli simili a cinture. Ogni rotolo
era composto da piume rosse provenienti da centinaia di minuscoli
honeyeaters scarlatti, insieme a linfa e ad altre sostanze. Il risultato
era una forma di denaro in piume più resistente e originale. Tuttavia,
per sua natura, questo tipo di denaro è intrinsecamente limitato nella
fungibilità e nella liquidità, confinato a una giurisdizione ristretta
sia dal punto di vista geografico che culturale.

\textbf{Perle africane}

Le perle da scambio venivano utilizzate in alcune regioni dell'Africa
occidentale come forma di denaro per molti secoli, risalendo almeno al
XIV secolo e anche a epoche precedenti. Potevano essere impiegati vari
materiali preziosi, come corallo, ambra e vetro. Le perle in vetro
veneziano, infatti, giunsero gradualmente nell'Africa sub-sahariana
grazie al commercio. Uno dei primi riferimenti documentati a tale
pratica proviene da Ibn Battuta, il famoso viaggiatore marocchino del
XIV secolo, i cui itinerari attraversarono ampie zone dell'Africa e
dell'Asia.

Emil Sandstedt, nel suo libro \emph{Money Dethroned: A Historical
Journey}, ha citato Ibn Battuta riguardo all'osservazione fatta da
Battuta sulle pratiche monetarie nell'Africa occidentale:

\begin{quote}
Un viaggiatore in questo paese non porta provviste, né cibo semplice né
condimenti, e né oro né argento. Porta soltanto pezzi di sale e
ornamenti di vetro, che la gente chiama perle, insieme ad alcune merci
aromatiche.\footnote{Emil Sandstedt, \emph{Money Dethroned: A Historical
  Journey}, 43}
\end{quote}

Si trattava di società pastorali, spesso in movimento, per cui poter
indossare il denaro sotto forma di fili di bellissime perle risultava
molto utile. Queste perle mantenevano elevati rapporti stock-to-flow
perché venivano conservate e scambiate come moneta, pur essendo
difficili da produrre con il livello tecnologico dell'epoca.

Col tempo, gli europei iniziarono a viaggiare e ad accedere con maggiore
frequenza all'Africa occidentale, notarono questo uso delle perle da
scambio e ne trassero profitto. Dotati della tecnologia per la
lavorazione del vetro, gli europei potevano produrre perle di notevole
bellezza con modesto sforzo. Così facevano scambiando tonnellate di
queste perle con merci e altri beni (e, purtroppo, anche con schiavi
umani).\footnote{Emil Sandstedt, ``Racconti sul denaro morbido --- Il
  sentiero delle perle'', \emph{Medium}, 26 maggio 2019.}

A causa di questa asimmetria tecnologica, gli europei svalutavano le
perle di vetro aumentando la loro offerta in tutta l'Africa occidentale,
estraendo così molto valore da quei sistemi. Gli abitanti dell'Africa
occidentale continuavano a scambiare rari beni locali, che andavano da
merci importanti a vite umane inestimabili, per perle di vetro molto più
abbondanti di quanto si rendessero conto.\footnote{Laure Dussubieux et
  al., ``Il commercio europeo in Malawi: le evidenze delle perle di
  vetro.''} Di conseguenza, hanno scambiato i loro veri beni di valore
per beni fittizi. Scegliere il tipo di denaro sbagliato in questo modo
può avere conseguenze disastrose.

Non fu affatto semplice come si potrebbe sospettare per gli europei
realizzare questo trucco, perché le preferenze degli africani per certi
tipi di perline cambiavano nel tempo, e le diverse tribù avevano gusti
differenti. Ciò ricordava l'esperienza con le pietre rai, dove una volta
che le nuove forniture di pietre rai iniziarono ad arrivare più
rapidamente grazie alla tecnologia industriale, il popolo di Yap
cominciò a dare maggior valore a quelle vecchie rispetto a quelle nuove.
In questo caso, i gusti dell'Africa occidentale parevano variare in base
all'estetica e alla scarsità. Tuttavia, questa pratica assegnava a
quella forma di denaro un punteggio inferiore in termini di fungibilità,
riducendone così l'affidabilità come mezzo di scambio.

\textbf{Denaro da invasione giapponese}

Sebbene non si trattasse propriamente di una moneta-merce, il Giappone
imperiale utilizzava una valuta debole per acquisire i beni e i servizi
scarsi delle regioni sotto il suo controllo.

Durante la Seconda Guerra Mondiale, quando il Giappone imperiale invase
diverse aree dell'Asia, confiscava la valuta forte ai locali e immetteva
in circolazione una carta speciale in sostituzione, denominata ``denaro
da invasione.''\footnote{Dazmin Daud, ``Uno studio su due varianti della
  moneta da \$100 del Malaya durante l'invasione giapponese (campione
  n.~M8A)'', 43.} Questi popoli conquistati erano costretti a
risparmiare e utilizzare una valuta che non aveva alcuna garanzia né
scarsità d'offerta e che, nel tempo, perdeva ogni valore. Era un modo
per il Giappone di sottrarre i risparmi dei propri sudditi, pur
mantenendo una temporanea unità di conto in quelle regioni per
consentire il proseguimento delle transazioni economiche.

In misura meno estrema --- come descriverò più avanti in questo libro
---, purtroppo, ciò che accade ancora oggi in molti paesi in via di
sviluppo è che le persone continuano a risparmiare nella loro valuta
fiat locale, la quale, ogni generazione o giù di lì, viene drasticamente
svalutata, dirottando i risparmi verso i governanti e la classe agiata.

\textbf{Grano}

Nell'antica Babilonia, lo scekel d'argento veniva usato come unità
monetaria chiave, ma il grano era spesso impiegato anche come forma di
pagamento. Il grano costituiva l'alimento base della regione, veniva
frequentemente utilizzato per pagare i salari giornalieri e fungeva
spesso da unità di conto per altre transazioni.\footnote{Goetzmann,
  \emph{Money Changes Everything}, pp.~59--69.}

Il Codice di Hammurabi, che ha quasi quattro mila anni, imponeva il
grano come moneta a corso legale:

\begin{quote}
108. Se un venditore di vino non riceve il grano come prezzo di una
bevanda, ma se ne riceve denaro tramite la grande pietra, oppure riduce
la misura della bevanda in favore del grano, quel venditore di vino sarà
chiamato a rendere conto e verrà gettato in acqua.

111. Se un venditore di vino concede 60 KA di bevanda a credito, al
tempo del raccolto dovrà ricevere 50 KA di grano.

114. Se un uomo non ha un debito di grano o di denaro nei confronti di
un altro, e lo aggredisce per il debito, per ogni sequestro dovrà pagare
un terzo di mana d'argento.

115. Se un uomo ha un debito di grano o di denaro nei confronti di un
altro, e lo aggredisce per il debito, e il soggetto aggredito muore
nella casa di chi lo ha aggredito, in tal caso non sarà prevista alcuna
pena.\footnote{Hammurabi, \emph{The Code of Hammurabi, King of Babylon},
  pp.~37--39.}
\end{quote}

La difficoltà nell'utilizzare un prodotto agricolo come denaro risiede
nel fatto che l'offerta monetaria varia notevolmente nel corso
dell'anno. La stagione del raccolto genera una grande quantità di nuova
moneta in grano, mentre nel resto dell'anno quella fornitura si riduce,
poiché il grano viene trasformato in pane e birra. Gli agricoltori
spesso facevano affidamento sul debito per effettuare i pagamenti,
utilizzando poi il raccolto (se favorevole) per saldare i debiti
contratti durante le stagioni non di raccolto.

In molte società come questa, i raccolti falliti potevano significare la
rovina finanziaria per l'agricoltore. L'agricoltore e/o i suoi familiari
potevano finire per diventare schiavi del debito. Tuttavia, vari re
perdonavano periodicamente i debiti o imponevano limiti ai creditori,
esentando il debitore in determinate circostanze oppure limitando
l'ammontare della servitù richiesta per estinguere i debiti.

\begin{quote}
48: Se un uomo ha un debito, ed una tempesta inonda il suo campo
portando via il raccolto, oppure, per la mancanza d'acqua il grano non
cresce nel campo, in quell'anno non dovrà restituire alcun grano al
creditore, dovrà modificare il suo contratto e non pagherà interessi in
quell'anno.
\end{quote}

\begin{quote}
117: Se un uomo è indebitato e vende sua moglie, suo figlio o sua
figlia, o li vincola al servizio, essi lavoreranno per tre anni nella
casa del loro acquirente o padrone; nel quarto anno verrà loro concessa
la libertà.\footnote{Hammurabi, \emph{Code of Hammurabi}, 27, 41.}
\end{quote}

Babilonia fornì tra i primi esempi noti di pesi e misure formalizzati,
di denaro merce formalizzato, di credito scritto formalizzato, di
contratti di custodia formalizzati e di moneta a corso legale
formalizzata. I re stabilivano le regole fondamentali del commercio e
affrontavano eventuali squilibri strutturali nel sistema non appena si
manifestavano durante il loro regno, mentre i templi fungevano da
sistemi amministrativi affinché il commercio formalizzato potesse
avvenire.

\textbf{Denaro nei videogiochi}

Uno degli aspetti affascinanti dei giochi online multiplayer di massa è
che essi danno origine a vari esperimenti economici non intenzionali. In
sostanza, ripropongono diversi ambienti economici con nuove regole,
dando luogo a studi di casi sugli aspetti emergenti delle economie.

Un esempio noto di ciò fu la formazione monetaria emergente nel
videogioco online \emph{Diablo II}, un popolare gioco multiplayer
d'azione-fantasy uscito nel 2000 e venduto a milioni di copie (io stesso
l'ho acquistato da adolescente). Negli anni numerosi hanno analizzato
l'economia interna al gioco, ma fu l'autore di \emph{21
Lessons},\footnote{Gigi, \emph{21 Lessons: What I've Learned from
  Falling Down the Bitcoin Rabbit Hole.}} Gigi, a portarmene notizia con
un articolo del 2022.\footnote{Gigi, ``Bitcoin Is Digital Scarcity,''
  DerGigi.com, 2 ottobre 2022. Vedi anche Solomon Stein, ``The Origins
  of Money in \emph{Diablo II}.''}

\emph{Diablo II} disponeva di una valuta interna che (come ci si
aspetterebbe in un'ambientazione fantasy) veniva chiamata ``gold''.
Tuttavia, il gold era programmato in modo tale da impedirgli,
inavvertitamente, di rappresentare la forma di denaro migliore del
gioco. In primo luogo, il gold era abbondante ai livelli avanzati di
gioco, ma il personaggio poteva trasportarne solo una quantità limitata.
In secondo luogo, ogni volta che il personaggio moriva si perdeva una
parte del gold posseduto, pur potendo recuperare tutti gli altri
oggetti. Naturalmente, con tali limitazioni sul gold, i giocatori
cercavano di conservare la propria ricchezza in oggetti di valore.

Inoltre, essendo un gioco multiplayer con numerosi oggetti rari (e
modalità per creare oggetti di livello superiore partendo da quelli
inferiori), i giocatori desideravano scambiarsi merce. C'erano druidi,
barbari, paladini, streghe e altre classi di personaggi. All'interno di
ciascuna classe, ogni personaggio poteva essere personalizzato in
maniera diversa, con molteplici direzioni di sviluppo delle abilità e
dell'equipaggiamento. Di conseguenza, alcuni oggetti rari utili a un
giocatore non lo erano per un altro, dando vita a un vivace mercato
degli scambi.

Perché il baratto funzioni è necessario soddisfare una doppia
coincidenza dei desideri. Se un barbaro e una strega si incontrano per
scambiarsi oggetti è probabile che la transazione fallisca. Forse lui
desidera una possente ascia, mentre lei cerca un bastone magico. Tra le
centinaia di oggetti presenti nel gioco, quali sono le probabilità che
ciascuno possieda esattamente ciò che l'altro vuole?

Per soddisfare questa esigenza, tra i giocatori emerse rapidamente una
forma di denaro diversa dal ``gold'' del gioco. Naturalmente, alcuni
oggetti per le loro caratteristiche erano intrinsecamente più adatti a
fungere da denaro rispetto ad altri. I giocatori avevano un numero
limitato di slot per gli oggetti e quelli di dimensioni maggiori
occupavano più spazio. Di conseguenza, un oggetto monetario doveva
essere prezioso in relazione allo spazio che occupava. Inoltre, doveva
possedere un'ampia desiderabilità universale: non poteva trattarsi di un
oggetto di nicchia utilizzabile solo dai barbari, bensì doveva essere
qualcosa di utile per la maggior parte delle classi.

La risposta nei primi due anni di gioco fu che un anello raro, chiamato
Stone of Jordan (abbreviato in ``SoJ''), divenne la valuta diffusa in
\emph{Diablo II}. Il SoJ aumentava il mana e tutte le abilità del
giocatore che lo indossava, rendendolo utile a tutti, e in particolare
alle varie classi di incantatori per cui il mana rappresentava una
risorsa fondamentale. Un SoJ poteva non essere il pezzo migliore di
equipaggiamento per un certo personaggio, ma ogni personaggio poteva
trarne un uso significativo, e per alcuni era davvero un oggetto d'elite
da indossare. Oltre al suo impiego funzionale, tuttavia, i SoJ
acquisirono un premio monetario poiché molti li collezionavano come
risparmi e li usavano come merce estremamente commerciabile. Occupavano
al minimo spazio in termini di capacità di carico del personaggio, il
che conferiva loro un'elevata densità di valore. Armi ed equipaggiamenti
rari venivano quotati in SoJ; ad esempio, una spada magica rara poteva
valere otto SoJ e un arco magico raro cinque SoJ.

Di conseguenza, una strega poteva raccogliere oggetti rari durante i
suoi viaggi e venderli ad altri giocatori in cambio di SoJ, che lei
avrebbe poi conservato. Se un giorno incontrasse un barbaro che
possedesse il bastone magico raro che desiderava, potrebbe scambiare i
SoJ per il bastone. Il barbaro, a sua volta, potrebbe successivamente
trovare qualcuno che detiene l'imponente ascia che ambisce e scambiare i
SoJ per essa. Questo sistema è molto più semplice rispetto al tentativo
di accordarsi direttamente per uno specifico bastone magico in cambio di
una specifica ascia imponente.

I SoJ emersero naturalmente come denaro perché offrivano le migliori
caratteristiche monetarie tra gli oggetti del gioco. Non erano stati
concepiti dagli sviluppatori per essere usati come valuta, e non fu come
se i giocatori si fossero riuniti un giorno per sceglierli
arbitrariamente; attraverso rapide analisi e iterazioni, emerse tra
milioni di giocatori il semplice fatto che i SoJ erano il miglior denaro
in-game. E una volta diventati la valuta del gioco, i SoJ possedevano
una profondità di liquidità che altri oggetti non avevano. Molti li
conservavano come risparmi e molti li accettavano, aumentando così la
loro commerciabilità rispetto ad altri articoli. La maggior parte dei
giocatori non avrebbe potuto citare \emph{The Wealth of Nations} o altre
opere economiche per giustificare il loro ragionamento; capivano
intuitivamente che possedere un denaro in-game era utile per risolvere
il problema del baratto in un mondo senza credito, e che oggetti rari,
piccoli e di ampia utilità erano ideali a tal fine.

L'unico difetto dei SoJ era che erano piuttosto preziosi e indivisibili.
Così, emersero anche oggetti chiamati ``perfect skulls'' che divennero
una forma di valuta minore. Erano piccoli e di ampia utilità, ma non
altrettanto rari. Cinque perfect skulls potevano essere scambiati per un
SoJ, e un numero variabile di SoJ poteva essere scambiato per altre armi
ed equipaggiamenti magici leggendari. In altre parole, i SoJ erano come
le banconote per gli acquisti importanti, mentre i perfect skulls erano
come le monete per acquisti minori o per il resto.

Alla fine, i giocatori scoprirono dei bug nel gioco che permettevano
loro di duplicare i SoJ, e così i SoJ iniziarono ad inondare il mercato,
deprezzandosi. Gli sviluppatori tentarono di identificare i bug e
cancellare gli oggetti duplicati, e i giocatori si ritrovavano ad aprire
il gioco scoprendo che alcuni dei loro SoJ erano scomparsi. Da quel
momento, i SoJ smisero di essere una buona forma di denaro, proprio come
molte valute merce furono rese obsolete dai progressi tecnologici. La
``tecnologia'' della duplicazione fallata trasformò i SoJ in una cattiva
moneta.

Quando è uscita l'espansione di \emph{Diablo II}, gli sviluppatori del
gioco hanno introdotto ulteriori oggetti, tra cui le rune. Le rune
potevano essere inserite in vari equipaggiamenti per renderli più
potenti e potevano essere combinate in modi specifici per creare
equipaggiamenti completamente nuovi. Erano piccole, preziose e molto
versatili, e si presentavano in diversi tipi con differenti livelli di
rarità, che in pratica potevano funzionare come banconote di dimensioni
diverse. Da quel momento, grazie alla loro elevata commerciabilità, sono
emerse naturalmente come denaro in-game.

Questo studio di caso (e altri simili) rappresenta sostanzialmente un
esempio accelerato di come le forme di denaro possano emergere in una
società in modo naturale, grazie alle loro caratteristiche che le
rendono il bene più commerciabile --- per poi perdere gradualmente
favore man mano che le condizioni cambiano.

\section{Footnotes}\label{footnotes-1}

\bookmarksetup{startatroot}

\chapter{\texorpdfstring{Chapter 2: \textbf{The Evolution of Commodities
as
Money}}{Chapter 2: The Evolution of Commodities as Money}}\label{chapter-2-the-evolution-of-commodities-as-money}

Come illustrato nel capitolo precedente, gli esseri umani che vivono in
piccoli gruppi di parentela o amicizia non hanno realmente bisogno di
denaro; sono infatti in grado di organizzare le risorse tra di loro in
maniera diretta, ricorrendo al massimo a registri orali informali. Essi
sono capaci di tenere conto di chi apporta un surplus costante al gruppo
e di chi, al contrario, opera in modo sistematicamente deficitario.
All'interno di queste comunità, le persone ``risolvono in maniera
naturale il problema del baratto'' attraverso sistemi informali di
credito sociale, prima ancora che sorge l'effettiva necessità di scambi
diretti.

Tuttavia, quando i gruppi cominciano a commerciare regolarmente con
comunità esterne o, in seguito all'avvio dell'agricoltura, raggiungono
popolazioni statiche di dimensioni superiori a quelle tribali, si rende
inevitabile l'identificazione e l'utilizzo di una forma di denaro. Tale
strumento fornisce un'unità contabile più liquida, divisibile,
trasportabile e di accettazione universale, agevolando lo stoccaggio e
lo scambio di valore tra individui sconosciuti. In aggiunta al
persistere dei sistemi di credito sociale, questi gruppi si avvalgono
anche del ``registro della natura'', consentendo loro di eludere il
problema della doppia coincidenza dei desideri, che altrimenti
ridurrebbe notevolmente l'efficacia degli scambi.

L'impiego di proto-monne da collezione, che richiedono una notevole
quantità di lavoro per essere realizzate, può apparire arbitrario agli
osservatori esterni a quella cultura. Perché investire così tanto tempo
nella creazione di perline di conchiglia, ad esempio? Non sarebbe uno
spreco di risorse in un ambiente aspro, a bassa tecnologia, dove ogni
risorsa è preziosa e dove oltre un terzo dei bambini non raggiunge
nemmeno l'età adulta? Il punto è che tale attività rappresenta un
utilizzo efficiente delle risorse nei periodi di abbondanza, ripagandosi
ampiamente, poiché la presenza di un mezzo di scambio standardizzato e
credibile, nonché di una riserva di valore affidabile, rende tutte le
transazioni economiche successive decisamente più efficienti.

Man mano che l'economia diventa più complessa, aumenta il numero di
possibili combinazioni di baratto tra differenti tipi di beni e servizi.
Ad esempio, in un'economia che produce cinque prodotti differenti, si
possono formare 10 coppie di scambio uniche. Se l'economia produce 20
prodotti diversi, il numero di coppie di scambio uniche sale a 190; in
un sistema con 100 prodotti la cifra arriva a 4.950. A questo punto,
ogni tipo di baratto, ad eccezione di quello relativo ai beni di prima
necessità, risulterebbe estremamente inefficiente.

Se una società richiede interazioni più complesse o basate sulla
mancanza di fiducia rispetto a quelle che un sistema di credito sociale
flessibile può garantire, allora è indispensabile adottare una unità di
conto standard -- in altre parole, una forma di denaro -- che funga da
polo di scambio per ogni altro bene o servizio.

In pratica, tra gli asset scambiati in una società, uno o due dei più
scarci, divisibili, durevoli, trasportabili e liquidi tendono a emergere
come forme predominanti di denaro. Prendiamo, ad esempio, una
coltivatrice di mele che ha bisogno di strumenti (per un fabbro), di
carne (da un allevatore di bestiame), di riparazioni (da un falegname) e
di medicinali per i propri figli (da un medico). Non può permettersi di
perdere tempo a cercare, di casa in casa, chi possiede ciò che le
occorre e, al contempo, desidera una grande quantità di mele proprio in
quel momento. Un sistema di baratto esteso tra vicini non si sviluppa
spontaneamente; ciò che le serve realmente è la possibilità di vendere
le sue mele, altamente stagionali e deperibili, in cambio di una riserva
di valore durevole e ampiamente accettata, che potrà poi utilizzare,
gradualmente e secondo necessità, per ottenere i beni e i servizi di cui
ha bisogno.

Nel 1776, Adam Smith affrontò il tema dell'emergere del denaro come
soluzione al problema del baratto nel suo \emph{Wealth of Nations}. Pur
essendo oggetto di critiche da parte dei teorici del credito, che
contestano quell'esempio e la sequenza degli eventi legati al baratto in
generale -- argomentazioni che vengono analizzate in dettaglio nel
Capitolo 4 di questo volume -- la discussione di Smith ha alimentato la
successiva focalizzazione sul denaro-merce da parte della scuola
austriaca di economia, fondata da Carl Menger nel XIX secolo e
ulteriormente sviluppata da Ludwig von Mises, Friedrich Hayek e altri
studiosi.

In questa prospettiva, il denaro ideale deve essere divisibile,
trasportabile, durevole, fungibile, verificabile e scarso. Di norma (pur
non in tutti i casi) esso possiede anche una certa utilità intrinseca. I
vari tipi di denaro, infatti, possono essere valutati in base a questi
parametri:\\
- Con ``divisibilità'' si intende la capacità del denaro di essere
suddiviso in unità di dimensioni diverse, adeguate a transazioni di
importo variabile.

\begin{itemize}
\tightlist
\item
  Portabilità: il denaro deve essere facilmente trasportabile anche su
  lunghe distanze, il che implica che esso racchiuda una notevole
  quantità di valore in un formato compatto e di poco peso.
\item
  Durabilità: per essere efficace, il denaro deve poter essere
  conservato nel tempo senza deteriorarsi; non deve marcire, arrugginire
  o rompersi facilmente, garantendo così una protezione del suo valore.
\item
  Fungibilità: le unità costitutive del denaro devono essere
  sostanzialmente intercambiabili, ovvero ogni unità deve essere
  equivalente a qualsiasi altra, senza differenze sostanziali che ne
  compromettano l'utilità.
\item
  Verificabilità: è fondamentale che chi vende beni o servizi possa
  accertarsi in modo semplice e rapido che il denaro ricevuto
  corrisponda effettivamente a quanto dichiarato, confermandone così
  l'autenticità.
\item
  Scarsità: il sistema monetario deve prevedere una crescita contenuta
  dell'offerta di denaro, in modo da evitare che un rapido aumento possa
  minare la sua funzione di riserva di valore.
\end{itemize}

L'utility significa che il denaro possiede un valore intrinseco: può
essere consumato oppure avere un valore estetico, a titolo
esemplificativo.

Sommando queste caratteristiche, il denaro risulta essere il ``bene più
commerciabile'' all'interno di una società, ovvero quello che risulta il
più vendibile, quello in grado di essere scambiato in maniera più
efficace. Esso rappresenta il bene più universale nel senso che le
persone lo desiderano, o riconoscono la possibilità di scambiarlo,
sapendo che potranno a loro volta convertirlo in qualcos'altro di cui
hanno bisogno, in modo semplice e affidabile. Nel suo articolo \emph{On
the Origin of Money}, Menger affermava che il denaro ideale trasporta
valore attraverso lo spazio e il tempo, ovvero che possa essere
trasportato su lunghe distanze in maniera efficiente oppure conservato
per essere speso in futuro.\footnote{Carl Menger, ``On the Origin of
  Money.''} Un altro aspetto fondamentale della commerciabilità è la
liquidità, ovvero la capacità di acquistare o vendere ingenti quantità
di denaro con relativa facilità, senza subire perdite eccessive dovute a
grandi differenziali di prezzo o a volumi di scambio insufficienti. In
molte accezioni, la liquidità rappresenta una misura di accettabilità:
quanto più qualcosa è ampiamente riconosciuto e detenuto, tanto più
liquido risulterà il mercato in cui viene scambiato.

La scarsità è spesso l'elemento determinante che fa prevalere una forma
di denaro basata su commodity rispetto a un'altra. Tuttavia, non si
tratta solamente di quanto l'asset sia raro; infatti, un'eccessiva
rarità può compromettere la liquidità, trasformando una commodity in una
forma di denaro poco commerciabile. Un concetto importante in questo
contesto è il rapporto stock-to-flow, che mette in relazione la quantità
attualmente disponibile (lo stock) con la nuova offerta che può essere
prodotta in un anno (il flow).

Ad esempio, i minatori d'oro aggiungono ogni anno circa l'1,5\% di nuova
produzione alla stima della fornitura d'oro attualmente disponibile
sopra il suolo,\footnote{Nuno Palma, ``The Real Effects of Monetary
  Expansions: Evidence from a Large-Scale Historical Experiment'';
  Saifedean Ammous, \emph{The Bitcoin Standard: The Decentralized
  Alternative to Central Banking}, 28--29.} e, a differenza di molte
altre commodity, la maggior parte dell'oro non viene consumata: viene
continuamente fusa e conservata in varie forme e luoghi.

L'oro, infatti, non marcisce, non arrugginisce né si corrode con la
stessa facilità di molti altri materiali. Essendo chimicamente inerte,
forma pochissimi composti. Può essere ri-fuso innumerevoli volte e
perfino dissolto in particolari tipi di acido per poi essere nuovamente
recuperato mediante filtrazione. Anche se venisse disperso o fatto
volare via, i frammenti non si disintegrerebbero come accade con altri
materiali, rimanendo così recuperabili. A eccezione di minime quantità
disperse nei circuiti elettronici o sommerse insieme ai relitti di navi,
la maggior parte dell'oro estratto dall'uomo è ancora sotto il nostro
controllo (e persino quelle quantità perdute sono, in linea teorica,
recuperabili a un giusto prezzo). È praticamente
indistruttibile.\footnote{Ammous, \emph{The Bitcoin Standard: The
  Decentralized Alternative to Central Banking}, 27--28.}

La combinazione tra l'estrazione continua dell'oro e la quasi totale
conservazione del quantitativo estratto ha prodotto un rapporto
stock-to-flow medio di circa 100/1,5 = 67, il più elevato tra tutte le
commodity. Secondo le stime del World Gold Council, il patrimonio
globale di oro equivale a 67 anni di produzione annua media. Negli
ultimi cento anni il tasso di crescita dell'offerta si è mantenuto tra
l'1\% e il 2\%, una fascia relativamente bassa e stabile.\footnote{World
  Gold Council, ``Above-Ground Stock.''} Persino negli anni '70, quando
il prezzo dell'oro in dollari aumentò di un ordine di grandezza, ciò non
influenzò sostanzialmente il tasso di crescita annuale in percentuale
rispetto alle scorte esistenti. Prima di quel periodo, le uniche
occasioni in cui la fornitura di oro raffinato aumentò a ritmi
accelerati si verificarono quando le società industriali scoprirono
nuovi continenti con depositi accessibili o idearono tecniche innovative
per estrarre profittevolmente depositi precedentemente considerati non
convenzionali.

Se un asset possiede un premio monetario oltre al suo valore d'uso
intrinseco, i partecipanti al mercato sono fortemente incentivati a
produrne di maggiori quantità. Solo quei beni estremamente resistenti a
incrementi dell'offerta rispetto alle scorte già esistenti possono
affrontare tale sfida, diventando e rimanendo così la moneta accettata a
livello globale.

D'altra parte, se un bene è così raro da essere posseduto da pochissimi,
esso può avere un valore elevato in virtù della sua utilità, ma risulta
poco adatto come moneta: non è liquido, non è ampiamente detenuto né
accettato, e ciò si traduce in elevati costi di transazione per
compravendite. Alcuni elementi come il rodio, ad esempio, sono più rari
dell'oro ma presentano bassi rapporti stock-to-flow, poiché l'industria
li consuma rapidamente quanto vengono estratti. Un lingotto o una moneta
di rodio può essere acquistato come oggetto da collezione o riserva di
valore, ma non risulta pratico come moneta accettata su vasta scala; lo
stesso discorso vale per i meteoriti o altre entità straordinariamente
rare. Ad oggi, negli Stati Uniti sono stati identificati
1.878\footnote{Randy Korotev, ``Meteorite Numbers in the United States,
  Canada, and Mexico,'' Washington University in St.~Louis.} meteoriti,
mentre in altre giurisdizioni ne sono state scoperte decine di migliaia:
questo li rende oggetti da collezione di grande valore, ma non adatti
all'uso monetario, in quanto non dispongono della necessaria liquidità o
divisibilità.

Un elevato rapporto stock-to-flow, costante nel tempo, risulta pertanto
l'indicatore più efficace per misurare la scarsità di un bene da
considerare come moneta -- unitamente agli altri attributi già esaminati
-- rispetto alla mera rarità assoluta. Una commodity caratterizzata da
un alto rapporto stock-to-flow è difficile da produrre, ma una quantità
rilevante è già stata generata e si distribuisce ampiamente, proprio
perché non viene consumata con rapidità o, in alcuni casi, non viene
consumata affatto. Questo insieme di caratteristiche relativamente raro
consente a un bene di essere utilizzato come denaro, anziché limitarsi a
essere un oggetto da collezione.

Nel corso della storia diverse materie hanno svolto la funzione
monetaria -- pietre, perline, piume, conchiglie, sale, pellicce,
tessuti, zucchero, noci di cocco, bestiame, rame, argento, oro e altri
ancora -- ciascuno caratterizzato da differenti punteggi nei vari
attributi del denaro, con specifici punti di forza e fragilità. È stato
frequente, infatti, il contesto in cui almeno due forme di moneta
coesistevano, poiché nessun singolo sistema monetario riusciva a
soddisfare perfettamente tutte le esigenze funzionali.

Il sale, ad esempio, è divisibile, duraturo, verificabile, fungibile e
dotato di una rilevante utilità; tuttavia, il suo valore per unità di
peso è modesto e non è particolarmente raro, pertanto non ottiene
punteggi elevati in termini di portabilità e scarsità. La parola
``salary'' deriva dal termine latino \emph{salarium}, che
originariamente indicava un compenso in denaro misurato in sale.

L'oro, invece, primeggia su quasi tutti i parametri ed è la commodity
che possiede il più alto rapporto stock-to-flow in assoluto. La sua
unica debolezza, se paragonata alle altre materie prime, risiede nella
scarsa divisibilità: anche una piccola moneta d'oro supera in valore la
maggior parte degli acquisti quotidiani, rappresentando essenzialmente
un notevole ammontare di lavoro. È il re delle commodities. L'oro,
inteso non solo come forma ideale di ornamento, si configura
sostanzialmente come una versione tecnologicamente superiora dei
tradizionali grani ornamentali realizzati con conchiglie. La sua
applicazione più comune consiste nel fungere da ricchezza da indossare o
esporre in molteplici culture, offrendo così un mezzo facilmente
trasportabile per comunicare il proprio status sociale.

Per gran parte della storia umana, l'argento ha prevalso per l'uso
quotidiano. Pur ottenendo il secondo miglior punteggio rispetto all'oro
per la maggior parte delle proprietà monetarie, incluso il secondo
rapporto stock-to-flow più elevato, l'argento supera l'oro nella sua
capacità di essere suddiviso, dato che le piccole monete d'argento sono
ideali per le transazioni di tutti i giorni. È, dunque, la regina delle
commodities. E, come nel gioco degli scacchi in cui il re è il pezzo più
importante, la regina rappresenta il pezzo più versatile ed efficace.

Di conseguenza, l'oro veniva spesso detenuto dai benestanti come riserva
di valore a lungo termine -- oltre che come simbolo esibito di status --
e utilizzato come mezzo di scambio in transazioni di grande entità,
mentre l'argento svolgeva un ruolo più tattico, impiegato dalla maggior
parte della classe lavoratrice sia come riserva di valore che come mezzo
di scambio. Questa dualità determinava la diffusione di sistemi monetari
bimetallici in molte regioni del mondo, in cui la limitata divisibilità
dell'oro veniva compensata dall'utilizzo complementare dell'argento,
nonostante le difficoltà insite in tale approccio multi-valuta.

Perché dunque oro e argento hanno prevalso su tutte le altre commodity
adibite a denaro, giungendo all'era moderna come forme di moneta
effettivamente utilizzabili? La risposta risiede nel fatto che questi
due metalli sono stati in grado di mantenere rapporti stock-to-flow
sufficientemente elevati, nonostante il progresso tecnologico umano, e
di assorbire un sostanziale premio monetario. Hanno saputo preservare la
loro rarità nel tempo, pur essendo ampiamente accettati e detenuti,
duraturi, portatili, divisibili e facilmente riconfigurabili.

Il potere d'acquisto di una moneta merceologica può essere
concettualmente suddiviso in due elementi: il valore d'uso e un premio
monetario che si aggiunge al valore d'uso stesso. Il valore d'uso
corrisponde all'effettivo impiego della merce per uno scopo economico
(sia per il consumo che per la produzione), mentre il premio monetario
rappresenta un valore aggiuntivo attribuito alla merce in virtù del
fatto che essa viene detenuta a scopo di risparmio, in mancanza di
opzioni migliori. Ciò che distingue una merce ordinaria dalla moneta
derivante dalla merce è proprio il fatto che chi detiene unità di tale
moneta non le utilizza esclusivamente per il loro fine originario, bensì
le trattiene come una forma di risparmio, proprio per la loro elevata
commerciabilità, che ne permette una facile rivendita nel tempo. Al
contrario, le merci non destinate a fini monetari, come il greggio,
vengono solitamente valutate solo in funzione del loro valore d'uso:
esiste una domanda pratica e una relativa offerta produttiva, e
pochissimi le persone le detengono per lunghi periodi. Le monete
merceologiche dominanti in una determinata area -- come l'oro, per
esempio -- sono invece caratterizzate da una domanda in eccesso,
generata dalla larga detenzione da parte di coloro che non sono
consumatori finali; questo fenomeno ne incrementa notevolmente il valore
complessivo sul mercato. Una moneta in oro, pertanto, non viene detenuta
perché il possessore intenda usarla direttamente, ma perché è
consapevole che l'oro ha molteplici impieghi e, detenendolo, conserva
valore in un bene che possiede elevata liquidità e accettabilità a
livello globale.

Questo premio monetario -- ovvero il valore in eccesso rispetto alla
semplice utilità -- funge da imponente e permanente annuncio che stimola
le persone a produrre ulteriori quantità della stessa merce. Solo le
merci più rare, ossia quelle con il più elevato rapporto stock-to-flow,
possono sostenere tale ``pubblicità'' nel lungo periodo. Premi monetari
analoghi possono essere attribuiti ad altri asset, come le proprietà
fronte mare o le opere d'arte di pregio, poiché essi vengono spesso
detenuti più come forma di risparmio che per il piacere intrinseco che
il loro uso diretto potrebbe fornire. Il problema, tuttavia, è che tali
asset non monetari mancano, per loro natura, della portabilità, della
liquidità, della fungibilità e della divisibilità proprie dell'oro e di
altre forme monetarie consolidate.

Molti considerano il denaro una sorta di ``illusione condivisa''. In
questo senso, una società potrebbe scegliere qualunque oggetto come
denaro, purché la maggior parte dei suoi membri ne condivida la
convinzione. Per esempio, anche delle graffette potrebbero costituire
denaro se tutti ne concordassero il valore. Sebbene tale ragionamento
possa apparire inizialmente verosimile, esso non è sostenibile: se
l'offerta di quel ``denaro'' potesse essere aumentata rapidamente, i
risparmi di tutti verrebbero diluiti in modo altrettanto veloce.
Inoltre, la presenza di un premio monetario incentiva notevolmente la
produzione aggiuntiva del bene in questione. Di conseguenza, se il
denaro non viene scelto in modo oculato all'interno di una società,
basta che pochi individui si rendano conto della mancanza di scarsità
intrinseca di tale denaro per produrne in eccesso e, così facendo,
estrarre valore dagli altri. In alternativa, soggetti appartenenti a una
società esterna potrebbero sfruttare l'illusione condivisa di quella
comunità. Dunque, le uniche tipologie di denaro in grado di mantenere
un'utilità duratura all'interno di una società sono quelle
caratterizzate da un autentico elevato grado di scarsità.

Le monete a base di conchiglie, ad esempio, sono durate per migliaia di
anni in svariate regioni, ma col tempo sono divenute impraticabili in
vista della rivoluzione industriale. Pellicce, bestiame, sale, tabacco e
altre forme di moneta hanno svolto il loro ruolo utile in determinati
periodi, ma la crescente capacità tecnica della civiltà ha infine reso
operativi anche questi sistemi inadeguati come strumenti monetari. Essi
funzionavano finché la tecnologia non ha imposto dei limiti tali da
impossibilitarne il proseguimento. Le sezioni che seguono illustrano
vari esempi a sostegno di questo concetto.

\textbf{Shells}

Come già descritto, il denaro in conchiglie è stato utilizzato per
lunghi periodi in varie regioni delle Americhe, dell'Africa e dell'Asia.
In certe aree l'impiego era prevalentemente cerimoniale, mentre in altre
la conchiglia veniva usata in senso più letterale come mezzo di scambio.

La varietà di perline in conchiglia nota come wampum, comune lungo la
costa orientale del Nord America, aveva una valenza principalmente
cerimoniale per le tribù originarie che l'avevano istituita. Tuttavia,
già nei primi anni del 1600 i coloni del New England la integrarono nel
loro sistema monetario, stabilendo un tasso di cambio fisso in base al
quale un certo numero di conchiglie equivaliva al valore delle loro
monete.\footnote{Claire Priest, ``Currency Policy and Legal Development
  in Colonial New England,'' \emph{The Yale Law Journal}, 1324--26; Glyn
  Davies, \emph{A History of Money: From Ancient Times to the Present
  Day}, 40--41.} Il wampum viola, essendo più raro, veniva dunque
attribuito il doppio del valore rispetto a quello bianco.

Con il tempo, le norme sul tasso di cambio fisso vennero abolite, e il
wampum passò a essere valutato sul mercato. Nel 1812 John Campbell fondò
il Campbell Wampum Mill nel New Jersey e, mediante l'adozione di moderne
tecniche di perforazione, fu possibile produrre in massa le perline di
wampum a un ritmo molto più rapido rispetto alle produzioni
precedenti.\footnote{Kristin Beuscher, ``From Pasack to the Plains.''
  \emph{Northern Valley Press}, 21 maggio 2019.} John Jacob Astor, della
American Fur Company, acquistò il wampum prodotto industrialmente e lo
impiegò per commerciare pellicce con le popolazioni indigene canadesi.

Nel corso del tempo, il denaro in conchiglie, in tutte le sue forme e
nelle varie aree geografiche, divenne inefficace di fronte ai progressi
dell'industria. L'introduzione di utensili in metallo e altre tecnologie
rese il denaro in conchiglie un mezzo inadeguato per conservare il
valore. Tuttavia, in alcune zone, la tradizione della realizzazione
artigianale del wampum è stata mantenuta viva dai discendenti delle
popolazioni indigene, che lo hanno restituito al suo significato
cerimoniale e complesso come strumento per preservare la tradizione
culturale.

\textbf{Tobacco}

Nei primi anni del 1600, Virginia, Maryland e Carolina del Nord
iniziarono a utilizzare il tabacco come denaro, includendolo persino tra
i mezzi di pagamento legali imposti dal governo. Col passare del tempo,
però, sorsero delle problematiche in questo sistema che ricordavano, in
qualche modo, l'industrializzazione delle conchiglie di wampum.

Dato che al tabacco veniva attribuito un premio monetario superiore al
suo normale valore d'uso, si creava un forte incentivo a coltivare
sempre di più questa pianta, nel tentativo di ``catturare'' -- e di
fatto erodere -- tale premio.\footnote{Milton Friedman, ``Understanding
  Inflation,'' 3:01--5:28.} A differenza dell'oro, il tabacco non
possiede una scarsità naturale sufficiente a impedire che questo premio
venga sfruttato e progressivamente dissipato. Il risultato di questo
fenomeno fu un notevole incremento dell'offerta di tabacco, che portò a
una marcata inflazione dei prezzi di beni e servizi espressi in termini
di questa merce. Per fronteggiare l'eccesso di offerta, i governi
coloniali introdussero restrizioni sulla coltivazione, limitando ad
esempio la produzione a specifici gruppi, al fine di creare una scarsità
artificiale in un bene che, in natura, non lo era.\footnote{Ron
  Mitchener, ``Money in the American Colonies,'' \emph{EH.net}.} In
altre parole, solamente i gruppi privilegiati dallo Stato potevano
fungere da ``stampatori di denaro a base di tabacco''. Si trattava
chiaramente di una soluzione imperfetta e difficile da mantenere a tempo
indeterminato.

Un ulteriore problema era dato dalla fungibilità non perfetta del
tabacco: esistono infatti diverse qualità, alcune superiori ad altre. Se
tutto il tabacco venisse valutato secondo un tasso di cambio standard,
ci sarebbe un forte incentivo a consumare localmente il tabacco di
qualità inferiore, mentre quello pregiato verrebbe destinato
all'esportazione, dove poteva essere valutato in maniera più adeguata.
Di conseguenza, si instaurò la pratica del magazzinaggio, che prevedeva
lo stoccaggio e la classificazione del tabacco, unitamente all'emissione
di titoli cartacei standardizzati basati su di esso. In questo modo si
creò un sistema di ``standard tabaccario'': per chi possedeva il titolo
cartaceo, si generava un rischio di controparte, oltre al rischio già
presente di svalutazione del tabacco sottostante.\footnote{Sharon Ann
  Murphy, \emph{Other People's Money: How Banking Worked in the Early
  American Republic}.}

In definitiva, il problema essenziale del tabacco stava proprio nel
fatto che non riusciva a resistere all'invito insito nel premio
monetario a produrne sempre di più -- un onere molto pesante per
qualsiasi merce. Se l'oro è riuscito a sopportare questa sfida per
millenni, il tabacco non ci è riuscito. Nei primi tempi, nelle colonie
meridionali, il tabacco svolse un ruolo utile poiché queste realtà erano
piccole, scarsamente sviluppate e carenti di specie metalliche;
tuttavia, una volta che esse crebbero e si svilupparono, il sistema
monetario basato sul tabacco perse di senso. Vari tentativi governativi
di prolungare la vita utile di questa forma di denaro ritardarono
l'inevitabile dismissione del sistema, che, col tempo, si rivelò
nettamente inadeguato rispetto a forme di moneta più solide, fino a
essere completamente abbandonato.\footnote{Farley Grubb, ``Colonial
  Virginia's Paper Money, 1755--1774''; Barry Eichengreen,
  \emph{Exorbitant Privilege: The Rise and Fall of the Dollar and the
  Future of the International Monetary System}, 9--11.}

\textbf{Cocoa}

In molte regioni dell'America centrale e meridionale le civiltà antiche
impiegavano il cacao come moneta. Tale prassi era già consolidata al
tempo dell'arrivo degli europei, e numerosi murales testimoniano che
essa risaliva a secoli precedenti.\footnote{Stefania Moramarco e Loreto
  Nemi, ``Nutritional and Health Effects of Chocolate,'' 134--35.} I
semi di cacao, per la loro dimensione contenuta, l'ottima fungibilità e
la capacità di conservarsi per un periodo ragionevole, venivano scelti
non solo per il loro valore economico, ma anche perché il loro gusto era
largamente apprezzato. Queste proprietà rendevano il cacao una forma di
denaro efficace.\footnote{Ingrid Fromm, ``From Small Chocolatiers to
  Multinationals to Sustainable Sourcing: A Historic Review of the Swiss
  Chocolate Industry,'' 73.}

Come molte società preindustriali, anche queste civiltà adottavano forme
flessibili di credito sociale e praticavano il baratto, orientandosi
verso l'impiego di uno o due beni facilmente commerciabili per agevolare
gli scambi. In situazioni in cui il credito non bastava, l'uso di uno o
due beni scarsi e liquidi tendeva a evolversi in una forma monetaria.
Gli Aztechi, ad esempio, utilizzavano anche il rame come moneta, con
unità sagomate nella forma di una zappa decorata o di un'ascia opaca
ornamentale. Migliaia di semi di cacao potevano essere scambiati con una
singola unità di rame, permettendo così di effettuare transazioni di
maggior rilievo o di custodire una ricchezza liquida in un'unità ridotta
e facilmente trasportabile per periodi più lunghi.\footnote{Dudley
  Easby, ``Early Metallurgy in the New World,'' 77.}

All'arrivo degli europei nelle Americhe, sia il cacao che il rame furono
impiegati come denaro; tuttavia, similmente ad altri contesti nel mondo,
questa pratica fu gradualmente soppiantata da forme monetarie ancora più
scarse.

\textbf{Rai Stones}

Gli abitanti di un'isola del Sud Pacifico, Yap, utilizzavano enormi
pietre come forma di moneta. Le cosiddette ``rai stones'' (o ``fei
stones'') venivano scolpite in dischi circolari in pietra, dotati di un
foro centrale, e si presentavano in varie dimensioni, da pochi pollici
di diametro fino ad oltre dieci piedi. Molte di esse misuravano almeno
qualche piede e, di conseguenza, pesavano centinaia di libbre, mentre le
più imponenti superavano i dieci piedi di diametro e raggiungevano un
peso dell'ordine di migliaia di libbre.\footnote{Milton Friedman,
  \emph{Monetary Mischief}, 3--7.}

È interessante notare che ho visto questo esempio adottato sia da un
economista austriaco che da un economista MMT. Ciò risulta
particolarmente rilevante poiché le due correnti di pensiero offrono
concezioni radicalmente diverse della natura del denaro. Gli economisti
austriaci tendono a sottolineare il denaro in quanto merce, mentre i
Chartalisti (e, nella loro forma attuale, gli economisti MMT)
enfatizzano il denaro quale registro pubblico.\footnote{William Luther e
  Alexander Salter, ``Synthesizing State and Spontaneous Order Theories
  of Money.''} Queste visioni possono essere riconciliate se si
comprende che i denari-commodity operano come registri, nella cui
amministrazione la natura stessa funge da intermediario. Tale
riconciliazione viene esaminata con maggior dettaglio nel Capitolo 4.

Ciò che rendeva uniche le pietre rai era il fatto che esse erano
scolpite da un particolare tipo di calcare assente sull'isola di Yap.
Gli abitanti di quest'isola viaggiavano per 250 miglia fino a un'isola
vicina, detta Palau, per estrarre il calcare e poi trasportarlo a Yap.

Una squadra composta da numerose persone veniva inviata su quell'isola
remota per estrarre la roccia in enormi lastre e poi riportarla su
barche di legno. Immaginate di dover trasportare una pietra del peso di
diverse migliaia di libbre attraverso 250 miglia di oceano aperto,
utilizzando un'imbarcazione in legno: nel corso degli anni, questo
procedimento si è rivelato tanto arduo da causare numerosi decessi,
richiedendo un incredibile dispendio di tempo, sforzi e coraggio.

Una volta che le pietre rai venivano realizzate a Yap, quelle di grandi
dimensioni non venivano più spostate. Essendo Yap una piccola isola e
avendo la proprietà delle pietre registrata esclusivamente tramite la
tradizione orale, il cambio di proprietà avveniva mediante il semplice
annuncio alla comunità che un bene era passato a mano di un altro
individuo, anziché tramite il fisico spostamento della pietra
stessa.\footnote{Friedman, \emph{Monetary Mischief}, 4.}

In questo senso, le pietre rai costituivano un vero e proprio sistema
contabile, non dissimile dal nostro attuale sistema monetario, poiché il
registro tracciava in modo chiaro la titolarità dei beni. In
particolare, questo registro era distribuito oralmente, una modalità
che, chiaramente, è praticabile solo in una comunità di dimensioni
ridotte.

Al momento in cui gli europei documentarono il fenomeno, a Yap
esistevano migliaia di pietre rai, testimonianza di generazioni intere
caratterizzate dall'estrazione, dal trasporto e dalla lavorazione di
questi preziosi manufatti. Le pietre rai possedevano infatti un elevato
rapporto stock-to-flow, qualità fondamentale che ne consentiva l'impiego
come moneta.

Verso la fine del XIX secolo, un irlandese di nome David O'Keefe giunse
sull'isola e colse l'essenza del fenomeno. Grazie alla tecnologia più
avanzata a sua disposizione, era in grado di estrarre pietra da Palau e
trasportarla a Yap per realizzare nuove pietre rai, facendosi così
promuovere al rango di uomo più ricco dell'isola, potendo ingaggiare i
residenti per lavori e scambiare in cambio una varietà di beni. Un
articolo pubblicato su \emph{Smithsonian Magazine} da Mike Dash,
intitolato \emph{David O'Keefe: The King of Hard Currency}, riassumeva
la vicenda nel seguente modo:

\begin{quote}
Man mano che l'Irlandese conosceva meglio Yap, si rese conto che
esisteva una sola merce, e una sola, che suscitava in modo particolare
il desiderio delle popolazioni locali: il ``denaro in pietra'', per cui
l'isola era celebre e che veniva impiegato in quasi tutte le transazioni
di alto valore. Queste monete venivano estratte dall'aragonite, una
particolare varietà di calcare che brilla alla luce e che era
considerata preziosa proprio perché non reperibile sull'isola. Il genio
di O'Keefe consisteva nel riconoscere che, importando le pietre per i
suoi nuovi amicissimi, poteva scambiarle con manodopera nelle
piantagioni di cocco di Yap. I locali, i cosiddetti yapesi, non si
interessavano particolarmente a sudare per ottenere quei modesti oggetti
di scambio, comuni in altre parti del Pacifico (come giustamente
osservava un visitatore, ``quando cibo, bevande e abiti sono prontamente
disponibili, non ha senso barattare né indebitarsi''), ma erano disposti
a lavorare instancabilmente in cambio del denaro in pietra.\footnote{Mike
  Dash, ``David O'Keefe: The King of Hard Currency,'' \emph{Smithsonian
  Magazine}, 28 luglio 2011.}
\end{quote}

In sostanza, l'introduzione di tecnologie più sofisticate finì per
rompere l'equilibrio del rapporto stock-to-flow delle pietre rai,
incrementando drasticamente la loro immissione sul mercato. Stranieri
come O'Keefe, armati di strumenti avanzati, potevano recarsi sull'isola
con un numero illimitato di pietre, raggiungere la vetta della ricchezza
locale e, conseguentemente, aumentare l'offerta, depauperando col tempo
il valore economico delle pietre stesse.

Tuttavia, anche i residenti dimostrarono notevole ingegno, riuscendo a
contenere progressivamente tale meccanismo. Iniziarono ad attribuire un
valore superiore alle pietre più antiche -- quelle che erano
inequivocabilmente state estratte a mano decenni o addirittura secoli fa
-- poiché costituivano un sottoinsieme scarso e intangibile. È un po'
come il caso dell'arte: per quanto ne venga prodotta di nuova, Vincent
van Gogh non realizza ulteriori opere, per cui il valore dei suoi
dipinti tende ad aumentare invece che essere diluito da una fornitura
fresca di opere altrui. Comunque, il messaggio era inequivocabile: le
pietre rai non rappresentavano più un sistema monetario sostenibile.

Le cose presero poi una piega decisamente più oscura. Come ulteriormente
descritto nell'articolo di \emph{Smithsonian} di Dash:

\begin{quote}
Con la morte di O'Keefe e con i tedeschi ormai saldamente insediati, le
sorti dei Yapesi cominciarono a deteriorarsi dopo il 1901. I nuovi
governanti arruolarono gli isolani per scavare un canale attraverso
l'arcipelago e, quando i Yapesi dimostrarono riluttanza, iniziarono a
confiscare il loro denaro in pietra, deturpando le monete con croci nere
dipinte e informando i sudditi che esse potevano essere riscattate solo
mediante lavoro. Ciò che aggravò ulteriormente la situazione fu
l'introduzione, da parte dei tedeschi, di una legge che vietava ai
Yapesi di allontanarsi per più di 200 miglia dalla propria isola. Tale
provvedimento interruppe immediatamente l'estrazione dei fei, sebbene la
valuta continuasse a circolare anche dopo che le isole furono occupate
dai giapponesi e, successivamente, dagli Stati Uniti nel
1945.\footnote{Dash, ``O'Keefe.''}
\end{quote}

Durante la Seconda Guerra Mondiale molti di queste pietre furono
prelevate e utilizzate dai giapponesi invasori come ancore improvvisate
o materiali da costruzione, riducendo sensibilmente il numero di pietre
presenti sull'isola.

Le pietre Rai rappresentavano una forma peculiare di valuta, notevole
proprio perché non avevano alcuna utilità pratica se non quella
estetica. Esse costituivano un mezzo per esibire e registrare la
ricchezza, e ben poco altro. In sostanza, possiamo considerarle una
delle prime versioni di un registro pubblico formale, dal momento che
gran parte delle pietre rimaneva immobile e solo le tradizioni orali (o,
in un secondo momento, le marcature fisiche apportate dai tedeschi)
indicavano a chi appartenessero.

\textbf{Feathers}

Le piume venivano spesso utilizzate come oggetti dal valore simile al
denaro da popolazioni tribali in tutto il mondo. Molte culture
raccoglievano piume appartenenti a uccelli maestosi, come aquile e
pappagalli, note per le loro dimensioni eccezionali o per la bellezza
particolare.

In alcuni casi, queste piume assumevano un valore cerimoniale, come ad
esempio quelle d'aquila impiegate come copricapi dai capi tribali; altre
volte venivano raccolte in maniera informale, semplicemente per il loro
fascino, e poi occasionalmente utilizzate negli scambi.\footnote{David
  Jones, \emph{Native North American Armor, Shields, and Fortification},
  41.} Un aspetto negativo delle piume risiede nella loro fragilità: con
il tempo diventa facile consumarle e rovinarle, specialmente se
costantemente trasportate.

Nelle Isole Salomone, degli artigiani tribali concepivano una sorta di
valuta in piume, trasformandole in rotoli simili a cinture. Ogni rotolo
era composto da piume rosse raccolte da centinaia di minuscoli melidieri
scarlatti, mescolate a sève e altre sostanze, creando così una forma di
denaro in piume più resistente e distintiva. Tuttavia, per la sua
natura, questa tipologia di valuta presenta limitazioni intrinseche in
termini di fungibilità e liquidità, confinandosi a una giurisdizione
culturalmente e geograficamente ristretta.

\textbf{African Beads}

Le perline da scambio furono impiegate come denaro in alcune zone
dell'Africa occidentale per molti secoli, risalendo almeno al XIV secolo
o addirittura prima. Venivano utilizzati vari materiali rari, come
corallo, ambra e vetro. Le perline in vetro, prodotte a Venezia, si
diffusero gradualmente fino all'Africa subsahariana attraverso rotte
commerciali. Una delle prime testimonianze documentate in merito ci
proviene da Ibn Battuta, il celebre viaggiatore marocchino del XIV
secolo, le cui esplorazioni lo portarono a percorrere vasti territori in
Africa e in Asia.

Emil Sandstedt, nel suo libro \emph{Money Dethroned: A Historical
Journey}, cita Ibn Battuta in riferimento alle osservazioni del
viaggiatore sulle pratiche monetarie in Africa occidentale:

\begin{quote}
Un viaggiatore in questo paese non porta con sé provviste, né alimenti
semplici né condimenti, e non possiede oro o argento. Accontentandosi,
porta solo pezzi di sale, ornamenti di vetro -- che la gente chiama
perline -- e alcune spezie aromatiche.\footnote{Emil Sandstedt,
  \emph{Money Dethroned: A Historical Journey}, 43}
\end{quote}

Si trattava di società prevalentemente pastorali, caratterizzate da uno
spostamento continuo, per le quali il potersi adornare di fili di
perline rappresentava un vantaggio funzionale. Queste perline, scambiate
e custodite come denaro, garantivano un elevato rapporto stock-to-flow,
nonostante la loro produzione risultasse particolarmente complessa con
il livello tecnologico dell'epoca.

Col passare del tempo, un incremento nei viaggi europei verso l'Africa
occidentale portò alla scoperta e al rapido sfruttamento di questo
particolare sistema di scambio basato sulle perline. Gli europei, dotati
di una consolidata tecnologia di lavorazione del vetro, riuscivano a
produrre perline di elevata bellezza con minimi sforzi, tanto da poterle
utilizzare in grandi quantità per ottenere merci e altri beni --
schiavizzando, purtroppo, anche esseri umani nell'ambito di questo
commercio.\footnote{Emil Sandstedt, ``Racconti sul denaro morbido --- Il
  sentiero delle perle'', \emph{Medium}, 26 maggio 2019.}

Questa asimmetria tecnologica permise agli europei di svalutare le
perline aumentando la loro offerta in tutta l'Africa occidentale,
estraendo così enormi ricchezze dalle società locali. Di conseguenza,
gli africani continuarono a scambiare bene locali, che spaziavano da
preziose materie prime a vite umane di inestimabile valore, con perline
di vetro prodotte in eccesso rispetto alle effettive necessità del
mercato locale.\footnote{Laure Dussubieux et al., ``Il commercio europeo
  in Malawi: le evidenze delle perle di vetro.''} Tale scambio comportò,
in sostanza, la cessione dei loro autentici valori in cambio di un
valore fittizio, evidenziando come la scelta di una tipologia di moneta
inadeguata possa determinare conseguenze drammatiche.

Non fu affatto semplice come si potrebbe supporre per gli europei
compiere questo trucco, poiché le preferenze degli africani per
determinate tipologie di perline mutavano nel tempo e vari gruppi
tribali avevano gusti differenti. Questo fenomeno ricordava l'esperienza
con le pietre rai: con l'arrivo più rapido di nuove pietre, favorito dai
progressi industriali, gli abitanti di Yap cominciarono a attribuire
maggiore valore a quelle vecchie rispetto alle nuove. Nel contesto
dell'Africa occidentale, i gusti mutavano in base all'estetica e alla
scarsità. Tuttavia, tale pratica comportava un punteggio inferiore per
la fungibilità di quella forma di moneta, riducendone così
l'affidabilità come mezzo di scambio. Analogamente alle pietre rai, le
perline commerciali non riuscirono a mantenere il loro elevato rapporto
stock-to-flow di fronte al progresso tecnologico e, pertanto, furono
progressivamente sostituite come mezzo di pagamento.

\textbf{Japanese Invasion Money}

Sebbene non si tratti propriamente di una moneta-commodità, l'Impero
giapponese impiegò una valuta debole per acquisire i beni e i servizi
scarsi nelle regioni sotto il suo controllo. Durante la Seconda Guerra
Mondiale, quando il Giappone imperiale invase diverse zone dell'Asia,
veniva confiscata la valuta forte dei locali e sostituita da una carta
moneta speciale, denominata ``invasion money''.\footnote{Dazmin Daud,
  ``Uno studio su due varianti della moneta da \$100 del Malaya durante
  l'invasione giapponese (campione n.~M8A)'', 43.} Le popolazioni
conquistate erano costrette a risparmiare e a utilizzare una valuta
priva di qualsiasi garanzia o scarsità nell'offerta, che col tempo
perdeva completamente il suo valore. In questo modo, il Giappone
riusciva a depredare i risparmi dei sudditi pur mantenendo un'unità di
conto temporanea nelle regioni interessate, così da consentire il
proseguimento delle transazioni economiche.

In misura meno estrema -- come verrà descritto in seguito in questo
libro -- ciò avviene purtroppo in molti paesi in via di sviluppo: le
persone continuano a risparmiare nella valuta fiat locale, la quale, a
ogni generazione o giù di lì, viene drasticamente svalutata, portando i
loro risparmi a finire nelle mani dei governanti e della classe agiata.

\textbf{Grain}

Nell'antica Babilonia lo shekel d'argento costituiva l'unità monetaria
principale, sebbene il grano venisse usato altrettanto frequentemente
come forma di pagamento. Il grano, alimento fondamentale per la regione,
veniva impiegato non solo per soddisfare il fabbisogno alimentare, ma
anche per corrispondere i salari giornalieri e fungeva da unità di conto
nelle varie transazioni.\footnote{Goetzmann, \emph{Money Changes
  Everything}, pp.~59--69.}

Il Codice di Hammurabi, che risale a quasi quattro mila anni fa,
imponeva l'uso del grano come moneta legale:

\begin{quote}
\begin{enumerate}
\def\labelenumi{\arabic{enumi}.}
\setcounter{enumi}{107}
\item
  Se un venditore di vino non riceve in grano il prezzo di una bevanda,
  ma ottiene denaro tramite la ``grande pietra'' oppure riduce la misura
  della bevanda a favore del grano, verrà chiamato a rispondere e
  gettato in acqua.
\item
  Se un venditore di vino concede 60 KA di bevanda a credito, al momento
  del raccolto dovrà ottenere in compenso 50 KA di grano.
\item
  Se un uomo non possiede un debito in grano o denaro nei confronti di
  un altro e lo sequestra per il debito, per ogni sequestro dovrà
  corrispondere un terzo di mana d'argento.
\item
  Se un uomo detiene un debito in grano o denaro nei confronti di un
  altro e procede al sequestro, e il soggetto sequestrato muore nella
  casa di chi lo ha sequestrato, in tal caso non sussiste alcuna
  pena.\footnote{Hammurabi, \emph{The Code of Hammurabi, King of
    Babylon}, pp.~37--39.}
\end{enumerate}
\end{quote}

La difficoltà nell'utilizzare un prodotto agricolo come denaro risiede
nel fatto che l'offerta monetaria varia in modo considerevole nel corso
dell'anno. Infatti, la stagione del raccolto genera un notevole afflusso
di ``denaro grano'', mentre il resto dell'anno tale offerta si riduce
man mano che il grano viene trasformato in pane e birra. Per far fronte
alle esigenze di pagamento durante le stagioni di scarsità, gli
agricoltori facevano spesso affidamento sul ricorso al credito, usando
il fruttuoso raccolto per estinguere i debiti contratti nei periodi meno
produttivi.

In molte società come questa, un raccolto fallito poteva significare la
rovina finanziaria del contadino. Il contadino e/o i membri della sua
famiglia rischiavano di cadere nella condizione di schiavitù per debiti.
Tuttavia, diversi sovrani, a intervalli periodici, concedevano
l'amnistia sui debiti o introducevano dei limiti per i creditori,
scusando il debitore per particolari eventi o riducendo la quantità di
lavoro in servitù necessaria per saldare i debiti.

\begin{quote}
48: Se un uomo ha un debito e una tempesta inonda il suo campo, portando
via il raccolto, o se la scarsità d'acqua impedisce la crescita di
grano, in quell'anno egli non dovrà consegnare alcuna parte del suo
grano al creditore, dovrà modificare il contenuto del suo tavolo di
contratto e non pagherà interessi per quell'anno.
\end{quote}

\begin{quote}
117: Se un uomo è indebitato e vende sua moglie, suo figlio o sua
figlia, o li lega al servizio per tre anni, essi dovranno lavorare nella
casa del compratore o del padrone; al quarto anno dovranno essere
liberati.\footnote{Hammurabi, \emph{Code of Hammurabi}, 27, 41.}
\end{quote}

Babilonia rappresenta uno dei primi esempi noti in cui furono
formalizzati sistemi di pesi e misure, il denaro merce, il credito
scritto, i contratti di custodia e la moneta a corso legale. I re
stabilivano le regole fondamentali del commercio e correggevano
eventuali squilibri strutturali che emergessero durante il loro regno,
mentre i templi fungevano da centri amministrativi all'interno dei quali
si svolgeva il commercio formalizzato.

\textbf{Video Game Money}

Uno degli sviluppi più affascinanti dei giochi online multiplayer di
massa è che essi danno origine a numerosi esperimenti economici non
intenzionali. In sostanza, questi giochi ricreano vari ambienti
economici, seppur con regole nuove, fornendo così casi studio sui
fenomeni emergenti nelle economie.

Uno degli esempi più noti di questo fenomeno fu l'emergente formazione
monetaria osservata nel videogioco online \emph{Diablo II}, un
action-fantasy multiplayer di grande successo, uscito nel 2000 e capace
di vendere milioni di copie (io stesso l'ho acquistato da adolescente).
Numerosi studiosi hanno analizzato l'economia interna del gioco nel
corso degli anni, ma fu proprio Gigi, autore di \emph{21
Lessons},\footnote{Gigi, \emph{21 Lessons: What I've Learned from
  Falling Down the Bitcoin Rabbit Hole.}} a richiamarne la mia
attenzione con un articolo pubblicato nel 2022.\footnote{Gigi, ``Bitcoin
  Is Digital Scarcity,'' DerGigi.com, 2 ottobre 2022. Vedi anche Solomon
  Stein, ``The Origins of Money in \emph{Diablo II}.''}

In \emph{Diablo II} la valuta interna, in linea con il contesto
fantastico, era denominata ``oro''. Tuttavia, il sistema di
programmazione dell'oro prevedeva dinamiche che, in maniera
involontaria, ne compromise l'efficacia come strumento monetario ideale.
Da un lato, l'oro risultava abbondante nelle fasi avanzate della
partita, pur essendo possibile trasportarne soltanto una quantità
limitata; dall'altro, ogni volta che il personaggio moriva, una parte
dell'oro detenuto andava persa, mentre tutti gli altri oggetti venivano
restituiti. Di conseguenza, a fronte di tali restrizioni, i giocatori
preferivano accumulare la loro ricchezza in beni di maggior valore.

Inoltre, essendo \emph{Diablo II} un gioco multiplayer caratterizzato
dalla presenza di numerosi oggetti rari -- e da opportunità di combinare
oggetti minori per crearne di più pregiati -- i partecipanti si
trovarono naturalmente a scambiare beni tra loro. Il gioco metteva a
disposizione differenti classi di personaggi, quali druidi, barbari,
paladini, streghe e altri, e ciascun personaggio poteva essere
personalizzato in maniera univoca seguendo molteplici direzioni di
sviluppo sia in termini di abilità che di equipaggiamento. Di
conseguenza, un oggetto raro che risultava estremamente utile a un
giocatore poteva essere del tutto inadeguato per un altro, dando origine
a una vibrante economia basata sullo scambio diretto.

Affinché il baratto possa funzionare efficacemente è necessario che si
verifichi il cosiddetto ``doppio incontro di bisogni''. Ad esempio, se
un barbaro e una strega tentano di scambiarsi oggetti, molto
probabilmente l'operazione fallirà: il barbaro potrebbe desiderare
un'ascia imponente, mentre la strega ambisce a un bastone magico. Tra le
centinaia di oggetti presenti nel gioco, le probabilità che ciascuno
possieda esattamente ciò di cui l'altro ha bisogno sono ridotte.

Per superare questo problema, tra i giocatori si sviluppò rapidamente
una forma di denaro alternativa all'oro di gioco. Naturalmente, alcune
tipologie di oggetti risultavano intrinsecamente più idonee a fungere da
``moneta'' rispetto ad altre, a causa delle loro caratteristiche. I
giocatori disponevano infatti di un numero limitato di slot per
l'inventario, e gli oggetti di dimensioni maggiori occupavano più
spazio; di conseguenza, un oggetto ad uso monetario doveva garantire un
valore elevato in rapporto allo spazio che richiedeva. Inoltre, tale
oggetto doveva risultare universalmente desiderabile, ossia non poteva
essere un bene di nicchia adatto esclusivamente a determinate classi
(come i barbari), ma doveva poter essere utilizzato dalla maggior parte
dei personaggi.

Nei primi due anni di vita del gioco, la risposta comune era che un raro
anello, il cosiddetto Stone of Jordan (``SoJ''), divenne la valuta de
facto di \emph{Diablo II}. Il SoJ aumentava il mana e potenziava tutte
le abilità del personaggio che lo indossava, rendendolo estremamente
utile a tutti, soprattutto alle classi di incantatori per cui il mana
rappresentava una risorsa fondamentale. Pur non essendo necessariamente
il miglior equipaggiamento per ogni personaggio, ogni classe poteva
trarne un vantaggio significativo, e per alcuni rappresentava
addirittura un oggetto di prim'ordine. Oltre alla sua funzione pratica,
il SoJ acquisì un valore monetario elevato, poiché molti giocatori lo
accumulavano come riserva e lo utilizzavano come bene facilmente
scambiabile; occupava ben poco spazio nella capacità di carico del
personaggio, offrendo così un'elevata densità di valore. Di conseguenza,
oggetti come armi e armature rare venivano quotati in SoJ: una spada
magica rara, ad esempio, poteva valere otto SoJ, mentre un arco magico
raro ne poteva valere cinque.

In questo contesto, una strega poteva procurarsi, durante i suoi viaggi,
alcuni oggetti rari da vendere ad altri giocatori in cambio di SoJ, che
poi avrebbe trattenuto. Se un giorno incontrava un barbaro in possesso
del raro bastone magico che desiderava, poteva negoziare i suoi SoJ per
ottenerlo; il barbaro, a sua volta, avrebbe potuto poi scambiare i SoJ
per procurarsi l'immensa ascia ambita. Tale meccanismo risultava
decisamente più semplice rispetto all'organizzare uno scambio diretto,
in cui un preciso bastone magico sarebbe stato scambiato per una
specifica ascia imponente.

I SoJ emersero spontaneamente come mezzo di scambio perché possedevano
le caratteristiche ideali rispetto a tutti gli altri oggetti presenti
nel gioco. Non erano stati concepiti dagli sviluppatori con l'intento di
fungere da valuta, e non fu una decisione arbitraria presa in qualche
riunione tra giocatori; attraverso un rapido iter di analisi, divenne
evidente, fra milioni di utenti, che i SoJ erano il miglior strumento
per facilitare gli scambi in-game. Una volta consolidatosi questo ruolo,
i SoJ acquisirono una profondità di liquidità ineguagliabile: essi
venivano accumulati come risparmio da molti e accettati da quasi tutti,
conferendo loro una commerciabilità superiore rispetto agli altri
oggetti. La maggior parte dei giocatori non poteva affermare di
conoscere testi economici quali \emph{The Wealth of Nations} per
giustificare la loro scelta; l'intuizione collettiva era che una valuta
interna al gioco fosse essenziale per superare i limiti del baratto in
un mondo privo di credito, e che oggetti rari, ridotti nelle dimensioni
e di utilità trasversale fossero ideali a tale scopo.

L'unico limite dei SoJ consisteva nel fatto che erano molto pregiati e,
al contempo, indivisibili. Per ovviare a ciò, sorsero altri oggetti,
denominati ``perfect skulls'', che si affermarono come forme di moneta
di secondaria importanza. Pur essendo piccoli e di ampia applicabilità,
non godevano della stessa rarità dei SoJ: cinque perfect skulls potevano
essere scambiati con un SoJ, e, a loro volta, un numero variabile di SoJ
era richiesto per ottenere diversi oggetti leggendari e magici. In
termini analogici, i SoJ funzionavano come banconote per acquisti di
grande entità, mentre i perfect skulls operavano come monete per
transazioni minori o per fornire il resto.

Col tempo, i giocatori scoprirono dei bug che permettevano di duplicare
i SoJ, provocando un'inarrestabile inondazione dell'oggetto nel mercato
e, di conseguenza, una sua devalorizzazione. Gli sviluppatori tentarono
di individuare tali anomalie e cancellare gli oggetti duplicati, facendo
sì che, ad ogni accesso al gioco, molti giocatori si trovassero a dover
constatare la scomparsa di parte dei loro SoJ. Da quel momento, i SoJ
smisero di essere una forma valida di denaro, similmente a come molte
valute basate su commodity vennero rese obsolete dal progresso
tecnologico. La ``tecnologia'' insita nei bug di duplicazione aveva
trasformato i SoJ in una cattiva forma di moneta.

Quando fu lanciata l'espansione di \emph{Diablo II}, gli sviluppatori
introdussero nuovi oggetti, fra i quali spiccavano le rune. Queste
potevano essere infuse in diversi equipaggiamenti per aumentarne la
potenza e, se combinate in maniera specifica, davano origine a oggetti
completamente nuovi. Le rune, per via delle loro dimensioni contenute,
del valore intrinseco e della versatilità d'uso, si presentavano in
diverse tipologie e livelli di rarità che, in pratica, richiamavano
l'idea di banconote di taglie differenti. Grazie all'elevata
commerciabilità, esse si imposero naturalmente come forma di moneta
interna al gioco.

Questo caso studio (così come altri simili) costituisce, in sostanza, un
esempio accelerato di come le forme monetarie possano emergere
spontaneamente in una società, fondandosi sulle proprietà che le rendono
il bene più facilmente commerciabile, per poi cadere in disuso man mano
che le condizioni evolvono.

\section{Footnotes}\label{footnotes-2}

\bookmarksetup{startatroot}

\chapter{\texorpdfstring{Capitolo 3: \textbf{Come l'oro ha vinto la
guerra delle
commodity}}{Capitolo 3: Come l'oro ha vinto la guerra delle commodity}}\label{capitolo-3-come-loro-ha-vinto-la-guerra-delle-commodity}

In ogni società che utilizza una forma di moneta merce, i capitoli
precedenti hanno dimostrato come sia la natura a controllare il
registro. La natura stabilisce i limiti alla difficoltà con cui viene
prodotta la moneta, determinando così la sua resistenza alla
svalutazione. Tra i partecipanti dotati di un livello di sviluppo
tecnologico relativamente equiparabile, nessuno può imbrogliare il
registro: tutti devono impiegare lavori simili per creare nuove unità di
moneta.

Tuttavia, quando una società industrializzata entra in contatto con una
società pre-industriale, quella industrializzata controlla di fatto il
registro della società pre-industriale. Essa possiede infatti la
capacità tecnologica di diluire la moneta merce usata dalla società
pre-industriale, mentre il contrario non è vero. La comprensione di tale
capacità impiega tempo a diffondersi nella società pre-industriale, il
che, purtroppo, consente a quella industrializzata di sfruttarla per
ottenere risorse preziose.

Il successo o il fallimento delle diverse monete merce possiede quindi
un aspetto di filtro naturale, con le monete meno scarse che
gradualmente scompaiono e quelle più rare che continuano ad esistere.
Man mano che, nel corso del tempo, i vari gruppi umani del nostro mondo
si sono incontrati, il numero di monete merce in uso si è ridotto a
poche.

Risparmiare la propria ricchezza in una forma di moneta non ideale ed
essere incapaci di misurare o comprendere correttamente la crescita
dell'offerta della moneta utilizzata può avere conseguenze disastrose,
sia a livello individuale che collettivo. Purtroppo questo problema si
estende anche alle banconote fisiche e ai sistemi di registro
elettronico, venendo addirittura amplificato da tali tecnologie; ne
verrà trattato più avanti nel libro. Dopo migliaia di anni, due
commodity hanno superato tutte le altre in termini di capacità di
mantenere le loro proprietà monetarie su più aree geografiche: oro e
argento. Solo esse sono riuscite a conservare un rapporto stock-to-flow
sufficientemente elevato da poter fungere da moneta e mantenere un
premio monetario, nonostante le civiltà abbiano costantemente migliorato
le proprie capacità tecnologiche in tutto il mondo nel corso dei secoli.

L'umanità ha scoperto come produrre o procurarsi praticamente tutte le
perle, le conchiglie, le pietre, le piume, il sale, le pellicce, il
bestiame, i cereali e i metalli industriali di cui abbiamo bisogno
grazie ai nostri strumenti migliorati, riducendo così i loro rapporti
stock-to-flow e facendo sì che cadessero in disuso come denaro.
Tuttavia, nonostante tutti i nostri progressi tecnologici, non riusciamo
ancora a ridurre in modo significativo i rapporti stock-to-flow di oro e
argento --- tranne in rari casi in cui il mondo industrializzato ha
scoperto nuovi continenti inesplorati o ha inventato nuove tecniche,
come il processo di estrazione con cianuro d'oro.\footnote{Alan
  Lougheed, ``The Discovery, Development, and Diffusion of New
  Technology.''} L'oro ha mantenuto un rapporto stock-to-flow compreso
tra 25x e 100x lungo la storia moderna, mediamente intorno a 50x o più,
scendendo brevemente a non meno di 16x durante la corsa all'oro della
metà del XIX secolo\footnote{Hugh Rockoff, ``Some Evidence on the Real
  Price of Gold,'' 621.}. In altre parole, a parte la scoperta di nuovi
continenti o altri eventi isolati, storicamente non siamo riusciti a far
crescere l'offerta di oro di oltre circa il 2\% annuo in maniera
sostenuta, anche quando il prezzo è aumentato di oltre dieci volte in un
decennio, come avvenne negli anni '70.\footnote{Greg Cipolaro e Ross
  Stevens, ``The Power of Bitcoin's Network Effect,'' 6.} L'argento,
invece, ha generalmente un rapporto stock-to-flow di 10x o superiore,
che rimane piuttosto elevato.

La maggior parte delle altre materie prime presenta un rapporto
stock-to-flow inferiore a 1 o 2. Anche gli elementi più rari, come il
platino e il rodio, hanno rapporti stock-to-flow piuttosto bassi a causa
della rapidità con cui vengono consumati dall'industria.

L'umanità è diventata più abile nell'estrazione dell'oro grazie alle
nuove tecnologie, ma esso è intrinsecamente raro e abbiamo già sfruttato
i giacimenti superficiali più facili. Rimangono solo i depositi profondi
e difficili da raggiungere, che fungono da continuo adeguamento della
difficoltà in contrapposizione ai nostri progressi tecnologici. Un
giorno potremmo finalmente rompere questo ciclo con l'estrazione da
asteroidi tramite droni, l'estrazione in acque profonde o qualche
tecnologia di livello fantascientifico simile, ma fino ad allora l'oro
conserverà il suo elevato rapporto stock-to-flow. Tali ambienti sono
così inospitali che il costo per ricavarvi l'oro sarebbe probabilmente
estremamente elevato.

In sostanza, ogni volta che un qualsiasi denaro basato su commodity si è
scontrato con oro e argento nella competizione per il denaro, sono
sempre stati proprio oro e argento a prevalere. Altre commodity potevano
rimanere denaro per periodi di tempo in regioni specifiche, ma oro e
argento hanno dimostrato la capacità di competere a livello globale come
mezzi di pagamento, vincendo ogni volta. Questo perché, ogni volta che
civiltà e forme di denaro si incontravano, i possessori di oro e argento
disponevano di una capacità tecnologica sufficiente a svalutare le altre
forme di denaro, mentre i detentori di conchiglie, perle, bestiame,
sale, tessuti e metalli inferiori non riuscivano a farlo.

\textbf{Monetazione dei metalli preziosi}

Le autorità hanno ulteriormente valorizzato oro e argento come denaro
creando unità standard, solitamente sotto forma di monete. La coniazione
di monete in metalli preziosi si è diffusa in molte regioni, e la Lidia
(l'attuale Turchia) fu una delle prime civiltà a produrle, già nel VI
secolo a.C.

Il vantaggio delle monete emesse da un'autorità ampiamente riconosciuta
(che all'epoca era generalmente un regno o un impero) è che esse
permettono di standardizzare le unità in termini di dimensioni, peso e
purezza, facilitando così gli scambi commerciali. Se si dovessero
utilizzare quantità arbitrarie di oro e argento, sarebbe necessario
pesarle affinché il metallo potesse essere impiegato nelle singole
transazioni, mentre l'utilizzo di unità standardizzate elimina questo
passaggio. Il ritratto di un imperatore coniato sulla moneta, magari
accompagnato da scanalature lungo i bordi per evitare l'alterazione del
metallo, conferisce un grado di verificabilità sulla qualità e sul
contenuto della moneta. Ancora oggi, se si acquistano monete d'oro
sovrane moderne come gli American Eagle, si paga un premio per oncia
rispetto a una maggiore massa d'oro, e ciò perché si sa che si sta
ottenendo oro reale, facilmente rivendibile in futuro.

Inoltre, un regno assegnava tipicamente a tali monete lo status di corso
legale, permettendo loro di essere scambiate a un valore superiore
rispetto al semplice contenuto in metallo puro e oltre il valore di
monete straniere simili all'interno della stessa giurisdizione, grazie
alla certezza della loro accettazione e liquidità. In altre parole,
possiamo considerare le monete in metalli preziosi ad corso legale come
aventi tre livelli di valore. Il primo livello è rappresentato dal
contenuto del metallo prezioso stesso. Il secondo livello consiste nel
premio di verifica e comodità che la coniazione offre, rispetto ai
semplici frammenti di metallo grezzo, e si applica in misura variabile
sia alle monete domestiche sia a quelle straniere. Il terzo livello è un
premio di liquidità che solo le monete domestiche possiedono, grazie
alla loro ampia (e spesso obbligatoria) accettazione da parte dei
commercianti della giurisdizione come corso legale. I salari e i prezzi
denominati in unità coniate tendevano ad essere piuttosto ``rigidi'',
nel senso che impiegavano del tempo a variare in tutta la società, anche
se la fornitura di metalli preziosi o di monete fluttuava nel breve
periodo.

Questi livelli di valore sono stati ripetutamente abusati nel corso dei
secoli. I governi, trovando i bilanci in difficoltà a causa di guerre,
avidità o cattiva gestione, alla fine cedevano alla tentazione della
svalutazione. Ad esempio, un re può raccogliere 1.000 monete d'oro puro
in tasse, fonderle e coniare nuove monete composte per il 90\% da oro
(con il restante 10\% costituito da un metallo riempitivo economico),
per poi immettere nell'economia 1.111 monete d'oro, pur mantenendo la
stessa quantità totale di oro. Apparirebbero sostanzialmente invariate
agli occhi della maggioranza; salari e prezzi cambierebbero lentamente,
e il re potrebbe persino obbligare all'accettazione al precedente
standard di conto, ad esempio stabilendo l'unità con cui paga i suoi
soldati. Anni dopo, se il re continua ad avere deficit, potrebbe
rifondere le entrate fiscali e coniare monete al 80\% di oro,
immettendone 1.250 nell'economia. E se ciò non bastasse, lui (o i suoi
successori) potrebbe ridurle ulteriormente al 70\% di oro, e così via.

Inizialmente, queste monete leggermente deprezzate al 90\% sarebbero
state spesso accettate al valore nominale precedente, specialmente se
imposte tramite decreto legale. Ciò avviene perché l'unità stessa è in
parte astratta dal metallo sottostante.\footnote{Thomas Marmefelt,
  \emph{The History of Money and Monetary Arrangements}, x e cap. 3.}
Con il prolungarsi della circolazione e l'aumentare del numero, a causa
della diluizione, diventava evidente, e molto più difficile, far
rispettare il valore originario. I prezzi sarebbero aumentati e i salari
e i risparmi delle persone sarebbero diminuiti, poiché si sarebbero
pagate tasse con riserve di denaro più puro, mentre il reddito corrente
veniva erogato in questo denaro appena deprezzato. I salari sarebbero
poi saliti nel tempo, poiché le persone avrebbero avuto bisogno di più
monete per coprire le proprie spese. I commercianti avrebbero cercato di
trattenere le monete d'oro più pure e di vendere quelle diluite,
indebolendo così la fungibilità dell'intera base monetaria, poiché le
monete non sarebbero più state standardizzate. I commercianti stranieri,
al di fuori della giurisdizione del governo che le conia, sarebbero
rapidamente stati inclini a richiedere prezzi più elevati per queste
monete d'oro deprezzate, valutandole più strettamente per il loro
contenuto metallico. Le monete più pure sarebbero infine uscite di
circolazione, a causa di una combinazione di fattori: il re le avrebbe
richiamate come tasse, le persone le avrebbero accumulate o i
commercianti stranieri le avrebbero prelevate dalla giurisdizione.

Il denaro svalutato in questo modo impiegava tipicamente anni e decenni.
Ad esempio, il denario romano d'argento fu introdotto come una piccola
moneta d'argento con una purezza superiore al 95\% nel 211 a.C.
Successivamente, fu ridotto di taglia, pur rimanendo ad una purezza
superiore al 95\%. Brevemente, sotto il regno di Tiberio, la sua purezza
venne aumentata, ma intorno al 64 d.C. fu ulteriormente ridotto e
conteneva meno del 94\% d'argento. Rimase alla medesima grandezza per
secoli, declinandone gradualmente la purezza di alcuni punti percentuali
qua e là. Alla fine, la purezza cominciò a crollare rapidamente, fino a
scendere a circa il 5\% d'argento nell'anno 274. Vari salari, compresi
quelli statali, non venivano adeguati immediatamente per tener conto del
denario lievemente svalutato ad ogni emissione, permettendo così
all'imperatore di ottenere più valore in argento con ogni
devalorizzazione.\footnote{Colin Elliott, ``The Acceptance and Value of
  Roman Silver Coinage in the Second and Third Centuries AD.''} Col
tempo, con un crescente afflusso di monete nel mercato, i prezzi
finirono per salire e i soldati reclamarono salari più elevati.

Alla fine, in tutto il mondo, i progressi nel settore bancario --- che
verranno descritti più avanti in questo libro --- ridussero la necessità
di monete e migliorarono la limitata divisibilità dell'oro. Le persone
potettero depositare il loro oro nelle banche e ricevere dei crediti
cartacei rappresentativi di pretese riscattabili su tale oro. Le banche,
sapendo che non tutti avrebbero riscattato immediatamente il loro oro,
iniziarono a emettere pretese superiori alla quantità di oro detenuta,
dando inizio alla pratica della riserva frazionaria.\footnote{Stephen
  Quinn, ``Goldsmith Banking: Mutual Acceptance and Interbank Clearing
  in Restoration London,'' \emph{Explorations in Economic History} 34:
  411--414.} Successivamente il sistema bancario si consolidò nel tempo
in una struttura di banche centrali in vari paesi, con dei certificati
nazionali in carta che rappresentavano una pretesa su una determinata
quantità d'oro.\footnote{Charles Goodhart, \emph{Evolution of Central
  Banks}.}

Durante quel periodo, dal tardo XIX secolo ai primi del XX, l'oro
prevalse sull'argento come mezzo di pagamento. L'argento perse parte del
suo premio monetario e, di conseguenza, si svalutò rispetto all'oro, al
contrario di quanto avveniva nei millenni precedenti.

Nel suo libro \emph{Globalizing Capital: A History of the International
Monetary System}, Barry Eichengreen spiega come il gold standard abbia
avuto la meglio sullo standard argentario: fu in gran parte un caso
fortuito. Nel 1717, il Maestro della Zecca inglese (null'altro che Sir
Isaac Newton in persona) fissò ufficialmente un rapporto argento-oro
troppo basso, secondo Eichengreen.\footnote{Barry Eichengreen,
  \emph{Globalizing Capital: A History of the International Monetary
  System,} 5--10.} Di conseguenza, la maggior parte delle monete
d'argento uscì dalla circolazione e l'oro divenne la moneta indiscussa
del regno.\footnote{Fay, ``Newton,'' 111.} Poi, con il Regno Unito che
si affermò come l'impero più forte del XIX secolo, gli effetti di rete
del gold standard si diffusero in tutto il mondo, con la maggior parte
dei paesi che ancorarono le proprie valute all'oro. I paesi che
persistevano nello standard argentario per troppo tempo, come India e
Cina, videro indebolirsi la loro valuta poiché la domanda d'argento
diminuì in Nord America e in Europa.

D'altro canto, Saifedean Ammous, nel suo libro \emph{The Bitcoin
Standard}, sottolinea come la tecnologia bancaria abbia migliorato la
divisibilità dell'oro.\footnote{Ammous, \emph{The Bitcoin Standard},
  28--29.} Come già accennato, l'oro ottiene punteggi pari o superiori a
quelli dell'argento nella maggior parte delle caratteristiche del
denaro, ad eccezione della divisibilità. L'argento, infatti, si presta
meglio alla suddivisione, rendendolo la moneta d'uso quotidiano per
migliaia di anni, mentre l'oro era prevalentemente riservato a re,
mercanti e ordini religiosi. La tecnologia dei sistemi bancari e delle
banconote in diverse denominazioni, garantite dall'oro, migliorò la sua
reale divisibilità. E poi, oltre a scambiare carta, le persone poterono
infine ``inviare'' denaro attraverso linee di telecomunicazione ad altre
parti del mondo, utilizzando banche e i loro registri come intermediari
custodi. Questo costituiva il gold standard --- il supporto delle valute
cartacee e dei sistemi di comunicazione finanziaria con l'oro. A quel
punto vi era meno motivo di usare l'argento, considerando che l'oro era
il metallo più scarso e resistente, con un rapporto stock-to-flow più
elevato, e, grazie a questa seconda astrazione, praticamente altrettanto
divisibile quanto l'argento.

Penso che vi sia un fondo di verità in entrambe le spiegazioni, sebbene
ritenga quella di Ammous più completa, poiché si fonda su un assioma più
profondo riguardante la natura stessa del denaro. Le banconote hanno
reso l'oro più divisibile e, col tempo, la moneta solida ha prevalso,
anche se gli effetti di rete derivanti da decisioni politiche possono
influenzare la tempistica e la diffusione geografica di tali
cambiamenti.

Anche dopo che oro e argento sono stati demonetizzati dal sistema
bancario globale nella seconda metà del XX secolo, l'oro ha mantenuto il
suo nuovo premio monetario rispetto all'argento come forma ideale di
risparmio. Per migliaia di anni l'oro veniva scambiato a un multiplo
compreso tra 10 e 16 volte il valore dell'argento in diverse aree
geografiche.\footnote{J.B. Maverick, ``A Historical Guide to the
  Gold-Silver Ratio,'' \emph{Investopedia}, 27 luglio 2022.} Negli
ultimi cento anni, tuttavia, il rapporto prezzo oro/argento ha mediato
attorno a 50. L'argento ha perso in modo strutturale gran parte del suo
storico premio monetario rispetto all'oro poco dopo l'introduzione e la
diffusione dei registri bancari collegati dai sistemi di
telecomunicazione intercontinentali negli anni '60 del 1800, e non credo
che ciò sia una coincidenza.

Figure 3‑A\footnote{Silvan Frank, ``Gold to Silver Ratio.''}

Quando l'oro e l'argento uscirono dall'uso come mezzi di scambio, la
superiore divisibilità dell'argento divenne quasi irrilevante. Le
proprietà dell'oro, in quanto metallo più scarso e duraturo con un
rapporto stock-to-flow più elevato, divennero le caratteristiche più
rilevanti per il risparmio, e per questo è probabile che l'oro abbia
assorbito parte del premio monetario dell'argento. Le banche centrali di
tutto il mondo conservano ancora oro nelle loro casseforti, e molte di
esse acquistano ogni anno ulteriore oro come parte delle loro riserve in
valuta estera. Le posizioni gestite su questa scala comprendono
centinaia --- o addirittura migliaia --- di tonnellate, e perciò il
valore molto più denso dell'oro rispetto all'argento risulta
particolarmente utile per questi risparmiatori a lungo termine.
Pertanto, sebbene la moneta emessa dal governo non sia più garantita da
una quantità specifica di oro, quest'ultimo rimane un elemento indiretto
e importante del sistema monetario globale in qualità di attivo di
riserva delle banche centrali. Non esiste una commodity naturale
migliore con cui sostituirlo.

Se la migliore forma di denaro presenta una limitazione intrinseca --
come la limitata divisibilità dell'oro -- allora ciò consente la
coesistenza di diverse tipologie di denaro. Gli standard bimetallico e
perfino trimetallico sono stati necessari per lunghi periodi per ovviare
alle limitazioni di divisibilità dell'oro. D'altra parte, se la migliore
forma di denaro non presenta limitazioni, essa tende a diventare
dominante, escludendo le altre. Ormai né oro né argento sono ampiamente
usati come mezzi di scambio, ma entrambi continuano a essere impiegati
come riserve di valore a lungo termine; l'oro, grazie alla sua maggiore
durabilità, al più elevato rapporto stock-to-flow e a un maggiore valore
per unità di massa e volume, rappresenta la scelta più attraente per la
maggior parte dei grandi detentori.

Ancora oggi, sia l'oro che l'argento godono di un significativo
riconoscimento monetario in tutto il mondo. Pur non essendo largamente
accettati come mezzo di pagamento, se portassi con me in quasi ogni
paese del mondo alcune monete d'oro o dei gioielli in oro, riuscirei a
trovare un rivenditore o un commerciante disposto a comprarli in valuta
locale a un prezzo di mercato ragionevole, di solito senza grandi
difficoltà. L'oro fisico, che si presenta sotto forma di monete,
lingotti o gioielli, rimane uno dei mezzi migliori a disposizione di chi
desidera conservare valore in un asset denso, liquido e fungibile, senza
alcun rischio di controparte.

\section{Footnotes}\label{footnotes-3}

\bookmarksetup{startatroot}

\chapter{\texorpdfstring{Chapter 3: \textbf{How Gold Won the Commodity
War}}{Chapter 3: How Gold Won the Commodity War}}\label{chapter-3-how-gold-won-the-commodity-war}

In ogni società che adotta una forma di denaro merce, i capitoli
precedenti hanno evidenziato come sia la natura a regolamentare il
registro contabile. La natura stabilisce i limiti alla difficoltà con
cui la moneta può essere prodotta e, di conseguenza, determina la sua
resistenza alla svalutazione. Tra soggetti con livelli tecnologici
comparabili, nessuno può imbrogliare il sistema: ciascuno deve impiegare
un simile quantitativo di lavoro per generare nuove unità di denaro.

Tuttavia, quando una società industrializzata si confronta con una
realtà pre-industriale, la prima acquisisce un controllo effettivo sul
registro della seconda. Essa possiede la capacità tecnologica di diluire
il denaro merce adottato dalla società pre-industriale, mentre il
contrario non è possibile. Tale capacità impiega del tempo per essere
compresa e diffondersi all'interno della società meno sviluppata,
offrendo così alla realtà industrializzata l'occasione di sfruttare le
risorse preziose della controparte.

Il successo o l'insuccesso delle diverse forme di denaro merce è dunque
soggetto a un processo di selezione naturale, in cui le monete meno rare
tendono a scomparire progressivamente, lasciando spazio a quelle
caratterizzate da una scarsità elevata. Con l'incontro e lo scambio tra
i differenti gruppi umani nel corso della storia, il numero delle valute
basate su beni si è ridotto fino a rimanere poche sole.

Accumulare ricchezza in una forma monetaria non ideale, senza poter
misurare né comprendere appieno l'espansione dell'offerta della moneta
adottata, può avere conseguenze gravissime, sia a livello individuale
che collettivo. Questo problema si estende purtroppo anche alle
banconote fisiche e ai sistemi elettronici di registro, dove le
problematiche vengono amplificate dalle stesse tecnologie -- un aspetto
che verrà esaminato più nel dettaglio nelle pagine successive di questo
libro. Dopo migliaia di anni, solo due materie prime hanno saputo
distinguersi per il mantenimento delle loro funzioni monetarie su scala
globale: l'oro e l'argento. Queste, infatti, sono riuscite a conservare
un rapporto stock-to-flow sufficientemente elevato da operare come
denaro e a mantenere un premio monetario, nonostante il continuo
progresso delle capacità tecnologiche delle civiltà nel corso dei
secoli.\footnote{C.R. Fay, ``Newton and the Gold Standard,'' 117--18.}

L'umanità ha imparato a produrre o ad acquisire praticamente tutte le
perle, le conchiglie, le pietre, le piume, il sale, le pellicce, il
bestiame, i cereali e i metalli industriali necessari grazie
all'evoluzione dei nostri strumenti, riducendo così i rispettivi
rapporti stock-to-flow, tanto da abbandonarli come forme di moneta.
Tuttavia, nonostante i notevoli progressi tecnologici, non siamo
riusciti a ridurre in maniera significativa il rapporto stock-to-flow di
oro e argento -- salvo rarissime eccezioni in cui il mondo
industrializzato ha scoperto interi continenti inesplorati o ha ideato
nuove tecniche, come il processo di estrazione con cianuro
d'oro.\footnote{Alan Lougheed, ``The Discovery, Development, and
  Diffusion of New Technology.''} Nel corso della storia moderna l'oro
ha mantenuto un rapporto stock-to-flow compreso tra 25x e 100x,
mediamente attorno a 50x o superiore, scendendo brevemente fino a 16x
durante la corsa all'oro della metà del XIX secolo.\footnote{Hugh
  Rockoff, ``Some Evidence on the Real Price of Gold,'' 621.} In altre
parole, a meno di eccezioni straordinarie quali la scoperta di nuovi
continenti o eventi isolati, non siamo storicamente in grado di
aumentare l'offerta di oro più di circa il 2\% annuo in maniera
sostenuta, anche quando il suo prezzo può decuplicarsi in un decennio,
come accadde negli anni '70.\footnote{Greg Cipolaro e Ross Stevens,
  ``The Power of Bitcoin's Network Effect,'' 6.} L'argento, dal canto
suo, presenta generalmente un rapporto stock-to-flow pari a 10x o
superiore, valore che, pur risultando inferiore a quello dell'oro,
rimane comunque piuttosto elevato.

La maggior parte delle altre materie prime, infatti, possiede rapporti
stock-to-flow inferiori a 1 o 2. Anche gli elementi più rari, come il
platino e il rodio, hanno rapporti relativamente bassi a causa del
rapido consumo da parte dell'industria.

Sebbene l'umanità abbia perfezionato le tecniche estrattive dell'oro
mediante nuove tecnologie, la sua intrinseca rarità, unita al fatto che
i giacimenti superficiali più accessibili sono ormai esauriti, costringe
oggi l'estrazione a dover puntare verso depositi profondi e
difficilmente raggiungibili. Questo meccanismo funge da naturale limite
al progresso tecnologico. Un giorno potremmo rompere questo ciclo con
l'adozione di tecnologie avanzatissime, come l'estrazione da asteroidi
tramite droni o il mining in acque profonde, soluzioni finora confinate
al regno della fantascienza; ma fino a quel momento l'oro conserverà il
suo elevato rapporto stock-to-flow, soprattutto perché gli ambienti
necessari per tale estrazione sono estremamente ostili e i costi
associati sarebbero esorbitanti.

In sostanza, ogni volta che una forma di moneta basata su una merce si
trovava a competere con oro e argento, era sempre quest'ultimi a
prevalere. Altre materie prime potevano fungere da moneta in particolari
regioni e per periodi limitati, ma oro e argento dimostravano
costantemente la capacità di competere a livello globale e di
assicurarsi il successo in ogni confronto. Ciò avveniva perché, in ogni
incontro tra civiltà e sistemi monetari, i detentori di oro e argento
possedevano le capacità tecnologiche necessarie per svalutare gli altri
strumenti monetari, mentre chi possedeva conchiglie, perle, bestiame,
sale, tessuti o metalli inferiori non riusciva a far altrettanto.

\textbf{Precious Metal Coinage}

Le autorità hanno ulteriormente rafforzato il ruolo dell'oro e
dell'argento come denaro introducendo unità standardizzate, solitamente
sotto forma di monete. La coniazione di monete in metalli preziosi si
diffuse in numerose regioni, e la civiltà della Lidia (nell'attuale
Turchia) risulta essere una delle prime conosciute a produrle, già nel
VI secolo a.C.

Il principale vantaggio delle monete emanate da un'autorità ampiamente
riconosciuta -- che all'epoca corrispondeva tipicamente a un regno o a
un impero -- risiedeva nella possibilità di standardizzare le unità in
termini di dimensioni, peso e purezza, facilitando così le transazioni
commerciali. Infatti, senza una misura standardizzata, sarebbe stato
necessario pesare quantità arbitrariamente determinate di oro e argento
per utilizzarle in singole operazioni; la definizione di taglie precise
eliminava questo passaggio. Inoltre, l'immagine di un imperatore
impressa sulle monete -- spesso accompagnata da scanalature lungo i
bordi per evitare la raschiatura del metallo -- conferiva un ulteriore
grado di certificazione della qualità e del contenuto della moneta.
Questo meccanismo di fiducia è ancora evidente oggi: acquistando monete
sovrane moderne in oro, come le \emph{American Eagles}, si deve
corrispondere un premio per oncia rispetto a una quantità maggiore di
oro, poiché gli acquirenti sanno di ottenere oro autentico, facilmente
rivendibile in futuro.

Una moneta, emessa da un regno e dotata dello status di corso legale,
tendeva ad essere scambiata a un valore superiore rispetto al semplice
valore intrinseco del metallo che la componeva, e persino a valori
maggiori rispetto a monete straniere analoghe, grazie alla rassicurante
accettazione e liquidità garantite. In altre parole, le monete in
metallo prezioso riconosciute come corso legale possedevano un valore
stratificato in tre livelli: il primo rappresentava il contenuto
effettivo di metallo prezioso; il secondo era costituito dal premio
legato alla verifica e alla praticità offerta dalla coniazione, che
superava la mera disponibilità di metallo grezzo e variava sia per le
monete nazionali sia per quelle estere; il terzo livello, infine, era un
premio di liquidità esclusivo delle monete domestiche, derivante
dall'ampia -- e spesso obbligatoria -- accettazione come corso legale da
parte dei commercianti della giurisdizione. Di conseguenza, salari e
prezzi espressi in unità monetarie standardizzate tendevano a essere
piuttosto ``rigidi'', ossia a variare lentamente in tutta la società,
anche qualora l'offerta di metalli preziosi o di monete cambiasse in
breve tempo.

Questi vari livelli di valore vennero, nel corso dei secoli,
frequentemente sfruttati dai governi alle prese con budget in crisi a
causa di guerra, avidità o cattiva gestione. Per far fronte a tali
difficoltà, essi cedettero alla tentazione della svalutazione. Ad
esempio, un re poteva prelevare 1.000 monete d'oro puro in tasse,
fonderle e coniare nuove monete composte al 90\% da oro (integrando il
restante 10\% con un metallo di riempimento a basso costo), immettendo
nell'economia 1.111 monete che, pur contenendo la medesima quantità
totale di oro, apparivano sostanzialmente invariate agli occhi della
popolazione. Poiché salari e prezzi tendevano a variare lentamente, il
monarca poteva persino imporre che queste nuove monete fossero accettate
al medesimo valore unitario precedente, per esempio stabilendo l'unità
con cui venivano pagati i soldati. Anni dopo, se il re si trovava ancora
in deficit, poteva nuovamente raccogliere le entrate fiscali, fonderle e
coniare monete al 80\% di purezza, immettendone nell'economia 1.250; se
ciò risultava ancora insufficiente, egli o i suoi successori potevano
ulteriormente ridurre la purezza al 70\%, e così via.

In un primo momento, queste monete leggermente svalutate al 90\%
venivano spesso accettate al valore nominale che avevano
precedentemente, soprattutto se imposte per decreto legale, poiché
l'unità monetaria era in parte astratta rispetto al metallo
sottostante.\footnote{Thomas Marmefelt, \emph{The History of Money and
  Monetary Arrangements}, x e cap. 3.} Tuttavia, con l'allungarsi del
tempo di circolazione e l'aumentare del numero di monete in circolazione
per effetto della diluizione, divenne evidente -- e sempre più difficile
da far rispettare -- il loro valore originario. I prezzi aumentarono,
mentre i salari e il potere d'acquisto dei risparmi diminivano, poiché
le tasse venivano pagate utilizzando riserve in denaro più puro mentre
il reddito corrente era percepito in denaro appena svalutato. Di
conseguenza, i salari tendevano ad aumentare nel tempo, in quanto era
necessario disporre di un numero maggiore di monete per far fronte alle
spese quotidiane. I commercianti cercavano quindi di trattenere le
monete d'oro più pure, vendendo quelle diluite, situazione che
indeboliva la fungibilità dell'intera base monetaria, con il risultato
che le monete non risultavano più standardizzate. Allo stesso tempo, i
mercanti stranieri, operanti al di fuori della giurisdizione che coniava
le monete, richiedevano rapidamente prezzi maggiorati per le monete
d'oro svalutate in cambio dei loro beni e servizi, attribuendone un
valore più rigorosamente legato al contenuto metallico. Col tempo, le
monete più pure finirono per uscire dalla circolazione, a causa della
combinazione di riscossione fiscale da parte del re, accumulo privato e
prelievo da parte dei commercianti stranieri.

La svalutazione della moneta, cioè il debasement, richiedeva solitamente
anni e decenni. Ad esempio, il denario romano in argento fu introdotto
nel 211 a.C. come una piccola moneta d'argento di oltre il 95\% di
purezza; successivamente, pur riducendone le dimensioni, la purezza
rimase superiore al 95\%. Durante il breve regno di Tiberio la purezza
venne persino aumentata, ma intorno al 64 d.C. la moneta subì un
ulteriore ridimensionamento, arrivando a contenere meno del 94\% di
argento. Pur mantenendo costanti le dimensioni per secoli, la purezza
diminuì gradualmente di pochi punti percentuali finché, verso la fine,
crollò rapidamente fino a raggiungere solo circa il 5\% nel 274 d.C. Dal
momento che diversi salari, compresi quelli erogati dallo Stato, non
venivano immediatamente adeguati al lieve deprezzamento del denario,
l'imperatore riusciva a ottenere maggior valore in argento avviando una
svalutazione.\footnote{Colin Elliott, ``The Acceptance and Value of
  Roman Silver Coinage in the Second and Third Centuries AD.''} Con il
tempo, l'aumento dell'offerta di monete sul mercato spinse i prezzi
verso l'alto e i soldati cominciarono a richiedere salari maggiori.

Con il passare degli anni, i progressi nel sistema bancario -- di cui si
parlerà più avanti in questo libro -- ridussero la necessità di
utilizzare monete fisiche e contribuirono a superare il problema della
limitata divisibilità dell'oro. Le persone poterono depositare il
proprio oro presso le banche, ricevendo in cambio un credito cartaceo
che rappresentava una pretesa esigibile sul metallo. Le istituzioni
bancarie, consapevoli che non tutti i depositanti avrebbero richiesto il
riscatto simultaneamente, iniziarono a emettere più crediti rispetto
alla quantità d'oro effettivamente detenuta, dando inizio alla pratica
della banca a riserva frazionaria.\footnote{Stephen Quinn, ``Goldsmith
  Banking: Mutual Acceptance and Interbank Clearing in Restoration
  London,'' \emph{Explorations in Economic History} 34: 411--414.} Nel
corso del tempo il sistema bancario si consolidò progressivamente in
forme di banca centrale in vari Paesi, mediante l'emissione di titoli
nazionali che rappresentavano una pretesa su una determinata quantità
d'oro.\footnote{Charles Goodhart, \emph{Evolution of Central Banks}.}

Durante il periodo che va dalla fine del XIX secolo all'inizio del XX
secolo, l'oro si impose definitivamente sull'argento come mezzo di
pagamento. L'argento perse parte del suo allora pregio monetario,
deprezzandosi rispetto all'oro rispetto ai millenni precedenti.

Nel libro \emph{Globalizing Capital: A History of the International
Monetary System}, Barry Eichengreen spiega che il motivo per cui il gold
standard ebbe la meglio sul silver standard fu in gran parte dovuto a
una coincidenza. Nel 1717, il Direttore della Zecca inglese --
nientemeno che Sir Isaac Newton -- fissò, secondo Eichengreen, un
rapporto ufficiale tra argento e oro troppo basso.\footnote{Barry
  Eichengreen, \emph{Globalizing Capital: A History of the International
  Monetary System,} 5--10.} Di conseguenza, la maggior parte delle
monete in argento uscì dalla circolazione, facendo dell'oro la moneta
indiscussa del regno.\footnote{Fay, ``Newton,'' 111.} Con il Regno Unito
che, nel XIX secolo, si affermò come impero dominante, gli effetti di
rete del gold standard si diffusero in tutto il pianeta, inducendo la
maggior parte degli Stati ad ancorare le proprie valute all'oro. I Paesi
che rimasero troppo a lungo legati al silver standard, come India e
Cina, videro indebolirsi la loro moneta, a seguito del calo della
domanda di argento in Nord America e in Europa.

D'altra parte, Saifedean Ammous, nel libro \emph{The Bitcoin Standard},
sottolinea come la tecnologia bancaria abbia migliorato la divisibilità
dell'oro.\footnote{Ammous, \emph{The Bitcoin Standard}, 28--29.} Come
già accennato, l'oro si dimostra pari o superiore all'argento in molti
degli attributi che definiscono una buona moneta, ad eccezione della
divisibilità. L'argento, infatti, offre una maggiore divisibilità, il
che lo rese il mezzo di pagamento quotidiano per migliaia di anni,
mentre l'oro era riservato prevalentemente a re, mercanti e ordini
religiosi. L'introduzione dei sistemi bancari e delle banconote in varie
denominazioni, garantite dall'oro, migliorò significativamente la sua
effettiva divisibilità. Inoltre, oltre allo scambio di carta, le persone
poterono progressivamente trasferire denaro lungo linee di comunicazione
verso altre parti del mondo, utilizzando le banche e i relativi registri
come intermediari di custodia. Questo rappresentava il gold standard --
il sostegno delle valute cartacee e dei sistemi di comunicazione
finanziaria mediante l'oro -- e riduceva notevolmente l'esigenza di
utilizzare l'argento, poiché l'oro, essendo il metallo più scarso,
resistente e con un rapporto stock-to-flow più elevato, divenne
sostanzialmente altrettanto divisibile dell'argento grazie a questo
ulteriore livello di astrazione.

Ritengo che in entrambe le spiegazioni vi sia un elemento di verità,
sebbene io consideri quella di Ammous più completa, in quanto parte da
un assioma più profondo sulla natura stessa del denaro. Le banconote
hanno reso l'oro più divisibile e, pertanto, la moneta ``dura'' ha
prevalso nel tempo; tuttavia, gli effetti di rete derivanti da decisioni
politiche possono influenzare sia la tempistica che la diffusione
geografica di tali cambiamenti.

Anche dopo che l'oro e l'argento furono demonetizzati dal sistema
bancario globale nella seconda metà del XX secolo, l'oro mantenne il suo
rinnovato premio monetario rispetto all'argento, rivelandosi così nella
forma ideale di risparmio. Per migliaia di anni e in diverse aree
geografiche l'oro veniva scambiato ad un multiplo compreso tra 10 e 16
volte il valore dell'argento.\footnote{J.B. Maverick, ``A Historical
  Guide to the Gold-Silver Ratio,'' \emph{Investopedia}, 27 luglio 2022.}
Negli ultimi cento anni, tuttavia, il rapporto prezzo oro/argento è
mediamente intorno a 50. L'argento perse in modo strutturale gran parte
del suo storico premio monetario rispetto all'oro subito dopo
l'introduzione e l'implementazione dei registri bancari collegati da
sistemi di telecomunicazione intercontinentali negli anni '60 del XIX
secolo -- non credo sia un caso.

Figure 3‑A\footnote{Silvan Frank, ``Gold to Silver Ratio.''}

Con il declino dell'uso dell'oro e dell'argento come mezzi di scambio,
la superiorità in termini di divisibilità dell'argento divenne quasi
irrilevante. Le proprietà dell'oro, inteso come metallo più scarso, più
resistente e con un rapporto stock-to-flow maggiore, divennero gli
attributi decisivi per il risparmio, tanto che l'oro probabilmente
assorbì parte del premio monetario storicamente riservato all'argento.
Le banche centrali di tutto il mondo conservano tuttora oro nei loro
caveau e molte di esse continuano ad acquistarlo annualmente come parte
delle riserve valutarie estere. Tali detenzioni, che si quantificano in
centinaia -- o addirittura in migliaia -- di tonnellate, richiedono
quell'alta densità di valore che l'oro offre rispetto all'argento,
rendendolo particolarmente adatto a risparmi a lungo termine di grande
entità. Pertanto, sebbene la moneta emessa dallo Stato non sia più
garantita da una quantità specifica di oro, questo metallo rimane un
elemento indiretto e fondamentale del sistema monetario globale in
qualità di asset di riserva per le banche centrali. Non esiste alcun
altro bene naturale che possa sostituirlo.

Se la forma migliore di denaro presenta una limitazione intrinseca --
come ad esempio la divisibilità limitata dell'oro -- ciò consente la
coesistenza di diverse tipologie monetarie. Standard bimetallici e
perfino trimetallici furono adottati per lunghi periodi al fine di
superare le restrizioni dovute alla limitata divisibilità dell'oro.
Viceversa, se la forma ottimale di denaro è priva di tali limitazioni,
essa tende a prevalere e a soppiantare le alternative. Oggi né l'oro né
l'argento sono largamente utilizzati come mezzi di scambio, pur
continuando a servire da riserva di valore a lungo termine; tuttavia,
l'oro risulta più attraente per i grandi detentori grazie alla sua
superiore durabilità, a un rapporto stock-to-flow maggiormente
favorevole e a un valore intrinseco per unità di massa e volume più
elevato.

Ancora oggi, sia l'oro che l'argento godono di un riconoscimento
monetario significativo a livello globale. Pur non essendo comunemente
accettati nei pagamenti quotidiani, se dovessi recarmi in quasi ogni
Paese munito di monete d'oro o gioielli in oro, saprei trovare
facilmente un negoziante o un commerciante disposto ad acquistarli in
valuta locale a un prezzo di mercato ragionevole, generalmente senza
grosse difficoltà. L'oro fisico -- che si tratti di monete, lingotti o
gioielli -- rimane una delle modalità migliori a disposizione degli
individui per conservare valore in forma d'asset portante, denso,
liquido e fungibile, senza incorrere nel rischio di controparte.

\section{Footnotes}\label{footnotes-4}

\bookmarksetup{startatroot}

\chapter{\texorpdfstring{Capitolo 4: \textbf{Una teoria unificata del
denaro}}{Capitolo 4: Una teoria unificata del denaro}}\label{capitolo-4-una-teoria-unificata-del-denaro}

Le descrizioni relative alla definizione e alle origini del denaro
tendono a dividersi in due principali orientamenti economici, con varie
sottoscuole attorno ad essi.

Una di queste scuole è la teoria del denaro come merce, mentre l'altra,
opposta, è la teoria del denaro come credito. I capitoli precedenti di
questo libro hanno fatto riferimento a entrambe, e il presente capitolo
esamina in modo più specifico come le due teorie possano essere
riconciliate.

Nella letteratura economica occidentale, la teoria del denaro come merce
affonda le sue radici almeno in \emph{Politica} di Aristotele
nell'Antica Grecia; venne poi elaborata e popolarizzata in \emph{La
ricchezza delle nazioni} di Adam Smith nel 1776, per essere
successivamente sviluppata nei minimi dettagli da Carl Menger, Ludwig
von Mises e altri economisti che costituirono le basi della scuola
austriaca di economia nel XIX e XX secolo. Il concetto chiave di questa
teoria è che il baratto risulta inefficiente, poiché richiede la doppia
coincidenza dei bisogni, e per questo motivo un bene altamente
commerciabile, resistente alla svalutazione (ad es. oro o argento),
emerge naturalmente in una società come unità di conto, mezzo di scambio
e riserva di valore, al fine di ridurre le frizioni nel commercio.

Nel suo libro \emph{Principi di economia}, Carl Menger sosteneva:

\begin{quote}
Il denaro non è un'invenzione dello Stato. Non è il prodotto di un atto
legislativo. Neanche la sanzione da parte dell'autorità politica è
necessaria per la sua esistenza. Alcune merci sono divenute denaro in
modo del tutto naturale, come conseguenza di rapporti economici
indipendenti dal potere dello Stato.\footnote{Carl Menger,
  \emph{Principles of Economics}, 262.}
\end{quote}

La teoria del credito del denaro è più recente in termini di esposizione
completa, risalendo a Henry Dunning Macleod e Georg Friedrich Knapp
nella seconda metà del XIX secolo, e viene spesso proposta come
controparte alla teoria della moneta basata sulle merci. Alfred
Mitchell-Innes ha esposto in maniera concisa questa teoria del denaro ai
primi del '900, e John Maynard Keynes ne fu influenzato, tanto da
integrarla nelle sue prescrizioni economiche.\footnote{John Maynard
  Keynes, ``A. Mitchell-Innes. \emph{What Is Money?}''} La Modern
Monetary Theory, formulata da Abba Lerner negli anni '40\footnote{Abba
  Lerner, ``Money as a Creature of the State.''} e rivitalizzata negli
anni '90 da economisti quali Warren Mosler, Bill Mitchell e Larry
Randall Wray, ha ulteriormente ampliato questo modo di
pensare.\footnote{Dylan Matthews, ``Modern Monetary Theory, explained,''
  \emph{Vox}, 16 aprile 2019.} In seguito, la popolarità degli scritti
dell'antropologo David Graeber sulla storia del debito ha generalmente
mostrato una visione favorevole alla teoria del credito rispetto a
quella basata sulle merci. Il tema centrale della teoria del credito del
denaro è che il credito costituisce il nucleo stesso di ciò che il
denaro è, piuttosto che le merci. In ``The Credit Theory of Money'', il
secondo dei suoi saggi influenti degli anni 1910, Alfred Mitchell-Innes
ha sintetizzato quanto segue:

\begin{quote}
In breve, la teoria del credito afferma che una vendita e un acquisto
rappresentano lo scambio di una merce contro credito. Da questa teoria
principale scaturisce la sotto-teoria secondo cui il valore del credito
o del denaro non dipende dal valore di un metallo o dei metalli, ma dal
diritto che il creditore acquisisce al ``pagamento'', ossia alla
soddisfazione del credito, e dall'obbligo del debitore di ``pagare'' il
proprio debito e, viceversa, dal diritto del debitore di liberarsi dal
debito mediante l'offerta di un debito equivalente dovuto dal creditore,
e dall'obbligo del creditore di accettare tale offerta in sostituzione
del proprio credito.
\end{quote}

\begin{quote}
Questa è la teoria fondamentale, ma in pratica non è necessario che un
debitore acquisisca crediti dalle stesse persone a cui è debitore. Siamo
tutti sia acquirenti che venditori, per cui siamo allo stesso tempo
debitori e creditori gli uni degli altri, e grazie alla macchina
meravigliosamente efficiente delle banche a cui vendiamo i nostri
crediti -- che diventano così le camere di compensazione del commercio
-- i debiti e i crediti dell'intera comunità vengono centralizzati e
compensati tra loro. In pratica, dunque, ogni buon credito salda ogni
debito.\footnote{Alfred Mitchell-Innes, ``The Credit Theory of Money,''
  152.}
\end{quote}

Ciò che complica la rivalità tra queste scuole di pensiero e il
dibattito attuale è che esse sono intrinsecamente soggette a
politicizzazione. Gli economisti che preferiscono un ruolo limitato del
governo tendono a orientarsi verso il concetto di denaro come fenomeno
emergente dal basso. Nella misura in cui lo Stato possa essere coinvolto
nell'emissione di valuta, i sostenitori di questa visione generalmente
sostengono che la creazione di moneta da parte dello Stato debba essere
vincolata a qualche forma di scarsità naturale, come il supporto e la
riscossione di quantità specifiche di oro. D'altra parte, gli economisti
che optano per un ruolo più ampio per lo Stato tendono a valorizzare il
concetto di denaro come prodotto dall'alto, o perlomeno come qualcosa
intrinsecamente legato all'organizzazione sociopolitica in tutte le sue
forme. Secondo molti di loro, lo Stato non dovrebbe essere vincolato
dalla scarsità naturale, ma disporsi di un elevato grado di flessibilità
nell'offerta della valuta emessa, per poter perseguire i propri
obiettivi.

Come discusso nel primo capitolo di questo libro sulle società di
cacciatori-raccoglitori, sia il concetto di proto-moneta basata sulle
merci che quello di credito risalgono alle interazioni umane più antiche
e fondamentali. Sia la teoria della moneta basata sulle merci che la
teoria del credito contribuiscono a una comprensione olistica della
definizione e delle origini del denaro sul piano più basilare.
Confrontando queste differenti linee di ragionamento, possiamo definire
il denaro nel modo più ampio e preciso possibile.

\textbf{Mettere le linee temporali in ordine}

Per conciliare queste due teorie opposte del denaro e individuare il
fondamento sottostante, dobbiamo iniziare esaminando le loro differenze,
valutando ciò che ciascuna teoria sembra affermare correttamente e dove
inciampa.

Per questo esercizio possiamo tornare a \emph{Wealth of Nations}, in cui
Adam Smith descrive il problema del baratto e spiega perché, in modo del
tutto naturale, sorge il denaro merceologico per risolvere tale
problema:

\begin{quote}
Ma quando la divisione del lavoro cominciò ad affermarsi, questa
capacità di scambio doveva spesso incontrare notevoli ostacoli nel suo
funzionamento. Supponiamo che un uomo abbia di una certa merce più di
quanto esso stesso necessiti, mentre un altro ne abbia meno. Il primo,
di conseguenza, sarebbe contento di liberarsene; e il secondo di
acquistarne una parte. Ma se costui non ha niente che il primo richieda,
tra loro non può essere effettuato alcuno scambio. Il macellaio ha in
negozio più carne di quanta ne possa consumare, e sia l'imbottitore che
il fornaio sarebbero disposti ad acquistarne una parte. Tuttavia, loro
non hanno nulla da offrire in cambio, se non le differenti produzioni
dei rispettivi mestieri, e il macellaio dispone già di tutto il pane e
la birra di cui ha immediata necessità. In tal caso, nessuno scambio può
essere realizzato tra di loro. Egli non può essere il loro mercante, né
loro i suoi clienti; e tutti quanti risultano così reciprocamente meno
utili gli uni agli altri. Per evitare l'inconveniente di tali
situazioni, ogni uomo prudente, in ogni epoca della società, dopo il
primo consolidamento della divisione del lavoro, doveva naturalmente
tendere a gestire i propri affari in modo da avere sempre a
disposizione, oltre al prodotto peculiare della propria industria, una
certa quantità di una determinata merce, tale da sembrare a suo parere,
da scambiare, difficile da rifiutare in cambio del prodotto del lavoro
altrui. Molte merci diverse, probabilmente, furono concepite ed
impiegate successivamente a questo scopo. Nei primordi della società, si
dice che il bestiame fosse lo strumento comune del commercio; e, sebbene
debba essere stato un mezzo alquanto scomodo, in passato le cose
venivano spesso valutate in base al numero di bestiame offerto in
cambio. L'armatura di Diomede, dice Omero, costava solo nove buoi;
mentre quella di Glauco valeva cento buoi. Si narra, inoltre, che in
Abissinia il sale fosse lo strumento comune per il commercio e gli
scambi; in alcune parti della costa indiana si usassero specie di
conchiglie; a Terranova il merluzzo essiccato; in Virginia il tabacco;
in alcune colonie delle Indie Occidentali lo zucchero; in altri paesi si
preferissero pelli o cuoio conciato; e oggi, in un villaggio della
Scozia, non è raro, mi è stato detto, che un operaio si rechi alla
panetteria o alla taverna portando chiodi al posto del denaro.
\end{quote}

\begin{quote}
Tuttavia, in tutti i paesi sembra che, per motivi irresistibili, gli
uomini abbiano infine deciso di privilegiare, per questo impiego, i
metalli rispetto a ogni altra merce. I metalli non solo possono essere
conservati con perdite minime, essendo tra le merci meno deperibili, ma
possono anche essere divisi, senza alcuna perdita, in un numero
qualunque di parti, poiché tali parti possono facilmente essere riunite
per fusione; una qualità che nessun'altra merce altrettanto durevole
possiede, e che, più di ogni altra, li rende idonei ad essere gli
strumenti del commercio e della circolazione.\footnote{Adam Smith,
  \emph{Wealth of Nations}, Libro I, Capitolo IV, 37--39.}
\end{quote}

Ripensando a questo passaggio secoli dopo, la descrizione complessiva di
Smith si è rivelata duratura, e in molti aspetti egli ha centrato il
punto. In primo luogo, descriveva il denaro merceologico come un
fenomeno che emerge naturalmente per risolvere la difficoltà insita nel
soddisfare la doppia coincidenza dei bisogni in un mondo caratterizzato
dalla specializzazione del lavoro. In secondo luogo, egli spiegava
perché i metalli preziosi, in particolare, abbiano resistito meglio nel
tempo per questa funzione rispetto ad altre merci -- grazie alle loro
proprietà uniche. In altre parole, non fu un caso che culture di tutto
il mondo tendessero a considerare oro e argento come le forme ideali di
denaro merceologico.

Divertentemente, l'emergere, dal basso verso l'alto, dei sigilli SoJ
come denaro in \emph{Diablo II}, come descritto nel Capitolo 2, costituì
un esperimento osservabile che confermava la proposta di Smith
sull'origine del denaro. Questo, insieme ad altri esempi simili, permise
di osservare direttamente il processo anziché limitarlo a mere teorie.
Gli sviluppatori di \emph{Diablo II} crearono l'ambiente di gioco e
milioni di giocatori valutavano quell'ambiente, gravitandovi rapidamente
e naturalmente verso i sigilli SoJ come forma principale di risparmio,
pagamento e unità di conto, per evitare di dover affrontare il problema
della doppia coincidenza dei bisogni (cioè il problema del baratto)
nello scambio di oggetti di alto valore, benché questa non fosse
l'intenzione degli sviluppatori. Una limitazione fondamentale di questo
ambiente era che il gioco venisse praticato da persone di tutto il
mondo, per lo più sconosciute tra loro, e pertanto il meccanismo del
credito non poteva essere generalmente utilizzato. In un mondo di
sconosciuti, con un'esistenza quasi nulla di credito, il denaro
merceologico emerge naturalmente dai singoli partecipanti al mercato, e
il denaro che si impone è quello che possiede i migliori attributi per
svolgere la funzione di denaro.\footnote{Vedi, ad esempio, Narayana
  Kocherlakota, ``Money is Memory''; Stefano Ugolini, \emph{The
  Evolution of Central Banking: Theory and History}, 165--175.}

Tuttavia, le evidenze antropologiche successive all'epoca di Smith hanno
dimostrato che probabilmente egli aveva capito al contrario un aspetto
fondamentale: lo scambio tramite baratto tra lavoratori specializzati
non ha preceduto il denaro nel modo da lui ipotizzato. Non vi fu mai un
periodo in cui fosse comune che lavoratori specializzati---come
macellai, birrai e panettieri---riscontrassero che i loro possibili
scambi risultassero ``ingorgati e imbarazzanti'' a causa della mancata
definizione del concetto di denaro. Infatti, una forma di denaro era già
stata ideata all'epoca in cui emersero forme di lavoro specializzato,
poiché un sistema flessibile di credito sociale aveva già parzialmente
eliminato la necessità di soddisfare la doppia coincidenza dei desideri
tra individui che si conoscevano. Il concetto di credito è infatti più
antico della coniazione e rivaleggia con l'età dei primi proto-denari da
collezione, come le conchiglie.\footnote{David Graeber, \emph{Debt: The
  First 5,000 Years}; George Selgin, ``The Myth of the Myth of Barter.''}

Nella più ampia interpretazione possibile, l'idea di debito e credito si
estende a tempi antecedenti la nostra specie. Quando una scimmia toglie
insetti dal dorso di un'altra e poi le due si scambiano il favore, ciò
rappresenta una forma temporanea di debito e credito. La scimmia che ha
ricevuto il beneficio potrebbe non ricambiare. Gli esseri umani,
naturalmente, hanno ampliato enormemente questa dinamica, sia in termini
di complessità che di durata dei tipi di credito e debito presenti
all'interno di una struttura sociale.

Alfred Mitchell-Innes, nel suo saggio del 1913 ``What is Money?'',
ribalta l'ordine cronologico di Smith in questo modo:

\begin{quote}
La posizione di Adam Smith dipende dalla verità dell'affermazione
secondo cui, se il panettiere o il birraio desidera carne dal macellaio,
ma non ha (dato che quest'ultimo è sufficientemente provvisto di pane e
birra) nulla da offrire in cambio, non può esserci scambio tra di loro.
Se ciò fosse vero, forse la dottrina del mezzo di scambio sarebbe
corretta. Ma è davvero così?
\end{quote}

\begin{quote}
Supponendo che il panettiere e il birraio siano uomini onesti, e onestà
non è una virtù moderna, il macellaio potrebbe ricevere da loro un
riconoscimento che attesti quanto hanno acquistato da lui, e basta
supporre che la comunità riconosca l'obbligo, per il panettiere e il
birraio, di rimborsare tali riconoscimenti in pane o birra al valore
relativo vigente nel mercato del villaggio ogni qualvolta vengano
presentati, ed avremo così una valuta buona e sufficiente. Una vendita,
secondo questa teoria, non è lo scambio di una merce per un intermedio
denominato ``mezzo di scambio'', bensì lo scambio di una merce per un
credito.
\end{quote}

\begin{quote}
Non sussiste alcuna ragione per ipotizzare l'esistenza di un dispositivo
così maldestro come il mezzo di scambio quando un sistema tanto semplice
riesce a soddisfare tutte le esigenze. Ciò che dobbiamo provare non è un
insolito accordo generale ad accettare oro e argento, ma un senso
condiviso della sacralità di un'obbligazione. In altre parole, la teoria
attuale si fonda sull'antichità della legge del debito.
\end{quote}

\begin{quote}
Siamo fortunati ad avere qui solide basi storiche. Fin dai primordi, di
cui abbiamo documentazione, siamo in presenza di una legge del debito, e
quando troveremo---come indubbiamente accadrà---documenti risalenti a
epoche ancora più remote rispetto a quelle del grande re Hammurabi, che
compilò il suo codice di leggi babilonese 2000 anni a.C., non dubito che
emergeranno ancora tracce della medesima legge. La sacralità di
un'obbligazione è, in effetti, il fondamento di tutte le società non
solo in ogni tempo, ma in ogni fase della civiltà; e l'idea che nei
popoli da noi definiti selvaggi il credito sia sconosciuto e che si
faccia esclusivamente baratto è infondata.\footnote{Alfred
  Mitchell-Innes, \emph{What is Money?}, 391.}
\end{quote}

In altre parole, lungo il suo saggio Mitchell-Innes sostiene, sulla base
delle evidenze antropologiche, che il credito sociale flessibile aveva
già parzialmente risolto il problema del baratto e ridotto le frizioni
nel commercio tra vicini già nella fase dei cacciatori-raccoglitori,
cioè ben prima che il lavoro specializzato raggiungesse il punto
descritto da Smith.

Il birraio e il panettiere possono procurarsi della carne promettendo di
pagare in seguito, molto probabilmente tramite i prodotti che essi
stessi realizzano (ad esempio, ``grazie per la carne, ecco un credito
per cinque pagnotte di pane, che puoi scambiare con qualcun altro se
vuoi, e io onorerò chiunque lo riscatti.''). Allo stesso modo, il
macellaio può ottenere birra e pane impegnandosi a pagare in un secondo
momento, e molti dei loro scambi possono essere compensati
reciprocamente. Un macellaio che detiene un credito per del pane, in
seguito ad aver venduto della carne al panettiere, potrebbe facilmente
cedere quel credito a qualcun altro per ottenere pane. L'uso del credito
risolve il problema del baratto tra persone che condividono un certo
grado di continuità e fiducia reciproca, oppure che si affidano entrambe
a un'autorità locale in grado di far rispettare il credito.

Il denaro-merce, invece, emerge ripetutamente soprattutto per ridurre
gli attriti del commercio tra sconosciuti o per integrare e migliorare i
sistemi che si basano su un credito sociale flessibile, fungendo da
forma di regolamento finale e da risparmio a lungo termine. Le persone
appartenenti allo stesso gruppo sociale possono cavarsela, fino a un
certo livello di complessità, con piccoli registri informali controllati
dagli uomini, mentre gli estranei traggono vantaggio dall'essere in
grado di effettuare il regolamento finale sul posto. Ed è proprio in
questo contesto che il denaro-merce è necessario per sostituire o
integrare il credito sociale flessibile.\footnote{Ugolini, \emph{Central
  Banking}, 169--171.}

Prendendo ad esempio l'antico Babilonia ai tempi di Hammurabi, si
usavano il grano e l'argento come denaro-merce, ma si impiegavano anche
registri in argilla per mantenere il concetto di credito. Dopotutto, i
cereali sono merci fortemente stagionali. Un agricoltore si affidava al
credito per acquistare vari beni fino alla stagione del raccolto,
momento in cui poteva (si spera) saldare tutti i suoi debiti
raccogliendo e vendendo il raccolto. Inoltre, lo studio dei
cacciatori-raccoglitori e le evidenze archeologiche testimoniano l'ampio
uso di sistemi di credito informali basati sull'onore e sulla parentela,
così come l'impiego di oggetti da collezione quali perline di conchiglia
usate come proto-denaro, il che significa che sia il denaro-credito che
il denaro-merce, nelle loro diverse forme, esistevano ancor prima
dell'avvento di forti specializzazioni lavorative.

\textbf{Dove la teoria del credito come denaro sbaglia}

I sostenitori della teoria del credito come denaro hanno fatto bene ad
esaminare le evidenze antropologiche sul concetto di credito e a
criticare i primi difensori della teoria del denaro-merce per quanto
riguarda l'ordine degli eventi. Il credito, infatti, si colloca quasi
all'origine del commercio e del denaro, piuttosto che essere un'aggiunta
successiva, e questa osservazione rappresenta una correzione utile a
\emph{La ricchezza delle nazioni}. L'interazione umana è, nel suo
nucleo, una sequenza di accrediti e addebiti formali o informali nei
confronti degli altri, organizzata tramite rituali e regole che
affondano le radici nei nostri istinti evolutivi, nelle prime religioni
e nelle strutture di governo più antiche.

Tuttavia, i sostenitori della teoria del credito del denaro tendono a
spingersi troppo oltre, trascurando spesso l'importanza del
denaro-mercanzia e dipingendo in genere un quadro troppo ottimistico
della capacità di una società di amministrare un registro flessibile,
basato sul credito, su larga scala e per lunghi periodi di tempo.

Come primo punto di critica, possiamo osservare le difficoltà che i
commercianti di nicchia incontrerebbero in un simile sistema basato sul
credito. Il macellaio, il birraio e il fornaio potrebbero distribuire
crediti in unità di scambio convenienti (ad esempio, una libbra di manzo
equivalente a tre pinte di birra, equivalenti a cinque pagnotte di
pane), ma che dire dei fornitori di servizi di alto valore e di nicchia,
come i chirurghi? Se una chirurga acquista del pane, consegna allora un
credito che le permetta di usufruire in seguito dei suoi servizi
chirurgici? Quante pagnotte di pane valgono un intervento chirurgico, e
di che tipo? Gli interventi chirurgici sono servizi di alto valore, di
nicchia, poco fungibili e non comunemente richiesti. Diventa evidente
quindi la necessità di una piccola unità di conto per rendere gli scambi
più agevoli e, storicamente, tale unità era spesso legata a una merce
specifica, come alcuni grammi d'argento o il valore di un pasto in
cereali. Altrimenti, si cadrebbe in un baratto astratto, cercando di
scambiare ogni sorta di diversi crediti per beni senza un'unità di conto
standardizzata. Storicamente, anche quando il credito veniva utilizzato
come strumento effettivo di scambio, tale strumento era di solito
espresso in un'unità merceologica vendibile che si era formata
naturalmente a causa delle sue proprietà intrinseche.

Come secondo punto di critica, ci si può chiedere: cosa accade se
qualcuno lascia una comunità per unirsi a un'altra? Raramente esiste una
comunità veramente chiusa; le comunità interagiscono tra loro sin dai
primordi dell'umanità. Affinché la ricchezza possa essere trasferita tra
comunità, essa deve esistere in una forma più fisica o universale. Il
credito può funzionare all'interno di una comunità per gli scambi
quotidiani, ma chi desidera viaggiare necessita di una forma di
ricchezza più fondamentale, riconosciuta da una nuova
comunità.\footnote{Brian Albrecht e Andrew Young, ``Wampum: The
  Political Economy of an Institutional Tragedy''; Lawrence White,
  \emph{Better Money: Gold, Fiat, or Bitcoin?}, 17.} Le monete naturali
fungono da collegamento tra ecosistemi creditizi altrimenti chiusi e
circolari.

Come terzo punto di critica, è intuitivo che ciò che funziona su piccola
scala non necessariamente sia efficace su larga scala. I piccoli sistemi
creditizi sociali, basati sull'onore e su trattative individuali tra
persone conosciute, non possono essere applicati allo stesso modo agli
Stati-nazione, i quali devono governare milioni di individui per lo più
sconosciuti tra loro. Il concetto di fiducia funziona solo se si conosce
e si ha realmente fiducia nella persona in questione. L'onore, pur
essendo un concetto estremamente importante nelle interazioni umane, non
si adatta bene a una burocrazia impersonale.\footnote{Seabright,
  \emph{Company of Strangers}, 86; Avner Greif, ``The Fundamental
  Problem of Exchange,'' 261--262.}

Mitchell-Innes sosteneva nel suo saggio ``Credit Theory'' del 1914 che
il nostro denaro si apprezzerebbe se lo si dissociasse dall'oro:

\begin{quote}
Immaginiamo che, mantenendo l'oro a un prezzo fisso, stiamo preservando
il valore della nostra unità monetaria, mentre in realtà facciamo
esattamente il contrario. Più a lungo manteniamo l'oro al suo prezzo
attuale, mentre il metallo continua ad essere abbondante come ora, tanto
più deprezziamo il nostro denaro.\footnote{Mitchell-Innes, ``Credit
  Theory,'' 160.}
\end{quote}

Naturalmente, accadde esattamente il contrario. Al momento della stesura
di questo testo, il dollaro statunitense e la sterlina britannica hanno
perso, rispettivamente, oltre il 98\% e il 99\% del loro valore in
cambio dell'oro, da quando furono dissociati dall'oro nei decenni
successivi al saggio di Mitchell-Innes. Per la maggior parte dei paesi
il calo fu ancora maggiore, e la svalutazione delle valute rispetto
all'oro si verificò in ogni paese che abbia mai lanciato una moneta
fiat.

Nonostante questo grave errore di previsione, il ragionamento che
Mitchell-Innes utilizzò nel 1914 per sostenere la sua tesi non appariva
per nulla improbabile in superficie. Egli sosteneva infatti che non
fosse lo svalutamento arbitrario \emph{per sé} a essere alla radice
della storica devalorizzazione del denaro emesso dallo Stato, ma bensì
la guerra, la peste e altri distruttori della produttività a condurre
allo svalutamento. Se soltanto potessimo enfatizzare ripetutamente la
pace e l'organizzazione, sosteneva, il denaro statale resisterebbe allo
svalutamento:

\begin{quote}
Non sono il re Gian, il re Filippo, Edward o Enrico a essere stati i
deprezzatori del denaro, ma è il re Guerra, il grande creatore di
debiti, coadiuvato dai suoi luogotenenti -- la peste, il morbo e i
raccolti rovinati -- qualunque cosa, in effetti, impedisca ai debiti di
essere estinti puntualmente. Non sono gli atti di ricoinazione a essere
stati i restauratori del valore del denaro, ma la Pace, il grande
creatore di crediti, e su questa verità invariabile deve fondarsi in
larga misura la teoria del credito del denaro.\footnote{Mitchell-Innes,
  ``Credit theory,'' 157.}
\end{quote}

Su questo punto aveva in gran parte ragione. Ad eccezione di casi di
sovrariamente corrotti o di governanti con gravi problemi mentali, un re
solitamente non si sveglia un giorno e decide impulsivamente di
svalutare la monetazione del suo regno utilizzando metalli più economici
senza motivo. La guerra, le pestilenze e altri distruttori della
produttività sono, infatti, alla base del motivo per cui i re finiscono
solitamente per svalutare il loro denaro. Per rimanere al potere, i
governanti cercano di rafforzare la loro posizione politica, placare i
sudditi e far fronte ai problemi che inevitabilmente sorgono durante il
loro regno. La svalutazione della valuta è un metodo a cui un re può
ricorrere per effettuare pagamenti maggiorati senza dover aumentare le
tasse, trasferendo così, nel tempo, il costo su coloro che accettano le
nuove monete svalutate al vecchio valore nominale, nonostante non
abbiano più lo stesso contenuto di metallo o la stessa scarsità
dell'offerta di un tempo.

Tuttavia, ciò che, a mio avviso, Mitchell-Innes ha tralasciato è che la
capacità di deprezzare la moneta \emph{contribuisce} alla probabilità
che si scatenino guerre e altre forme di produttività danneggiata sin
dall'inizio. La tentazione per un sovrano di deprezzare la monetazione è
troppo forte per essere ignorata, poiché è solitamente la via di minor
resistenza quando si presenta un problema. Se il re sa che pagare una
guerra aumentando direttamente le tasse probabilmente porterebbe a una
rivoluzione, mentre finanziare la guerra tramite un graduale
deprezzamento della monetazione no, potrà giustificare il finanziamento
della guerra affidandosi a questo secondo metodo. Se lui e il possibile
avversario in guerra fossero entrambi costretti al primo metodo, ovvero
pagare una guerra con tasse aggiuntive invece che con il deprezzamento,
la guerra potrebbe tramutarsi in una possibilità remota, dato che i
sudditi potrebbero rivoltarsi se essa avvenisse. I costi della guerra
risulterebbero così immediatamente più trasparenti e impopolari. Al
contrario, la possibilità di deprezzare la monetazione per finanziare
una guerra consente che essa avvenga innanzitutto, con i costi
parzialmente posticipati, aumentando così sia la probabilità che scoppia
una guerra sia la scala dell'evento. Se il deprezzamento \emph{può}
avvenire, alla fine \emph{succederà} per una miriade di ragioni. La
possibilità di deprezzamento esiste, sempre, ovunque e in modo così
invitante, da rappresentare una risorsa a cui un governo può ricorrere
quando non è in grado di spendere in maniera trasparente per ciò che
desidera.

Nel lungo termine, dal punto di vista di chi risparmia, quasi sempre
conviene detenere direttamente una moneta basata su una merce scarsa
piuttosto che affidarsi indefinitamente alla promessa offerta da un
regno, un impero o uno stato-nazione. La prima è soggetta alle leggi
inesorabili della natura, mentre la seconda è vincolata alla fallibilità
dell'umanità.

Possiamo evidenziare l'errore nel ragionamento di Mitchell-Innes
osservando come egli descriva il credito come la forma di proprietà più
preziosa nel suo saggio ``What is Money?'':

\begin{quote}
Un credito di prima classe è la forma di proprietà più preziosa. Non
avendo un'esistenza corporea, non ha peso e non occupa spazio. Può
essere trasferito con facilità, spesso senza alcuna formalità. È
trasportabile a volontà da un luogo all'altro con un semplice ordine, il
cui costo si limita a una lettera o a un telegramma. Può essere
utilizzato immediatamente per soddisfare qualsiasi esigenza materiale e
può essere protetto dalla distruzione e dal furto a costi contenuti. È
la forma di proprietà più facilmente gestibile e una delle più
permanenti. Vive con il debitore e condivide le sue fortune, e quando
questi muore, passa agli eredi della sua eredità. Finché l'eredità
esiste, l'obbligazione persiste, e in circostanze favorevoli e in uno
stato di commercio sano non sembra esserci alcuna ragione per cui essa
debba mai deteriorarsi.\footnote{Mitchell-Innes, ``What is money?'' 392.}
\end{quote}

Il problema sta nel fatto che l'assunzione ``in circostanze favorevoli e
in uno stato di commercio sano'' è piuttosto elevata se estesa
all'intera vita, per non parlare delle generazioni. Durante il corso
della vita e della governance, inevitabilmente sorgono problemi e vari
debiti vengono deprezzati, estinti o cadono in default. Nel secolo
scorso, durante il quale le valute hanno trascorso la maggior parte del
tempo scollegate dalla scarsità naturale dei metalli preziosi, un
``credito di prima classe'' si è rivelato uno degli asset meno
desiderabili da detenere rispetto alle alternative. In dozzine di paesi
in tutto il mondo, a seguito dei saggi di Mitchell-Innes, i crediti e le
relative valute sottostanti sono stati letteralmente distrutti
dall'iperinflazione. Nei paesi di maggior successo, che sono stati dalla
parte vincente delle grandi guerre e disponevano di istituzioni
finanziarie solide, i crediti di prima classe hanno generalmente evitato
il destino dell'iperinflazione, ma hanno comunque registrato performance
inferiori rispetto agli immobili, al capitale d'impresa, ai metalli
preziosi, alle opere d'arte pregiate e ai vini pregiati.\footnote{Òscar
  Jordà et al., ``The rate of return on everything, 1780--2015.''}

In altre parole, i sostenitori della teoria del credito del denaro,
quando applicano la loro analisi a un organo governativo
sufficientemente ampio, si basano generalmente sull'assunzione di
disporre di una catena ininterrotta di amministratori altamente
competenti e altruisti del registro pubblico. Questa è un'assunzione
che, volta dopo volta, cultura dopo cultura, secolo dopo secolo, si è
rivelata del tutto inefficace. Ignorando i metalli preziosi o qualunque
vincolo naturale come un metodo inutile o goffo per mantenere la
disciplina del registro pubblico, trascurano un aspetto fondamentale: le
monete merce hanno resistito al tempo per migliaia di anni perché
nessuno può istantaneamente produrne di più, nemmeno quando
apparentemente ha una valida ragione per farlo. Inoltre, esse
rappresentano una liquidazione finale anziché un'affidarsi in
ininterrotta misura alle promesse di entità centralizzate.

È interessante notare che, nonostante le varie affermazioni sul denaro,
Mitchell-Innes era ben consapevole del fatto che, in tutta la storia
finanziaria, le unità monetarie definite dall'uomo si deprezzano
strutturalmente e non sembrano mai apprezzarsi in modo strutturale. Come
scrisse in \emph{The Credit Theory of Money}:

\begin{quote}
Ma sebbene l'unità monetaria possa deprezzarsi, essa non sembra mai
apprezzarsi. Un aumento generale dei prezzi, a volte rapido e talvolta
lento, è la caratteristica comune di tutta la storia finanziaria; e
mentre un aumento rapido può essere seguito da una caduta, quest'ultima
appare come un semplice ritorno a uno stato di equilibrio. Dubito che
esistano casi in cui il ribasso porti a un prezzo inferiore a quello
precedente all'aumento, e qualsiasi cosa che possa assomigliare a un
ribasso persistente dei prezzi, indicante un aumento continuo del valore
del denaro, appare sconosciuta.\footnote{Mitchell-Innes, ``Credit
  theory,'' 159.}
\end{quote}

In questo modo, possiamo paragonare i registri centralizzati,
controllati dall'uomo, alla seconda legge della termodinamica. Questa
legge afferma che l'entropia (fondamentalmente il termine scientifico
per ``disordine'') di un qualsiasi sistema chiuso può solo aumentare nel
tempo; non può mai diminuire. Niente, al di là di un sistema
perfettamente efficiente, privo di attrito e di perdita di calore (cose
che, in realtà, non esistono), può evitare la crescita continua
dell'entropia. Analogamente, nulla, a parte una catena ininterrotta di
governanti impeccabili, può mantenere un sistema monetario flessibile
senza svalutazione, e una catena così perfetta non esiste. Problemi
inevitabilmente sorgono in ogni ambito, e, di volta in volta, le
autorità ricorrono alla creazione di ulteriore moneta per mitigare tali
problemi e svalutare in maniera poco trasparente diversi debiti.

David Graeber, che per lo più può essere collocato nel campo della
teoria del credito, osservò il rapporto tra il livello di fiducia
sociale e il tipo di moneta in uso nel suo libro \emph{Debt: The First
5000 Years}:

\begin{quote}
Di conseguenza, mentre i sistemi di credito tendono a dominare in
periodi di relativa pace sociale, o nelle reti di fiducia (sia create
dagli Stati o, nella maggior parte dei periodi, da istituzioni
transnazionali come corporazioni di mercanti o comunità di fede), in
periodi caratterizzati da guerre diffuse e saccheggi, essi tendono a
essere sostituiti dai metalli preziosi.\footnote{Graeber, \emph{Debt},
  215.}
\end{quote}

Nel suo libro, Graeber tendeva a descrivere i metalli preziosi in
termini per lo più negativi, sostenendo, ad esempio, che una delle
ragioni per cui venivano impiegati durante i conflitti era che
risultavano abbondanti a causa dei saccheggi. I soldati del lato
vincente di una guerra saccheggiavano ogni sorta di riserva in metallo
prezioso e ornamenti dai caveau e dai templi del lato sconfitto, per poi
immettere quel bottino nella circolazione, sia direttamente che
attraverso la creazione di ulteriori coniazioni.

Tuttavia, un'analisi più neutrale può mettere in luce il variare dei
livelli di fiducia all'interno della società. In tempi in cui i registri
sociali sono affidabili e l'offerta e la domanda di beni e servizi
relativamente stabili, il credito può espandersi più facilmente,
risultando conveniente. Al contrario, in periodi in cui i registri
sociali sono inaffidabili e l'offerta e la domanda di beni e servizi
instabili, il credito diventa rischioso e soggetto a insolvenza o
svalutazione, mentre i metalli preziosi mantengono la loro scarsità e
desiderabilità, potendo così essere scelti come mezzo di scambio
preferito e riserva di valore.

\textbf{Una teoria unificata del denaro}

Piuttosto che aderire esclusivamente alla teoria-merce del denaro o a
quella del credito, una teoria più completa deve individuare la logica
sottostante che accomuna entrambe. E ciò che condividono è che
rappresentano modi per tenere un registro, sebbene con differenti
soggetti a gestirlo.

Nella teoria del denaro basata sul credito, gli esseri umani tengono
registri utilizzando metodi che si fondano sulla fiducia. In piccoli
gruppi ciò può essere fatto in maniera informale, basandosi su legami di
parentela, amicizie e relazioni d'onore. In gruppi numerosi, che
coinvolgono sconosciuti, i registri basati sul credito vengono mantenuti
da uno stato amministrativo centralizzato e dal rispetto della legge, e
sono storicamente soggetti a vari reset e svalutazioni quando
inevitabilmente sorgono problemi o squilibri.

Nella teoria del denaro basata sulle merci, gli esseri umani impiegano
un metodo che minimizza la fiducia, lasciando che la natura e le sue
leggi fisiche mantengano il registro. Lo scambio fisico di merci
altamente commerciabili è ciò che regola immediatamente il registro tra
entità non fidate, e lo stato completo del registro in ogni momento è
garantito dal possesso fisico. Nessuna autorità umana può svalutare il
denaro con il semplice tratto di una penna: altrimenti, dovrebbe usare
la forza per convincere le persone a cederlo, oppure impiegare risorse
per trovarne e produrne di nuove attraverso il mining.

Pertanto, la sintesi delle due teorie può essere descritta come una
``teoria dei registri del denaro'', poiché essa esprime la logica
profonda o il fondamento su cui si basano entrambe. Sia il credito
sociale flessibile che i proto-denari da collezione affondano le loro
radici nell'alba dell'umanità. Entrambi implicano che gruppi di varie
dimensioni mantengano un registro tra di loro per evitare la necessità
di soddisfare la doppia coincidenza dei desideri, per ridurre l'attrito
negli scambi benefici e per fungere da forma di risparmio liquido. La
differenza essenziale risiede nell'autorità cui viene affidato il
mantenimento del registro.

In contesti in cui la fiducia è elevata, come all'interno di un piccolo
gruppo o di uno stato centralizzato ben funzionante, le persone si
sentono a loro agio nell'utilizzare registri, sia basati sull'onore che
scritti legalmente, per effettuare pagamenti e mettere da parte
risparmi. Questi sistemi registrativi tendono a garantire elevata
convenienza ed efficienza, ma sono soggetti a un degrado nel lungo
periodo e a occasionali default o ristrutturazioni massicce. In contesti
in cui la fiducia è bassa, come tra gruppi distinti o quando i registri
fiduciari hanno recentemente fallito, le persone si affidano invece a
registri che minimizzano la fiducia, come le merci monetarie, per
pagamenti e risparmi, pur accettando una minore convenienza ed
efficienza.

Una teoria dei registri del denaro osserva che la maggior parte delle
forme di scambio viene migliorata dalla presenza di un'unità di conto
commerciabile che può essere detenuta e trasferita nel tempo e nello
spazio, e che tale unità implica l'esistenza di un registro, sia in
senso letterale che astratto. Queste unità monetarie e il registro che
le definisce si basano o sugli amministratori umani o sulle leggi
naturali per mantenere la loro stabilità nel tempo e nello spazio.

\section{Note a piè di pagina}\label{note-a-piuxe8-di-pagina-3}

\bookmarksetup{startatroot}

\chapter{\texorpdfstring{Capitolo 4: \textbf{Una teoria unificata del
denaro}}{Capitolo 4: Una teoria unificata del denaro}}\label{capitolo-4-una-teoria-unificata-del-denaro-1}

Le definizioni e le origini del denaro sono tradizionalmente inquadrate
in due grandi filoni economici, ognuno con propri sottogruppi
interpretativi.

Un filone si concentra sulla teoria del denaro come merce, mentre
l'altro abbraccia la teoria del denaro come credito. I capitoli
precedenti hanno già menzionato entrambe queste prospettive, mentre il
presente capitolo si sofferma in modo più specifico su come queste due
teorie possano essere integrate.

Nella tradizione economica occidentale, la teoria del denaro come merce
affonda le sue radici almeno fino al \emph{Politics} di Aristotele
nell'Antica Grecia; fu poi ulteriormente elaborata e diffusamente
popolarizzata nel \emph{Wealth of Nations} di Adam Smith nel 1776; ed
infine, nel corso del XIX e XX secolo, fu sviluppata in maggior
dettaglio da Carl Menger, Ludwig von Mises e altri economisti che
costituirono il nucleo portante della scuola austriaca di economia. Il
concetto fondamentale che emerge da tale teoria è che il baratto risulta
inefficiente, in quanto richiede la doppia coincidenza dei bisogni, e
per tale ragione, in una società si affaccia in maniera naturale un bene
di alta commerciabilità, immune alla svalutazione (ad esempio, oro o
argento), che assume il ruolo di unità di conto, mezzo di scambio e
riserva di valore, riducendo così le attriti nel commercio. In
\emph{Principles of Economics}, Carl Menger sosteneva:

\begin{quote}
Money is not an invention of the state. It is not the product of a
legislative act. Even the sanction of political authority is not
necessary for its existence. Certain commodities came to be money quite
naturally, as the result of economic relationships that were independent
of the power of the state.\footnote{Carl Menger, \emph{Principles of
  Economics}, 262.}
\end{quote}

La teoria creditizia della moneta ha una piena esposizione relativamente
recente, risalendo a Henry Dunning Macleod e Georg Friedrich Knapp nella
seconda metà del XIX secolo, e viene spesso proposta in contrapposizione
alla teoria delle monete merci. Già nei primi decenni del Novecento,
Alfred Mitchell-Innes articolava in modo conciso questa concezione della
moneta, la quale influenzò anche John Maynard Keynes, il quale se ne
ispirò nelle sue prescrizioni economiche.\footnote{John Maynard Keynes,
  ``A. Mitchell-Innes. \emph{What Is Money?}''} La Modern Monetary
Theory, formulata da Abba Lerner negli anni '40\footnote{Abba Lerner,
  ``Money as a Creature of the State.''} e rivitalizzata negli anni '90
da economisti come Warren Mosler, Bill Mitchell e Larry Randall Wray, ha
ulteriormente ampliato questo approccio al pensiero.\footnote{Dylan
  Matthews, ``Modern Monetary Theory, explained,'' \emph{Vox}, 16 aprile
  2019.} In seguito, i saggi dell'antropologo David Graeber sulla storia
del debito, che in generale favorivano la teoria creditizia rispetto a
quella delle monete merci, contribuirono ulteriormente alla sua
diffusione.

Il tema centrale della teoria creditizia della moneta è che il credito
costituisce il fulcro stesso della moneta, a differenza dell'idea che
siano le merci ad essere la base del valore monetario. In \emph{The
Credit Theory of Money}, il secondo dei suoi influenti saggi degli anni
1910, Alfred Mitchell-Innes riassumeva:

\begin{quote}
In sintesi, la teoria creditizia afferma che un atto di vendita e
acquisto equivale allo scambio di una merce con credito. Da questa tesi
principale deriva la sottoteoria secondo cui il valore del credito o
della moneta non dipende dal valore di un metallo o di metalli, ma dal
diritto che il creditore acquisisce al ``pagamento'', cioè alla
soddisfazione del credito, e dall'obbligo del debitore di ``salvare'' il
proprio debito, mentre contestualmente il debitore ha il diritto di
estinguersi dal proprio debito tramite la presentazione di un debito
equivalente dovuto dal creditore, il quale ha l'obbligo di accettare
tale proposta come pagamento del credito.

Questa è la tesi fondamentale, ma in pratica non è necessario che un
debitore accerti crediti nei confronti degli stessi soggetti a cui è
debitore. Poiché siamo tutti contemporaneamente acquirenti e venditori,
ciascuno di noi è, allo stesso tempo, debitore e creditore degli altri;
e grazie all'immensa efficienza degli istituti bancari a cui vendiamo i
nostri crediti, che fungono da centri di compensazione commerciale, i
debiti e i crediti dell'intera comunità vengono centralizzati e
compensati reciprocamente. In sostanza, nella pratica, un buon credito
può saldare qualsiasi debito.\footnote{Alfred Mitchell-Innes, ``The
  Credit Theory of Money,'' 152.}
\end{quote}

Ciò che complica il dibattito tra questi orientamenti e la discussione
odierna è la tendenza intrinseca a politizzare il tema. Gli economisti
che sostengono un ruolo limitato dello Stato tendono ad orientarsi verso
l'idea che la moneta emerga dal basso, in modo spontaneo; se lo Stato
dovesse intervenire nell'emissione della moneta, i suoi sostenitori
solitamente sostengono che la creazione di moneta debba essere vincolata
a una scarsità naturale, ad esempio tramite il sostegno e il riscatto di
quantitativi specifici di oro. Al contrario, gli economisti favorevoli a
un intervento statale più consistente tendono a concepire la moneta come
un prodotto che scaturisce dall'alto, al centro dell'organizzazione
sociopolitica in tutte le sue forme. Secondo questa visione, lo Stato
non dovrebbe essere limitato da vincoli di scarsità naturale, ma godere
di ampie flessibilità nell'offerta della moneta emessa, al fine di
perseguire i propri obiettivi.

Come discusso nel primo capitolo di questo libro dedicato alle società
di cacciatori-raccoglitori, sia il concetto di proto-monete merce che
quello del credito affondano le loro radici nelle interazioni umane più
antiche e basilari. Sia la teoria delle monete merci sia quella
creditizia contribuiscono a fornire una visione olistica della
definizione e delle origini della moneta a un livello fondamentale.
Confrontando queste diverse linee di ragionamento, possiamo definire la
moneta nella maniera più ampia e precisa possibile.

\textbf{Setting the Timelines Straight}

Per riconciliare queste due teorie contrastanti sulla moneta e
individuare il fondamento comune, occorre iniziare esaminando le loro
differenze, valutando con attenzione quali aspetti ciascuna teoria
afferma correttamente e dove, invece, si verificano errori.

A tal proposito, possiamo ritornare a \emph{Wealth of Nations}, in cui
Adam~Smith illustra il problema del baratto e spiega perché, per
superare tale difficoltà, si sviluppa naturalmente una forma di
moneta--merce con elevata commerciabilità:

\begin{quote}
Ma quando, per la prima volta, ebbe luogo la divisione del lavoro, il
potere di scambiare doveva essere spesso ostacolato e complicato nel suo
funzionamento. Supponiamo che un individuo possieda un eccesso di una
certa merce rispetto alle proprie necessità, mentre un altro ne abbia
meno di quanto occorra. Il primo, dunque, sarebbe incline a liberarsene,
e il secondo desidererebbe acquistarne una parte. Tuttavia, se questo
ultimo non possedesse nulla che soddisfi il fabbisogno del primo, lo
scambio non potrebbe aver luogo. Ad esempio, il macellaio potrebbe avere
in negozio una quantità di carne superiore al proprio consumo, mentre il
mastro birraio e il fornaio sarebbero disposti ad acquistare una
porzione di tale eccedenza, pur non avendo nulla da offrire in cambio,
se non i prodotti delle rispettive lavorazioni; e il macellaio, essendo
già provvisto di tutto il pane e la birra di cui ha immediata necessità,
non potrebbe fare da intermediario. Così, egli non si configura né come
mercante né come cliente per loro, rendendoli reciprocamente meno
proficui gli uni per gli altri. Per evitare tali situazioni scomode,
ogni individuo prudente, in ogni epoca della società, subito dopo
l'istituzione della divisione del lavoro, ha naturalmente cercato di
organizzare le proprie faccende in modo da possedere, oltre ai prodotti
della propria industria, una certa quantità di una merce ritenuta quasi
universalmente accettabile come mezzo d'oro scambio. È molto probabile
che, a seguito l'una dell'altra, siano state pensate e impiegate diverse
merci a questo scopo. Nei primi stadi della civiltà, si narra che il
bestiame fosse lo strumento comune del commercio; e, sebbene tale scelta
risultasse alquanto scomoda, è noto che in antichità le cose venivano
spesso valutate in base al numero di animali scambiati per esse.
L'armatura di Diomede, afferma Omero, costava appena nove buoi, mentre
quella di Glaucus arrivava a costare cento buoi. Feltramente il sale
veniva impiegato come mezzo d'oro scambio in Abissinia; in alcune aree
costiere dell'India si utilizzavano particolari tipologie di conchiglie;
a Terranova il merluzzo essiccato, in Virginia il tabacco, in alcune
colonie delle Indie Occidentali lo zucchero, in altri Paesi le pelli o
il cuoio conciato; e ancor oggi si racconta di un villaggio scozzese
dove non è raro, secondo le voci, che un artigiano si rechi dal fornaio
o all'osteria portando chiodi anziché denaro.
\end{quote}

\begin{quote}
In ogni paese, tuttavia, gli uomini sembrano essere giunti, per ragioni
irresistibili, alla decisione di privilegiare, per questo impiego, i
metalli rispetto a qualsiasi altra merce. I metalli non solo possono
essere conservati con perdite minime -- pochissima cosa risulta meno
deperibile -- ma possono anche essere, senza subire alcuna perdita,
suddivisi in un numero indefinito di parti, in quanto queste, grazie al
processo di fusione, possono essere facilmente riunite; una qualità che
nessun'altra merce altrettanto durevole possiede e che, più di ogni
altra, li rende perfetti come strumenti per il commercio e la
circolazione.\footnote{Adam Smith, \emph{Wealth of Nations}, Libro I,
  Capitolo IV, 37--39.}
\end{quote}

Guardando a questo passaggio secoli dopo, l'analisi complessiva di Smith
appare rimasta valida e molti dei suoi ragionamenti si sono dimostrati
corretti. Innanzitutto, egli descrisse la moneta--merce come un fenomeno
che emerge spontaneamente per risolvere il problema della doppia
coincidenza dei bisogni in un mondo caratterizzato dalla
specializzazione della mano d'opera. In secondo luogo, spiegò perché i
metalli preziosi, grazie alle loro proprietà uniche, hanno resistito nel
tempo in tale funzione meglio di altre merci. In altre parole, non fu un
caso che culture di ogni parte del mondo orientassero le proprie scelte
verso oro e argento, considerate le forme ideali di moneta--merce.

Curiosamente, l'emergere spontaneo, dal basso, degli anelli SoJ come
forma monetaria in \emph{Diablo II}, descritto nel Capitolo~2, ha
offerto un esperimento osservabile a conferma della proposta di Smith
sull'origine della moneta. Insieme ad altri esempi analoghi, ciò ha
permesso di assistere in prima persona a tale processo, anziché
limitarlo a mere teorie. Gli sviluppatori di \emph{Diablo II} crearono
un ambiente di gioco nel quale milioni di utenti, valutandolo
attentamente, si orientarono rapidamente e spontaneamente verso gli
anelli SoJ, adottandoli come forma principale di risparmio, pagamento e
unità di conto, al fine di evitare la necessità di risolvere il problema
della doppia coincidenza dei bisogni -- ovvero, il problema del baratto
-- nel momento in cui venivano scambiati oggetti di alto valore,
nonostante questa non fosse l'intenzione originaria dei creatori. Un
limite fondamentale di questo ambiente risiedeva nel fatto che il gioco
era praticato da persone provenienti da ogni parte del mondo, per lo più
sconosciute tra loro, rendendo generalmente inapplicabile il meccanismo
del credito. In un contesto di sconosciuti, dove il credito scarseggia o
è inesistente, la moneta--merce emerge infatti in maniera naturale dai
singoli partecipanti al mercato, e ciò che si impone è semplicemente ciò
che possiede le migliori qualità per essere usato come
denaro.\footnote{Vedi, ad esempio, Narayana Kocherlakota, ``Money is
  Memory''; Stefano Ugolini, \emph{The Evolution of Central Banking:
  Theory and History}, 165--175.}

Tuttavia, evidenze antropologiche raccolte dopo l'epoca di Smith
indicano che egli probabilmente ebbe un'importante inversione di
paradigma: il baratto fra lavoratori specializzati non ha avuto origine
in precedenza al denaro, come lui aveva ipotizzato. Non vi fu un'epoca
in cui artigiani come macellai, birrai e panettieri si trovassero
comunemente nell'imbarazzo di uno scambio ``intasato'' per via
dell'assenza del concetto di denaro. Già, in ogni sua forma, il denaro
era stato concepito quando cominciarono a emergere tali forme di lavoro
specializzato. In particolare, un sistema di credito sociale flessibile
aveva in parte eluso l'esigenza di soddisfare la doppia coincidenza dei
desideri tra persone che si conoscevano. Il concetto stesso di credito,
infatti, precede l'introduzione della moneta fisica e si confronta, in
termini temporali, con le prime proto-monete da collezione, come nel
caso delle conchiglie.\footnote{David Graeber, \emph{Debt: The First
  5,000 Years}; George Selgin, ``The Myth of the Myth of Barter.''}

In senso ancora più ampio, l'idea di debito e credito si estende ancor
prima della nostra specie. Quando una scimmia libera insetti dalla
schiena di un'altra e, in seguito, inverte i ruoli per ricambiare il
favore, si assiste a una forma estremamente transitoria di debito e
credito. Naturalmente, la scimmia che beneficia del gesto potrebbe non
sentirsi obbligata a ricambiare. Gli esseri umani, al contrario, hanno
notevolmente ampliato la complessità e la durata dei rapporti di credito
e debito all'interno delle loro strutture sociali.

Alfred Mitchell-Innes, nel suo saggio del 1913 \emph{What is Money?},
ribalta l'ordine degli eventi proposto da Smith nel modo seguente:

\begin{quote}
La posizione di Adam Smith si fonda sulla veridicità dell'affermazione
che, se il panettiere o il birraio desidera della carne dal macellaio,
ma non ha (perché quest'ultimo è già abbondantemente provvisto di pane e
birra) nulla da offrire in cambio, allora non si può realizzare alcuno
scambio fra loro. Se ciò fosse vero, la dottrina di un mezzo di scambio
potrebbe, forse, avere senso. Ma è davvero così?
\end{quote}

\begin{quote}
Supponendo che panettiere e birraio siano uomini onesti --- e l'onestà
non è una virtù dell'epoca moderna --- il macellaio potrebbe ottenere da
loro il riconoscimento di avergli comprato una certa quantità di carne,
e tutto ciò che dobbiamo ipotizzare è che la comunità riconosca
l'obbligo del panettiere e del birraio a riscuotere questi
riconoscimenti in cambio di pane o birra, alle valute relative vigenti
nel mercato del villaggio, ogniqualvolta vengano presentati. In tal modo
si ottiene immediatamente una valuta buona e sufficiente. Secondo questa
teoria, una vendita non consiste nello scambiare una merce con un
qualche intermediario, il cosiddetto ``mezzo di scambio'', ma
semplicemente nello scambiare una merce con un credito.
\end{quote}

\begin{quote}
Non sussiste alcuna ragione per ipotizzare l'esistenza di un dispositivo
così goffo come un mezzo di scambio quando un sistema tanto semplice
sarebbe in grado di soddisfare appieno le necessità. Ciò che dobbiamo
provare non è l'esistenza di un assurdo accordo generale finalizzato ad
accettare oro e argento, ma piuttosto un senso diffuso della sacralità
di un'obbligazione. In altre parole, la teoria attuale si basa
sull'antichità della legge del debito.
\end{quote}

\begin{quote}
Fortunatamente, ci troviamo su basi storiche solide. Dai primordi dei
documenti storici in nostro possesso, si evidenzia la presenza di una
legge del debito, e quando, ne sono certo, troveremo testimonianze
antecedenti ai tempi del grande re Hammurabi, che compilò il Codice di
leggi della Babilonia intorno al 2000 a.C., non dubiterò che emergeranno
ancora tracce della medesima legge. La sacralità dell'obbligazione
costituisce, in effetti, il fondamento di tutte le società non solo in
epoche diverse, ma in ogni stadio della civiltà; l'idea, pertanto, che
tra quei popoli che siamo soliti definire selvaggi il credito fosse
ignoto e si praticasse esclusivamente il baratto, non trova alcun
fondamento. Dal mercante cinese al nativo americano, dall'arabo del
deserto all'Hottentoto del Sudafrica o al Maori della Nuova Zelanda,
debiti e crediti sono altrettanto familiari a tutti, e il tradire la
parola data o rifiutarsi di adempiere a un'obbligazione è considerato
altrettanto grave.\footnote{Alfred Mitchell-Innes, \emph{What is
  Money?}, 391.}
\end{quote}

In sintesi, Mitchell-Innes sostiene, basandosi su evidenze
antropologiche, che il credito sociale flessibile aveva già in parte
risolto il problema del baratto e ridotto gli attriti nelle transazioni
tra vicini già nella fase di cacciatori-raccoglitori, ben prima che il
lavoro specializzato raggiungesse il livello descritto da Smith.

Il birraio e il fornaio possono procurarsi della carne facendo una
promessa di pagamento posticipato, probabilmente utilizzando come
compenso i prodotti che essi stessi realizzano (ad esempio: ``grazie per
la carne, ecco un credito per cinque pagnotte di pane, che potrai cedere
a chi desideri; io riconoscerò il credito a chi lo riscuoterà'').
Analogamente, il macellaio può ottenere birra e pane impegnandosi a
saldare il debito in un secondo momento, e molti di questi scambi
possono essere compensati reciprocamente. Così, un macellaio che
possiede un credito per del pane conseguente alla vendita di carne al
fornaio potrebbe agevolmente cedere tale credito a terzi. L'uso del
credito risolve il problema del baratto tra soggetti che godono di una
certa continuità e fiducia reciproca, oppure che si affidano a
un'autorità locale in grado di far rispettare gli impegni creditizi.

Al contrario, la moneta merci emerge ripetutamente soprattutto per
ridurre gli attriti nel commercio tra sconosciuti o per rafforzare e
migliorare sistemi fondati sul credito sociale flessibile, fungendo da
strumento di regolamento definitivo e di risparmio a lungo termine.
All'interno dello stesso gruppo sociale, le persone possono cavarsela --
sino a un certo livello di complessità -- usando piccoli registri
informali gestiti direttamente dagli individui, mentre gli estranei
traggono vantaggio dalla capacità di effettuare pagamenti finali
immediati. Ed è proprio in questa circostanza che la moneta merci
diventa necessaria per integrare o sostituire il credito sociale
flessibile.\footnote{Ugolini, \emph{Central Banking}, 169--171.}

Se osserviamo l'antica Babilonia ai tempi di Hammurabi -- uno dei primi
esempi noti --, essa impiegava il grano e l'argento come moneta merci,
avvalendosi altresì di registri in argilla per mantenere il concetto di
credito. Il grano, essendo una merce fortemente stagionale, consentiva a
un agricoltore di vivere a credito per acquistare vari beni fino al
momento del raccolto, quando -- si spera -- poteva estinguere i propri
debiti vendendo l'intera produzione. Inoltre, lo studio delle società
dei cacciatori-raccoglitori e le evidenze archeologiche testimoniano
l'ampio uso di sistemi informali di credito basati sull'onore e sui
legami familiari, nonché l'impiego di oggetti da collezione, come le
perle di conchiglia, usate come forme proto-moneta, il che indica che
diverse forme di denaro -- sia a credito che merci -- esistevano ben
prima dell'avvento di una marcata specializzazione del lavoro.

\textbf{Where the Credit Theory of Money Goes Wrong}

I sostenitori della teoria creditizia del denaro hanno ragione nel
riesaminare l'evidenza antropologica sul concetto di credito e nel
mettere in discussione le prime affermazioni della teoria della moneta
merci, in relazione all'ordine cronologico degli eventi da loro
proposto. Il credito, infatti, rappresenta quasi la radice stessa del
commercio e della nascita del denaro, piuttosto che essere una scoperta
successiva; questa osservazione costituisce una correzione utile a
\emph{The Wealth of Nations}. Alla base dell'interazione umana vi è, in
effetti, una serie di accrediti e addebiti -- formali o informali --
verso gli altri, regolati da riti e norme che trovano fondamento nei
nostri istinti evolutivi, nelle prime religioni e nelle forme ancestrali
di governo.

Tuttavia, i fautori della teoria creditizia del denaro tendono a
spingersi oltre, ignorando spesso l'importanza fondamentale della
moneta-merce e dipingendo un quadro eccessivamente ottimistico della
capacità di una società di gestire, su vasta scala e per lunghi periodi,
un sistema contabile flessibile basato sul credito.

Come primo punto di critica, occorre evidenziare le difficoltà che
incontrerebbero i mercanti di nicchia in un sistema fondato
esclusivamente sul credito. Il macellaio, il birraio e il fornaio
potrebbero, ad esempio, distribuire crediti in unità scambiabili in modo
agevole---una libbra di carne potrebbe essere equiparata a tre pinte di
birra e a cinque pagnotte di pane---ma come si comporterebbero invece
operatori di servizi di alto valore e specializzati, come i chirurghi?
Se una chirurga acquista del pane, le verrebbe riconosciuto un credito
utilizzabile in futuro per usufruire delle sue prestazioni chirurgiche?
Quante pagnotte di pane equivarrebbero a un'operazione e, ancor più, di
che tipologia si tratterebbe? Le prestazioni chirurgiche, essendo
servizi preziosi e poco fungibili, necessitano di una misura di conto
più ridotta per rendere lo scambio agevole. Storicamente, infatti, tale
unità era spesso legata a una merce specifica---alcuni grammi d'argento
o la quantità di cereali necessaria per un pasto---altrimenti si
rischierebbe di cadere in un baratto astratto in cui crediti di vario
genere venissero scambiati senza una misura standardizzata. Anche quando
il credito costituiva lo strumento principale di scambio, esso veniva
solitamente espresso in unità derivanti naturalmente da una merce
commerciabile, grazie alle sue proprietà intrinseche.

Come secondo punto di critica, si può domandare: cosa accade se qualcuno
abbandona una comunità per unirsi a un'altra? Raramente esiste una
comunità veramente chiusa, in quanto sin dagli albori dell'umanità le
comunità hanno interagito tra loro. Affinché la ricchezza possa essere
trasferita da un gruppo all'altro, essa deve assumere una forma più
tangibile o universale. Il credito può funzionare efficacemente per gli
scambi quotidiani all'interno di una comunità, ma per chi desidera
allargare i propri orizzonti è indispensabile una forma di ricchezza
primordiale e riconosciuta universalmente.\footnote{Brian Albrecht e
  Andrew Young, ``Wampum: The Political Economy of an Institutional
  Tragedy''; Lawrence White, \emph{Better Money: Gold, Fiat, or
  Bitcoin?}, 17.} Le monete naturali, pertanto, fungono da ponte
indispensabile tra ecosistemi di credito altrimenti chiusi e circolari.

Come terzo punto di critica, appare evidente che ciò che risulta
efficace su scala ridotta non necessariamente si adatta a contesti più
ampi. I piccoli sistemi di credito sociale, basati sull'onore e su
rapporti personali tra individui noti, non possono essere trasposti in
maniera identica alla gestione di Stati nazionali, che richiedono
l'amministrazione di milioni di persone per lo più sconosciute tra loro.
Il concetto di fiducia funziona soltanto se si conosce e si ripone
fiducia nell'interlocutore, mentre l'onore, pur costituendo un elemento
cruciale nelle interazioni umane, tende a perdere efficacia in una
burocrazia impersonale.\footnote{Seabright, \emph{Company of Strangers},
  86; Avner Greif, ``The Fundamental Problem of Exchange,'' 261--262.}

Mitchell-Innes sosteneva, nel suo saggio \emph{Credit Theory} del 1914,
che il nostro denaro si apprezzerebbe se venisse scollegato dall'oro:

Immaginiamo che, mantenendo l'oro a un prezzo fisso, stiamo preservando
il valore della nostra unità monetaria, quando in realtà stiamo facendo
esattamente il contrario. Più a lungo insistiamo nel mantenere l'oro al
prezzo attuale, pur essendo il metallo ancora abbondante come oggi,
maggiore è la svalutazione del nostro denaro.\footnote{Mitchell-Innes,
  ``Credit Theory,'' 160.}

Naturalmente, si è verificato esattamente il rotto opposto. Al momento
della stesura di questo testo, il dollaro statunitense e la sterlina
britannica hanno perso, rispettivamente, oltre il 98\% e il 99\% del
loro tasso di cambio rispetto all'oro, da quando sono state desvinculate
da esso nei decenni successivi al saggio di Mitchell-Innes. Per la
maggior parte dei Paesi la caduta è stata ancor più marcata, e la
svalutazione della moneta rispetto all'oro ha interessato ogni nazione
che abbia mai introdotto una valuta fiat.

Nonostante questo evidente insuccesso previsivo, il ragionamento da cui
si ispirò Mitchell-Innes nel 1914 per sostenere la sua tesi non
appariva, a prima vista, del tutto infondato. Egli sosteneva infatti che
alla radice della decadenza storica del denaro emesso dallo Stato non vi
fosse una mera svalutazione arbitraria \emph{per se} stessa, ma fattori
quali guerra, pestilenza e altre calamità produttrici di inefficienza
che condusse alla devalorizzazione. Se solo si potesse reiterare
l'enfasi sulla pace e sull'organizzazione, concluse, il denaro statale
sarebbe riuscito a resistere alla svalutazione:

\begin{quote}
Non sono stati re Giovanni, re Filippo, Edward o Enrico a svalutare il
denaro, bensì la Guerra, il grande generatore di debiti, affiancata dai
suoi luogotenenti -- la pestilenza, la morbosità del bestiame e i
raccolti rovinati -- insomma, da tutto ciò che impedisce il puntuale
adempimento dei debiti. Non sono gli atti di recoinage a ripristinare il
valore del denaro, ma la Pace, il grande creatore di crediti, e su
questa ineludibile verità deve fondarsi in gran parte la credit theory
of money.\footnote{Mitchell-Innes, ``Credit theory,'' 157.}
\end{quote}

Su questo punto, Mitchell-Innes aveva in gran parte ragione. Salvo casi
eccezionali di sovrani particolarmente corrotti o afflitti da gravi
disturbi mentali, un re non si sveglia giorno per giorno e decide
capricciosamente di svalutare il conio del proprio regno con
l'introduzione di metalli meno pregiati senza una ragione valida.
Guerra, pestilenze e altre catastrofi che intaccano la produttività
sono, infatti, le cause principali per cui i sovrani, nel tentativo di
rafforzare la propria posizione, conciliare gli interessi della
popolazione e gestire le inevitabili emergenze del regno, ricorrono alla
svalutazione della moneta. Questo strumento consente loro di
incrementare i pagamenti senza dover aumentare le tasse, spostando di
fatto, nel tempo, l'onere sui soggetti che continuano ad accettare le
monete svalutate al vecchio valore nominale, nonostante esse abbiano
perso il corrispettivo contenuto metallico e la scarsità che le
caratterizzava una volta.

Tuttavia, ciò che secondo me Mitchell-Innes ha trascurato è che la
possibilità di svalutare la moneta \emph{contribuisce} ad aumentare la
probabilità che scoppino guerre e si verifichino altre forme di
inefficienza produttiva sin dall'inizio. La tentazione per un sovrano di
svalutare la moneta è così forte da risultare quasi inevitabile, poiché
rappresenta generalmente la via di minor resistenza quando si è di
fronte a un problema. Se un re sa che finanziare una guerra aumentando
direttamente le tasse potrebbe facilmente scatenare una rivolta, mentre
optare per una svalutazione graduale della moneta permette di evitare un
immediato malcontento popolare, potrà legittimarsi nel pagare il
conflitto mediante quest'ultima via. Se invece lui e il potenziale
avversario in guerra fossero costretti entrambi a pagare il conflitto
attraverso un oneroso prelievo fiscale aggiuntivo anziché con la
svalutazione, il conflitto potrebbe non avvenire del tutto, poiché i
sudditi, di fronte a costi immediatamente trasparenti e impopolari,
potrebbero ribellarsi. Al contrario, la possibilità di svalutare la
moneta per finanziare una guerra permette che il conflitto abbia inizio
e che i costi vengano parzialmente rinviati nel tempo, il che incrementa
sia la probabilità che la guerra abbia luogo sia la sua eventuale
portata. In sostanza, se la svalutazione \emph{può} verificarsi, lo
farà, per innumerevoli ragioni. Tale possibilità esiste in ogni momento
e in ogni luogo, invitando i governi a ricorrervi ogni volta che non
possono finanziare apertamente le proprie spese.

In definitiva, nel lungo arco temporale, dal punto di vista
dell'investitore risparmiatore risulterà quasi sempre preferibile
detenere direttamente una moneta basata su una merce scarsa, piuttosto
che affidarsi indefinitamente alla promessa di un regno, di un impero o
di uno stato-nazione. Il primo caso è vincolato alle leggi immutabili
della natura, mentre il secondo subisce inevitabilmente le debolezze e
la fallibilità dell'essere umano.

Possiamo individuare più chiaramente l'errore nel ragionamento di
Mitchell-Innes osservando come egli descriva il credito come la forma di
proprietà più preziosa nel suo saggio ``What is Money?'':

\begin{quote}
A first class credit is the most valuable kind of property. Having no
corporeal existence, it has no weight and takes no room. It can easily
be transferred, often without any formality whatever. It is movable at
will from place to place by a simple order with nothing but the cost of
a letter or a telegram. It can be immediately used to supply any
material want, and it can be guarded against destruction and theft at
little expense. It is the most easily handled of all forms of property
and is one of the most permanent. It lives with the debtor and shares
his fortunes, and when he dies, it passes to the heirs of his estate. As
long as the estate exists, the obligation continues, and under favorable
circumstances and in a healthy state of commerce there seems to be no
reason why it should ever suffer deterioration.\footnote{Mitchell-Innes,
  ``What is money?'' 392.}
\end{quote}

Il problema sta nell'assumere che ``in circostanze favorevoli e in uno
stato di commercio sano'' persista per tutta la vita, per non parlare di
generazioni. Nel corso dell'esistenza personale e della governance,
infatti, insorgono inevitabilmente problematiche e vari debiti vengono
inevitabilmente svalutati, estinti o vengono oggetto di inadempienza.
Nel secolo scorso, in cui le valute hanno trascorso la maggior parte del
tempo disaccoppiate dalla naturale scarsità dei metalli preziosi, un
\emph{first class credit} si è rivelato essere uno degli asset peggiori
da detenere rispetto alle alternative. In decine di paesi nel mondo, a
partire dai saggi di Mitchell-Innes, i crediti e le relative valute
sottostanti sono stati letteralmente sterminati dall'iperinflazione. Nei
paesi più riusciti -- quelli schierati dalla parte vincente in tutti i
grandi conflitti e dotati di istituzioni finanziarie robuste -- i
crediti di prima categoria hanno generalmente evitato l'incoronarsi
all'iperinflazione, ma hanno comunque registrato performance
notevolmente inferiori rispetto al patrimonio immobiliare, al capitale
d'impresa, ai metalli preziosi, alle opere d'arte di pregio e ai vini
pregiati.\footnote{Òscar Jordà et al., ``The rate of return on
  everything, 1780--2015.''}

In altre parole, i sostenitori della teoria del credito del denaro,
applicando la loro analisi a un organismo governativo sufficientemente
esteso, fanno generalmente affidamento sull'assunzione di disporre di
una catena ininterrotta di amministratori del registro pubblico
altamente competenti e altruisti. Tale ipotesi, tuttavia, si è rivelata
ripetutamente inefficace, cultura dopo cultura e secolo dopo secolo.
Trascurando l'importanza dei metalli preziosi o di qualsiasi vincolo
naturale come strumento necessario per garantire la disciplina del
registro pubblico, si perde di vista un aspetto fondamentale che ha
permesso alle monete merce di resistere alla prova del tempo per
migliaia di anni: nessuno può, in un attimo, crearne ulteriori, anche
quando sembrerebbe giustificato farlo. Inoltre, esse rappresentano una
liquidazione definitiva anziché una dipendenza perpetua dalle promesse
di entità centralizzate.

È interessante notare che, pur sostenendo varie tesi sul denaro,
Mitchell-Innes era ben consapevole del fatto che, in tutta la storia
finanziaria, le unità monetarie definite dall'uomo tendono
strutturalmente a deprezzarsi e non sembrano mai apprezzarsi in maniera
intrinseca. Come scrive in \emph{The Credit Theory of Money}:

\begin{quote}
Ma sebbene l'unità monetaria possa deprezzarsi, non pare mai
apprezzarsi. Un generale aumento dei prezzi -- a tratti rapido e a
tratti lento -- è la caratteristica comune di tutta la storia
finanziaria; e se un'impennata viene seguita da un ribasso, quest'ultimo
appare semplicemente come un ritorno a uno stato di equilibrio. Dubito
che esistano casi in cui il calo porti a un prezzo inferiore a quello
che regnava prima del rialzo, e nulla che assomigli a una diminuzione
persistente dei prezzi, indicativa di un continuo apprezzamento del
valore del denaro, sembra essere noto.\footnote{Mitchell-Innes, ``Credit
  theory,'' 159.}
\end{quote}

In questo modo, possiamo paragonare i registri centralizzati controllati
dall'uomo alla seconda legge della termodinamica, la quale stabilisce
che l'entropia (termine che indica essenzialmente il ``disordine'') di
ogni sistema chiuso può solo aumentare con il tempo, senza possibilità
di diminuzione. Nulla, se non un sistema perfettamente efficiente, privo
di attriti e di dispersione di calore -- condizione inaccessibile nella
realtà -- sarebbe in grado di arginare la crescita continua
dell'entropia. Analogamente, nulla se non una catena ininterrotta di
governanti perfetti potrebbe mantenere un sistema monetario flessibile
senza incorrere nel deterioramento del suo valore; una tale catena
ideale, però, non esiste. Di conseguenza, in ogni ambito insorgono
inevitabilmente problemi e, ancora una volta, le autorità tendono a
ricorrere alla creazione di ulteriore moneta per alleviarli e svalutare,
in maniera poco trasparente, vari debiti.

David Graeber, il quale per lo più può essere inquadrato nel campo della
teoria del credito, ha osservato la relazione tra il livello di fiducia
sociale e il tipo di denaro in uso nel suo libro \emph{Debt: The First
5000 Years}:

Di conseguenza, mentre i sistemi di credito tendono a predominare in
periodi di relativa pace sociale o all'interno di reti di fiducia --
siano esse create dagli Stati o, nella maggior parte dei periodi, da
istituzioni transnazionali quali le corporazioni mercantili o le
comunità di fede -- in tempi segnati da guerre diffuse e saccheggi
vengono solitamente sostituiti da metalli preziosi.\footnote{Graeber,
  \emph{Debt}, 215.}

Nel suo libro, Graeber si esprimeva spesso in termini critici nei
confronti dei metalli preziosi, sostenendo, ad esempio, che uno dei
motivi del loro impiego durante i periodi bellici fosse la loro
abbondanza, conseguenza dei saccheggi. Infatti, i soldati della parte
vincente abbatterebbero le casseforti e i templi della parte sconfitta,
saccheggiando ogni sorta di riserva e ornamento in metallo prezioso, per
poi immettere quel bottino in larga circolazione, direttamente o
attraverso l'emissione di nuova moneta.

Una lettura più neutrale, invece, può evidenziare come il livello di
fiducia presente nella società vari in base al contesto. In periodi in
cui i registri sociali sono affidabili e l'offerta e la domanda di beni
e servizi risultano relativamente stabili, il credito può espandersi
agevolmente, essendo pratico e conveniente. Al contrario, in tempi
caratterizzati da registri sociali inaffidabili e fluttuazioni
nell'offerta e nella domanda, il credito diventa rischioso, soggetto a
insolvenze o svalutazioni, mentre i metalli preziosi, mantenendosi
scarni e desiderabili, vengono scelti come mezzo di scambio e come
deposito di valore preferito.

\textbf{A Unified Theory of Money}

Invece di aderire in maniera esclusiva alla teoria della moneta-merce o
a quella del credito, una teoria più completa deve individuare il
principio logico comune a entrambe. Ciò che queste due concezioni
condividono è proprio il fatto di costituire modalità alternative per
mantenere un registro contabile, sebbene con gestori differenti.

Nella teoria creditizia del denaro, gli esseri umani si affidano alla
fiducia per gestire i propri registri contabili. In piccole comunità ciò
può avvenire in maniera informale, basandosi su legami di parentela,
amicizia e rapporti d'onore, mentre in gruppi numerosi e composti da
sconosciuti i registri basati sul credito sono tenuti da un apparato
statale centralizzato e regolati dallo stato di diritto. Tali sistemi,
tuttavia, sono storicamente soggetti a periodici reset e svalutazioni
quando, in modo inevitabile, si verificano problemi o squilibri.

Al contrario, secondo la teoria delle merci del denaro, gli esseri umani
affidano il mantenimento del registro contabile alle leggi fisiche della
natura, riducendo al minimo il bisogno di fiducia. In questo caso, lo
scambio fisico di merci altamente commerciabili compie la funzione di
``regolamento'' immediato del registro, garantendo, di fatto, che lo
stato complessivo del registro in ogni momento sia determinato
attraverso il possesso materiale. Nessuna autorità umana, infatti, può
svalutare il denaro con una semplice firma: per ottenere ciò sarebbe
necessario ricorrere alla forza per costringere la cessione oppure
investire risorse nel ritrovare e produrre ulteriori unità mediante il
processo di mining.

Quindi, l'integrazione di entrambe le teorie si può definire una
``teoria dei registri del denaro'', in quanto essa svela la logica
profonda e le fondamenta su cui poggiano le due concezioni. Sia il
credito flessibile che i proto-denari da collezione affondano le loro
radici nell'alba dell'umanità: in ogni caso, gruppi di dimensioni varie
si sono sempre organizzati per tenere un registro condiviso, evitando la
doppia coincidenza dei desideri, riducendo gli attriti negli scambi
vantaggiosi e fungendo da riserva di valore liquida. La differenza
fondamentale risiede nell'autorità o nell'entità di fiducia cui viene
delegato il compito di mantenere il registro.

In contesti caratterizzati da alta fiducia --- come all'interno di
piccoli gruppi o in uno stato centralizzato efficiente --- le persone si
sentono a loro agio nell'utilizzare registri basati sull'onore o sulla
scrittura giuridica per i pagamenti e i risparmi. Tali sistemi offrono
elevata convenienza ed efficienza, pur essendo soggetti a degrado nel
lungo periodo e a episodiche crisi o ristrutturazioni massicce. Al
contrario, in contesti dove la fiducia è scarsa, ad esempio tra gruppi
distinti o quando i registri ``di fiducia'' hanno recentemente fallito,
si ricorre a registri in cui la fiducia è minimizzata, come nel caso
delle merci usate come denaro, anche se questo accade a costo di minore
comodità ed efficienza.

Una teoria dei registri del denaro osserva che la maggior parte delle
forme di scambio trae vantaggio dall'avere un'unità di conto
commerciabile che possa essere detenuta e trasferita nel tempo e nello
spazio. Tale unità implica inevitabilmente l'esistenza di un registro,
sia esso preso in senso letterale o astratto. Queste unità monetarie e
il registro che le definisce fanno affidamento, rispettivamente, su
amministratori umani o sulle leggi naturali, per garantirne la stabilità
nel corso del tempo e nelle diverse località.

Mitchell-Innes, in ``Credit Theory'', analizza in modo approfondito i
meccanismi che regolano il credito e il suo impatto sulle dinamiche
economiche, come evidenziato a pagina 157 \footnote{Mitchell-Innes,
  ``Credit theory,'' 157.}. Nel saggio ``What is Money?'' dello stesso
autore, viene esaminata con rigore la natura della moneta e le sue
funzioni fondamentali, con una trattazione approfondita che si conclude
a pagina 392 \footnote{Mitchell-Innes, ``What is money?'' 392.}. In uno
studio empirico di Òscar Jordà e collaboratori, intitolato ``The Rate of
Return on Everything, 1780--2015'', si viene a conoscere l'evoluzione
storica dei rendimenti sugli investimenti, offrendo uno spaccato
illuminante dell'andamento dei mercati finanziari nel lungo periodo
\footnote{Òscar Jordà et al., ``The rate of return on everything,
  1780--2015.''}. Ulteriori argomentazioni su questi temi, che
illustrano criticamente le teorie sul credito e le loro implicazioni
economiche, sono descritte ancora una volta da Mitchell-Innes in
``Credit Theory'' a pagina 159 \footnote{Mitchell-Innes, ``Credit
  theory,'' 159.}. Infine, Graeber, in \emph{Debt}, propone una visione
interdisciplinare del debito, in cui vengono messi in luce gli aspetti
sociali, filosofici e storici che ne determinano il ruolo nelle società
contemporanee, come riportato a pagina 215 \footnote{Graeber,
  \emph{Debt}, 215.}.

\bookmarksetup{startatroot}

\chapter{Part 2}\label{part-2}

\textbf{La nascita delle banche}

\emph{``Lo scopo fondamentale del commercio consiste nell'accumulare
crediti. Il banchiere è colui che centralizza i debiti dell'umanità e li
bilancia l'uno con l'altro, per cui le banche assumono il ruolo di
camere di compensazione del commercio.''}\footnote{Mitchell-Innes,
  \emph{Credit Theory}, 168.}

-Alfred Mitchell-Innes

\section{Note a piè di pagina}\label{note-a-piuxe8-di-pagina-4}

\bookmarksetup{startatroot}

\chapter{\texorpdfstring{Capitolo 5: \textbf{Proto-Banking e il Sistema
Hawala}}{Capitolo 5: Proto-Banking e il Sistema Hawala}}\label{capitolo-5-proto-banking-e-il-sistema-hawala}

Gli istituti bancari nella forma in cui li conosciamo oggi ebbero
origine nelle città-stato italiane all'alba del Rinascimento, ma la
storia delle proto-banche risale a migliaia di anni fa, interessando
diverse aree del mondo. In senso lato, l'attività bancaria rappresenta
una serie di passaggi legali e tecnologici che le società hanno
progressivamente sviluppato attorno al denaro merceologico.

Il Codice di Hammurabi, antico testo legislativo babilonese di quasi
quattro millenni fa, contiene diverse sezioni dedicate alle norme su
prestiti e depositi\footnote{Goetzmann, \emph{Money Changes Everything,}
  46--48.}. Il Libro del Deuteronomio, invece, autorizza l'addebito di
interessi agli stranieri, pur vietandolo tra gli stessi
israeliti\footnote{Edward Chancellor, \emph{The Price of Time: The Real
  Story of Interest}, 5--6, 17--20.}. Già nell'antica Grecia, oltre
2.500 anni fa, si trovava una forma embrionale di attività bancaria,
rappresentata dai trapeziti, così chiamati per via dei trapeza (tavoli)
che utilizzavano nelle loro transazioni\footnote{Goetzmann, \emph{Money
  Changes Everything}, pp.~82--83.}.

Un'importante evoluzione nella storia del credito formale fu
rappresentata dall'introduzione della suftaja e dei suoi vari
precursori. La suftaja era una lettera di credito adottata in Nord
Africa, in Medio Oriente e lungo la Via della Seta, impiegata da
mercanti musulmani ed ebrei almeno a partire dall'VIII secolo. Essa
nacque come strumento indispensabile per risolvere problematiche quali
la prevenzione del furto e l'incremento dell'efficienza nei
trasferimenti di denaro su lunghe distanze, nonché per delegare il
compito di trasmettere fondi tramite un incaricato\footnote{Ghislaine
  Lydon, ``Strumenti cartacei nelle prime economie africane e il
  controverso ruolo della Suftaja.''}.

Ghislaine Lydon, docente di storia con un focus sugli studi africani e
mediorientali, ha documentato la storia della suftaja nel suo saggio del
2019, \emph{Paper Instruments in Early African Economies and the Debated
Role of the Suftaja}. In questo lavoro la suftaja viene descritta come
segue:

In quanto contratto di debito, la suftaja consentiva il trasferimento di
fondi tra mercanti operanti in mercati distanti, i quali offrivano
servizi come finanziatori internazionali all'interno di reti commerciali
consolidate. Essa aveva fondamentalmente due funzioni: innanzitutto,
permetteva di inviare pagamenti o saldare debiti a lunga distanza, in un
modo analogo a un bonifico bancario; in secondo luogo, similmente a un
traveler's check, agevolava i viaggi senza il peso e l'ingombro
derivanti dal trasporto di enormi quantitativi di denaro
contante.\footnote{Lydon, ``Paper Instruments'', paragrafo 21.}

Proseguì fornendo un esempio:

Un mercante si trova a viaggiare per affari in una carovana che collega
Awdaghust a Sijilmasa. Per tutelare il proprio capitale, in Awdaghust
egli acquista una suftaja dietro pagamento di una commissione a favore
del Mercante A, il quale ha un parente e/o partner commerciale, il
Mercante B, stabilito a Sijilmasa. In via preliminare, i mercanti A e B
hanno instaurato, in forma ex ante, un rapporto di fiducia fondato sullo
scambio di servizi finanziari e commerciali a lunga distanza. Il
viaggiatore, alla presenza di testimoni, deposita il proprio capitale
presso il Mercante A, paga una commissione e riceve in cambio la
suftaja. Tale documento, spesso redatto sotto forma di lettera, contiene
l'istruzione diretta al Mercante B di corrispondere al viaggiatore
l'esatto ammontare del capitale depositato. Al suo arrivo, il
viaggiatore incassa questo assegno. Lo stesso meccanismo veniva usato
per realizzare pagamenti a distanza, con la sola differenza che la
lettera contenente le istruzioni di pagamento veniva inviata tramite un
messaggero. Una volta ricevuta, il Mercante B eseguiva il pagamento a
favore di una terza parte. In altri termini, un pagamento tramite
suftaja effettuato dal Mercante B poteva estinguere, totalmente o
parzialmente, un debito preesistente nei confronti del Mercante A.
Viceversa -- come spesso sostenuto da alcuni giuristi, di cui si parlerà
più avanti -- il mercante viaggiante, o la parte che intendeva
effettuare un pagamento internazionale, concede a Mercante A un
``prestito'' che viene successivamente rimborsato o erogato in altra
sede dal suo associato, Mercante B, sia al viaggiatore sia a una quarta
parte. Poiché i mercanti A e B si scambiano regolarmente corrispondenza
e conducono transazioni, i loro saldi finanziari vengono compensati nel
corso di tali scambi bilaterali.\footnote{Lydon, ``Paper Instruments'',
  paragrafo 22.}

Lydon, citando la letteratura disponibile in materia, risale l'origine
del termine suftaja all'VIII secolo, quando veniva impiegato in questo
modo dai mercanti musulmani ed ebrei del Medio Oriente e di alcune aree
dell'Africa. Essa fa inoltre riferimento a numerosi esempi di contratti
di debito basati su papiro dell'Antico Egitto, risalenti almeno al III
secolo a.C., che probabilmente costituirono i precursori di questo
specifico metodo di scambio. Riporta, inoltre, l'esempio di un documento
risalente al IV secolo d.C., usato in relazione a transazioni tra
Uzbekistan e Cina lungo la Via della Seta, adibito a documento di
trasferimento. Esiste, infine, una solida letteratura sullo sviluppo
degli hundi in India medievale, termine che essi utilizzavano per
indicare una cambiale.

Per quanto riguarda la denominazione comune del denaro nella suftaja,
Lydon fa riferimento ai metalli preziosi:

Il dinaro rappresentava la denominazione internazionale più comune tra i
musulmani medievali, giocando un ruolo centrale nelle pratiche suftaja.
I dinari si caratterizzavano per peculiarità sia nella coniatura sia sul
mercato, manifestando varianti locali come quelle Maghribi e Baghdadi, e
venivano ancorati al prezzo locale dell'oro.\footnote{Lydon, ``Paper
  Instruments'', paragrafo 29.}

Lo sviluppo delle cambiali basate sul papiro (e successivamente sulla
carta) si è gradualmente evoluto in una forma primitiva di attività
bancaria, conosciuta come sistema hawala, che affonda le sue radici
oltre 1.200 anni fa nei commerci degli antichi mercanti indiani e arabi.
Sebbene il sistema si sia diffuso approssimativamente in concomitanza
con l'espansione geografica e temporale dell'Islam -- dall'Africa a
parti d'Europa, attraversando il Medio Oriente e giungendo fino
all'India -- esso è stato (ed è tutt'oggi) adottato tanto da musulmani
quanto da non musulmani.

Hawala costituisce una rete decentralizzata di intermediari finanziari
specializzati, denominati hawaladars, che operano sulla base della
fiducia e della reputazione reciproca. La rete, che ancora oggi impiega
tecnologie moderne quali le e-mail e le telefonate, gestisce volumi
milionari pari a centinaia di miliardi di dollari ogni anno.

Il funzionamento attuale del sistema è il seguente: una ``Persona A'' si
presenta presso un ``Hawaladar A'' e consegna una somma di denaro
unitamente a una password, specificando che questa somma debba essere
erogata a una ``Persona B''. Contestualmente, la Persona A comunica la
password alla Persona B tramite un mezzo di comunicazione -- ad esempio
via e-mail. Successivamente, l'Hawaladar A contatta un ``Hawaladar B''
situato nel Paese in cui risiede la Persona B, trasmettendogli la
medesima password attraverso canali simili. Infine, la Persona B si reca
dall'Hawaladar B, fornisce la password e riceve il denaro. Gli hawaladar
applicano una modesta commissione per il servizio offerto. Così facendo,
la Persona A riesce a inviare denaro a livello internazionale alla
Persona B, senza che effetti fisici di valuta attraversino i confini e
senza doversi affidare a istituzioni bancarie formali. I vari hawaladar
si limitano ad aggiornare un registro basato sui canali di
comunicazione: l'Hawaladar A ora è debitore nei confronti dell'Hawaladar
B dell'importo trasferito, che potrà essere regolare in un secondo
momento. Questi intermediari si conoscono e si fidano personalmente,
anche operando su distanze notevoli, oppure instaurano relazioni di
fiducia indirette tramite altri colleghi nella rete, permettendo così di
mantenere un sistema di credito che non sarebbe replicabile tra soggetti
privi di fiducia reciproca.

In epoche antecedenti, i canali comunicativi erano di natura fisica: ad
esempio, un mercante poteva consegnare personalmente a un altro una
password o un documento scritto nel contesto dello scambio di beni
fisici, evitando così il trasporto di ingenti quantità di monete. Gli
hawaladar potevano accompagnare il mercante durante il viaggio oppure
far circolare la password mediante una rete di spostamenti di breve
durata. Oggi, tuttavia, la trasmissione delle informazioni avviene
attraverso Internet.

In questo sistema, la Persona A e la Persona B non devono riporre
fiducia l'una nell'altra, ma è indispensabile che confidino nei
hawaladar. Questi ultimi, a loro volta, devono avere fiducia reciproca;
in particolare, l'Hawaladar B deve essere certo che l'Hawaladar A sia in
grado di saldare il debito, poiché B ha anticipato il denaro a favore
della Persona B e ora si aspetta di riceverlo da A. Il funzionamento di
questo sistema, fondato sulla fiducia, risiede nel fatto che i hawaladar
sono mercanti professionisti la cui attività si sostiene grazie alla
reputazione: se uno di essi non onora una transazione valida, rischia di
perdere la fiducia dei colleghi e di essere escluso dalla rete.

I hawaladar sono in grado di regolare scambi monetari a lunga distanza
in maniera molto più sicura ed efficiente rispetto al comune cittadino.
Essi gestiscono un elevato numero di transazioni e possono aggregare gli
scambi, compensandoli. Ad esempio, in epoca medievale l'Hawaladar A
potrebbe inviare una nota del valore di dieci monete d'oro per conto
della Persona A, acquisendo così un debito di dieci monete nei confronti
dell'Hawaladar B, che aveva erogato tali fondi alla Persona B. La
settimana successiva, la Persona C si rivolge allo stesso Hawaladar B,
richiedendo il trasferimento di sei monete d'oro a favore della Persona
D, tramite l'Hawaladar A. Di conseguenza, A risulta in debito solo per
quattro monete, poiché le sei possono essere compensate con le dieci
precedenti. In un anno, potrebbero essere eseguite decine di transazioni
reciproche, per poi liquidarsi in un'unica operazione fisica di
regolamento. Lo stesso meccanismo può essere applicato utilizzando
dollari, rupie o altre unità monetarie.

Questo sistema permette il trasferimento di pagamenti su lunghe
distanze, pur comportando uno spostamento fisico della moneta con minore
frequenza. Complessivamente, i hawaladar costituiscono un registro
decentralizzato e un sistema di pagamenti a canale, accessibile agli
utenti comuni tramite il proprio hawaladar locale. Nessuno detiene o
monitora l'intero registro della rete; non esiste un hawaladar centrale
cui tutti debbano fare riferimento. Invece, il sistema funziona grazie a
singoli hawaladar che mantengono scrupolosamente i loro conti per
ciascun canale instaurato con gli altri, supportati da una reputazione
consolidata a livello regionale che li rende noti a molti colleghi.

Nel XXI secolo il sistema hawala è stato impiegato prevalentemente per i
trasferimenti internazionali di denaro ed è solitamente effettuato in
valuta fiat. Esso bypassa il sistema bancario formale, superando le
barriere delle frontiere internazionali, e offre servizi analoghi a
quelli bancari per le persone non bancarizzate. Ad esempio, un
lavoratore migrante indiano negli Emirati Arabi Uniti potrebbe
desiderare di inviare parte dei propri guadagni alla famiglia in India,
utilizzando proprio il sistema hawala.

In alcuni Paesi, tuttavia, il sistema hawala è scoraggiato o addirittura
illegale, poiché consente il trasferimento anonimo di fondi oltre i
confini. Sebbene il sistema stesso affondi le proprie radici in epoca
medievale, è stato associato in certi contesti al finanziamento del
terrorismo, dato che i terroristi tendono ad utilizzare ogni strumento a
loro disposizione. In altri stati, invece, come negli Emirati Arabi
Uniti, la pratica è consentita e regolamentata, trasformando tali Paesi
in veri e propri hub per la versione moderna di questa rete.

È stata una sfida per gli storici stabilire con precisione l'etimologia
e la sequenza esatta degli eventi relativi a termini specifici o
tecnologie impiegate nelle forme di scambio basate su carta. Una delle
ragioni invocate per questa mancanza di chiarezza è che, nel Medioevo, i
studiosi musulmani spesso dibattevano e scoraggiavano l'uso dei suftaja
e/o del più ampio sistema hawala a causa del suo impiego di debito e
arbitraggio, sebbene tale sistema fosse ampiamente diffuso nelle loro
regioni. Di conseguenza, in molti casi il sistema operava al di fuori
dei registri ufficiali. In un articolo del 2007, corredato da un ampio
elenco di citazioni e intitolato \emph{Misplaced Blame: Islam,
Terrorism, and the Origins of Hawala},\footnote{Edwina Thomson,
  ``Misplaced Blame: Islam, Terrorism and the Origins of
  \emph{Hawala}.''} Edwina Thompson riporta le osservazioni di Richard
Grasshoff nel classificare il rapporto tra hawala e suftaja:

\begin{quote}
Grasshoff dimostra che il termine hawala si riferisce al concetto
giuridico di delega del debito, piuttosto che a un'applicazione
concreta, mentre il termine suftaja, al contrario, indica una cambiale,
intesa come uno degli strumenti commerciali possibili derivanti dal
sistema hawala. {[}\ldots{]} Tecnicamente, si potrebbe quindi
argomentare che i clienti operano a livello del suftaja, mentre i
commercianti gestiscono in modo più accurato un sistema fondato sul
hawala.\footnote{Thomson, ``Origins of \emph{Hawala},'' 294.}
\end{quote}

Successivamente, i contatti economici e militari tra musulmani e
cristiani diffusero l'uso di queste tecnologie e di sistemi monetari
affini in Europa. In un rapporto della Federal Reserve Bank di Atlanta
dal titolo \emph{The Evolution of the Check as a Means of Payment: A
Historical Survey}, gli autori Stephen Quinn e William Roberds
descrivono la diffusione delle innovazioni nei sistemi di pagamento nel
seguente modo:

\begin{quote}
Le cambiali sembrano essere state di uso comune nel Mediterraneo
orientale durante il primo millennio. Entro il decimo secolo, esse erano
ampiamente impiegate nel mondo musulmano (Ashtor 1972). Al contrario, i
sistemi monetari in Europa in quel periodo erano estremamente
rudimentali: esistevano pochissime monete di valore affidabile e non si
contavano le banche, per non parlare delle cambiali (Usher 1934;
Spufford 1988).
\end{quote}

\begin{quote}
Durante le Crociate, l'incremento del contatto tra europei e mondo
musulmano portò all'adozione, con opportune modifiche, dei sistemi
bancari e monetari allora in voga nel Mediterraneo orientale. Nel
tredicesimo secolo comparvero banche rudimentali nelle città commerciali
di Barcellona, Firenze, Genova e Venezia, il cui scopo principale era
facilitare i pagamenti tra mercanti locali, anziché fornire
credito.\footnote{Stephen Quinn and William Roberds, ``The Evolution of
  the Check as a Means of Payment: A Historical Survey,'' 2.}
\end{quote}

In tale contesto, anche l'ordine dei monaci guerrieri cattolici del XII
secolo, noti come Cavalieri Templari, sembra aver accolto e adattato
queste pratiche.\footnote{History.com Editors, ``Knights Templar,''
  \emph{History.com}, 13 luglio 2017.} Con sede a Gerusalemme, i
Templari gestivano una rete estesa che agevolava le Crociate dei
cristiani contro i musulmani. I nobili europei, intenzionati a
partecipare a queste spedizioni, potevano depositare in Europa i loro
beni preziosi presso i Templari, ricevendo in cambio un documento
specifico, per poi riscattarlo in Gerusalemme ottenendo una somma
equivalente a quella depositata.

Alla fine di questo capitolo ci poniamo la domanda: ``chi controlla il
registro?'' In questi sistemi la risposta è che il controllo del
registro è detenuto dai hawaladar -- unità che comprendono anche vari
mercanti, templari e altri proto-banchieri che operano attraverso canali
specifici. Gli utenti della rete devono affidarsi al corretto
funzionamento di ogni singolo hawaladar, e questi, a loro volta, devono
riporre fiducia gli uni negli altri.

\section{Footnotes}\label{footnotes-5}

\bookmarksetup{startatroot}

\chapter{Capitolo 6: L'innovazione della contabilità a partita
doppia}\label{capitolo-6-linnovazione-della-contabilituxe0-a-partita-doppia}

Nel 1494, Luca Pacioli d'Italia scrisse \emph{Summa de arithmetica},
un'opera in cui, tra le altre cose, forniva una descrizione dettagliata
della contabilità a partita doppia. Questo testo lo rese noto come il
``Padre della Contabilità'', poiché il suo contributo rivoluzionò le
pratiche contabili e il sistema bancario in tutta Europa.\footnote{Alan
  Sangster et al., ``The Market for Luca Pacioli's \emph{Summa
  Arithmetica}.''}

La contabilità a partita doppia si fonda sulla divisione del libro
mastro in due sezioni, le quali si riconciliano reciprocamente. Ad
esempio, se un soggetto prende in prestito 10 monete d'oro da una banca,
tale importo verrà contabilizzato come passività per il debitore e,
parallelamente, come attivo per l'istituto bancario. In questo modo,
ciascuna parte del registro, relativa rispettivamente al soggetto e alla
banca, si bilancia, con le attività di uno che corrispondono ai passivi
dell'altro. Questo sistema ha permesso lo sviluppo di meccanismi
finanziari molto più complessi di quelli possibili prima dell'adozione
di questa tecnica, consentendo alle banche di gestire insiemi articolati
di attivi e passivi e di offrire servizi finanziari elaborati.

Va precisato, tuttavia, che Pacioli non fu l'inventore della contabilità
a partita doppia. Già in alcune zone d'Italia, prima della pubblicazione
della sua opera, si praticavano forme embrionali della tecnica, e parte
del materiale da lui presentato fu ripreso dal collega Piero della
Francesca. Inoltre, tecniche analoghe erano state utilizzate
precedentemente dai mercanti islamici -- ad esempio, attraverso il
sistema hawala -- e giunsero in Italia mediante i contatti commerciali,
come già descritto nel capitolo precedente. Se si guarda ancora più
indietro, è possibile rintracciare gli sviluppi del sistema numerico
indiano, che resero operativa la contabilità a partita doppia, e le
prime forme di contabilità risalgono addirittura alla Mesopotamia, come
illustrato nel primo capitolo dedicato ai registri contabili.
L'organizzazione e la diffusione di queste tecniche operata da Pacioli
assunse un'importanza storica fondamentale, contribuendo a
standardizzare e a propagare la prassi contabile da quel momento in poi.

Queste innovazioni, inoltre, ebbero un impatto rilevante sul
miglioramento dei sistemi di pagamento a Venezia, Firenze e in altre
regioni dell'Italia moderna. I prestatori di denaro esistevano da
millenni, ma i contabili delle città-stato italiane elevarono tale
attività a un livello completamente nuovo, dando origine al moderno
sistema bancario. Il termine ``bank'' deriva infatti dall'italiano
\emph{banco}, che significa ``panca'' o ``banco di legno''. Venezia e
Firenze godevano di un commercio relativamente libero e aperto rispetto
ad altre regioni europee e vantavano rapporti commerciali significativi
con i mercanti arabi. I contabili sedevano su vere e proprie panche
nelle piazze dei mercanti, svolgendo il ruolo di banchieri per la classe
mercantile.\footnote{Ugolini, \emph{Central Banking}, 11.}

Lo sviluppo del sistema bancario ha permesso ai mercanti di non doversi
più munire di ingenti quantità di monete, riducendo così gli attriti e i
rischi legati agli scambi commerciali. Se due mercanti possiedono un
conto presso lo stesso banchiere, possono concludere una transazione
semplicemente comunicando al banchiere di aggiornare il libro mastro.
Quest'ultimo, infatti, provvederà a modificare i saldi dei due clienti,
detraendo una somma dall'acquirente, aggiungendola al venditore e
trattenendo una commissione per il servizio offerto. Nel caso in cui un
mercante desideri regolare i propri conti con il banchiere mediante il
prelievo o il deposito di oro fisico, ciò potrà avvenire con una cadenza
minore e in condizioni di maggiore sicurezza. Da qui in poi, i banchieri
potettero combinare tale prassi con l'emissione di banconote,
evolvendosi in grandi istituzioni finanziarie.

Strumenti finanziari cartacei erano già stati concepiti in diverse
regioni. Come descritto nel capitolo precedente, papiro e cambiali
basate su supporto cartaceo venivano utilizzati sin dall'Antico Egitto e
dalla Cina antica, nonché lungo la Via della Seta. Queste prime cambiali
erano generalmente associate a una persona specifica. Per fare un
esempio semplificato, si possedeva una ricevuta cartacea che dichiarava:
``Lyn Alden ha diritto al pagamento di cinque once d'oro da parte di
XYZ''. Solo il beneficiario, o un incaricato autorizzato legalmente a
suo nome, poteva presentare tale documento per prelevare l'oro da quel
determinato individuo.

Col tempo, molti di questi documenti si trasformarono, per ragioni di
comodità, in titoli al portatore con importi standardizzati: ciò
significava che chiunque fosse in possesso del documento poteva
richiedere l'oro.\footnote{Markus Denzel, ``The European Bill of
  Exchange.''} Sempre in termini semplificati a titolo esemplificativo,
un documento del genere reciterebbe: ``Il portatore di questa banconota
ha diritto a prelevare cinque once d'oro presso l'istituto di deposito
XYZ''. I mercanti potevano utilizzare tali titoli, sostituendo l'oro
fisico durante gli scambi, oltre a gestire i conti correnti presso le
banche. A differenza del sistema su canali tipico del suftaja/hawala,
l'accettazione di queste banconote basate su titoli al portatore divenne
diffusa in quanto mezzo di scambio generale e fu associata alla
reputazione di un'istituzione finanziaria di ampio rilievo.\footnote{Jim
  Bolton e Francesco Guidi-Bruscoli, ``\,`Your Flexible Friend': The
  Bill of Exchange in Theory and Practice in the Fifteenth Century,''}

In ambito tecnologico dei pagamenti, possiamo riassumere la transizione
dal proto-banking al sistema bancario a servizio completo in tre fasi
principali, caratterizzate da crescenti livelli di ``negoziabilità''.
Nel linguaggio finanziario, il termine negoziabile indica che uno
strumento cartaceo può essere trasferito a una diversa controparte.
Nella prima fase, uno strumento semplice e non negoziabile poteva essere
riscattato per denaro solo da una parte specifica, come indicato nel
documento originario. Nella seconda fase, lo strumento, pur essendo
emesso a favore di una parte determinata (a cui era destinato il
riscatto), veniva reso negoziabile e quindi poteva essere fisicamente
girato a favore di un'altra parte, che ne avrebbe potuto poi procedere
al riscatto. Questo passaggio richiedeva una rete finanziaria più
complessa e di maggior fiducia, coinvolgendo un numero elevato di
controparte. Nella terza fase, uno strumento come una banconota era
intrinsecamente un titolo al portatore---privo dell'indicazione di un
nome specifico---e poteva essere liberamente scambiato tra le parti
senza necessità di apporre una firma o di effettuare ulteriori formalità
per il trasferimento della proprietà, ad eccezione del semplice possesso
fisico. Tale terza forma richiede e si fonda sulla presenza di
istituzioni vaste e ampiamente riconosciute.\footnote{Larry Neal,
  \emph{The Rise of Financial Capitalism}, 7--16; John Munro, ``Rentes
  and the European `Financial Revolution','' 236.}

L'integrazione dei conti correnti con le banconote, nonché l'espansione
dei sistemi cartacei non negoziabili basati su canali in sistemi diffusi
di titoli al portatore negoziabili, ha notevolmente migliorato, nel
tempo, la portabilità, la liquidità e la divisibilità effettiva
dell'oro. Grazie all'astrazione, il possesso legale dell'oro poteva ora
variare di frequente, indipendentemente dal movimento dell'oro fisico
sottostante.\footnote{Donald McCloskey e Richard Zacher, ``How the Gold
  Standard Worked.''} Tale sviluppo ha accresciuto la comodità e la
sicurezza nel trattare con ingenti somme di denaro, aprendo però anche
la porta al rischio di controparte e all'arbitraggio. Possiamo
visualizzare l'evoluzione di queste reti quasi come una versione
cartacea di Internet: le prime connessioni su canali si trasformarono
progressivamente in un complesso e interconnesso insieme di entità che
si riconoscevano reciprocamente e operavano in sinergia. Tutti questi
attivi cartacei rappresentavano pretese sull'oro, pur richiedendo la
fiducia che il custode dell'oro lo mantenesse in modo
responsabile.\footnote{Eichengreen, \emph{Exorbitant Privilege}, 15--16.}
Inoltre, le monete d'argento continuavano ad avere utilità in quel
periodo, poiché il processo bancario comportava costi considerevoli,
rendendolo pertanto inadatto a tutti --- in particolare a coloro
appartenenti alla fascia inferiore della distribuzione della ricchezza.

In molte regioni dell'Europa medievale era possibile, in una certa
misura, mantenere una contabilità in partita doppia anche tramite l'uso
dei tally sticks. Se un creditore concedeva un prestito a un debitore,
venivano annotati i dettagli dell'operazione mediante una serie di segni
tracciati su un apposito bastoncino di legno, che veniva poi spezzato
longitudinalmente in due metà. Creditore e debitore conservavano
ciascuno la propria metà, in modo da poterle ricongiungere e dimostrare
così che nessuna delle due era stata alterata. I tally rappresentavano
un ulteriore metodo di tenuta dei registri contabili, caratterizzato da
un'elevata resistenza alle manomissioni, sebbene dal punto di vista
logistico risultassero poco efficienti per il creditore, che era
costretto a custodire numerosi bastoncini individuali.\footnote{Martin
  Slater, \emph{The National Debt: A Short History}, 15--44.}

\textbf{Una tendenza verso il sistema di riserva frazionaria}

Il modello bancario più elementare si configura come un custode al 100\%
garantito dagli asset. Le persone depositano oro o altri beni monetari,
e la banca custode emette per questi depositi delle credenziali
cartacee; l'istituto, infatti, non fa altro che conservarli in
sicurezza, addebitando in cambio una commissione per il servizio
offerto.

Esempi attuali analoghi includono la custodia fisica dell'oro in
cassaforte, per la quale viene generalmente richiesto il pagamento di
una tassa periodica, o l'utilizzo di una cassaforte di deposito, così
come la gestione di un exchange-traded fund composto da azioni che
comporta una tariffa amministrativa. Questi servizi rappresentano
tipologie di custodia e amministrazione a riserva piena: anziché
produrre reddito prestando i propri asset -- mettendo così a rischio la
restituzione -- l'istituzione si limita a conservarli, remunerandosi
attraverso commissioni che coprono i costi di gestione e le garantiscono
un utile.

In un mercato libero, infatti, le banche competono naturalmente per
quota di mercato applicando diversi livelli di commissione.
Inevitabilmente, però, gli operatori osservano che la maggior parte
dell'oro depositato non viene ritirata in un'unica soluzione, ma permane
in custodia. Immaginate, ad esempio, un banchiere che, riflettendo su
dieci anni di attività della propria banca a riserva piena, constata che
il prelievo aggregato massimo effettuato dai clienti non ha mai superato
il 40\% dell'oro custodito. Di conseguenza, se si dispone di almeno
l'80\% dell'oro a portata di mano, la situazione può considerarsi
sicura. Il banchiere decide così di impiegare il restante 20\%
concedendolo, in modo ponderato, a prestito a fronte di interessi,
ottenendo in tal modo ulteriori profitti e potendo offrire un servizio
privo di commissioni, con il risultato di attrarre un numero sempre
maggiore di depositi. È in questo modo che si è sviluppato il sistema
della banca a riserva frazionaria.

Se una banca non informa i propri clienti delle operazioni in corso, ciò
configura una frode, poiché questi non sono al corrente che il 20\% del
loro oro viene prestato e messo a rischio. Se, invece, tale operazione
viene esplicitata e i clienti acconsentono, si tratta allora di una
scelta consapevole. Dal punto di vista di un potenziale depositante, può
apparire ragionevole che il gestore disponga l'80\% dei depositi in
forma liquida e investa il restante 20\% in prestiti illiquidi per
generare un reddito extra ed eliminare le commissioni. Se questo scambio
risponde alle esigenze di mercato, altri istituti saranno sicuramente
costretti a operare con riserva frazionaria al fine di eliminare le
spese, pur mantenendo in circolazione banche a riserva totale, basate su
commissioni, per servire i soggetti più avversi al rischio, consapevoli
dei problemi che tale sistema può comportare.

Se la maggior parte degli istituti adotta questa modalità, il risultato
sarà che le pretese sull'oro presenti sul mercato supereranno la
quantità fisica disponibile. Inizialmente, l'espansione del credito
probabilmente stimolerà un boom economico, tanto da guadagnarsi il
favore dei governanti locali, che potrebbero persino incentivare tale
prassi.

Tuttavia, le banche possono spingersi oltre. Se il regime con l'80\% di
riserva e l'eliminazione delle commissioni risulta attraente, perché non
ipotizzare una situazione in cui le riserve si riducano al 60\% e,
anziché limitarsi a cancellare le spese, la banca offra ai depositanti
una piccola quota dei profitti derivanti dai prestiti, sotto forma di
interessi sui depositi? Sicuramente ciò attirerebbe una massa imponente
di depositi, ma quali sono le probabilità che, nel complesso, i clienti
tentino contemporaneamente di ritirare più del 60\% del loro oro? In un
tale sistema, i banchieri tenderanno a spingere continuamente i limiti,
riducendo progressivamente la percentuale di riserva rispetto ai
depositi, così da premiare---in maniera consapevole o meno---coloro che
si assumono maggiori rischi con l'attività di prestito custodiale.
Questa struttura è intrinsecamente instabile, in quanto si fonda sulla
falsa promessa che i depositanti a vista possano prelevare il proprio
denaro in qualsiasi momento, anche se, se molti tentassero di farlo
simultaneamente, non sarebbe possibile soddisfare tale esigenza.

È importante sottolineare che una singola banca a riserva frazionaria,
ammesso che non sia insolvente a causa di prestiti inadempienti,
possiede comunque un ammontare di attivi almeno pari alle sue passività;
il problema risiede nel fatto che non tutti tali attivi sono in forma
liquida e immediatamente disponibili per il prelievo. Il vero problema
emerge infatti in un sistema finanziario composto da molteplici banche a
riserva frazionaria: in esso l'ammontare complessivo dei depositi supera
abbondantemente la quantità di oro fisico di base. Il denaro prestato da
un istituto può essere depositato presso un altro e immediatamente (e
parzialmente) prestato nuovamente, generando così un fenomeno di doppia,
tripla, quadruplice---e così via---conteggiatura dei depositi rispetto
alla base monetaria. A quel punto, le richieste di riscatto in oro
superano la quantità effettivamente esistente, rendendo illusoria la
ricchezza dei depositanti. Ciò conferisce al sistema una naturale
instabilità, facendolo sprofondare facilmente in una serie di corse agli
sportelli, in cui il fallimento in una banca può rapidamente innescare
il tracollo di molte altre. In sostanza, il sistema bancario a riserva
frazionaria somiglia a un gioco delle sedie musicali: funziona per un
po', ma non appena la musica si ferma, tutto può crollare in maniera
repentina.

A complicare ulteriormente il quadro, gli incentivi che regolano le
corse agli sportelli nelle banche a riserva frazionaria sono ben più
problematici di quanto sembri a una prima occhiata. Immaginiamo che una
banca, nell'ambito della propria struttura patrimoniale, eroghi prestiti
e che alcuni di questi entrino in default. In tal caso, la banca avrebbe
coperto solo il 90\% dei depositi con oro o altri attivi, perdendo il
restante 10\% a causa dei prestiti inadempienti. In un primo momento,
questa situazione potrebbe non sembrare eccessivamente critica: i
depositanti godranno degli anni trascorsi con basse commissioni o
addirittura con una ripartizione degli interessi, accettando una perdita
contenuta del 10\% dovuta a una cattiva gestione del rischio. Purtroppo,
però, se la situazione viene lasciata a se stessa, le conseguenze
diventeranno tutt'altro che benigne. Non appena alcuni depositanti ben
collegati percepiscono segni di insolvenza, ritireranno rapidamente i
propri fondi. Il fenomeno, a sua volta, innesca il ritiro collettivo:
chi osserva tali movimenti seguirà l'esempio. Senza misure correttive,
l'intero sistema di depositi rischia di essere svuotato, lasciando
coloro che esitano al prelievo senza alcun capitale residuo. Inoltre,
non tutti subiscono una perdita del 10\% in modo uniforme: chi preleva
anticipatamente può evitare la perdita, mentre chi agisce per ultimo
rischia di perdere tutto, poiché, a quel punto, le riserve saranno
esaurite. Questi incentivi intrinseci, pertanto, favoriscono le corse
agli sportelli al minimo segnale di insolvenza, premiando coloro che
riescono a ritirare i depositi per primi. A seguito delle ripetute
crisi, le autorità hanno dovuto implementare schemi regolatori e
assicurativi per distribuire il rischio e scoraggiare tale comportamento
da parte dei depositanti.

Questi problemi di stabilità possono essere affrontati in maniera più
fondamentale mettendo in perfetto allineamento le durate dei depositi e
dei prestiti. In un sistema di questo tipo, i ``depositi a vista'' e le
banconote devono poter essere ritirati o riscattati in qualsiasi momento
e, pertanto, devono essere completamente garantiti da oro. Al contrario,
i certificati di deposito vincolano i fondi dei depositanti per periodi
più lunghi, configurandosi come contratti d'investimento o ``depositi a
termine'', e permettono alla banca di concedere prestiti con la stessa
durata oppure inferiore. Questo metodo evita di assumere impegni di
liquidità nei confronti dei depositanti a vista che, in caso contrario,
potrebbero rivelarsi insostenibili, e impedisce l'eccessiva riipoteca e
il disallineamento delle durate su cui si fonda il sistema bancario a
riserva frazionaria. Tuttavia, nella pratica le società non hanno
adottato sistematicamente tale approccio. Storicamente, i banchieri --
insieme a regolatori e clienti -- hanno preferito il sistema di riserva
frazionaria, convivendo con il disallineamento sottostante delle durate
e confidando nel fatto che non si verificasse un contingente eccessivo
di richieste di riscatto simultaneo. La pratica pervasiva della riserva
frazionaria ha, infatti, contribuito in misura notevole alla frequenza
delle crisi finanziarie in ogni giurisdizione.

Le banche moderne, solitamente, detengono dai 5 ai 10 volte l'ammontare
dei depositi rispetto alle loro riserve liquide, il che equivale a una
leva finanziaria del 500\% o del 1.000\%, mentre il resto del loro
patrimonio è composto da vari titoli e prestiti. Inoltre, la maggior
parte delle riserve liquide non si presenta sotto forma di denaro
fisico, ma come riserve astratte detenute presso la banca
centrale.\footnote{Federal Reserve Economic Data, ``Deposits, All
  Commercial Banks''; ``Cash Assets, All Commercial Banks''} Basta
infatti che soltanto una piccola frazione di clienti decida di ritirare
i propri fondi per generare una carenza di liquidità. A differenza di un
sistema bancario sostenuto dall'oro, in questo regime moderno la banca
centrale può creare ulteriore base monetaria ogniqualvolta sia
necessario, contrastando così le corse agli sportelli attraverso una
diluizione universale del denaro.\footnote{Michael McLeay et al.,
  ``Money Creation in the Modern Economy,'' 21--25.}

La Figura 6-A illustra il rapporto storico tra i depositi delle banche
commerciali statunitensi e la cassa liquida detenuta da tali istituti.
Dagli anni '80 fino al 2008, il rapporto è aumentato da circa 6 a 23, il
che significa che al suo picco esisteva un dollaro di cassa per ogni 23
dollari depositati, equivalente a una leva del 2.300\%. Successivamente,
con lo scoppio della crisi finanziaria globale, la Federal Reserve ha
immesso una notevole quantità di nuova cassa bancaria per acquistare
attività dalle banche, facendo così scendere in maniera sostanziale il
rapporto depositi/cassa, che da allora si è attestato intorno a 5--6
volte (ossia il sistema risulta essere ``solo'' levato al 500\%--600\%).

Figura 6-A\footnote{Federal Reserve Economic Data, ``Deposits, All
  Commercial Banks''; ``Cash Assets, All Commercial Banks.''}

Alla fine del 2022, le banche statunitensi detenevano depositi per circa
18 trilioni di dollari che dovevano ai clienti, mentre possedevano poco
più di 3 trilioni di dollari in liquidità immediata.

Questo dato appare (leggermente) meno inquietante se si considera che,
alla stessa data, le banche vantavano attivi complessivi per circa 22,6
trilioni di dollari, di cui 3 trilioni rappresentavano denaro liquido.
Gli altri attivi erano costituiti da mutui, titoli di Stato e prestiti
alle imprese. Nel complesso, le banche possiedono un patrimonio
superiore alle passività, ma la maggior parte dei loro attivi è data da
crediti e titoli piuttosto che da denaro contante a disposizione. Il
sistema finanziario, infatti, non sarebbe in grado di sostenere una
ritirata massiccia e simultanea da parte di una buona fetta dei
depositanti: se tale evenienza dovesse verificarsi, le banche
negherebbero l'accesso ai fondi.\footnote{Jiang et al., ``U.S. Bank
  Fragility.''} In realtà, alla fine del 2022, le banche negli Stati
Uniti disponevano solamente di circa 100 miliardi di dollari in denaro
fisico; il resto della liquidità si presentava sotto forma di riserve
bancarie intangibili, registrate come attivi sul libro mastro della
banca centrale. Durante il 2023 si sono verificati, infatti, alcuni casi
di corse agli sportelli contro istituti che avevano gestito male parte
dei propri attivi durante il rapido inasprimento della politica
monetaria da parte della Federal Reserve, causando alcuni dei più
imponenti fallimenti bancari della storia americana.

La Parte 4 di questo libro offre ulteriori esempi di prestiti a riserva
frazionaria e di creazione di depositi, nel contesto di un sistema
bancario basato su valuta fiat, anziché su una copertura aurea. Per il
momento, concentrandoci sui sistemi bancari tradizionali garantiti
dall'oro, possiamo porci la domanda: ``Chi controlla il libro mastro?''

La risposta è che ogni banca gestisce il proprio sottomastro.\footnote{Michael
  McLeay e colleghi, \emph{Money Creation in the Modern Economy},
  pp.~18--20.} Quando i clienti depositano fondi presso una banca, essi
confidano nell'etica e nella competenza dell'istituto, assicurandosi che
il denaro venga custodito integralmente (come avviene nei servizi di
custodia a riserva piena) oppure, se l'istituto si espone al rischio,
che esso sia assunto con prudenza (come nel sistema a riserva
frazionaria).

Inoltre, si può affermare che il governo esercita un controllo parziale
sul libro mastro complessivo. Poiché il numero di banche presenti in un
Paese è limitato, le autorità statali possono facilmente rivolgersi a
ciascun istituto per impartire ordini --- ad esempio, congelare o
sequestrare i depositi di un individuo. La banca è tenuta a conformarsi
a questi comandi, indipendentemente dal fatto che il provvedimento sia
giustificato o meno. Il governo potrebbe, ad esempio, prendere di mira
una persona per motivi religiosi, politici, legati all'orientamento
sessuale o per aver espresso verità scomode, situazione che sarebbe
estremamente deplorevole; oppure potrebbe procedere contro un furfante o
truffatore, operando in maniera ragionevole nell'ambito dell'azione
legale. Inoltre, un governo può costringere tutte le banche della
propria giurisdizione a consegnare l'oro in loro possesso a un'autorità
centrale, rilasciando in cambio delle note contabili, come è accaduto
inevitabilmente in vari Paesi nel corso della storia. È decisamente più
semplice per lo Stato ottenere l'oro da un ristretto numero di banche
piuttosto che da ogni singola famiglia.

Nel complesso, è la combinazione tra il potere detenuto dalle banche e
quello dello Stato sul registro contabile che costituisce la forma di
moneta con cui la maggior parte delle persone opera in qualsiasi sistema
bancario. In un sistema bancario garantito dall'oro, l'unica parte del
registro sulla quale gli utenti individuali esercitano un controllo
diretto sono le monete in metallo prezioso che conservano personalmente;
per mantenere l'integrità del registro, essi si affidano alle proprietà
intrinseche di quel bene naturale. Non appena tali monete vengono
consegnate al sistema bancario, il controllo del denaro passa
inevitabilmente a una gerarchia di intermediari, con cui si stabilisce
un affidamento crescente sul loro operato.

\section{Footnotes}\label{footnotes-6}

\bookmarksetup{startatroot}

\chapter{\texorpdfstring{Capitolo 7: \textbf{Free Banking vs Central
Banking}}{Capitolo 7: Free Banking vs Central Banking}}\label{capitolo-7-free-banking-vs-central-banking}

Nel corso dei secoli si sono sperimentati diversi modelli bancari. Nel
contesto di questo capitolo, è possibile classificare i sistemi bancari
nazionali in due categorie principali: il free banking e il central
banking.

Le banche descritte nel capitolo precedente sono esempi tipici di free
banks. Queste istituzioni detengono la moneta base (ad esempio, l'oro) e
accusano passività nei confronti dei depositanti, i quali detengono
delle pretese sulla moneta stessa. Inoltre esse possono emettere
banconote che incarnano tali pretese, fungendo da strumenti al portatore
non assegnati a singoli individui, ma che garantiscono al possessore
l'accesso diretto alla moneta base. Le singole banconote circolano come
vera e propria valuta al portatore e vengono utilizzate solo
occasionalmente per prelevare la moneta sottostante dalla
banca.\footnote{Quinn, ``Goldsmith Banking.''}

La Figura 7-A mostra un esempio di sistema di free banking a riserva
piena supportato da oro.

Figura 7‑A

Nella maggior parte dei contesti, tuttavia, le banche libere adottavano
il modello a riserva frazionaria. Esse raccoglievano depositi in oro,
conservandone una quota nei propri caveau e concedendo in prestito il
resto, applicando un interesse. Grazie ai guadagni derivanti da tali
prestiti, la banca poteva ridurre le commissioni o corrispondere
interessi ai depositanti. Tale sistema, però, comportava rischi sia per
l'istituto stesso che per i clienti, poiché un'improvvisa richiesta
massiccia di restituzione dell'oro avrebbe costretto la banca a
sospendere i prelievi.\footnote{Lawrence White, \emph{The Theory of
  Monetary Institutions}, 57--73.}

La Figura 7‑B mostra un esempio di sistema bancario libero garantito da
oro con una riserva frazionaria pari al 50\%.

Figura 7‑B

La risposta alla domanda «chi controlla il registro?» in questo scenario
è data da una combinazione fra la natura e le singole banche. In un
sistema bancario libero fondato sull'oro o su altra moneta di base
naturale, la quantità di moneta fondamentale presente in un Paese è
determinata dalle forze geologiche e dai flussi del commercio
internazionale. L'offerta d'oro può aumentare con la scoperta di nuovi
giacimenti (come accadde durante la corsa all'oro in California), oppure
essa può espandersi o restringersi a seconda dei surplus o dei deficit
commerciali. Se, nel complesso, la popolazione di un Paese continua ad
acquistare più di quanto venda all'estero, il Paese registrerà un
deficit commerciale strutturale e l'oro fluirà costantemente verso le
casse dei partner commerciali stranieri. Viceversa, se il Paese è
altamente produttivo e i suoi abitanti esportano più beni e servizi di
quanti ne importino, si verificherà un surplus commerciale strutturale,
con un corrispondente afflusso continuo di oro.

Nel corso dei secoli XVIII e XIX il free banking era diffuso in numerosi
Paesi, ottenendo notevole successo in Canada, Svezia e Scozia,
nonostante il ricorso al duration mismatching. L'era turbolenta del free
banking negli Stati Uniti, compresa tra gli anni '30 e '60 del XIX
secolo, fu segnata da numerosi fallimenti bancari e viene spesso citata
come prova dell'insuccesso del free banking come concetto; tuttavia,
quel periodo rappresentò una fase limitata e problematica nell'ambito
della più ampia e globale storia del free banking, che generalmente si
può definire come avendo conseguito risultati misti. Il libro del 1988
di George Selgin sulla teoria del free banking fornì importanti spunti
di ricerca sulle pratiche del free banking del XIX secolo,\footnote{George
  Selgin, \emph{The Theory of Free Banking: Money Supply Under
  Competitive Note Issue}.} mentre Selgin ha continuato a produrre, per
decenni, ulteriori studi -- tra libri e articoli accademici --
sull'argomento fino ai giorni nostri.

Nel suo saggio sulla storia del denaro, Glyn Davies ha esposto esempi di
regolamentazioni del sistema bancario libero suddivisi per stato. Ad
esempio, il Massachusetts consentiva a quasi chiunque di fondare una
banca, applicando requisiti minimi, mentre a New York erano imposti
standard più rigorosi, comprensivi di requisiti di capitale e
dell'obbligo di riservare in monete preziose almeno il 12,5\% delle
banconote in circolazione. In Louisiana, inoltre, si prevedeva un
vincolo ancora più stringente, richiedendo che le banconote fossero
garantite da una riserva pari ad almeno un terzo in monete preziose,
unitamente ad altre regole relative al capitale e alla
liquidità.\footnote{Davies, \emph{A History of Money}, pp.~461--468,
  482.}

In netto contrasto con il free banking, la banca centrale standardizza e
centralizza il registro nazionale e le banconote. In tale sistema, una
banca centrale riconosciuta o istituita dal governo funge da referente:
ogni banca adotta il registro contabile della banca centrale come base
per la propria moneta. Piuttosto che detenere le proprie riserve sotto
forma di oro custodito in cassaforte, le banche le registrano come voci
contabili sul registro della banca centrale, mentre quest'ultima -- o il
governo ad essa collegato -- detiene l'oro, almeno se, per il momento,
il sistema è ancora garantito dall'oro, come in passato. Così la banca
centrale sostituisce le banche individuali nell'emissione delle
banconote.\footnote{Goodhart, \emph{Evolution}, pp.~85--99; Lawrence
  White, \emph{Free Banking in Britain: Theory, Experience, and Debate,
  1800--1845}, pp.~21--63.}

La banca centrale, a sua volta, possiede sia attività che passività. Le
sue passività sono costituite principalmente dalle riserve che le banche
individuali vi depositano, nonché dalle banconote in circolazione
emesse. Le attività, invece, possono variare a seconda del periodo
storico: in passato, ad esempio, l'oro costituiva la principale fonte
attiva, mentre nell'era moderna fiat le banche centrali impiegano
principalmente titoli di Stato.

Figura 7‑C mostra un esempio di sistema bancario centrale a riserva
frazionaria supportato dall'oro.

Figura 7‑C

In questo contesto, la banca centrale stabilisce principalmente
l'ammontare della moneta base (la somma di valuta fisica e riserve
bancarie) presente nel sistema. In caso di crisi, essa può immettere
ulteriore moneta base e svolgere il ruolo di prestatore di ultima
istanza grazie a tale flessibilità. In pratica, la banca centrale gode
di una certa autonomia nel decidere quale percentuale della moneta base
debba essere effettivamente garantita dall'oro fisico, purché il sistema
funzioni correttamente. In molti contesti, durante la fine del XIX
secolo e l'inizio del XX secolo, le banche centrali erano obbligate a
mantenere almeno il 35\%--40\% della loro base monetaria coperta da oro,
cercando pertanto di rispettare costantemente tale parametro\footnote{David
  Wheelock, ``Monetary Policy in the Great Depression,'' pp.~14--19.}.
In termini semplificati, se troppi soggetti ritiravano oro dal sistema,
la banca centrale poteva aumentare i tassi d'interesse (più precisamente
i tassi di sconto) per incentivare il rimpatrio del metallo, sia in
ambito nazionale che dall'estero. Al contrario, disponendo di una
sostanziosa quantità d'oro, la banca centrale era in grado di ridurre i
tassi (tassi di sconto) al fine di stimolare l'espansione del credito e
la crescita economica.

Il sistema bancario centrale si articola in molteplici livelli di
astrazione e centralizzazione. I depositanti individuali, infatti,
risultano complessivamente impotenti in questo tipo di assetto, e ogni
banca, nella realtà, è priva di potere decisionale poiché la totalità
delle sue attività si riduce ad semplici crediti (IOU). Il potere
risiede infatti interamente nella banca centrale, la quale può essere
persino soggetta al controllo del governo. Se ci poniamo la domanda
ricorrente -- ``chi controlla il registro contabile?'' -- notiamo come
la risposta possa variare notevolmente passando dal sistema di free
banking a quello di banking centrale.

In un sistema bancario libero garantito dall'oro, il controllo del
registro contabile è nelle mani della natura e delle banche individuali,
mentre lo Stato ha la facoltà, in misura variabile, di influenzare o
addirittura subentrare nelle banche. Le proprietà intrinseche della
natura continuano infatti a garantire la scarsità del metallo prezioso
sottostante, e gli individui possono comunque detenere direttamente
monete, lingotti e gioielli in metallo prezioso. Nella misura in cui
essi versano denaro in una banca, quest'ultima controlla la relativa
porzione del registro.

In un sistema bancario centralizzato garantito dall'oro, invece, la
natura fornisce ancora scarsità al metallo prezioso, ma tale risorsa si
allontana progressivamente dalle operazioni quotidiane del sistema. Le
banche individuali hanno ormai un controllo marginale, poiché, anziché
custodire direttamente oro fisico, depositano le loro riserve come voci
nel registro della banca centrale. Esse possono subire perdite --
compresi i depositi dei clienti -- a seguito di prestiti deteriorati, ma
non hanno alcuna possibilità di influire sul valore delle proprie
riserve, essendo separati dall'oro sottostante da un ulteriore livello
di astrazione. Ora è la banca centrale a detenere il controllo del
registro dei depositi e delle riserve sottostanti su scala nazionale.
L'unico controllo diretto che gli individui mantengono riguarda la parte
di denaro da essi posseduta sotto forma di monete in metallo prezioso.

\textbf{A Brisk Walk Through American Monetary History}

Nel corso del tempo, i sistemi bancari hanno progressivamente tenduto
verso una maggiore centralizzazione. Questo fenomeno si è verificato in
vari paesi europei e, in seguito, Stati Uniti e altre nazioni hanno
seguito un percorso simile. La storia relativamente recente e continua
degli Stati Uniti offre un utile esempio di come la centralizzazione si
concretizzi.

La fondazione degli Stati Uniti e del suo sistema monetario avvenne in
fasi distinte. La guerra d'indipendenza ebbe inizio nel 1775, mentre nel
1776 fu redatta la Dichiarazione d'Indipendenza. Nel periodo bellico, il
Congresso Continentale emise banconote note come „Continentals'', che in
seguito subirono un'inflazione incontrollata.\footnote{Farley Grubb,
  ``The Continental Dollar: What Happened to It after 1779?''} La
Costituzione entrò in vigore nel 1789. Il Coinage Act del 1792 stabilì
il dollaro d'argento come unità di conto base negli Stati Uniti e
istituì lo U.S. Mint per emettere monete a corso legale standardizzate.
All'interno di tale normativa, un dollaro d'argento fu definito come
equivalente a 371,25 grani (24,1 grammi) d'argento, attorno al quale fu
strutturato un sistema decimale tri-metallico come segue:

\begin{itemize}
\item
  Eagles: sono composti da 247,5 grani d'oro e hanno un valore nominale
  di \$10,00.
\item
  Half eagles: contengono 123,75 grani d'oro e sono denominati \$5,00.
\item
  Quarter eagles: comprendono 61,875 grani d'oro e hanno un valore
  nominale di \$2,50.
\item
  Dollars: sono formati da 371,25 grani d'argento e corrispondono ad un
  valore di \$1,00.
\item
  Half dollars: consistono in 185,625 grani d'argento e sono valutati
  \$0,50.
\item
  Quarter dollars: 92,8125 grani d'argento, con valore nominale di
  \$0,25.
\item
  Dimes: 37,125 grani d'argento, con valore nominale di \$0,10.
\item
  Half dimes: 18,5625 grani d'argento, con valore nominale di \$0,05.
\item
  Cents: 264 grani di rame, con valore nominale di \$0,01.
\item
  Half cents: 132 grani di rame, con valore nominale di
  \$0,005.\footnote{U.S. Mint, \emph{Coinage Act of April 2, 1792}.}
\end{itemize}

La prima banca degli Stati Uniti fu istituita nel 1791, anche se essa
non rappresentava un'autentica banca centrale e operava in ambiti molto
ristretti, a causa del profondo dibattito allora in corso
sull'opportunità di istituire una banca centrale. Essa godeva di una
carta statutaria limitata a 20 anni, scaduta nel 1811. Nel 1816 nacque
la Seconda Banca degli Stati Uniti, anch'essa con funzioni circoscritte,
che operò per altri 20 anni fino al 1836.\footnote{Andrew Hill,
  \emph{The Second Bank of the United States}, 2--5.}

Contemporaneamente esistevano numerosi istituti bancari commerciali, i
quali erano autorizzati all'emissione di banconote. Questi istituti
utilizzavano metalli preziosi e altri beni come riserva, garantendo così
il riscatto delle banconote emesse. In questo quadro, il ruolo del
governo era essenzialmente quello di standardizzare e coniare la moneta
per l'unità di conto nazionale, lasciando agli istituti privati
l'incarico di emettere le banconote. Il denaro messo in circolazione
direttamente dallo Stato era vincolato dalla disponibilità di metalli
preziosi, poiché non si poteva semplicemente creare moneta dal nulla. Se
una banca decideva di utilizzare diversi tipi di garanzia per le proprie
banconote, essa doveva comunque essere in grado di soddisfare, su
richiesta, il riscatto in metallo prezioso per mantenere la propria
credibilità.

Gli Stati regolavano individualmente le banche presenti nei rispettivi
territori. Spesso, la gestione degli istituti bancari era soggetta a
elevati livelli di corruzione, visto che controllare l'emissione di
banconote a riserva frazionaria era estremamente redditizio. Inoltre, in
molti casi venivano imposte restrizioni all'apertura di filiali in più
stati, impedendo così alle banche di diversificare in modo ottimale
depositi e prestiti su scala geografica. Ciò le rendeva particolarmente
vulnerabili a corse agli sportelli e a crisi di liquidità e solvibilità.
Rispetto a quanto avveniva in Canada, Svezia e Scozia, il sistema di
free banking negli Stati Uniti era di gran lunga più limitato, e nel
complesso, questi istituti risultavano meno sicuri al di fuori di
specifiche giurisdizioni.\footnote{Kurt Schuler, \emph{The World History
  of Free Banking}, 14--37.}

Negli anni '60, durante la Guerra Civile Americana e sotto la guida del
Presidente Abraham Lincoln, il paese iniziò un processo di
centralizzazione del sistema bancario. I National Banking Acts del 1863
e del 1864 crearono una rete di banche nazionali soggette a
regolamentazioni più rigide, istituirono una carta di moneta nazionale
-- emessa dalle banche nazionali e parzialmente garantita da titoli di
Stato -- e ampliarono le facoltà del governo federale nell'emissione di
obbligazioni di guerra. In fase di costituzione, ogni banca nazionale
era costretta ad acquistare titoli governativi e a depositarli presso il
Comptroller of the Currency. Una successiva normativa del 1865 fece
praticamente scomparire le banconote statali, consolidando così il quasi
monopolio sull'emissione monetaria nelle mani delle banche
nazionali.\footnote{George Selgin and Lawrence White, \emph{Monetary
  Reform and the Redemption of National Bank Notes, 1863--1913}.}

Per finanziare la Guerra Civile americana negli anni '60, il governo
federale cominciò a emettere i ``greenbacks'' come moneta fiat --
inizialmente sotto forma di demand notes e successivamente come United
States notes. In questa fase, il governo statunitense sfruttava il
fenomeno del seigniorage, potendo emettere moneta e titoli a costo quasi
nullo. Ciò gli consentiva di assorbire i risparmi della popolazione -- a
condizione di mantenere un certo grado di credibilità e reputazione -- e
di indirizzarli verso lo sforzo bellico. Pur subendo fluttuazioni nel
loro rapporto col valore aureo, le caratteristiche monetarie dei
greenbacks venivano gestite con maggior disciplina rispetto ai
Continentals, evitando così fenomeni iperinflazionistici. Sul versante
opposto del conflitto, anche gli Stati Confederati d'America emisero
moneta fiat per convogliare i risparmi verso la guerra, ma la loro
esperienza portò a un'inflazione incontrollata quando persero il
conflitto.\footnote{Ben Baack, \emph{America's First Monetary Policy:
  Inflation and Seigniorage During the Revolutionary War}.}

Dopo la guerra, emerse un conflitto duraturo tra creditori e debitori. I
primi, rappresentanti soprattutto della più agiata classe finanziaria
del Nord‐Est, sostenevano la necessità di restringere il getto monetario
al massimo, abolendo i greenback fiat e demonetizzando l'argento, in
modo che il dollaro fosse ancorato esclusivamente all'oro. I debitori,
costituiti principalmente da agricoltori e da alcuni gruppi della classe
operaia che si organizzavano sotto il movimento del ``Free Silver'',
erano invece favorevoli a mantenere i greenback in circolazione e a
conservare sia oro che argento come moneta, consentendo così un'ampia
offerta di dollari.\footnote{Gretchen Ritter, \emph{Goldbugs and
  Greenbacks: The Antimonopoly Tradition and the Politics of Finance in
  America, 1865--1896.}} Questa situazione evidenziava un aspetto
fondamentale: i dollari erano semplicemente un'astrazione di valore,
riscattabile in cambio di \emph{qualcosa} che possedeva valore, il che
implicava che la definizione stessa di dollaro potesse essere modificata
favorevolmente da un gruppo politico attraverso la contrazione o
l'espansione della sua offerta. I risparmiatori e i creditori,
naturalmente, auspicavano un dollaro più forte, mentre i debitori, che
si trovavano in debito in dollari, preferivano un dollaro più debole. In
questo caso, prevalette la faccia dura della moneta, tanto che il
Coinage Act del 1873 e il Gold Standard Act del 1900 demonetizzarono
l'argento, posizionando il Paese su un regime aureo fino al 1933.

Numerose analisi sull'inflazione condannano l'emissione monetaria da
parte dei governi e delle banche centrali -- un fenomeno che, in realtà,
si verifica frequentemente nella modernità. Tuttavia, è altresì
importante prestare attenzione alla distruzione centralizzata della
moneta. Quando i risparmiatori depositano i loro fondi in un'unità di
conto che si aspettano di rimanere stabile, e tale unità viene
rapidamente stampata o ridefinita in modo da perderne potere d'acquisto
a opera di un'autorità centrale, si configura di fatto una violazione
contrattuale nei confronti dei risparmiatori. Al contrario, quando i
debitori contraggono un prestito in un'unità di conto che si aspettano
di essere stabile e questa viene distrutta o ridefinita in maniera da
aumentarne il potere d'acquisto, si verifica una violazione contrattuale
nei confronti dei debitori.

Con il Federal Reserve Act del 1913 venne istituita la terza banca
nazionale degli Stati Uniti, ovvero la prima vera banca centrale del
paese: il Federal Reserve System. Quest'ultimo fu concepito come un
sistema composto da dodici banche della Federal Reserve, di proprietà
delle banche commerciali ma soggette alla vigilanza di un consiglio di
funzionari nominati a livello federale. La legge conferiva al Federal
Reserve il potere di vigilare sul sistema bancario, di fungere da
prestatore di ultima istanza e di emettere banconote della Federal
Reserve. Il Federal Reserve deteniva oro tra i suoi attivi e gestiva il
registro che costituiva la base monetaria del Paese, mentre le banche
commerciali conservavano le loro riserve sotto forma di annotazioni
contabili nel registro della Federal Reserve.\footnote{Allen Meltzer,
  \emph{A History of the Federal Reserve, Volume 1: 1913--1951},
  pp.~65--68.}

Durante la Grande Depressione del 1933, il presidente Roosevelt firmò
l'Executive Order 6102, che rese reato -- punibile con fino a dieci anni
di carcere -- il possesso di oro da parte degli americani, salvo piccole
quantità destinate ad usi ridotti, come gli anelli di fidanzamento. Ai
cittadini fu ordinato di consegnare il proprio oro, ricevendo in cambio
il tasso di cambio fisso in vigore, pari a 20,67 dollari per oncia.
Nell'anno successivo, il Gold Reserve Act del 1934 proibì a tutte le
banche di riscattare i dollari in cambio di oro e impose al Federal
Reserve di consegnare tutto il proprio oro al Tesoro degli Stati Uniti.
Grazie a questa combinazione di misure, ingenti quantità di oro furono
consegnate dal pubblico al governo federale in cambio di dollari
cartacei e depositi bancari.\footnote{Henry Mark Holzer, ``How Americans
  Lost Their Right To Own Gold And Became Criminals in the Process.''}

Dopo il Gold Reserve Act del 1934, il governo federale deprezzò
nettamente il dollaro rispetto all'oro. Un'oncia d'oro passò da un
valore di 20,67 dollari a 35 dollari. Ciò permise di espandere la base
monetaria in termini di dollari rispetto alle riserve auree, causando
nel contempo una svalutazione dei risparmi denominati in dollari -- e
una riduzione del valore di numerosi debiti, incluso quello federale. Il
governo organizzò quindi la creazione del United States Bullion
Depository a Fort Knox, trasferendo in tale struttura le proprie riserve
auree.

Sebbene per circa quattro decenni fosse illegale per gli americani
possedere oro, molti di loro lo facevano in segreto e il divieto, nella
pratica, risultava poco applicabile. Per uno Stato è relativamente
semplice svuotare le banche del loro oro con la firma di un
provvedimento, mentre sequestrare le piccole quantità detenute nelle
abitazioni private richiederebbe un'operazione molto più onerosa e
draconiana. Di conseguenza, le autorità si accontentarono di ottenere
una percentuale significativa dell'oro, prelevandolo principalmente
dalle banche e da quegli individui che, per evitare il rischio quasi
nullo di essere scoperti, lo consegnarono volontariamente sin dai primi
tempi.

Dal 1933 fino al 1971, i dollari potevano ancora essere cambiati in oro
da parte delle banche centrali straniere, al nuovo tasso devalorizzato,
mentre ciò non era possibile per i cittadini americani né per le entità
private straniere. Nel 1971 gli Stati Uniti cessarono di onorare anche i
cambi esteri, rendendo il dollaro privo di qualsiasi garanzia in oro.
Dopo quel momento, il deprezzamento del dollaro accelerò notevolmente.
Negli anni '80 e '90, un'oncia d'oro valeva circa dieci volte più
dollari (tra 300 e 400 dollari, a seconda dell'anno) rispetto al
passato, a fronte di una rapida svalutazione della moneta. Negli anni
2010 e 2020, invece, il valore di un'oncia d'oro superava
abbondantemente i 1.000 dollari, raggiungendo talvolta cifre superiori
ai 2.000 dollari.

Figure 7‑D mostra l'evoluzione del potere d'acquisto del dollaro
statunitense nel tempo, misurato dall'inflazione dei prezzi aggregati,
con i momenti chiave opportunamente annotati.

Figure 7‑D

Come illustrato dalla Figura 7-D, la Guerra del 1812 e la Guerra Civile
degli anni '60 provocarono una svalutazione temporanea della moneta;
tuttavia, grazie ai successi militari, all'incremento della produttività
nazionale e al meccanismo del peg dollar-oro, tale svalutazione venne
successivamente invertita, ripristinando il precedente
livello\footnote{Alioth Finance, ``Inflation Calculator,'' U.S. Official
  Inflation Data.}. A partire dal 1913, con la creazione della Federal
Reserve, seguita dalle due guerre mondiali, dalla sospensione nel 1933
della convertibilità interna in oro e, nuovamente, dalla cessazione nel
1971 della convertibilità internazionale in oro, il dollaro uscì
bruscamente dalla fascia storica di riferimento. Eppure, nonostante
tutto ciò, il dollaro statunitense si attestò come la seconda valuta a
miglior rendimentoe a livello mondiale in tale periodo, mentre la
maggior parte delle altre valute si deprezzò ancor più rapidamente.

Nel periodo in cui il dollaro manteneva una stabilità di valore, i
cittadini potevano tranquillamente detenere banconote fisiche emesse da
banche di elevata qualità per lunghi intervalli temporali, senza
preoccuparsi di una svalutazione persistente; il ricorso a depositi
bancari finalizzati a percepire interessi era una scelta opzionale e non
una necessità imprescindibile. Al contrario, nel mondo post-1913 e, in
maniera ancor più marcata, in quello post-1971, con il deprezzarsi
continuo e accelerato del dollaro, diventa invivibile detenere ingenti
quantitativi di banconote al di fuori degli istituti di credito, a causa
della costante perdita di potere d'acquisto. Se un risparmiatore sperava
di tenere il passo con l'inflazione, doveva necessariamente depositare
il denaro in banca per ottenere interessi; sebbene i depositi bancari
abbiano performato al di sotto del tasso d'inflazione dal 1913 ad oggi,
ciò è comunque avvenuto in misura \emph{minore} rispetto alla detenzione
di banconote fisiche che non generano alcun interesse. Di conseguenza,
questo sistema intrinsecamente inflazionistico ha rafforzato il potere
delle banche, rendendole indispensabili affinché ciascuno depositasse la
maggior parte dei propri risparmi. Inoltre, ha incrementato la
possibilità per il governo di monitorare i saldi e le transazioni dei
conti, di riscuotere tasse e di bloccare fondi su richiesta, dato che la
maggior parte del denaro si trova in banca piuttosto che in forma fisica
e privata. In sostanza, una moneta inflazionistica richiede l'impiego di
controparti e di leva finanziaria per cercare di compensare l'effetto
dell'inflazione mediante il guadagno d'interessi, situazione del tutto
differente rispetto a quella di una moneta ``solida''.

Nel 1970 il Congresso approvò il Bank Secrecy Act, il quale imponeva
alle banche di presentare comunicazioni alle autorità ogniqualvolta i
clienti effettuassero operazioni superiori a 10.000 dollari nell'arco di
una giornata. All'epoca tale soglia corrispondeva a un importo superiore
al reddito annuo mediano, motivo per cui le segnalazioni erano piuttosto
sporadiche. Tuttavia, poiché la normativa non fu adeguata al variare
dell'inflazione, nel corso di cinque decadi il governo ha
automaticamente abbassato il limite minimo per le segnalazioni,
espandendo così anno dopo anno il proprio mandato di sorveglianza senza
necessità di ulteriori provvedimenti legislativi. La combinazione della
limitazione del numero di banconote fisiche in circolazione, del rendere
svantaggioso il possesso prolungato di banconote (a causa
dell'inflazione senza interessi) e del controllo sui depositi bancari si
è rivelata un efficace binomio di controllo e sorveglianza.

La Figura 7-E presenta una versione semplificata dell'attuale assetto
del sistema, in cui l'oro è stato sostituito dai titoli di Stato e dai
prodotti garantiti da mutui, che costituiscono gli asset principali
della Federal Reserve.

Figure 7-E

La combinazione di queste azioni evidenzia perché la questione ``chi
controlla il registro'' riveste un'importanza cruciale. Allo stesso
tempo, mostra quanto rapidamente possa variare la risposta a tale
interrogativo.

Quando le persone possiedono direttamente dei metalli preziosi, si
affidano alla natura stessa per amministrare il loro registro, poiché è
la scarsità del metallo, economicamente estraibile, a stabilire in quale
misura i loro risparmi possano conservare il potere d'acquisto. Inoltre,
la possibilità di consegnare questi metalli a terzi consente di
realizzare transazioni private e resistenti alla censura, senza dover
dipendere da controparti esterne. Tuttavia, chi decide di custodire le
proprie monete a casa rimane vulnerabile al rischio di furto fisico.

Nel momento in cui una parte dei fondi viene depositata presso custodi o
banche ``libere'' che conservano parte dell'oro, l'utente si affida sia
alla natura che all'istituto bancario nella gestione del registro. In
tal caso, bisogna avere fiducia non solo nella scarsità intrinseca
dell'oro, ma anche nella capacità della banca di amministrare con
correttezza il proprio portafoglio prestiti e di evitare frodi.
Contemporaneamente, l'utente rinuncia a parte della sua privacy: il
banchiere conosce il patrimonio e le controparti di chi si affida a lui,
così come lo fa il governo, entrambi in grado di sequestrare i fondi se
lo ritenessero opportuno. In cambio, l'utente ottiene una maggiore
comodità, compresa la possibilità di effettuare transazioni rapide e a
lunga distanza. Rimane comunque la facoltà di detenere una parte di oro
o argento in proprio possesso, il che consente di decidere
strategicamente come bilanciare il rischio e il rendimento tra il
possesso autonomo e quello affidato alla banca.

Ogni volta che i saldi sono definiti in una unità di conto ancorata a
un'altra entità, le autorità che ne controllano il riferimento (o peg)
possono determinare il destino sia dei risparmiatori sia degli
indebitati. Una modifica delle regole, che porti ad ancorare l'unità a
qualcosa di più scarso, può arrecare danni considerevoli ai debitori;
viceversa, ancorarla a un bene più abbondante può compromettere
gravemente i risparmiatori.

Quando il governo istituisce una banca centrale -- e ancor più se ne
vieta la detenzione dell'oro -- sottrae il potere monetario alla
collettività, concentrandolo quasi interamente nelle mani delle banche e
delle autorità statali. In questo scenario, i cittadini hanno una
capacità limitata di custodire autonomamente i propri asset, sia per la
loro scarsità che per la loro elevata liquidità, e sono costretti a fare
affidamento sul registro delle banche centrali; ciò li espone
inevitabilmente al rischio di svalutazione della moneta e li obbliga a
rinunciare a gran parte della loro privacy. Gli incaricati statali,
infatti, possono ora appropriarsi più facilmente del potere d'acquisto
dei risparmiatori -- non solamente mediante forme di tassazione
trasparenti, ma anche attraverso un'inflazione non trasparente
dell'offerta monetaria -- reindirizzando tali risorse verso il
raggiungimento dei propri obiettivi. Inoltre, i governi possono
monitorare e verificare con maggiore facilità le finanze di ciascun
individuo.

\section{Footnotes}\label{footnotes-7}


\backmatter


\end{document}
